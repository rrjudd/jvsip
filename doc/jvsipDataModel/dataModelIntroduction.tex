\section{Introduction}
In this document I am proposing a standard data model for VSIPL. To be fair this is my notion of what a data model is. If there is some formal document describing what a data model is I don't know about it although there is a data model wikipedia page. The goal is for the data model to be language (as in programing language) independent.
\\[6pt]
I think the data model presented here is the same as in \cvl{} but I will call it the \jv{} data model so that if nobody else wants to participate in the discussion I will have a data model I can use for my own research interests.  I don't want to limit myself to just what is defined in the VSIPL 1p3 specification but I will try to generalize the ideas so they may be extended.
\\[6pt]
I don't expect this model to necessarily be complete as described here but I do expect it to be a starting point.  There may be some fuzzy sections. There may be some stuff that other folks disagree with. There may be some parts of the model which only apply for a particular language or platform.  But until somebody does a document and some discussion occurs the model will not go anywhere except for \jv{}.  I do eat my own cooking so to speak.
\\[6pt]
There are four parts to the model. The first is the scalar definitions which define the type of data that will be stored. The second is the block which is a memory abstraction providing an interface to memory allowing data to be stored in a machine independent way. The third is a \Vw{} which provides methods to allow an index set to be defined on a subset of the data stored in a block. For \cvl{} views are generally called vectors, matrices, or tensors but in general we could define other types of views. The final part of the data model is the interface between the data stored for use by \cvl{} functions (in blocks) and other functions which are not part of \cvl{}.
\paragraph{Single Thread Model}
The model I am proposing is for a library supporting a single thread. Under the covers the library might have multiple threads working; but from the user's perspective the code is a single thread.  If for instance pthreads are used, or a parallel model like MPI, then  VSIPL objects need to be protected from thread contention or race conditions the same as for any other multithreaded program using.
\subsection{Purpose, Approach, Goal}
\paragraph{My purpose} for writing this document is to further the discussion of the VSIPL data model to see if we can come up with a common agreement for a generalized VSIPL data model which has enough flexibility for use with any language which one might want to write a VSIPL library for.  
\paragraph{My approach} will be to discuss what I consider the \cvl{} data model to be and add information and insight from what I have learned writing \pyjv{}.  I will attempt some basic UML class diagrams.  I do not know UML so people who know UML need to complain and correct if they think it is needed. Perhaps we can bootstrap some UML documentation for VSIPL in this manner if enough people participate.
\\[6pt]
I also have an example for a Strassen matrix product routine from Golub and Van Loan Matrix Computations algorithm 1.3.1.  I previously implemented that in C++ VSIPL.  I do not know if that example is current with the current VSIPL++ library and I have not been able to install a working library from the current distribution on my systems.  In any case at one time it worked and should still be a valid example for my purposes.  I also have a run-able example for \cvl{} and I used the \cvl{} example as a template for a \pyjv{} example.  So for the current 3 languages of interest I have a common example function that I will use to discuss the data model.
\\[6pt]
There will be some discussion paragraphs where I talk about things I know little about, like the data model for C++ VSIPL, and ask questions or make comments designed to (hopefully) foster some discussion and illumination from the C++ \ttbf{Implementors} and \ttbf{Users}.
\paragraph{Comment}
I have never had much luck with UML either reading it or producing it. I think it may be much overblown as a methodology for specifying systems.  But in case I am wrong, and because so many other seem to think it important, I will attempt some diagrams.  
\paragraph{My goal} is to move the discussion and try to get a few people involved who may be interested in the topic.  If we want to move the spec we need some consensus. If I don't get much interest I will move off and just do my own thing.

\subsection{Notation}
In this section I define terminology and notation I use in this document. Some comes from previous VSIP work and some is specific to this paper study or my own idiosyncrasies.
\subsubsection{In General}
If I say \emph{in general} I am really saying we are restricted by the current \cvl{} specification but otherwise there is no reason not to extend the use to include more general functionality or cases. 
\subsubsection{Transpose VS Corner Turn}
I consider a transpose to be a mathematical operation on matrix views. Data may or may not move if a transpose is done. On the other hand I consider a corner turn to involve movement of data to achieve better data locality. Corner turns are frequently done to locate data to the proper node in a parallel process.
\subsubsection{\Blk{} and \Vw{} and \blk{} and \vw}
For this document a \Blk{} indicates the class including create/destroy operations and a \blk{} is the instantiated \Blk{} object. Similar for the \Vw{} being the class and \vw{} is the \Vw{} object.
\subsubsection{Users and Implementors}
In the \cvl{} specification we have two entities we specify for; the implementor and the user. 
\paragraph{The \ttbf{Implementor}} is the company, person, group, whatever who writes the library for use by others. This person is concerned about the requirements for the data model at a low level including how the \blk{} utilizes underlying memory resources.
\paragraph{The \ttbf{User}}is the person, company, group, whatever who will write software using the library as a tool. This person is concerned about managing data in the \blk{} and \vw{} format to efficiently process the data for the application the user is developing.
\paragraph{Comment}I note that ever since VSIPL was released the \ttbf{User} has been trying to pry open the \blk{} in order to directly access machine memory in the name of performance or convenience.  This may cause portability problems with user codes. I have not been convinced that the current blockbind/admit/release functionality will not work for most of the problems \ttbf{Users} are trying to solve.  Some use case diagrams (by \ttbf{Users} who want access to the block memory model) might be helpful to understand the problem. The problem seems to crop up frequently when moving data between a Host running VSIPL and a special processing device like a GPGPU. Still seems to me block bind should work.
\paragraph{Questions}It would be handy to have use case diagrams (or even a well written narrative) showing the need and use of the proposed direct data access method and why block bind will not work here.  Activity diagrams showing the work flow for the direct data access model would also be helpful.
\subsubsection{Function and type Naming Conventions}
I use the same notation as the \cvl{} specification but for those who are uninformed and don't want to read in the \cvl{} spec naming convention to produce a generic name an italic \emph{d} is used for some type of depth, an \emph{p}for some type of precision and an \emph{s} for some type of shape. Unfortunately in \cvl{} the matrix and tensor index are indicated in the \emph{p} spot even though they seem to me to be a depth. I am trying to generalize because I feel that perhaps at some time in the future we might have other depths beside complex and real.
\paragraph{Discusion}
In order to create the \Blk{} and \Vw{} objects the user needs to be able to communicate to the library the necessary metadata.  For a language such as C a lot of information is transferred via the function naming. Avoiding name collisions was the primary instigation for the \cvl{} naming convention.  For \pyjv{}, which is more object oriented, type information is indicated with strings using the \cvl{} naming convention; but the argument list is also used in \pyjv{} for function overloading.  For C++  information seems to be communicated with both templates and function naming.  For instance in \pyjv{} I just have a view class and the shape of  the \vw{} is indicated to the \Vw{} constructor argument list. The \Vw{} constructor is actually called by a \blk{} using the \ttbf{bind} method. So the scalar information needed to call the underlying \cvl{} constructors is provided by the block. Examples from the C++ proof of concept code done by Stefan use a \ttbf{Matrix} class so the shape is defined by the class constructor and the precision (and I suppose size) is defined by parameters. I am not sure if a \blk{} is involved at all for \ttbf{VSIPL++}. For \ttbf{VSIPL++} is a \blk{} only produced from a matrix or vector if an interface operation is being done?
\paragraph{Discussion}
While writing \pyjv{} I have noticed that for the current \cvl{} library all the information needed to create a view is the \ttbf{block} and the shape argument list.  The strings I have defined for view types are more for convenience than a necessity. Everything can be derived from the \vw{} attributes and the \blk{} type. I am not sure this would be true in general, but it seems to be true for \cvl{}.
\paragraph{Comment} In retrospect I am not sure the naming convention for \cvl{} was necessarily optimum. If we want to support more scalar types it can become unwieldily.  For instance I wonder if perhaps instead of \ttbf{vsip\_\emph{c}block\_\emph{p}} for complex we would have been better to use \ttbf{vsip\_block\_\emph{cp}}. In this way all the type information about the scalars the block stores reside in a scalar affix.  This affix would not necessarily need to be just one or two characters if more are needed for a particular implementation.  The point I am trying to make here is it a \emph{complex} \blk{} or is it a \blk{} that stores \emph{complex} scalars.
\subsubsection{Primary Functions and Convenience Functions}
Important functions, what I have termed primary, are the functions you can't do without. For instance in \cvl{} \ttbf{vsip\_\emph{ds}bind\_\emph{p}} is a primary function used to create a view and bind it to a (already created) \blk{}. The function \ttbf{vsip\_\emph{d}blockcreate\_\emph{p}} is a primary function used to create a \blk{}. The function \ttbf{vsip\_\emph{d}blockbind\_\emph{p}} is a primary function used to create an interface \blk{} and associate it with user defined memory. The function \ttbf{vsip\_\emph{d}blockrebind\_\emph{p}} is a primary function used to change an interface \blk{}s association to user memory.  There is no way to do these functions at a lower level with user code.
\\[6pt]
On the other hand a function like \ttbf{vsip\_\emph{ds}create\_\emph{p}} is a convenience function and will create a \blk{} and bind a \vw{} to it all in one step.  Functions like subview, row or column view, etc.  produce new view objects but could be done using primary functions. Functions which could be written by Users using primary function I call convenience functions.
\paragraph{Comment} It is my opinion that the primary functions define the data model, not the convenience functions.  In VSIPL++ it seems that all the functions are convenience functions.  It is difficult to find the data model when you only create matrices and vectors. Basically I don't understand the data model in VSIPL++ at all.  Do they even have a counterpart to \ttbf{bind}?  They seem to have blocks but I am not sure how they are associated with the matrix or vector.
%
