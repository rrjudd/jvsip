\afuncT{prodh}{Matrix hermitian product.}{matrixOperations}
\\\cvsiplh
\\ \hspace*{.8cm} \vspace*{.1cm} \textbf{Available Functions }
\\ \hspace*{0.03\textwidth} {
\ttfamily
\begin{tabular}[H]{l}
void vsip\_cmprodh\_d(\\*\hspace{.6cm}
    const vsip\_cmview\_d*, const vsip\_cmview\_d*, const vsip\_cmview\_d*);\Bs\\
void vsip\_cmprodh\_f(\\*\hspace{.6cm}
    const vsip\_cmview\_f*, const vsip\_cmview\_f*, const vsip\_cmview\_f*);\Bs\\
\end{tabular}
}
\\\pyjvsiph
\viewmthd{yes}{No}{No}{out=inOne.prodh(inTwo)}
\apyfunc{yes}{out = prodh(inOne,inTwo,out)}
\pyComment{
\item{The \ttbf{prodh} function works much the same as the C VSIPL version except that a convenience pointer to the output view is returned. This may not be done in-place.}
\item{The \ttbf{prodh} method creates and returns a new \ttbf{view} with the result.}
}
