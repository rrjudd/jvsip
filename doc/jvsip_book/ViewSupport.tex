\subsubsection*{View Objects\hypertarget{ViewObjects}{}\hspace*{\fill}\hyperlink{supportFunctions}{(up)}}\addcontentsline{toc}{subsubsection}{View Class}
\begin{table}[H]
\caption{View Support}
\label{tab:viewSupport}
\begin{center}
\begin{tabular}{|l|l|}\hline
\hlnkFunc{alldestroy} & Free both \ttbf{block} and \ttbf{view}.\\
\hlnkFunc{bind} & Bind a \ttbf{view} to a \ttbf{block}.\\
\hlnkFunc{cloneview} & Clone a \ttbf{view}.\\
\hlnkFunc{colview} & Return a column \ttbf{view} (vector) of a matrix \ttbf{view}\\
\hlnkFunc{create} & Create a \ttbf{view}.\\
\hlnkFunc{destroy} & Free a \ttbf{view}.\\
\hlnkFunc{get} & Get a value from a \ttbf{view}\\
\hlnkFunc{getblock} & Return \ttbf{block} associated with \ttbf{view}\\
\hlnkFunc{getattrib} & Get attribute structure associated with \ttbf{view}\\
\hlnkFunc{getlength} & Get get length of vector \ttbf{view}\\
\hlnkFunc{getcollength} & Get length of matrix \ttbf{view} column\\
\hlnkFunc{getrowlength} & Get length of matrix \ttbf{view} row\\
\hlnkFunc{getoffset} & Get get offset into block of vector \ttbf{view}\\
\hlnkFunc{getstride} & Get stride through block of vector \ttbf{view}\\
\hlnkFunc{getcolstride} & Get stride through block of matrix \ttbf{view} for columns.\\
\hlnkFunc{getrowstride} & Get stride through block of matrix \ttbf{view} for rows.\\
\hlnkFunc{getxlength} & Get get X length of tensor \ttbf{view}\\
\hlnkFunc{getxstride} & Get the X stride attribute of a tensor \ttbf{view}\\
\hlnkFunc{getylength} & Get get Y length of tensor \ttbf{view}\\
\hlnkFunc{getystride} & Get the Y stride attribute of a tensor \ttbf{view}\\
\hlnkFunc{getzlength} & Get get Z length of tensor \ttbf{view}\\
\hlnkFunc{getzstride} & Get the Z stride attribute of a tensor \ttbf{view}\\
\hlnkFunc{imagview} & Return \ttbf{view} of imaginary part of complex \ttbf{view}\\
\hlnkFunc{matrixview} & Create a matrix view of a 2-D slice of the tensor \ttbf{view}\\
\hlnkFunc{put} & Put a value into a \ttbf{view}\\
\hlnkFunc{putattrib} & Set attribute structure associated with \ttbf{view}.\\
\hlnkFunc{putlength} & Set length of vector \ttbf{view}.\\
\hlnkFunc{putcollength} & Set length of matrix \ttbf{view} column.\\
\hlnkFunc{putrowlength} & Set length of matrix \ttbf{view} row.\\
\hlnkFunc{putoffset} & Set offset into block of vector \ttbf{view}.\\
\hlnkFunc{putstride} & Set stride through block of vector\ttbf{view}.\\
\hlnkFunc{putcolstride} & Set stride through block of matrix \ttbf{view} column.\\
\hlnkFunc{putrowstride} & Set stride through block of matrix \ttbf{view} row.\\
\hlnkFunc{putxlength} & Set X length of tensor \ttbf{view}\\
\hlnkFunc{putxstride} & Set X stride through block of tensor\ttbf{view}\\
\hlnkFunc{putylength} & Set Y length of tensor \ttbf{view}\\
\hlnkFunc{putystride} & Set Y stride through block of tensor\ttbf{view}\\
\hlnkFunc{putzlength} & Set Z length of tensor \ttbf{view}\\
\hlnkFunc{putzstride} & Set Z stride through block of tensor\ttbf{view}\\
\hlnkFunc{realview} & Return \ttbf{view} of real part of complex \ttbf{view}.\\
\hlnkFunc{rowview} & Return a row \ttbf{view} (vector) of a matrix \ttbf{view}\\
\hlnkFunc{subview} & Create a sub-\ttbf{view} of a \ttbf{view}.\\
\hlnkFunc{transview} & Create a matrix \ttbf{view} as a transpose of a matrix\ttbf{view}\\
\hlnkFunc{vectview} & Create a vector \ttbf{view} of a 1-D slice of a tensor \ttbf{view}\\
\hline\end{tabular}
\end{center}
%\label{default}
\end{table}%
