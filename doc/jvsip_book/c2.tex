\chapter{Functions}
\section*{Introduction}
In this chapter I give basic usage information for the functions included in the JVSIP implementation of the C VSIPL specification and also related information for the \pyjv python module.  There are many functions so I may miss a few.

Usage information may also be found by reading the C VSIPL specification, either the old one included with the JVSIP distribution or the newer one developed by the HPEC working group of the OMG.  I currently recommend sticking with the old one included with the JVSIP distribution.  There is a lot of information about C VSIPL in the specification so C VSIPL information in this document will not be extensive; and since \pyjv has no specification document I will spend more time covering the \pyjv methodology.

I try and include information on the \pyjv methods and functions collocated with the corresponding C VSIPL information.  Reading the pyJvsip.py module file is also encouraged.  \ttbf{PyJvsip} includes some functionality not (directly) part of C VSIPL.  I will try and highlight these special cases.  

For python information the python help mechanism has also been supported somewhat; but keeping that information correct, up-to-date, and available for every function is a work in progress. 

Keep in mind this chapters main purpose is as a go-to reference for proper incantations when writing code. Except for the introductory sections it is probably not something you will want to read.

In order to have some reasonable ordering of the functions the alphabetical listing is based upon a root function name, not the actual vsip function. For instance the second function in the list is the \ilCode{add} function. There are several \ilCode{add} functions in the Core profile. All of them are placed together under \ilCode{add}.

When a C VSIPL function requires a special object it needs support functions to create the object, and destroy it, and perhaps query it for its attributes. For instance to do a discrete Fourier transform one needs a function to create an FFT object, a function to do the actual FFT using the FFT object, and a function to destroy the FFT object when it is no longer needed. The author calls functions which are designed to work together to do a single job function sets. Function sets are placed together under a single heading. For instance all the functions involved with doing an FFT are placed under the FFT heading.

As discussed in chapter one python supports polymorphism, and object oriented programing. A \pyjv object is an instantiation of a python class definition. The python object will contain a C VSIPL object as an instance variable as well as other information needed by \pyjv. For this reason the python garbage collector will destroy C VSIPL objects when no reference to the \pyjv object exists.

Because of the true object oriented nature of \pyjv there are methods defined for every class which accomplish most of the functionality of C VSIPL. \ttbf{PyJvsip} also defines many functions which operate on the \pyjv objects. Frequently you can use either a method or a function. This information is reflected in the JVSIP function list.

No attempt is made to be exhaustive in the function descriptions. Those interested in more detail are directed to the VSIPL specification document included with the JVSIP distribution. In addition various examples included in this document will provide more detail on the use of some of the more complicated functions.

\section*{C VSIPL Specification}
The main document on which \ttbf{JVSIP} is based is the \emph{VSIPL 1.3 API} as approved by the VSIPL Forum on January 31, 2008.  That document is included with the \ttbf{JVSIP} distribution.  The purpose of this section is to provide a roadmap for people who are familiar with this C VSIPL specification to get around in this \ttbf{JVSIP} manual.  Here I provide tables in an order matching the \emph{VSIPL 1.3 API} specification with links to the same information as presented  in the \ttbf{JVSIP} manual.
    \begin{table}[H]
\caption{VSIPL 1.3 API Chapters}
\label{tab:vsiplAPI}
\begin{center}
\begin{tabular}{l}
VSIPL INTRODUCTION\\
SUMMARY OF VSIPL TYPES\\
SUPPORT FUNCTIONS\\
SCALAR FUNCTIONS\\
RANDOM NUMBER GENERATION\\
VECTOR \& ELEMENTWISE OPERATIONS\\
SIGNAL PROCESSING FUNCTIONS\\
LINEAR ALGEBRA FUNCTIONS\\
IMPLEMENTATION DEPENDENT INPUT AND OUTPUT\\
VSIPL Addendum\\
\end{tabular}
\end{center}
\label{default}
\end{table}%
\subsection*{Summary of VSIPL Types}
\subsection*{Support Functions}
    \begin{table}[H]
\caption{Support Function Overview}
\label{tab:vsiplAPISupport}
\begin{center}
\begin{tabular}{l}
Initialization\\
Array and Block Object Functions\\
Vector View Object Functions\\
Matrix View Object Functions\\
Tensor Views\\
\end{tabular}
\end{center}
\label{default}
\end{table}%
\subsection*{Scalar Functions}
In general I do not define scalar functions in \pyjv.  Ease of use is a major goal of the \pyjv module and to further this goal I decided scalars used by or returned by \pyjv functions should be normal python scalars. Using scalar functions (such as $\cos$, $\sin$, etc.) imported from the math module or the numpy module should work fine. That said, you can always use the C VSIPL scalar functions directly since they are in the \ttbf{vsip} module which is included in the \pyjv module.
\subsection*{Random Number Generation}
\subsection*{Vector \& Elementwise Operations}
Many of the functions are simple operations which are done on each element in a matrix or vector. The tables referenced here list these with a link to the corresponding function page.
    \subsection*{Elementwise Operations
\hspace*{\fill}\hyperlink{VSIPspecHead}{(up)}\hypertarget{ElementwiseOperations}{}} \addcontentsline{toc}{subsection}{Elementwise Operations}
Elementwise operatons are simple operations which are done on each element in a matrix or vector. Most of the time, when more than one \ttbf{view} is input, the \ttbf{view} shapes will need to be the same since the operation is done to identically indexed elements for each input \ttbf{view} and the operation result is placed in an identically indexed element of the output \ttbf{view}. \\
The tables referenced in this section list elementwise operations with a link to the corresponding function page. Although the function pages are alphabetical, the lists here are in the same order (although not necessarily identical) to the order they appear in the C VSIPL specification.
\begin{table}[H]
\hypertarget{vectorAndElementwise}{}
\caption{Vector And Elementwise Operations}
\label{tab:elementwiseChapter}
\begin{center}
\begin{tabular}{|l|}\hline
\hyperlink{elementaryMath}{Elementary Math Functions}\\
\hyperlink{unaryOperations}{Unary Operations}\\
\hyperlink{binaryOperations}{Binary Operations}\\
\hyperlink{ternaryOperations}{Ternary Operations}\\
\hyperlink{logicalOperations}{Logical Operations}\\
\hyperlink{selectionOperations}{Selection Operations}\\
\hyperlink{bitwiseOperators}{Bitwise and Boolean Logical Operators}\\
\hyperlink{elementGenerationOperations}{Element Generation and Copy}\\
\hyperlink{manipulationOperations}{Manipulation Operations}\\
\hline\end{tabular}
\end{center}
%\label{default}
\end{table}%
%%%
      \subsubsection*{Elementary Math\hspace*{\fill}\hyperlink{ElementwiseOperations}{(up)}\hypertarget{elementaryMath}{}}\addcontentsline{toc}{subsubsection}{Elementary Math}
Elementary math functions constitute elementwise applications of elementary operations on \ttbf{view}s. The term \emph{elementary} is somewhat arbitrary but includes trigonometric functions, log functions, and exponential functions. Functions here (for elements) are defined by C 89 in the \ttbf{math.h} header file. \ttbf{JVSIP} generally uses this math library to do the calculations for these functions.
\begin{table}[H]
\caption{Elementary Math Functions \ref{tab:elementwiseChapter}}
\label{tab:elementaryMath}
\begin{center}
\begin{tabular}{|l|l|}
\hline
\hlnkFunc{acos} & Arccosine\\
\hlnkFunc{asin} & Arcsine\\
\hlnkFunc{atan} & Arctangent\\
\hlnkFunc{atan2} & Arctangent of Two Arguments\\
\hlnkFunc{cos} & Cosine\\
\hlnkFunc{cosh} & Hyperbolic Cosine\\
\hlnkFunc{exp} & Exponential\\
\hlnkFunc{exp10} & Exponential Base 10\\
\hlnkFunc{log} & Natural Log\\
\hlnkFunc{log10} & Base 10 Log\\
\hlnkFunc{sin} & Sine \\
\hlnkFunc{sinh} & Hyperbolic Sine\\
\hlnkFunc{sqrt} & Square Root\\
\hlnkFunc{tan} & Tangent\\
\hlnkFunc{tanh} & Hyperbolic Tangent\\
\hline
\end{tabular}
\end{center}
\label{default}
\end{table}%

      \begin{table}[H]
\caption{Unary Operations}
\label{tab:unaryOperations}
\begin{center}
\begin{tabular}{|l|l|}
\hlnkFunc{arg} & Argument\\
\hlnkFunc{ceil} & Ceiling\\
\hlnkFunc{conj} & Conjugate\\
\hlnkFunc{cumsum} & Cumulative Sum\\
\hlnkFunc{euler} & Euler\\
\hlnkFunc{floor} & Floor\\
\hlnkFunc{mag} & Magnitude\\
\hlnkFunc{cmagsq} & Complex Magnitude Squared\\
\hlnkFunc{meanval} & Mean Value\\
\hlnkFunc{meansqval} & Mean Square Value\\
\hlnkFunc{modulate} & Modulate\\
\hlnkFunc{neg} & Negate\\
\hlnkFunc{recip} & Reciprocal\\
\hlnkFunc{round} & Round\\
\hlnkFunc{rsqrt} & reciprocal Square Root\\
\hlnkFunc{sq} & Square\\
\hlnkFunc{sumval} & Sum Value\\
\hlnkFunc{sumsqval} & Sum of Squares Value\\
\end{tabular}
\end{center}
\label{default}
\end{table}%

      \begin{table}[H]
\caption{Binary Operations}
\label{tab:binaryOperations}
\begin{center}
\begin{tabular}{|l|l|}
\hlnkFunc{add} & Add\\
\hlnkFunc{div} & Divide\\
\hlnkFunc{expoavg} & Exponential Average\\
\hlnkFunc{hypot} & Hypotenuse\\
\hlnkFunc{jmul} & Conjugate Multiply\\
\hlnkFunc{mul} & Multiply\\
\hlnkFunc{vmmul} & Vector Matrix Multiply\\
\hlnkFunc{sub} & Subtract\\
\end{tabular}
\end{center}
\label{default}
\end{table}%

      \begin{table}[H]
\caption{Ternary Operations}
\label{tab:ternaryOperations}
\begin{center}
\begin{tabular}{|l|l|}
\hlnkFunc{am} & Add and multiply \\
\hlnkFunc{ma} & Multiply and add \\
\hlnkFunc{msb} & Multiply and subtract\\
\hlnkFunc{sbm} & Substract and multiply\\
\end{tabular}
\end{center}
\label{default}
\end{table}%

      \begin{table}[H]
\caption{Logical Operations}
\label{tab:logicalOperations}
\begin{center}
\begin{tabular}{|l|l|}
\hlnkFunc{add} & Add\\
\hlnkFunc{div} & Divide\\
\hlnkFunc{expoavg} & Exponential Average\\
\hlnkFunc{hypot} & Hypotenuse\\
\hlnkFunc{jmul} & Conjugate Multiply\\
\hlnkFunc{mul} & Multiply\\
\hlnkFunc{vmmul} & Vector Matrix Multiply\\
\hlnkFunc{sub} & Subtract\\
\end{tabular}
\end{center}
\label{default}
\end{table}%

      \subsubsection*{Selection Operations}\addcontentsline{toc}{subsubsection}{Selection Operations}
Selection operations involve some logical comparison and, based upon the result, an answer is \emph{selected} and returned; either as a scalar output (signified by \ttbf{val} ending the root name), or elementwise into an appropriately sized output \ttbf{view}. 
\begin{table}[H]
\caption{Selection Operations}
\label{tab:selectionOperations}
\begin{center}
\begin{tabular}{|l|l|}
\hlnkFunc{add} & Add\\
\hlnkFunc{div} & Divide\\
\hlnkFunc{expoavg} & Exponential Average\\
\hlnkFunc{hypot} & Hypotenuse\\
\hlnkFunc{jmul} & Conjugate Multiply\\
\hlnkFunc{mul} & Multiply\\
\hlnkFunc{vmmul} & Vector Matrix Multiply\\
\hlnkFunc{sub} & Subtract\\
\end{tabular}
\end{center}
\label{default}
\end{table}%

      \subsubsection*{Bitwise and Boolean Logical Operators}\addcontentsline{toc}{subsubsection}{Bitwise and Boolean Logical Operators}
This section provides support for standard logical operators. These will operate on integer precision \ttbf{view}s bitwise, or on \ttbf{view}s of precision \ttbf{bl} logically.
\begin{table}[H]
\caption{Bitwise and Boolean Logical Operators}
\label{tab:bitwiseOperators}
\begin{center}
\begin{tabular}{|l|l|}\hline
\hlnkFunc{and} & And operation\\
\hlnkFunc{not} & Not operation\\
\hlnkFunc{or} & Or operation\\
\hlnkFunc{xor} & Exclusive or operation\\
\hline\end{tabular}
\end{center}
\label{default}
\end{table}%

      \subsubsection*{Element Generation and Copy} \addcontentsline{toc}{subsubsection}{Element Generation and Copy}
This section has functions to copy data from one place to another. 
\begin{table}[H]
\caption{Element Generation and Copy}
\label{tab:elementGenerationOperations}
\begin{center}
\begin{tabular}{|l|l|}\hline
\hlnkFunc{copy} & Copy \ttbf{view} to \ttbf{view}\\
\hyperlink{copyto}{\texttt{copyto\_user}} & Copy data in a \ttbf{view} to user specified memory\\
\hyperlink{copyfrom}{\texttt{copyfrom\_user}} & Copy data from user specified memory to a \ttbf{view}\\
\hlnkFunc{fill} & Fill a \ttbf{view} with a constant value\\
\hlnkFunc{ramp} & In a vector \ttbf{view} create equally space \emph{ramp} data\\
\hline\end{tabular}
\end{center}
\label{default}
\end{table}%

      \subsubsection*{Manipulation Operations} \addcontentsline{toc}{subsubsection}{Manipulation Operations}
Manipulation operations are functions which copy \ttbf{view}s, or parts of \ttbf{view}s, from one location to another while doing some manipulation operation to convert the data. For instance the \ttbf{cmplx} function takes two real \ttbf{view}s and copies one \ttbf{view} to the imaginary part of a complex vector and the other \ttbf{view} to the real part of a complex vector. 
\begin{table}[H]
\caption{Manipulation Operations}
\label{tab:manipulationOperations}
\begin{center}
\begin{tabular}{|l|l|}\hline
\hlnkFunc{cmplx} & Complex\\
\hlnkFunc{gather} & Data Gather\\
\hlnkFunc{imag} & Imaginary Part\\
\hlnkFunc{polar} & Hypotenuse\\
\hlnkFunc{real} & Real Part\\
\hlnkFunc{rect} & Rectangular\\
\hlnkFunc{scatter} & Data Scatter\\
\hlnkFunc{swap} & Swap\\
binary & Not Supported\\
bool & Not Supported\\
mary & Not Supported\\
nary & Not Supported\\
serialmary & Not Supported\\
unary & Not Supported\\
\hline\end{tabular}
\end{center}
\label{default}
\end{table}%

    \subsubsection*{Elementary Math\hspace*{\fill}\hyperlink{ElementwiseOperations}{(up)}\hypertarget{elementaryMath}{}}\addcontentsline{toc}{subsubsection}{Elementary Math}
Elementary math functions constitute elementwise applications of elementary operations on \ttbf{view}s. The term \emph{elementary} is somewhat arbitrary but includes trigonometric functions, log functions, and exponential functions. Functions here (for elements) are defined by C 89 in the \ttbf{math.h} header file. \ttbf{JVSIP} generally uses this math library to do the calculations for these functions.
\begin{table}[H]
\caption{Elementary Math Functions \ref{tab:elementwiseChapter}}
\label{tab:elementaryMath}
\begin{center}
\begin{tabular}{|l|l|}
\hline
\hlnkFunc{acos} & Arccosine\\
\hlnkFunc{asin} & Arcsine\\
\hlnkFunc{atan} & Arctangent\\
\hlnkFunc{atan2} & Arctangent of Two Arguments\\
\hlnkFunc{cos} & Cosine\\
\hlnkFunc{cosh} & Hyperbolic Cosine\\
\hlnkFunc{exp} & Exponential\\
\hlnkFunc{exp10} & Exponential Base 10\\
\hlnkFunc{log} & Natural Log\\
\hlnkFunc{log10} & Base 10 Log\\
\hlnkFunc{sin} & Sine \\
\hlnkFunc{sinh} & Hyperbolic Sine\\
\hlnkFunc{sqrt} & Square Root\\
\hlnkFunc{tan} & Tangent\\
\hlnkFunc{tanh} & Hyperbolic Tangent\\
\hline
\end{tabular}
\end{center}
\label{default}
\end{table}%

    \begin{table}[H]
\caption{Unary Operations}
\label{tab:unaryOperations}
\begin{center}
\begin{tabular}{|l|l|}
\hlnkFunc{arg} & Argument\\
\hlnkFunc{ceil} & Ceiling\\
\hlnkFunc{conj} & Conjugate\\
\hlnkFunc{cumsum} & Cumulative Sum\\
\hlnkFunc{euler} & Euler\\
\hlnkFunc{floor} & Floor\\
\hlnkFunc{mag} & Magnitude\\
\hlnkFunc{cmagsq} & Complex Magnitude Squared\\
\hlnkFunc{meanval} & Mean Value\\
\hlnkFunc{meansqval} & Mean Square Value\\
\hlnkFunc{modulate} & Modulate\\
\hlnkFunc{neg} & Negate\\
\hlnkFunc{recip} & Reciprocal\\
\hlnkFunc{round} & Round\\
\hlnkFunc{rsqrt} & reciprocal Square Root\\
\hlnkFunc{sq} & Square\\
\hlnkFunc{sumval} & Sum Value\\
\hlnkFunc{sumsqval} & Sum of Squares Value\\
\end{tabular}
\end{center}
\label{default}
\end{table}%

    \begin{table}[H]
\caption{Binary Operations}
\label{tab:binaryOperations}
\begin{center}
\begin{tabular}{|l|l|}
\hlnkFunc{add} & Add\\
\hlnkFunc{div} & Divide\\
\hlnkFunc{expoavg} & Exponential Average\\
\hlnkFunc{hypot} & Hypotenuse\\
\hlnkFunc{jmul} & Conjugate Multiply\\
\hlnkFunc{mul} & Multiply\\
\hlnkFunc{vmmul} & Vector Matrix Multiply\\
\hlnkFunc{sub} & Subtract\\
\end{tabular}
\end{center}
\label{default}
\end{table}%

\subsection*{Signal Processing Functions}
    \subsection*{Signal Processing Functions \hyperlink{VSIPspecHead}{(up)}}\addcontentsline{toc}{subsection}{Signal Processing Functions}
\begin{table}[H]
\hypertarget{vsipljSignalProcessing}{}
\caption{Signal Processing Functions}
\label{tab:signalProcessingFunctions}
\begin{center}
\begin{tabular}{|l|}\hline
FFT Functions\ref{tab:fftFunctions}\\
Convolution/Correlation Functions\ref{tab:convCorrFunctions}\\
Window Functions\ref{windowFunctions}\\
Filter Functions\ref{tab:filterFunctions}\\
Miscellaneous Signal Processing Functions\ref{tab:miscSigProcFunctions}\\
\hline\end{tabular}
\end{center}
\label{default}
\end{table}%
\subsection*{Linear Algebra Functions}
    \subsection*{
    Linear Algebra Functions 
    \hspace*{\fill} \hyperlink{VSIPspecHead}{(up)}}
    \addcontentsline{toc}{subsection}{Linear Algebra Functions}
\begin{table}[H]
\hypertarget{linearAlgebraFunctions}{}
\caption{Linear Algebra Functions}
\label{tab:linearAlgebraFunctions}
\begin{center}
\begin{tabular}{|l|}\hline
\hyperlink{matrixOperations}{Matrix and Vector Operations}\\
\hyperlink{specialLinearSystemSolvers}{Special Linear System Solvers}\\
\hyperlink{generalSquareSolver}{General Square Linear System Solver}\\
\hyperlink{symmetricPositiveDefiniteSolver}{Symmetric Positive Definite Linear System Solver}\\
\hyperlink{overDeterminedSolver}{Over-determined Linear System Solver}\\
\hyperlink{singularValueDecompostion}{Singular Value Decomposition}\\
\hline\end{tabular}
\end{center}
\label{default}
\end{table}%
%%%
      \subsubsection*{Matrix and Vector Operations\hfill \hyperlink{linearAlgebraFunctions}{(up)}\hypertarget{matrixOperations}{}} \addcontentsline{toc}{subsubsection}{Matrix and Vector Operations}
\begin{table}[H]
\caption{Matrix and Vector Operations.}
\label{tab:matrixOperations}
\begin{center}
\begin{tabular}{|l|l|}\hline
\hlnkFunc{herm} & Matrix Hermitian\\
\hlnkFunc{jdot} & Complex Vector Conjugate Dot Product\\
\hlnkFunc{gemp} & General Matrix Product\\
\hlnkFunc{gems} & General Matrix Sum \\
\hlnkFunc{kron} & Kronecker Product \\
\hlnkFunc{prod3} & 3 by 3 Matrix Product\\
\hlnkFunc{prod4} & 4 by 4 Matrix Product\\
\hlnkFunc{prod} & Matrix product \\
\hlnkFunc{prodh} & Matrix Hermitian Product\\
\hlnkFunc{prodj} & Matrix Conjugate Product\\
\hlnkFunc{prodt} & Matrix Transpose Product\\
\hlnkFunc{trans} & Matrix Transpose\\
\hlnkFunc{dot} & Vector Dot Product\\
\hlnkFunc{outer} & Vector Outer Product\\
\hline\end{tabular}
\end{center}
%\label{default}
\end{table}
%
      \begin{table}[H]
\caption{Special Linear System Solvers}
\label{tab:specialLinearSystemSolvers}
\begin{center}
\begin{tabular}{|l|l|}\hline
\hlnkFunc{covsol} & Solve Covariance System\\
\hlnkFunc{llsqsol} & Solve Linear Least Squares Problem \\
\hlnkFunc{toepsol} & Solve Toeplitz System\\
\hline\end{tabular}
\end{center}
\label{default}
\end{table}%
      \subsubsection*{General Square Linear System Solver\hspace*{\fill} \hyperlink{linearAlgebraFunctions}{(up)}\hypertarget{generalSquareSolver}{}} \addcontentsline{toc}{subsubsection}{General Square Linear System Solver}
\begin{table}[H]
\caption{General Square Linear System Solver}
\label{tab:generalSquareSolver}
\begin{center}
\begin{tabular}{|l|l|}
\multicolumn{2}{c}{\hyperlink{ludFunc}{\rmfamily \bfseries LUD Function set}}\\
\hline
lud & LU Decomposition\\
lud\_create & Create LU Decomposition Object \\
lud\_destroy & Destroy LUD Object \\
lud\_getattr & LUD Get Attributes\\
lusol & Solve General Linear System \\
\hline\end{tabular}
\end{center}
\label{default}
\end{table}%%lud
      \subsubsection*{Symmetric Positive Definite Linear System Solver\hspace*{\fill} \hyperlink{linearAlgebraFunctions}{(up)}\hypertarget{symmetricPositiveDefiniteSolver}{}} \addcontentsline{toc}{subsubsection}{Symmetric Positive Definite Linear System Solver}
\begin{table}[H]
\caption{Symmetric Positive Definite (SPD) Linear System Solver  \ref{tab:linearAlgebraFunctions}}
\label{tab:symmetricPositiveDefiniteSolvers}
\begin{center}
\begin{tabular}{|l|l|}
\multicolumn{2}{c}{\hyperlink{choldFunc}{\rmfamily \bfseries Cholesky Decomposition Function set}}\\
\hline
chold & Cholesky Decomposition\\
chold\_create & Create Cholesky Decomposition Object\\
chold\_destroy & Destroy CHOLD Object\\
chold\_getattr & CHOLD Get Attributes\\
cholsol & Solve SPD Linear System\\
\hline\end{tabular}
\end{center}
%\label{default}
\end{table}%%chold
      \subsubsection*{Over-determined Linear System Solver\hspace*{\fill} \hyperlink{linearAlgebraFunctions}{(up)}\hypertarget{OverDeterminedSolver}{}} \addcontentsline{toc}{subsubsection}{Over-determined Linear System Solver}
\begin{table}[H]
\caption{Over-determined Linear System Solver}
\label{tab:overDeterminedSolver}
\begin{center}
\begin{tabular}{|l|l|}
\multicolumn{2}{c}{\hyperlink{qrdFunc}{\rmfamily \bfseries QRD Function Set}}\\
\hline
qrd & Cholesky Decomposition\\
qrd\_create & Create QR Decomposition Object\\
qrd\_destroy & Destroy QRD Object\\
qrd\_getattr & QRD Get Attributes\\
qrdprodq & Product with Q from QR Decomposition\\
qrdsolr &Solve Linear System Based on R from QRD\\
qrdsol & Solve Covariance or LLSQ System\\
\hline\end{tabular}
\end{center}
\label{default}
\end{table}%%qrd
      \subsubsection*{Singular Value Decomposition\hspace*{\fill} \hyperlink{linearAlgebraFunctions}{(up)}\hypertarget{singularValueDecompostion}{}} \addcontentsline{toc}{subsubsection}{Singular Value Decomposition}
\begin{table}[H]
\caption{Singular Value Decomposition}
\label{tab:singularValueDecompostion}
\begin{center}
\begin{tabular}{|l|l|}
\multicolumn{2}{c}{\hyperlink{svdFunc}{\rmfamily \bfseries SVD Function Set}}\\
\hline\Ts
svd & Matrix Singular Value Decomposition\Bs\\
svd\_create & Create Singular Value Decomposition Object \Bs\\
svd\_destroy & Destroy SVD Object\Bs\\
svd\_getattr & SVD Get Attributes\Bs\\
svd\_produ & Product with U from SV Decomposition\Bs\\
svdprodv & Product with V from SV Decomposition\Bs\\
svdmatu & Return with U from SV Decomposition\Bs\\
svdmatv & Return with V from SV Decomposition\Bs\\
\hline\end{tabular}
\end{center}
\label{default}
\end{table}%%svd
\subsection*{Implementation Dependent Input and Output}
   \ttbf{JVSIP} does not support implementation dependent IO at this time.
\subsection*{VSIPL Addendum}
Editing the VSIPL specification was becoming difficult because of it's length and instabilities in the MS Word source document. When functions were added to the specification for interpolation, permutation and sorting they were added as separate documents in a addendum and basically glued onto the pdf. This allowed for much less editing of the MS Word source document.
\subsubsection*{VSIPL Interpolation}
\subsubsection*{VSIPL Permute}
\subsubsection*{VSIPL Sort}
\section*{VSIP Types}
This section covers the enumerated types and special structures. These are declared in the public header file \ilCode{vsip.h}. 
\section*{JVSIP Function List}\addcontentsline{roc}{section}{JVSIP Function List} 
\afuncT{acos}{Inverse Cosine. An elementary math function.}{elementaryMath}
\\\cvsiplh
\begin{cfuncs}
vsip\_scalar\_f~vsip\_acos\_f(vsip\_scalar\_f);\\
vsip\_scalar\_d~vsip\_acos\_d(vsip\_scalar\_d);\\
void vsip\_macos\_d(const~vsip\_mview\_d*, const~vsip\_mview\_d*);\\
void vsip\_macos\_f(const~vsip\_mview\_f*, const~vsip\_mview\_f*);\\
void vsip\_vacos\_d(const~vsip\_vview\_d*, const~vsip\_vview\_d*);\\
void vsip\_vacos\_f(const~vsip\_vview\_f*, const~vsip\_vview\_f*);\\
\end{cfuncs}
\pyjvsiph
\viewmthd{yes}{yes}{yes}{inOut.acos}
\apyfunc{yes}{out = acos(in,out)}
\\\begin{minipage}{\textwidth}
\hspace*{.04\textwidth}\textbf{Comments}\\ 
\hspace*{.04\textwidth}\parbox{.95\textwidth}
{\vspace*{.1cm}
\begin{itemize}
\item{The \ttbf{acos} function works much the same as the C VSIPL version except that a convenience pointer to the output view is returned.}
\item{This may be done in-place if \ttbf{in==out}.}
\end{itemize}
}
\end{minipage}\afuncT{arg}{Compute the radian value argument of complex elements in the interval $[-\pi,\pi]$. An Unary Operation.}{unaryOperations}
\\\cvsiplh
\newline \hspace*{.8cm} \vspace*{.1cm} \textbf{Available Functions }
\newline \hspace*{1.1cm} {
\ttfamily
\begin{tabular}[H]{l}
vsip\_scalar\_d vsip\_arg\_d(vsip\_cscalar\_d);\\
vsip\_scalar\_f vsip\_arg\_f(vsip\_cscalar\_f);\\
void vsip\_marg\_d(\\*\hspace{.5cm}const vsip\_cmview\_d*, const vsip\_mview\_d*);\\
void vsip\_marg\_f(\\*\hspace{.5cm}const vsip\_cmview\_f*, const vsip\_mview\_f*);\\
void vsip\_varg\_d(\\*\hspace{.5cm}const vsip\_cvview\_d*, const vsip\_vview\_d*);\\
void vsip\_varg\_f(\\*\hspace{.5cm}const vsip\_cvview\_f*, const vsip\_vview\_f*);\\
\end{tabular}
}
\\\pyjvsiph
\viewmthd{yes}{yes}{No}{out=in.arg}
\apyfunc{yes}{out = arg(in,out)}
\newline \hspace*{.8cm}\textbf{Comment}\\
\hspace*{.8cm}\parbox{11cm}{\vspace*{.2cm}
\begin{itemize}
\item{Since \ttbf{arg} takes a view of \emph{depth} complex and outputs to a view of \emph{depth} real of the same \emph{shape} and \emph{precision} as the input view the \ttbf{arg} method will create a view of the proper type and size and return it.}
\item{The \ttbf{arg} function works the same as the C VSIPL function except a convenience pointer is returned to the output view}
\item{For the function limited in-place functionality exists with replacement of the real or imaginary view of the input with the output. For instance \ilCode{out=arg(in,in.realview)} works fine.}
\end{itemize}}
\afuncT{asin}{Inverse Cosine. An elementary math function.}{elementaryMath}
\\\cvsiplh
\begin{cfuncs}
vsip\_scalar\_f vsip\_asin\_f(vsip\_scalar\_f a);\Bs\\
vsip\_scalar\_d vsip\_asin\_d(vsip\_scalar\_d a);\Bs\\
void vsip\_masin\_d(const~vsip\_mview\_d*, const~vsip\_mview\_d*);\Bs\\
void vsip\_masin\_f(const~vsip\_mview\_f*, const~vsip\_mview\_f*);\Bs\\
void vsip\_vasin\_d(const~vsip\_vview\_d*, const~vsip\_vview\_d*);\Bs\\
void vsip\_vasin\_f(const~vsip\_vview\_f*, const~vsip\_vview\_f*);\Bs\\
\end{cfuncs}
\pyjvsiph
\viewmthd{yes}{yes}{yes}{inOut.asin}
\apyfunc{yes}{out = asin(in,out)}
\pyComment{
\item{The \ttbf{asin} function works much the same as the C VSIPL version except that a convenience pointer to the output view is returned. This may be done in-place if \ttbf{in==out}.}}
\afunc{atan}{Computes the principal radian value,[-+/2, +/2] of the arctangent for each element of a \ttbf{view}. See elementary math functions table \ref{tab:elementaryMath}.}
\cvsiplh
\pyjvsiph\afuncT{atan2}{Arctangent of Two Arguments; An elementwise function. Computes the four quadrant radian value, $[-\pi,\pi]$, of the arctangent of the ratio of the corresponding elements of two input views.}{elementaryMath}
\\\cvsiplh
\hspace*{.8cm} \vspace*{.1cm} \textbf{Available Functions } \\
\hspace*{1cm}
\texttt{
\begin{tabular}[H]{l}
vsip\_scalar\_d vsip\_atan2\_d(vsip\_scalar\_d, vsip\_scalar\_d);\\
vsip\_scalar\_f vsip\_atan2\_f(vsip\_scalar\_f, vsip\_scalar\_f);\\
void vsip\_matan2\_d(const vsip\_mview\_d*, const vsip\_mview\_d*, const vsip\_mview\_d*);\\
void vsip\_matan2\_f(const vsip\_mview\_f*, const vsip\_mview\_f*, const vsip\_mview\_f*);\\
void vsip\_vatan2\_d(const vsip\_vview\_d*, const vsip\_vview\_d*, const vsip\_vview\_d*);\\
void vsip\_vatan2\_f(const vsip\_vview\_f*, const vsip\_vview\_f*, const vsip\_vview\_f*);\\
\end{tabular}
}
\\\pyjvsiph
\viewmthd{no}{NA}{NA}{NA}
\apyfunc{yes}{out = atan2(inOne,inTwo,out)}
\pyComment{\item{The \ttbf{atan2} function works much the same as the C VSIPL version except that a convenience pointer to the output view is returned. This may be done in-place if \ttbf{inOne==out} or \ttbf{inTwo==out}.}}
\afunc{ceil}{Ceiling. Not currently supported in \jv. An unary operation. See table \ref{tab:unaryOperations}}
\cvsiplh
\pyjvsiph
\afuncT{cos}{Cosine; An elementary math function.}{elementaryMath}
\\\cvsiplh
\newline \hspace*{.8cm} \vspace*{.1cm} \textbf{Available Functions }
\newline \hspace*{1.1cm} {
\ttfamily
\begin{tabular}[H]{l}
vsip\_scalar\_f vsip\_cos\_f(vsip\_scalar\_f a);\\
vsip\_scalar\_d vsip\_cos\_d(vsip\_scalar\_d a);\\
void vsip\_mcos\_d(const vsip\_mview\_d*, const vsip\_mview\_d*);\\
void vsip\_mcos\_f(const vsip\_mview\_f*, const vsip\_mview\_f*);\\
void vsip\_vcos\_d(const vsip\_vview\_d*, const vsip\_vview\_d*);\\
void vsip\_vcos\_f(const vsip\_vview\_f*, const vsip\_vview\_f*);\\
\end{tabular}
}
\\\pyjvsiph
\viewmthd{yes}{yes}{yes}{inOut.cos}
\apyfunc{yes}{out = cos(in,out)}
\pyComment{\item{The \ttbf{cos} function works much the same as the C VSIPL version except that a convenience pointer to the output view is returned. This may be done in-place if \ttbf{in==out}.}}
\afuncT{cosh}{Hyperbolic Cosine; An elementwise function}{elementaryMath}
\\\cvsiplh
\afh
\\\hspace*{.04\textwidth} {
\ttfamily
\begin{tabular}[H]{l}
vsip\_scalar\_f vsip\_cosh\_f(vsip\_scalar\_f a);\\
vsip\_scalar\_d vsip\_cosh\_d(vsip\_scalar\_d a);\\
void vsip\_mcosh\_d(const vsip\_mview\_d*, const vsip\_mview\_d*);\\
void vsip\_mcosh\_f(const vsip\_mview\_f*, const vsip\_mview\_f*);\\
void vsip\_vcosh\_d(const vsip\_vview\_d*, const vsip\_vview\_d*);\\
void vsip\_vcosh\_f(const vsip\_vview\_f*, const vsip\_vview\_f*);\\
\end{tabular}
}
\\\pyjvsiph
\viewmthd{yes}{yes}{yes}{inOut.cosh}
\apyfunc{yes}{out = cosh(in,out)}
\pyComment{
\item{The \ttbf{cosh} function works much the same as the C VSIPL version except that a convenience pointer to the output view is returned. This may be done in-place if \ttbf{in==out}.}}
\afunc{ceil}{Ceiling. Not currently supported in \jv. An unary operation. See table \ref{tab:unaryOperations}}
\cvsiplh
\pyjvsiph\afunc{conj}{Conjugate}
\cvsiplh
\pyjvsiph\afuncT{cumsum}{Cumulative Sum}{unaryOperations}
\index{Cumulative Sum}
\\\cvsiplh
\\\pyjvsiph\afunc{copy}{Copy Data between two views. Some mixed types are supported so this method can be used to produce a copy of data of a new precision}{elementGenerationOperations}
\\\cvsiplh
\newline \hspace*{.8cm} \vspace*{.1cm} \textbf{Available Functions }
\newline \hspace*{1cm} {\ttfamily
\begin{tabular}[H]{l}
void vsip\_cmcopy\_d\_d(\\*
\hspace{1cm}const vsip\_cmview\_d*, const vsip\_cmview\_d*);\\
void vsip\_cmcopy\_d\_f(\\*
\hspace{1cm}const vsip\_cmview\_d*, const vsip\_cmview\_f*);\\
void vsip\_cmcopy\_f\_d(\\*
\hspace{1cm}const vsip\_cmview\_f*, const vsip\_cmview\_d*);\\
void vsip\_cmcopy\_f\_f(\\*
\hspace{1cm}const vsip\_cmview\_f*, const vsip\_cmview\_f*);\\
void vsip\_cvcopy\_d\_d\\*
\hspace{1cm}(const vsip\_cvview\_d*, const vsip\_cvview\_d*);\\
$\cdots$  \emph{etc.} \end{tabular}
}
\newline \hspace*{1cm}
\parbox{11cm}{There are many copy functions. To see all supported search the \ilCode{vsip.h} header file.\footnotemark}
\footnotetext{For instance \ttbf{grep copy\_ vsip.h} will list all available copy functions.}
\\\pyjvsiph
\viewmthd{yes}{yes}{no}{\parbox[t]{4cm}{out=in.copy\\out=in.copyrm\\out=in.copycm}}
\newline\hspace*{1cm}\parbox{11cm}{The \ttbf{copy} method creates a new view and data space that is the same shape, precision and depth as the input view and copies the data from the \ilCode{in} view to the \ilCode{out} view. The block in the \ilCode{out} view will be the exact size needed to hold the data and will be unit stride along the major direction of the \ilCode{in} view.\\The {\texttt{\bfseries{copycm}}} method is the same as the \ilCode{copy} method except the output view will always be row major independent of the input views major direction.\\The \ttbf{copyrm} method is the same as the \ilCode{copy} method except the output view will always be column major independent of the input views major direction.\\If the input view is a vector the three copy methods have identical results.}
\newline
\apyfunc{yes}{out = copy(in,out)}
\newline\hspace*{1cm}\parbox{11cm}{The \ttbf{copy} function works much the same as the C VSIPL version except that a convenience pointer to the output view is returned.}

\afuncT{euler}{Euler}{unaryOperations}
\\\cvsiplh
\afh
\\\hspace*{.04\textwidth} {
\ttfamily
}
\\\pyjvsiph\afuncT{exp}{Exponential; An elementwise function}{elementaryMath}
\\\cvsiplh
\\\pyjvsiph
\afunc{exp10}{Exponential Base 10; An elementwise function}
\\\cvsiplh
\\\pyjvsiph

\afuncT{log}{Natural logarithm; An element-wise function.}{elementaryMath}
\\\cvsiplh
\newline \hspace*{.8cm} \vspace*{.1cm} \textbf{Available Functions }
\newline \hspace*{1.1cm} {
\ttfamily
\begin{tabular}[H]{l}
\end{tabular}
}
\\\pyjvsiph
\viewmthd{yes}{yes}{yes}{inOut.sin}
\apyfunc{yes}{out = sin(in,out)}
\newline\hspace*{1.2cm}\parbox{10.8cm}{\vspace*{.1cm}The \ttbf{log} function works much the same as the C VSIPL version except that a convenience pointer to the output view is returned. This may be done in-place if \ttbf{in==out}.}
\afuncT{log10}{Compute the base ten logarithm; An element-wise function.}{elementaryMath}
\\\cvsiplh
\newline \hspace*{.8cm} \vspace*{.1cm} \textbf{Available Functions }
\newline \hspace*{1.1cm} {
\ttfamily
\begin{tabular}[H]{l}
vsip\_scalar\_d vsip\_log10\_d(vsip\_scalar\_d)\\
vsip\_scalar\_f vsip\_log10\_f(vsip\_scalar\_f)\\
void vsip\_mlog10\_d(\\*\hspace{1cm}const vsip\_mview\_d*, const vsip\_mview\_d*);\\
void vsip\_mlog10\_f(\\*\hspace{1cm}const vsip\_mview\_f*, const vsip\_mview\_f*);\\
void vsip\_vlog10\_d(\\*\hspace{1cm}const vsip\_vview\_d*, const vsip\_vview\_d*);\\
void vsip\_vlog10\_f(\\*\hspace{1cm}const vsip\_vview\_f*, const vsip\_vview\_f*);\\
\end{tabular}
}
\\\pyjvsiph
\viewmthd{yes}{yes}{yes}{inOut.sin}
\apyfunc{yes}{out = sin(in,out)}
\newline\hspace*{1.2cm}\parbox{10.8cm}{\vspace*{.1cm}The \ttbf{log10} function works much the same as the C VSIPL version except that a convenience pointer to the output view is returned. This may be done in-place if \ttbf{in==out}.}

\afunc{floor}{For each element in the input \ttbf{view} round to the largest integral value not greater than the input. An unary operation. See table \ref{tab:unaryOperations}.}
\cvsiplh
\pyjvsiph
\pyjvComment{
\item{The \ilCode{floor} function is not supported in \jv at this time}
}
\afunc{mag}{Arctangent of Two Arguments; An elementwise function}
\cvsiplh
\pyjvsiph\afuncT{magsq}{Arctangent of Two Arguments; An elementwise function}{unaryOperations}
\\\cvsiplh
\afh
\\\hspace*{.04\textwidth} {
\ttfamily
}
\\\pyjvsiph\afunc{meanval}{Arctangent of Two Arguments; An elementwise function}
\cvsiplh
\pyjvsiph\afuncT{meansqval}{Returns the mean value of all the elements of a view.}{unaryOperations}
\\\cvsiplh
\\ \hspace*{.8cm} \vspace*{.1cm} \textbf{Available Functions }
\\ \hspace*{1.1cm} {
\ttfamily
\begin{tabular}[H]{l}
vsip\_scalar\_d vsip\_cmmeansqval\_d(const vsip\_cmview\_d*);\\
vsip\_scalar\_d vsip\_cvmeansqval\_d(const vsip\_cvview\_d*);\\
vsip\_scalar\_d vsip\_mmeansqval\_d(const vsip\_mview\_d*);\\
vsip\_scalar\_d vsip\_vmeansqval\_d(const vsip\_vview\_d*);\\
vsip\_scalar\_f vsip\_cmmeansqval\_f(const vsip\_cmview\_f*);\\
vsip\_scalar\_f vsip\_cvmeansqval\_f(const vsip\_cvview\_f*);\\
vsip\_scalar\_f vsip\_mmeansqval\_f(const vsip\_mview\_f*);\\
vsip\_scalar\_f vsip\_vmeansqval\_f(const vsip\_vview\_f*);\\
\end{tabular}
}
\\\pyjvsiph
\viewmthd{Yes}{Yes}{NA}{msq=in.meansqval}
\apyfunc{No}{NA}
\pyComment{\item{There seemed to be no reason to include this as a separate function for \pyjv}}\afunc{modulate}{Arctangent of Two Arguments; An elementwise function}
\cvsiplh
\pyjvsiph
\afunc{neg}{Arctangent of Two Arguments; An elementwise function}
\cvsiplh
\pyjvsiph
\afunc{recip}{Arctangent of Two Arguments; An elementwise function}
\cvsiplh
\pyjvsiph\afunc{round}{Round to nearest integral value; An elementwise function. See unary operations table \ref{tab:unaryOperations}}
\cvsiplh
\pyjvsiph
\pyjvComment{
\item{The \ilCode{round} function is not supported in \jv at this time}
}
\afunc{rsqrt}{Arctangent of Two Arguments; An elementwise function}
\cvsiplh
\pyjvsiph
\afuncT{sin}{Sine; An element-wise function. Input \ttbf{view} elements are assumed to be in radians.}{elementaryMath}
\\\cvsiplh
\afh
\\\hspace*{.04\textwidth} {
\ttfamily
\begin{tabular}[H]{l}
vsip\_scalar\_f vsip\_sin\_f(vsip\_scalar\_f a);\\
vsip\_scalar\_d vsip\_sin\_d(vsip\_scalar\_d a);\\
void vsip\_msin\_d(\\*
\hspace{1cm}const vsip\_mview\_d*, const vsip\_mview\_d*);\\
void vsip\_msin\_f(\\*
\hspace{1cm}const vsip\_mview\_f*, const vsip\_mview\_f*);\\
void vsip\_vsin\_d(\\*
\hspace{1cm}const vsip\_vview\_d*, const vsip\_vview\_d*);\\
void vsip\_vsin\_f(\\*
\hspace{1cm}const vsip\_vview\_f*, const vsip\_vview\_f*);\\
\end{tabular}
}
\\\pyjvsiph
\viewmthd{yes}{yes}{yes}{inOut.sin}
\apyfunc{yes}{out = sin(in,out)}
\\\hspace*{.06\textwidth}\parbox{.9\textwidth}{\vspace*{.005\textheight}The \ttbf{sin} function works much the same as the C VSIPL version except that a convenience pointer to the output view is returned. This may be done in-place if \ttbf{in==out}.}
\afunc{sinh}{Hyperbolic Sine; An elementwise function. See elementary math functions table \ref{tab:elementaryMath}.}
\\\cvsiplh
\newline \hspace*{.8cm} \vspace*{.1cm} \textbf{Available Functions }
\newline \hspace*{1.1cm} {
\ttfamily
\begin{tabular}[H]{l}
vsip\_scalar\_f vsip\_sinh\_f(vsip\_scalar\_f a);\\
vsip\_scalar\_d vsip\_sinh\_d(vsip\_scalar\_d a);\\
void vsip\_msinh\_d(\\*
\hspace{1cm}const vsip\_mview\_d*, const vsip\_mview\_d*);\\
void vsip\_msinh\_f(\\*
\hspace{1cm}const vsip\_mview\_f*, const vsip\_mview\_f*);\\
void vsip\_vsinh\_d(\\*
\hspace{1cm}const vsip\_vview\_d*, const vsip\_vview\_d*);\\
void vsip\_vsinh\_f(\\*
\hspace{1cm}const vsip\_vview\_f*, const vsip\_vview\_f*);\\
\end{tabular}
}
\\\pyjvsiph
\viewmthd{yes}{yes}{yes}{inOut.sinh}
\apyfunc{yes}{out = sinh(in,out)}
\newline\hspace*{1.2cm}\parbox{10.8cm}{\vspace*{.1cm}The \ttbf{sinh} function works much the same as the C VSIPL version except that a convenience pointer to the output view is returned. This may be done in-place if \ttbf{in==out}.}
\afuncT{sqrt}{Square Root; An elementwise function.}{elementaryMath}
\\\cvsiplh
\afh
\\\hspace*{.04\textwidth} {
\ttfamily
\begin{tabular}[H]{l}
vsip\_scalar\_f vsip\_sqrt\_f(vsip\_scalar\_f a);\\
vsip\_scalar\_d vsip\_sqrt\_d(vsip\_scalar\_d a);\\
vsip\_cscalar\_d vsip\_csqrt\_d(vsip\_cscalar\_d);\\
vsip\_cscalar\_f vsip\_csqrt\_f(vsip\_cscalar\_f);\\
void vsip\_msqrt\_d(const vsip\_mview\_d*, const vsip\_mview\_d*);\\
void vsip\_msqrt\_f(const vsip\_mview\_f*, const vsip\_mview\_f*);\\
void vsip\_vsqrt\_d(const vsip\_vview\_d*, const vsip\_vview\_d*);\\
void vsip\_vsqrt\_f(const vsip\_vview\_f*, const vsip\_vview\_f*);\\
void vsip\_cmsqrt\_d(const vsip\_cmview\_d*, const vsip\_cmview\_d*);\\
void vsip\_cmsqrt\_f(const vsip\_cmview\_f*, const vsip\_cmview\_f*);\\
void vsip\_cvsqrt\_d(const vsip\_cvview\_d*, const vsip\_cvview\_d*);\\
void vsip\_cvsqrt\_f(const vsip\_cvview\_f*, const vsip\_cvview\_f*);\\
\end{tabular}
}
\\\pyjvsiph
\viewmthd{yes}{yes}{yes}{inOut.sqrt}
\apyfunc{yes}{out = sqrt(in,out)}
\pyComment{\item{The \ttbf{sqrt} function works much the same as the C VSIPL version except that a convenience pointer to the output view is returned. This may be done in-place if \ttbf{in==out}.}}
\afunc{sq}{Arctangent of Two Arguments; An elementwise function}
\cvsiplh
\pyjvsiph\afuncT{sumval}{Returns the sum of the the elements of a \ttbf{view}. Does not modify input.}{unaryOperations}
\\\cvsiplh
\afh
\\\hspace*{.04\textwidth} {
\ttfamily
\begin{tabular}[H]{l}
vsip\_cscalar\_d vsip\_cmsumval\_d(const vsip\_cmview\_d*);\\
vsip\_cscalar\_d vsip\_cvsumval\_d(const vsip\_cvview\_d*);\\
vsip\_cscalar\_f vsip\_cmsumval\_f(const vsip\_cmview\_f*);\\
vsip\_cscalar\_f vsip\_cvsumval\_f(const vsip\_cvview\_f*);\\
vsip\_scalar\_d vsip\_msumval\_d(const vsip\_mview\_d*);\\
vsip\_scalar\_d vsip\_vsumval\_d(const vsip\_vview\_d*);\\
vsip\_scalar\_f vsip\_msumval\_f(const vsip\_mview\_f*);\\
vsip\_scalar\_f vsip\_vsumval\_f(const vsip\_vview\_f*);\\
vsip\_scalar\_i vsip\_vsumval\_i(const vsip\_vview\_i*);\\
vsip\_scalar\_si vsip\_vsumval\_si(const vsip\_vview\_si*);\\
vsip\_scalar\_uc vsip\_vsumval\_uc(const vsip\_vview\_uc*);\\
vsip\_scalar\_vi vsip\_msumval\_bl(const vsip\_mview\_bl*);\\
vsip\_scalar\_vi vsip\_vsumval\_bl(const vsip\_vview\_bl*);\\
\end{tabular}
}
\\\pyjvsiph\afunc{sumsqval}{Returns the sum of the squares of all the elements of a \ttbf{view}. Does not modify input. See table \ref{tab:unaryOperations}}
\\\cvsiplh
\newline \hspace*{.8cm} \vspace*{.1cm} \textbf{Available Functions }
\newline \hspace*{1.1cm} {
\ttfamily
\begin{tabular}[H]{l}
vsip\_scalar\_d vsip\_msumsqval\_d(const vsip\_mview\_d* );\\
vsip\_scalar\_d vsip\_vsumsqval\_d(const vsip\_vview\_d* );\\
vsip\_scalar\_f vsip\_msumsqval\_f(const vsip\_mview\_f* );\\
vsip\_scalar\_f vsip\_vsumsqval\_f(const vsip\_vview\_f* );\\
\end{tabular}
}
\\\pyjvsiph
\viewmthd{yes}{yes}{NA}{aValue=in.sumsqval}
\apyfunc{No}{}
\newline \hspace*{.8cm} \textbf{Comments}
\newline\hspace*{.9cm}\parbox{10.8cm}{\vspace*{.1cm}\begin{itemize}
\item{Since the \ttbf{sumsqval} function returns a scalar without modifying the \ttbf{view} there seemed little point in supporting this as a separate function call for \pyjv.}
\end{itemize}
}

\afunc{tan}{Tangent; An elementwise function}
\cvsiplh
\pyjvsiph
\afunc{tanh}{Hyperbolic Tangent; An elementwise function}
\cvsiplh
\pyjvsiph
