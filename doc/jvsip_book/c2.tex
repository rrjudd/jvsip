\chapter{Functions}
\section*{Introduction}
In this chapter I give basic usage information for the functions included in the JVSIP implementation of the C VSIPL specification and also related information for the \pyjv python module.  There are many functions so I may miss a few.

Usage information may also be found by reading the C VSIPL specification, either the old one included with the JVSIP distribution or the newer one developed by the HPEC working group of the OMG.  I currently recommend sticking with the old one included with the JVSIP distribution.  There is a lot of information about C VSIPL in the specification so C VSIPL information in this document will not be extensive; and since \pyjv has no specification document I will spend more time covering the \pyjv methodology.

I try and include information on the \pyjv methods and functions collocated with the corresponding C VSIPL information.  Reading the pyJvsip.py module file is also encouraged.  \ttbf{PyJvsip} includes some functionality not (directly) part of C VSIPL.  I will try and highlight these special cases.  

For python information the python help mechanism has also been supported somewhat; but keeping that information correct, up-to-date, and available for every function is a work in progress. 

Keep in mind this chapters main purpose is as a go-to reference for proper incantations when writing code. Except for the introductory sections it is probably not something you will want to read.

In order to have some reasonable ordering of the functions the alphabetical listing is based upon a root function name, not the actual vsip function. For instance the second function in the list is the \ilCode{add} function. There are several \ilCode{add} functions in the Core profile. All of them are placed together under \ilCode{add}.

When a C VSIPL function requires a special object it needs support functions to create the object, and destroy it, and perhaps query it for its attributes. For instance to do a discrete Fourier transform one needs a function to create an FFT object, a function to do the actual FFT using the FFT object, and a function to destroy the FFT object when it is no longer needed. The author calls functions which are designed to work together to do a single job function sets. Function sets are placed together under a single heading. For instance all the functions involved with doing an FFT are placed under the FFT heading.

As discussed in chapter one python supports polymorphism, and object oriented programing. A \pyjv object is an instantiation of a python class definition. The python object will contain a C VSIPL object as an instance variable as well as other information needed by \pyjv. For this reason the python garbage collector will destroy C VSIPL objects when no reference to the \pyjv object exists.

Because of the true object oriented nature of \pyjv there are methods defined for every class which accomplish most of the functionality of C VSIPL. \ttbf{PyJvsip} also defines many functions which operate on the \pyjv objects. Frequently you can use either a method or a function. This information is reflected in the JVSIP function list.

No attempt is made to be exhaustive in the function descriptions. Those interested in more detail are directed to the VSIPL specification document included with the JVSIP distribution. In addition various examples included in this document will provide more detail on the use of some of the more complicated functions.
%
\section*{C VSIPL Specification}\addcontentsline{toc}{section}{C VSIPL Specification}
The main document on which \ttbf{JVSIP} is based is the \emph{VSIPL 1.3 API} as approved by the VSIPL Forum on January 31, 2008.  That document is included with the \ttbf{JVSIP} distribution.  The main purpose of this section is to provide a roadmap for people who are familiar with the C VSIPL specification to get around in this \ttbf{JVSIP} manual.  Here I provide tables in an order matching the \emph{VSIPL 1.3 API} specification with links to the same information as presented  in the \ttbf{JVSIP} manual.
    \begin{table}[H]
\caption{VSIPL 1.3 API Chapters}
\label{tab:vsiplAPI}
\begin{center}
\begin{tabular}{l}
VSIPL INTRODUCTION\\
SUMMARY OF VSIPL TYPES\\
SUPPORT FUNCTIONS\\
SCALAR FUNCTIONS\\
RANDOM NUMBER GENERATION\\
VECTOR \& ELEMENTWISE OPERATIONS\\
SIGNAL PROCESSING FUNCTIONS\\
LINEAR ALGEBRA FUNCTIONS\\
IMPLEMENTATION DEPENDENT INPUT AND OUTPUT\\
VSIPL Addendum\\
\end{tabular}
\end{center}
\label{default}
\end{table}% %table of
\subsection*{Summary of VSIPL Types}\addcontentsline{toc}{subsection}{Summary of VSIPL Types}
\subsection*{Support Functions}\addcontentsline{toc}{subsection}{Support Functions}
    \begin{table}[H]
\caption{Support Function Overview}
\label{tab:vsiplAPISupport}
\begin{center}
\begin{tabular}{l}
Initialization\\
Array and Block Object Functions\\
Vector View Object Functions\\
Matrix View Object Functions\\
Tensor Views\\
\end{tabular}
\end{center}
\label{default}
\end{table}%
\subsubsection*{Initialize and Finalize Operations}\addcontentsline{toc}{subsubsection}{Initialize and Finalize Operations}
    \subsubsection*{Initialize and Finalize Operations}\addcontentsline{toc}{subsubsection}{Initialize and Finalize Operations}
\begin{table}[H]
\caption{Initialization \ref{tab:vsiplAPISupport}}
\label{tab:initSupport}
\begin{center}
\begin{tabular}{|l|l|}\hline
\hlnkFunc{init} & Initialize the VSIP Library\\
\hlnkFunc{finalize} & Finalize the VSIP Library\\
\hline\end{tabular}
\end{center}
\label{default}
\end{table}%
 %table of
\subsubsection*{Block Objects}\addcontentsline{toc}{subsubsection}{Block Class}
    \subsubsection*{Block Objects}\addcontentsline{toc}{subsubsection}{Block Class}
\begin{table}[H]
\caption{Array and Block Object Functions}
\label{tab:blockSupport}
\begin{center}
\begin{tabular}{|l|l|}\hline
\hlnkFunc{blockadmit} & Admit block associated with user allocated\\*&memory.\\
\hlnkFunc{blockbind} & Create and bind a \cvl block to user allocated \\*&memory.\\
\hlnkFunc{blockcreate} & Creates a \cvl block and bind to VSIPL \\*&allocated memory.\\
\hlnkFunc{blockdestroy} & Free any memory allocated by \cvl associated \\*&with a block.\\
\hlnkFunc{blockfind} & Find the pointer to the data bound to a VSIPL \\*&released block object.\\
\hlnkFunc{blockrebind} &Rebind a VSIPL block to user allocated \\*&memory.\\
\hlnkFunc{blockrelease} & Release block associated with user allocated \\*&memory.\\
\hlnkFunc{complete} & Force all deferred VSIPL execution to complete.\\
\hlnkFunc{cstorage} & Returns the preferred complex storage format,\\*& for a precision type for this implementation.\\
\hline\end{tabular}
\end{center}
\label{default}
\end{table}%
 % table of
\subsubsection*{View Objects}\addcontentsline{toc}{subsubsection}{View Class}
    \begin{table}[H]
\caption{Vector View Object Functions}
\label{tab:vectorSupport}
\begin{center}
\begin{tabular}{|l|l|}\hline
\hlnkFunc{alldestroy} & Free both \ttbf{block} and \ttbf{view}\\
\hlnkFunc{bind} & Bind a \ttbf{view} to a \ttbf{block} \\
\hlnkFunc{cloneview} & Clone a \ttbf{view} \\
\hlnkFunc{create} & Create a \ttbf{view} \\
\hlnkFunc{destroy} & Free a \ttbf{view} \\
\hlnkFunc{get} & Get a value from a \ttbf{view}\\
\hlnkFunc{getblock} & Return \ttbf{block} associated with \ttbf{view}\\
\hlnkFunc{getattrib} & Get attribute structure associated with \ttbf{view}\\
\hlnkFunc{getlength} & Get get length of vector \ttbf{view}\\
\hlnkFunc{getoffset} & Get get offset into block of vector \ttbf{view}\\
\hlnkFunc{getstride} & Get stride through block of vector\ttbf{view}\\
\hlnkFunc{imagview} & Return \ttbf{view} of imaginary part of complex \ttbf{view}\\
\hlnkFunc{put} & Get a value from a \ttbf{view}\\
\hlnkFunc{putattrib} & Set attribute structure associated with \ttbf{view}\\
\hlnkFunc{putlength} & Set length of vector \ttbf{view}\\
\hlnkFunc{putoffset} & Set offset into block of vector \ttbf{view}\\
\hlnkFunc{putstride} & Set stride through block of vector\ttbf{view}\\
\hlnkFunc{realview} & Return \ttbf{view} of real part of complex \ttbf{view}\\
\hlnkFunc{subview} & Create a sub-\ttbf{view} of a \ttbf{view} \\
\hline\end{tabular}
\end{center}
%\label{default}
\end{table}%
% table of
    \begin{table}[H]
\caption{Matrix View Object Functions}
\label{tab:matrixSupport}
\begin{center}
\begin{tabular}{|l|l|}\hline
\hlnkFunc{alldestroy} & Free both \ttbf{block} and \ttbf{view}\\
\hlnkFunc{bind} & Bind a \ttbf{view} to a \ttbf{block} \\
\hlnkFunc{cloneview} & Clone a \ttbf{view} \\
\hlnkFunc{colview} & Return a column \ttbf{view} (vector) of a matrix \ttbf{view}\\
\hlnkFunc{create} & Create a \ttbf{view} \\
\hlnkFunc{destroy} & Free a \ttbf{view} \\
\hlnkFunc{diagview} & Return a diagonal \ttbf{view} (vector) of a matrix \ttbf{view}\\
\hlnkFunc{get} & Get a value from a \ttbf{view}\\
\hlnkFunc{getblock} & Return \ttbf{block} associated with \ttbf{view}\\
\hlnkFunc{getattrib} & Get attribute structure associated with \ttbf{view}\\
\hlnkFunc{getcollength} & Get get length of vector \ttbf{view}\\
\hlnkFunc{getrowlength} & Get get length of vector \ttbf{view}\\
\hlnkFunc{getoffset} & Get get offset into block of vector \ttbf{view}\\
\hlnkFunc{getcolstride} & Get stride through block of vector\ttbf{view}\\
\hlnkFunc{getrowstride} & Get stride through block of vector\ttbf{view}\\
\hlnkFunc{imagview} & Return \ttbf{view} of imaginary part of complex \ttbf{view}\\
\hlnkFunc{put} & Get a value from a \ttbf{view}\\
\hlnkFunc{putattrib} & Set attribute structure associated with \ttbf{view}\\
\hlnkFunc{putcollength} & Set length of vector \ttbf{view}\\
\hlnkFunc{putrowlength} & Set length of vector \ttbf{view}\\
\hlnkFunc{putoffset} & Set offset into block of vector \ttbf{view}\\
\hlnkFunc{putcolstride} & Set stride through block of vector\ttbf{view}\\
\hlnkFunc{putrowstride} & Set stride through block of vector\ttbf{view}\\
\hlnkFunc{realview} & Return \ttbf{view} of real part of complex \ttbf{view}\\
\hlnkFunc{rowview} & Return a row \ttbf{view} (vector) of a matrix \ttbf{view}\\
\hlnkFunc{subview} & Create a sub-\ttbf{view} of a \ttbf{view}\\
\hlnkFunc{transview} & Create a matrix \ttbf{view} as a transpose of a matrix\ttbf{view}\\
\hline\end{tabular}
\end{center}
\label{default}
\end{table}%
 % table of
    \begin{table}[H]
\caption{Tensor Views}
\label{tab:tensorSupport}
\begin{center}
\begin{tabular}{|l|l|}\hline
\hlnkFunc{alldestroy} & Free both \ttbf{block} and \ttbf{view}\\
\hlnkFunc{bind} & Bind a \ttbf{view} to a \ttbf{block} \\
\hlnkFunc{cloneview} & Clone a \ttbf{view} \\
\hlnkFunc{create} & Create a \ttbf{view} \\
\hlnkFunc{destroy} & Free a \ttbf{view} \\
\hlnkFunc{get} & Get a value from a \ttbf{view}\\
\hlnkFunc{getattrib} & Get attribute structure associated with \ttbf{view}\\
\hlnkFunc{getblock} & Return \ttbf{block} associated with \ttbf{view}\\
\hlnkFunc{getoffset} & Get get offset into block of vector \ttbf{view}\\
\hlnkFunc{getxlength} & Get get X length of tensor \ttbf{view}\\
\hlnkFunc{getxstride} & Get the X stride attribute of a tensor \ttbf{view}\\
\hlnkFunc{getylength} & Get get Y length of tensor \ttbf{view}\\
\hlnkFunc{getystride} & Get the Y stride attribute of a tensor \ttbf{view}\\
\hlnkFunc{getzlength} & Get get Z length of tensor \ttbf{view}\\
\hlnkFunc{getzstride} & Get the Z stride attribute of a tensor \ttbf{view}\\
\hlnkFunc{imagview} & Return \ttbf{view} of imaginary part of complex \ttbf{view}\\
\hlnkFunc{matrixview} & Create a matrix view of a 2-D slice of the tensor \ttbf{view}\\
\hlnkFunc{put} & Get a value from a \ttbf{view}\\
\hlnkFunc{putattrib} & Set attribute structure associated with \ttbf{view}\\
\hlnkFunc{putoffset} & Set offset into block of vector \ttbf{view}\\
\hlnkFunc{putxlength} & Set X length of tensor \ttbf{view}\\
\hlnkFunc{putxstride} & Set X stride through block of tensor\ttbf{view}\\
\hlnkFunc{putylength} & Set Y length of tensor \ttbf{view}\\
\hlnkFunc{putystride} & Set Y stride through block of tensor\ttbf{view}\\
\hlnkFunc{putzlength} & Set Z length of tensor \ttbf{view}\\
\hlnkFunc{putzstride} & Set Z stride through block of tensor\ttbf{view}\\
\hlnkFunc{realview} & Return \ttbf{view} of real part of complex \ttbf{view}\\
\hlnkFunc{subview} & Create a sub-\ttbf{view} of a \ttbf{view}\\
\hlnkFunc{transview} & Create a transposed of a tensor \ttbf{view}\\
\hlnkFunc{vectview} & Create a vector view of a 1-D slice of the tensor \ttbf{view}\\
\hline\end{tabular}
\end{center}
%\label{default}
\end{table}%
% table of
\subsection*{Scalar Functions}\addcontentsline{toc}{subsection}{Scalar Functions}
In general I do not define scalar functions in \pyjv.  Ease of use is a major goal of the \pyjv module and to further this goal I decided scalars used by or returned by \pyjv functions should be normal python scalars. Using scalar functions (such as $\cos$, $\sin$, etc.) imported from the math module or the numpy module should work fine. That said, you can always use the C VSIPL scalar functions directly since they are in the \ttbf{vsip} module which is included in the \pyjv module.
\subsection*{Random Number Generation}\addcontentsline{toc}{subsection}{Random Number Generation}
\subsection*{Elementwise Operations}\addcontentsline{toc}{subsection}{Elementwise Operations}
Elementwise operatons are simple operations which are done on each element in a matrix or vector. Most of the time, when more than one \ttbf{view} is input, the \ttbf{view} shapes will need to be the same since the operation is done to identically indexed elements for each input \ttbf{view} and the operation result is placed in an identically indexed element of the output \ttbf{view}. 

The tables referenced in this section list elementwise operations with a link to the corresponding function page. Although the function pages are alphabetical, the lists here are in the same order (although not necessarily identical) to the order they appear in the C VSIPL specification.
    \subsection*{Elementwise Operations
\hspace*{\fill}\hyperlink{VSIPspecHead}{(up)}\hypertarget{ElementwiseOperations}{}} \addcontentsline{toc}{subsection}{Elementwise Operations}
Elementwise operatons are simple operations which are done on each element in a matrix or vector. Most of the time, when more than one \ttbf{view} is input, the \ttbf{view} shapes will need to be the same since the operation is done to identically indexed elements for each input \ttbf{view} and the operation result is placed in an identically indexed element of the output \ttbf{view}. \\
The tables referenced in this section list elementwise operations with a link to the corresponding function page. Although the function pages are alphabetical, the lists here are in the same order (although not necessarily identical) to the order they appear in the C VSIPL specification.
\begin{table}[H]
\hypertarget{vectorAndElementwise}{}
\caption{Vector And Elementwise Operations}
\label{tab:elementwiseChapter}
\begin{center}
\begin{tabular}{|l|}\hline
\hyperlink{elementaryMath}{Elementary Math Functions}\\
\hyperlink{unaryOperations}{Unary Operations}\\
\hyperlink{binaryOperations}{Binary Operations}\\
\hyperlink{ternaryOperations}{Ternary Operations}\\
\hyperlink{logicalOperations}{Logical Operations}\\
\hyperlink{selectionOperations}{Selection Operations}\\
\hyperlink{bitwiseOperators}{Bitwise and Boolean Logical Operators}\\
\hyperlink{elementGenerationOperations}{Element Generation and Copy}\\
\hyperlink{manipulationOperations}{Manipulation Operations}\\
\hline\end{tabular}
\end{center}
%\label{default}
\end{table}%
%%%
      \subsubsection*{Elementary Math\hspace*{\fill}\hyperlink{ElementwiseOperations}{(up)}\hypertarget{elementaryMath}{}}\addcontentsline{toc}{subsubsection}{Elementary Math}
Elementary math functions constitute elementwise applications of elementary operations on \ttbf{view}s. The term \emph{elementary} is somewhat arbitrary but includes trigonometric functions, log functions, and exponential functions. Functions here (for elements) are defined by C 89 in the \ttbf{math.h} header file. \ttbf{JVSIP} generally uses this math library to do the calculations for these functions.
\begin{table}[H]
\caption{Elementary Math Functions \ref{tab:elementwiseChapter}}
\label{tab:elementaryMath}
\begin{center}
\begin{tabular}{|l|l|}
\hline
\hlnkFunc{acos} & Arccosine\\
\hlnkFunc{asin} & Arcsine\\
\hlnkFunc{atan} & Arctangent\\
\hlnkFunc{atan2} & Arctangent of Two Arguments\\
\hlnkFunc{cos} & Cosine\\
\hlnkFunc{cosh} & Hyperbolic Cosine\\
\hlnkFunc{exp} & Exponential\\
\hlnkFunc{exp10} & Exponential Base 10\\
\hlnkFunc{log} & Natural Log\\
\hlnkFunc{log10} & Base 10 Log\\
\hlnkFunc{sin} & Sine \\
\hlnkFunc{sinh} & Hyperbolic Sine\\
\hlnkFunc{sqrt} & Square Root\\
\hlnkFunc{tan} & Tangent\\
\hlnkFunc{tanh} & Hyperbolic Tangent\\
\hline
\end{tabular}
\end{center}
\label{default}
\end{table}%

      \begin{table}[H]
\caption{Unary Operations}
\label{tab:unaryOperations}
\begin{center}
\begin{tabular}{|l|l|}
\hlnkFunc{arg} & Argument\\
\hlnkFunc{ceil} & Ceiling\\
\hlnkFunc{conj} & Conjugate\\
\hlnkFunc{cumsum} & Cumulative Sum\\
\hlnkFunc{euler} & Euler\\
\hlnkFunc{floor} & Floor\\
\hlnkFunc{mag} & Magnitude\\
\hlnkFunc{cmagsq} & Complex Magnitude Squared\\
\hlnkFunc{meanval} & Mean Value\\
\hlnkFunc{meansqval} & Mean Square Value\\
\hlnkFunc{modulate} & Modulate\\
\hlnkFunc{neg} & Negate\\
\hlnkFunc{recip} & Reciprocal\\
\hlnkFunc{round} & Round\\
\hlnkFunc{rsqrt} & reciprocal Square Root\\
\hlnkFunc{sq} & Square\\
\hlnkFunc{sumval} & Sum Value\\
\hlnkFunc{sumsqval} & Sum of Squares Value\\
\end{tabular}
\end{center}
\label{default}
\end{table}%

      \begin{table}[H]
\caption{Binary Operations}
\label{tab:binaryOperations}
\begin{center}
\begin{tabular}{|l|l|}
\hlnkFunc{add} & Add\\
\hlnkFunc{div} & Divide\\
\hlnkFunc{expoavg} & Exponential Average\\
\hlnkFunc{hypot} & Hypotenuse\\
\hlnkFunc{jmul} & Conjugate Multiply\\
\hlnkFunc{mul} & Multiply\\
\hlnkFunc{vmmul} & Vector Matrix Multiply\\
\hlnkFunc{sub} & Subtract\\
\end{tabular}
\end{center}
\label{default}
\end{table}%

      \begin{table}[H]
\caption{Ternary Operations}
\label{tab:ternaryOperations}
\begin{center}
\begin{tabular}{|l|l|}
\hlnkFunc{am} & Add and multiply \\
\hlnkFunc{ma} & Multiply and add \\
\hlnkFunc{msb} & Multiply and subtract\\
\hlnkFunc{sbm} & Substract and multiply\\
\end{tabular}
\end{center}
\label{default}
\end{table}%

      \begin{table}[H]
\caption{Logical Operations}
\label{tab:logicalOperations}
\begin{center}
\begin{tabular}{|l|l|}
\hlnkFunc{add} & Add\\
\hlnkFunc{div} & Divide\\
\hlnkFunc{expoavg} & Exponential Average\\
\hlnkFunc{hypot} & Hypotenuse\\
\hlnkFunc{jmul} & Conjugate Multiply\\
\hlnkFunc{mul} & Multiply\\
\hlnkFunc{vmmul} & Vector Matrix Multiply\\
\hlnkFunc{sub} & Subtract\\
\end{tabular}
\end{center}
\label{default}
\end{table}%

      \subsubsection*{Selection Operations}\addcontentsline{toc}{subsubsection}{Selection Operations}
Selection operations involve some logical comparison and, based upon the result, an answer is \emph{selected} and returned; either as a scalar output (signified by \ttbf{val} ending the root name), or elementwise into an appropriately sized output \ttbf{view}. 
\begin{table}[H]
\caption{Selection Operations}
\label{tab:selectionOperations}
\begin{center}
\begin{tabular}{|l|l|}
\hlnkFunc{add} & Add\\
\hlnkFunc{div} & Divide\\
\hlnkFunc{expoavg} & Exponential Average\\
\hlnkFunc{hypot} & Hypotenuse\\
\hlnkFunc{jmul} & Conjugate Multiply\\
\hlnkFunc{mul} & Multiply\\
\hlnkFunc{vmmul} & Vector Matrix Multiply\\
\hlnkFunc{sub} & Subtract\\
\end{tabular}
\end{center}
\label{default}
\end{table}%

      \subsubsection*{Bitwise and Boolean Logical Operators}\addcontentsline{toc}{subsubsection}{Bitwise and Boolean Logical Operators}
This section provides support for standard logical operators. These will operate on integer precision \ttbf{view}s bitwise, or on \ttbf{view}s of precision \ttbf{bl} logically.
\begin{table}[H]
\caption{Bitwise and Boolean Logical Operators}
\label{tab:bitwiseOperators}
\begin{center}
\begin{tabular}{|l|l|}\hline
\hlnkFunc{and} & And operation\\
\hlnkFunc{not} & Not operation\\
\hlnkFunc{or} & Or operation\\
\hlnkFunc{xor} & Exclusive or operation\\
\hline\end{tabular}
\end{center}
\label{default}
\end{table}%

      \subsubsection*{Element Generation and Copy} \addcontentsline{toc}{subsubsection}{Element Generation and Copy}
This section has functions to copy data from one place to another. 
\begin{table}[H]
\caption{Element Generation and Copy}
\label{tab:elementGenerationOperations}
\begin{center}
\begin{tabular}{|l|l|}\hline
\hlnkFunc{copy} & Copy \ttbf{view} to \ttbf{view}\\
\hyperlink{copyto}{\texttt{copyto\_user}} & Copy data in a \ttbf{view} to user specified memory\\
\hyperlink{copyfrom}{\texttt{copyfrom\_user}} & Copy data from user specified memory to a \ttbf{view}\\
\hlnkFunc{fill} & Fill a \ttbf{view} with a constant value\\
\hlnkFunc{ramp} & In a vector \ttbf{view} create equally space \emph{ramp} data\\
\hline\end{tabular}
\end{center}
\label{default}
\end{table}%

      \subsubsection*{Manipulation Operations} \addcontentsline{toc}{subsubsection}{Manipulation Operations}
Manipulation operations are functions which copy \ttbf{view}s, or parts of \ttbf{view}s, from one location to another while doing some manipulation operation to convert the data. For instance the \ttbf{cmplx} function takes two real \ttbf{view}s and copies one \ttbf{view} to the imaginary part of a complex vector and the other \ttbf{view} to the real part of a complex vector. 
\begin{table}[H]
\caption{Manipulation Operations}
\label{tab:manipulationOperations}
\begin{center}
\begin{tabular}{|l|l|}\hline
\hlnkFunc{cmplx} & Complex\\
\hlnkFunc{gather} & Data Gather\\
\hlnkFunc{imag} & Imaginary Part\\
\hlnkFunc{polar} & Hypotenuse\\
\hlnkFunc{real} & Real Part\\
\hlnkFunc{rect} & Rectangular\\
\hlnkFunc{scatter} & Data Scatter\\
\hlnkFunc{swap} & Swap\\
binary & Not Supported\\
bool & Not Supported\\
mary & Not Supported\\
nary & Not Supported\\
serialmary & Not Supported\\
unary & Not Supported\\
\hline\end{tabular}
\end{center}
\label{default}
\end{table}%

    \subsubsection*{Elementary Math\hspace*{\fill}\hyperlink{ElementwiseOperations}{(up)}\hypertarget{elementaryMath}{}}\addcontentsline{toc}{subsubsection}{Elementary Math}
Elementary math functions constitute elementwise applications of elementary operations on \ttbf{view}s. The term \emph{elementary} is somewhat arbitrary but includes trigonometric functions, log functions, and exponential functions. Functions here (for elements) are defined by C 89 in the \ttbf{math.h} header file. \ttbf{JVSIP} generally uses this math library to do the calculations for these functions.
\begin{table}[H]
\caption{Elementary Math Functions \ref{tab:elementwiseChapter}}
\label{tab:elementaryMath}
\begin{center}
\begin{tabular}{|l|l|}
\hline
\hlnkFunc{acos} & Arccosine\\
\hlnkFunc{asin} & Arcsine\\
\hlnkFunc{atan} & Arctangent\\
\hlnkFunc{atan2} & Arctangent of Two Arguments\\
\hlnkFunc{cos} & Cosine\\
\hlnkFunc{cosh} & Hyperbolic Cosine\\
\hlnkFunc{exp} & Exponential\\
\hlnkFunc{exp10} & Exponential Base 10\\
\hlnkFunc{log} & Natural Log\\
\hlnkFunc{log10} & Base 10 Log\\
\hlnkFunc{sin} & Sine \\
\hlnkFunc{sinh} & Hyperbolic Sine\\
\hlnkFunc{sqrt} & Square Root\\
\hlnkFunc{tan} & Tangent\\
\hlnkFunc{tanh} & Hyperbolic Tangent\\
\hline
\end{tabular}
\end{center}
\label{default}
\end{table}%

    \begin{table}[H]
\caption{Unary Operations}
\label{tab:unaryOperations}
\begin{center}
\begin{tabular}{|l|l|}
\hlnkFunc{arg} & Argument\\
\hlnkFunc{ceil} & Ceiling\\
\hlnkFunc{conj} & Conjugate\\
\hlnkFunc{cumsum} & Cumulative Sum\\
\hlnkFunc{euler} & Euler\\
\hlnkFunc{floor} & Floor\\
\hlnkFunc{mag} & Magnitude\\
\hlnkFunc{cmagsq} & Complex Magnitude Squared\\
\hlnkFunc{meanval} & Mean Value\\
\hlnkFunc{meansqval} & Mean Square Value\\
\hlnkFunc{modulate} & Modulate\\
\hlnkFunc{neg} & Negate\\
\hlnkFunc{recip} & Reciprocal\\
\hlnkFunc{round} & Round\\
\hlnkFunc{rsqrt} & reciprocal Square Root\\
\hlnkFunc{sq} & Square\\
\hlnkFunc{sumval} & Sum Value\\
\hlnkFunc{sumsqval} & Sum of Squares Value\\
\end{tabular}
\end{center}
\label{default}
\end{table}%

    \begin{table}[H]
\caption{Binary Operations}
\label{tab:binaryOperations}
\begin{center}
\begin{tabular}{|l|l|}
\hlnkFunc{add} & Add\\
\hlnkFunc{div} & Divide\\
\hlnkFunc{expoavg} & Exponential Average\\
\hlnkFunc{hypot} & Hypotenuse\\
\hlnkFunc{jmul} & Conjugate Multiply\\
\hlnkFunc{mul} & Multiply\\
\hlnkFunc{vmmul} & Vector Matrix Multiply\\
\hlnkFunc{sub} & Subtract\\
\end{tabular}
\end{center}
\label{default}
\end{table}%

    \begin{table}[H]
\caption{Ternary Operations}
\label{tab:ternaryOperations}
\begin{center}
\begin{tabular}{|l|l|}
\hlnkFunc{am} & Add and multiply \\
\hlnkFunc{ma} & Multiply and add \\
\hlnkFunc{msb} & Multiply and subtract\\
\hlnkFunc{sbm} & Substract and multiply\\
\end{tabular}
\end{center}
\label{default}
\end{table}%

    \begin{table}[H]
\caption{Logical Operations}
\label{tab:logicalOperations}
\begin{center}
\begin{tabular}{|l|l|}
\hlnkFunc{add} & Add\\
\hlnkFunc{div} & Divide\\
\hlnkFunc{expoavg} & Exponential Average\\
\hlnkFunc{hypot} & Hypotenuse\\
\hlnkFunc{jmul} & Conjugate Multiply\\
\hlnkFunc{mul} & Multiply\\
\hlnkFunc{vmmul} & Vector Matrix Multiply\\
\hlnkFunc{sub} & Subtract\\
\end{tabular}
\end{center}
\label{default}
\end{table}%

    \subsubsection*{Selection Operations}\addcontentsline{toc}{subsubsection}{Selection Operations}
Selection operations involve some logical comparison and, based upon the result, an answer is \emph{selected} and returned; either as a scalar output (signified by \ttbf{val} ending the root name), or elementwise into an appropriately sized output \ttbf{view}. 
\begin{table}[H]
\caption{Selection Operations}
\label{tab:selectionOperations}
\begin{center}
\begin{tabular}{|l|l|}
\hlnkFunc{add} & Add\\
\hlnkFunc{div} & Divide\\
\hlnkFunc{expoavg} & Exponential Average\\
\hlnkFunc{hypot} & Hypotenuse\\
\hlnkFunc{jmul} & Conjugate Multiply\\
\hlnkFunc{mul} & Multiply\\
\hlnkFunc{vmmul} & Vector Matrix Multiply\\
\hlnkFunc{sub} & Subtract\\
\end{tabular}
\end{center}
\label{default}
\end{table}%

    \subsubsection*{Bitwise and Boolean Logical Operators}\addcontentsline{toc}{subsubsection}{Bitwise and Boolean Logical Operators}
This section provides support for standard logical operators. These will operate on integer precision \ttbf{view}s bitwise, or on \ttbf{view}s of precision \ttbf{bl} logically.
\begin{table}[H]
\caption{Bitwise and Boolean Logical Operators}
\label{tab:bitwiseOperators}
\begin{center}
\begin{tabular}{|l|l|}\hline
\hlnkFunc{and} & And operation\\
\hlnkFunc{not} & Not operation\\
\hlnkFunc{or} & Or operation\\
\hlnkFunc{xor} & Exclusive or operation\\
\hline\end{tabular}
\end{center}
\label{default}
\end{table}%

    \subsubsection*{Element Generation and Copy} \addcontentsline{toc}{subsubsection}{Element Generation and Copy}
This section has functions to copy data from one place to another. 
\begin{table}[H]
\caption{Element Generation and Copy}
\label{tab:elementGenerationOperations}
\begin{center}
\begin{tabular}{|l|l|}\hline
\hlnkFunc{copy} & Copy \ttbf{view} to \ttbf{view}\\
\hyperlink{copyto}{\texttt{copyto\_user}} & Copy data in a \ttbf{view} to user specified memory\\
\hyperlink{copyfrom}{\texttt{copyfrom\_user}} & Copy data from user specified memory to a \ttbf{view}\\
\hlnkFunc{fill} & Fill a \ttbf{view} with a constant value\\
\hlnkFunc{ramp} & In a vector \ttbf{view} create equally space \emph{ramp} data\\
\hline\end{tabular}
\end{center}
\label{default}
\end{table}%

    \subsubsection*{Manipulation Operations} \addcontentsline{toc}{subsubsection}{Manipulation Operations}
Manipulation operations are functions which copy \ttbf{view}s, or parts of \ttbf{view}s, from one location to another while doing some manipulation operation to convert the data. For instance the \ttbf{cmplx} function takes two real \ttbf{view}s and copies one \ttbf{view} to the imaginary part of a complex vector and the other \ttbf{view} to the real part of a complex vector. 
\begin{table}[H]
\caption{Manipulation Operations}
\label{tab:manipulationOperations}
\begin{center}
\begin{tabular}{|l|l|}\hline
\hlnkFunc{cmplx} & Complex\\
\hlnkFunc{gather} & Data Gather\\
\hlnkFunc{imag} & Imaginary Part\\
\hlnkFunc{polar} & Hypotenuse\\
\hlnkFunc{real} & Real Part\\
\hlnkFunc{rect} & Rectangular\\
\hlnkFunc{scatter} & Data Scatter\\
\hlnkFunc{swap} & Swap\\
binary & Not Supported\\
bool & Not Supported\\
mary & Not Supported\\
nary & Not Supported\\
serialmary & Not Supported\\
unary & Not Supported\\
\hline\end{tabular}
\end{center}
\label{default}
\end{table}%

\subsection*{Signal Processing Functions}\addcontentsline{toc}{subsection}{Signal Processing Functions}
    \subsection*{Signal Processing Functions \hyperlink{VSIPspecHead}{(up)}}\addcontentsline{toc}{subsection}{Signal Processing Functions}
\begin{table}[H]
\hypertarget{vsipljSignalProcessing}{}
\caption{Signal Processing Functions}
\label{tab:signalProcessingFunctions}
\begin{center}
\begin{tabular}{|l|}\hline
FFT Functions\ref{tab:fftFunctions}\\
Convolution/Correlation Functions\ref{tab:convCorrFunctions}\\
Window Functions\ref{windowFunctions}\\
Filter Functions\ref{tab:filterFunctions}\\
Miscellaneous Signal Processing Functions\ref{tab:miscSigProcFunctions}\\
\hline\end{tabular}
\end{center}
\label{default}
\end{table}%
\subsubsection*{Fast Fourier Transforms} \addcontentsline{toc}{subsubsection}{Fast Fourier Transforms}
Discrete Fourier transforms are done using an FFT algorithm.  Although the VSIPL 1.3 specification has definitions for two and three dimensional FFTs \jv only supports the one dimensional version.
\begin{table}[H]
\caption{FFT Functions \ref{tab:signalProcessingFunctions}}
\label{tab:fftFunctions}
\begin{center}
\begin{tabular}{|l|l|}
\multicolumn{2}{c}{\rmfamily \bfseries Discrete Fourier Transform Class}\\
\multicolumn{2}{c}{See function page \hyperlink{fftFunc}{\texttt{FFT}}} \\ \hline
fft & Execute FFT\\
fft\_create &Create FFT Object\\
fft\_setwindow &Set a window in the FFT object\\
fft\_destroy & Free FFT object\\
fft\_getattr & Get attributes of FFT object\\
\hline\end{tabular}
\end{center}
\label{default}
\end{table}%
%
\subsubsection*{Convolution and Correlation Functions} 
\addcontentsline{toc}{subsubsection}{Convolution and Correlation Functions}
\begin{table}[H]
\captionsetup{justification=centering}
\caption{Convolution and Correlation Functions \ref{tab:signalProcessingFunctions}}
\label{tab:convCorrFunctions}
\begin{center}
\begin{tabular}{|l|l|} 
\multicolumn{2}{c}{\rmfamily \bfseries Convolution Class}\\
\multicolumn{2}{c}{See function page \hyperlink{convFunc}{\texttt{CONV}}} \\ \hline
conv\_create & Create Convolution Object\\
conv\_destroy & Destroy Convolution Ojbect\\
conv\_attrib & Fill attribute structure with Convolution Object Attributes\\
convolve & Convolve with ttbf{view}\\
\hline
\multicolumn{2}{c}{\rmfamily \bfseries Correlation Class}\\
\multicolumn{2}{c}{See function page \hyperlink{corrFunc}{\texttt{CORR}}} \\ \hline
corr\_create & Create Correlation Object\\
corr\_destroy & Destroy Correlation Object\\
corr\_attrib & Fill attribute structure with Correlation Object Attributes\\
correlate & Do Correlation with \ttbf{view}\\
\hline\end{tabular}
\end{center}
\label{default}
\end{table}%
%
\subsubsection*{Window Functions\hspace*{\fill}\hyperlink{SignalProcessing}{(up)}\hypertarget{windowFunctions}{}} \addcontentsline{toc}{subsubsection}{Window Functions}
When windows were defined in the VSIPL specification they were defined as standalone functions to create a compact block with a vector \ttbf{view} and fill the view with the window coefficients. I think this was an unfortunate way to do it; we would have been better off to first create the \ttbf{view} and then call a function to fill the view with the coefficients.

The method of window creation in C VSIPL makes it difficult to encapsulate windows into the \pyjv{} methods and functions; and I don't want to create a special class just for windows. Consequently window creation has become part of the \ttbf{Block} class for \pyjv.
\begin{table}[H]
\caption{Window Functions}
\label{tab:windowFunctions}
\begin{center}
\begin{tabular}{|l|l|} \hline
\hlnkFuncT{Window}{blackman} & Blackman Window\\
\hlnkFuncT{Window}{cheby} & Chebyshev Window\\
\hlnkFuncT{Window}{hanning} & Hanning Window\\
\hlnkFuncT{Window}{kaiser} & Kaiser Window\\
\hline\end{tabular}
\end{center}
\label{default}
\end{table}%
%
\subsubsection*{Filter Functions} \addcontentsline{toc}{subsubsection}{Filter Functions}
\begin{table}[H]
\caption{Filter Functions \\* See signal processing table \ref{tab:signalProcessingFunctions}}
\label{tab:filterFunctions}
\begin{center}
\begin{tabular}{|l|l|}
\multicolumn{2}{c}{\rmfamily \bfseries Finite Impulse Response Filter Class}\\
\multicolumn{2}{c}{See function page \hyperlink{firFunc}{\texttt{FIR}}} \\ \hline
fir\_create & Create FIR Object\\
fir\_destroy & Free FIR Object\\
firflt & Filter Data\\
fir\_getattr & Get attributes of FIR Object\\
fir\_reset & Reset FIR Object to just created state\\ \hline
\multicolumn{2}{c}{\rmfamily \bfseries Infinite Impulse Response Filter Class} \\
\multicolumn{2}{c}{IIR not supported in \jv} \\ \hline
iir\_create & Create IIR Object\\
iir\_destroy & Free IIR Object\\
iirflt & Filter Data\\
iir\_getattr & Get attributes of IIR Object\\
iir\_reset & Reset IIR Object to just created state\\ \hline
\end{tabular}
\end{center}
\label{default}
\end{table}%
%
\subsubsection*{Miscellaneous Signal Processing Functions} \addcontentsline{toc}{subsubsection}{Miscellaneous Signal Processing Functions}
\input{miscSigProcFunctions}
%
\subsection*{Linear Algebra Functions}\addcontentsline{toc}{subsection}{Linear Algebra Functions}
    \subsection*{
    Linear Algebra Functions 
    \hspace*{\fill} \hyperlink{VSIPspecHead}{(up)}}
    \addcontentsline{toc}{subsection}{Linear Algebra Functions}
\begin{table}[H]
\hypertarget{linearAlgebraFunctions}{}
\caption{Linear Algebra Functions}
\label{tab:linearAlgebraFunctions}
\begin{center}
\begin{tabular}{|l|}\hline
\hyperlink{matrixOperations}{Matrix and Vector Operations}\\
\hyperlink{specialLinearSystemSolvers}{Special Linear System Solvers}\\
\hyperlink{generalSquareSolver}{General Square Linear System Solver}\\
\hyperlink{symmetricPositiveDefiniteSolver}{Symmetric Positive Definite Linear System Solver}\\
\hyperlink{overDeterminedSolver}{Over-determined Linear System Solver}\\
\hyperlink{singularValueDecompostion}{Singular Value Decomposition}\\
\hline\end{tabular}
\end{center}
\label{default}
\end{table}%
%%%
      \subsubsection*{Matrix and Vector Operations\hfill \hyperlink{linearAlgebraFunctions}{(up)}\hypertarget{matrixOperations}{}} \addcontentsline{toc}{subsubsection}{Matrix and Vector Operations}
\begin{table}[H]
\caption{Matrix and Vector Operations.}
\label{tab:matrixOperations}
\begin{center}
\begin{tabular}{|l|l|}\hline
\hlnkFunc{herm} & Matrix Hermitian\\
\hlnkFunc{jdot} & Complex Vector Conjugate Dot Product\\
\hlnkFunc{gemp} & General Matrix Product\\
\hlnkFunc{gems} & General Matrix Sum \\
\hlnkFunc{kron} & Kronecker Product \\
\hlnkFunc{prod3} & 3 by 3 Matrix Product\\
\hlnkFunc{prod4} & 4 by 4 Matrix Product\\
\hlnkFunc{prod} & Matrix product \\
\hlnkFunc{prodh} & Matrix Hermitian Product\\
\hlnkFunc{prodj} & Matrix Conjugate Product\\
\hlnkFunc{prodt} & Matrix Transpose Product\\
\hlnkFunc{trans} & Matrix Transpose\\
\hlnkFunc{dot} & Vector Dot Product\\
\hlnkFunc{outer} & Vector Outer Product\\
\hline\end{tabular}
\end{center}
%\label{default}
\end{table}
%
      \begin{table}[H]
\caption{Special Linear System Solvers}
\label{tab:specialLinearSystemSolvers}
\begin{center}
\begin{tabular}{|l|l|}\hline
\hlnkFunc{covsol} & Solve Covariance System\\
\hlnkFunc{llsqsol} & Solve Linear Least Squares Problem \\
\hlnkFunc{toepsol} & Solve Toeplitz System\\
\hline\end{tabular}
\end{center}
\label{default}
\end{table}%
      \subsubsection*{General Square Linear System Solver\hspace*{\fill} \hyperlink{linearAlgebraFunctions}{(up)}\hypertarget{generalSquareSolver}{}} \addcontentsline{toc}{subsubsection}{General Square Linear System Solver}
\begin{table}[H]
\caption{General Square Linear System Solver}
\label{tab:generalSquareSolver}
\begin{center}
\begin{tabular}{|l|l|}
\multicolumn{2}{c}{\hyperlink{ludFunc}{\rmfamily \bfseries LUD Function set}}\\
\hline
lud & LU Decomposition\\
lud\_create & Create LU Decomposition Object \\
lud\_destroy & Destroy LUD Object \\
lud\_getattr & LUD Get Attributes\\
lusol & Solve General Linear System \\
\hline\end{tabular}
\end{center}
\label{default}
\end{table}%%lud
      \subsubsection*{Symmetric Positive Definite Linear System Solver\hspace*{\fill} \hyperlink{linearAlgebraFunctions}{(up)}\hypertarget{symmetricPositiveDefiniteSolver}{}} \addcontentsline{toc}{subsubsection}{Symmetric Positive Definite Linear System Solver}
\begin{table}[H]
\caption{Symmetric Positive Definite (SPD) Linear System Solver  \ref{tab:linearAlgebraFunctions}}
\label{tab:symmetricPositiveDefiniteSolvers}
\begin{center}
\begin{tabular}{|l|l|}
\multicolumn{2}{c}{\hyperlink{choldFunc}{\rmfamily \bfseries Cholesky Decomposition Function set}}\\
\hline
chold & Cholesky Decomposition\\
chold\_create & Create Cholesky Decomposition Object\\
chold\_destroy & Destroy CHOLD Object\\
chold\_getattr & CHOLD Get Attributes\\
cholsol & Solve SPD Linear System\\
\hline\end{tabular}
\end{center}
%\label{default}
\end{table}%%chold
      \subsubsection*{Over-determined Linear System Solver\hspace*{\fill} \hyperlink{linearAlgebraFunctions}{(up)}\hypertarget{OverDeterminedSolver}{}} \addcontentsline{toc}{subsubsection}{Over-determined Linear System Solver}
\begin{table}[H]
\caption{Over-determined Linear System Solver}
\label{tab:overDeterminedSolver}
\begin{center}
\begin{tabular}{|l|l|}
\multicolumn{2}{c}{\hyperlink{qrdFunc}{\rmfamily \bfseries QRD Function Set}}\\
\hline
qrd & Cholesky Decomposition\\
qrd\_create & Create QR Decomposition Object\\
qrd\_destroy & Destroy QRD Object\\
qrd\_getattr & QRD Get Attributes\\
qrdprodq & Product with Q from QR Decomposition\\
qrdsolr &Solve Linear System Based on R from QRD\\
qrdsol & Solve Covariance or LLSQ System\\
\hline\end{tabular}
\end{center}
\label{default}
\end{table}%%qrd
      \subsubsection*{Singular Value Decomposition\hspace*{\fill} \hyperlink{linearAlgebraFunctions}{(up)}\hypertarget{singularValueDecompostion}{}} \addcontentsline{toc}{subsubsection}{Singular Value Decomposition}
\begin{table}[H]
\caption{Singular Value Decomposition}
\label{tab:singularValueDecompostion}
\begin{center}
\begin{tabular}{|l|l|}
\multicolumn{2}{c}{\hyperlink{svdFunc}{\rmfamily \bfseries SVD Function Set}}\\
\hline\Ts
svd & Matrix Singular Value Decomposition\Bs\\
svd\_create & Create Singular Value Decomposition Object \Bs\\
svd\_destroy & Destroy SVD Object\Bs\\
svd\_getattr & SVD Get Attributes\Bs\\
svd\_produ & Product with U from SV Decomposition\Bs\\
svdprodv & Product with V from SV Decomposition\Bs\\
svdmatu & Return with U from SV Decomposition\Bs\\
svdmatv & Return with V from SV Decomposition\Bs\\
\hline\end{tabular}
\end{center}
\label{default}
\end{table}%%svd
\subsection*{Implementation Dependent Input and Output}\addcontentsline{toc}{subsection}{Implementation Dependent Input and Output}
   \ttbf{JVSIP} does not support implementation dependent IO at this time.
\subsection*{VSIPL Addendum}\addcontentsline{toc}{subsection}{VSIPL Addendum}
Editing the VSIPL specification was becoming difficult because of its length and instabilities in the MS Word source document. When functions were added to the specification for interpolation, permutation and sorting they were added as separate documents in a addendum and basically glued onto the pdf. This allowed for much less editing of the MS Word source document.
\subsubsection*{VSIPL Interpolation}\addcontentsline{toc}{subsubsection}{VSIPL Interpolation}
\subsubsection*{VSIPL Permute}\addcontentsline{toc}{subsubsection}{VSIPL Permute}
\subsubsection*{VSIPL Sort}\addcontentsline{toc}{subsubsection}{VSIPL Sort}
\section*{VSIP Types}
This section covers the enumerated types and special structures. These are declared in the public header file \ilCode{vsip.h}.
 
\section*{JVSIP Function List}\addcontentsline{toc}{section}{JVSIP Function List}
The following pages are a list of available functions in JVSIP. The top part of each page will include a section of available C functions (basically extracted from \ilCode{vsip.h}). Since the VSIPL specification is the primary source of information for C VSIPL not much more is included. 

Following the list of available functions is information on how (and if) the function is supported by \pyjv. The \pyjv section of the function page is more extensive than the C information. Basically there is a line indicating if it is available (as a 	tbf{view} method), if it is a property, and if it is in-place. Then there is a line indicating if it is available as a \pyjv function. Finally there is a comment section with additional information.

Note that comments may follow the C VSIPL and/or pyJvsip section and may also follow both indicating the comments pertain to the entire page and not just C or Python.

\subsection*{PyJvsip Methods}\addcontentsline{toc}{subsection}{PyJvsip Methods}
We note that saying the method is a property means you call it without even an empty argument list. For instance if \ilCode{a} is a \pyjv \ttbf{view} then \ilCode{a.cos} will replace the values in \ilCode{a} with there cosine. Since it is a property we DON'T say \ilCode{a.cos()}. Frequently, but not always, view methods are done in-place and that is also indicated. If it is not done in-place then the method will construct an appropriate output \ttbf{view} and return it filled out with the appropriate values.

An example of a \ttbf{view} method that is not done in place is \ilCode{copy}. For instance \ilCode{b=a.copy} will produce a copy of \ilCode{a} in an appropriate view. Note the new \ttbf{b view} will be the same precision, shape and depth as the calling \ttbf{view} but the \ttbf block will be of an exact size and the stride information will be the minimum stride required for the \ttbf{view}. Additional information on copy is available on its function page.

This is the type of information included on the function pages. Since there seem to be many exceptions we won't provide a lot of rules; and instead refer to the function page.

\subsection*{PyJvsip Functions}\addcontentsline{toc}{subsection}{PyJvsip Functions}
In the \pyjv \ttbf{Function} section we provide information on function calls. For pyJvsip python function calls correspond closely with their C counterpart except that the shape, depth and precision are determined by the argument types used in the call and not by the actual name as is used by C VSIPL.

Not all C functions have a corresponding \pyjv function call. In particular most functions that return a value will be handled using a view method with no need for a function.
 
\afunc{add}{Compute the sum of a scalar and a \ttbf{view} or two \ttbf{view}s. A binary operation. See table \ref{tab:binaryOperations}.}
\cvsiplh
\vspace{.25cm}\newline
\hspace*{1cm}\texttt{
\begin{tabular}[H]{l}
\multicolumn{1}{c}{\rmfamily \bfseries Scalar Add Functions}\\ \hline
vsip\_cscalar\_d vsip\_cadd\_d(\\*\hspace{1cm}vsip\_cscalar\_d, vsip\_cscalar\_d);\\
vsip\_cscalar\_d vsip\_rcadd\_d(\\*\hspace{1cm}vsip\_scalar\_d, vsip\_cscalar\_d);\\
vsip\_cscalar\_f vsip\_cadd\_f(\\*\hspace{1cm}vsip\_cscalar\_f, vsip\_cscalar\_f);\\
vsip\_cscalar\_f vsip\_rcadd\_f(\\*\hspace{1cm}vsip\_scalar\_f, vsip\_cscalar\_f);\\
\end{tabular}
}
\clearpage
\hspace*{1cm}\texttt{
\begin{tabular}[H]{l}
\multicolumn{1}{c}{\rmfamily \bfseries Normal View - View Add Functions}\\ \hline
void vsip\_vadd\_d(\\*\hspace{1cm}const vsip\_vview\_d*, const vsip\_vview\_d*,\\*\hspace{1cm}const vsip\_vview\_d*);\\
void vsip\_vadd\_f(\\*\hspace{1cm}const vsip\_vview\_f*, const vsip\_vview\_f*,\\*\hspace{1cm}const vsip\_vview\_f*);\\
void vsip\_cvadd\_d(\\*\hspace{1cm}const vsip\_cvview\_d*, const vsip\_cvview\_d*,\\*\hspace{1cm}const vsip\_cvview\_d*);\\
void vsip\_cvadd\_f(\\*\hspace{1cm}const vsip\_cvview\_f*, const vsip\_cvview\_f*,\\*\hspace{1cm}const vsip\_cvview\_f*);\\
void vsip\_madd\_d(\\*\hspace{1cm}const vsip\_mview\_d*, const vsip\_mview\_d*,\\*\hspace{1cm}const vsip\_mview\_d*);\\
void vsip\_madd\_f(\\*\hspace{1cm}const vsip\_mview\_f*, const vsip\_mview\_f*,\\*\hspace{1cm}const vsip\_mview\_f*);\\
void vsip\_cmadd\_d(\\*\hspace{1cm}const vsip\_cmview\_d*, const vsip\_cmview\_d*,\\*\hspace{1cm}const vsip\_cmview\_d*);\\
void vsip\_cmadd\_f(\\*\hspace{1cm}const vsip\_cmview\_f*, const vsip\_cmview\_f*,\\*\hspace{1cm}const vsip\_cmview\_f*);\\
void vsip\_vadd\_i(\\*\hspace{1cm}const vsip\_vview\_i*, const vsip\_vview\_i*,\\*\hspace{1cm}const vsip\_vview\_i*);\\
void vsip\_madd\_i(\\*\hspace{1cm}const vsip\_mview\_i*, const vsip\_mview\_i*,\\*\hspace{1cm}const vsip\_mview\_i*);\\
void vsip\_vadd\_si(\\*\hspace{1cm}const vsip\_mview\_si*, const vsip\_mview\_si*,\\*\hspace{1cm}const vsip\_mview\_si*);\\
void vsip\_madd\_si(\\*\hspace{1cm}const vsip\_mview\_si*, const vsip\_mview\_si*,\\*\hspace{1cm}const vsip\_mview\_si*);\\
void vsip\_vadd\_uc(\\*\hspace{1cm}const vsip\_vview\_uc*, const vsip\_vview\_uc*,\\*\hspace{1cm}const vsip\_vview\_uc*);\\
void vsip\_vadd\_vi(\\*\hspace{1cm}const vsip\_vview\_vi*, const vsip\_vview\_vi*,\\*\hspace{1cm}const vsip\_vview\_vi*);\\
\end{tabular}
}
%
\clearpage
\hspace*{1cm}\texttt{
\begin{tabular}[H]{l}
\multicolumn{1}{c}{\rmfamily \bfseries Mixed Depth View - View Add Functions}\\ \hline
void vsip\_rcvadd\_d(\\*\hspace{1cm}const vsip\_vview\_d*, const vsip\_cvview\_d*,\\*\hspace{1cm}const vsip\_cvview\_d*);\\
void vsip\_rcvadd\_f(\\*\hspace{1cm}const vsip\_vview\_f*, const vsip\_cvview\_f*,\\*\hspace{1cm}const vsip\_cvview\_f*);\\
void vsip\_rcmadd\_d(\\*\hspace{1cm}const vsip\_mview\_d*, const vsip\_cmview\_d*,\\*\hspace{1cm}const vsip\_cmview\_d*);\\
void vsip\_rcmadd\_f(\\*\hspace{1cm}const vsip\_mview\_f*, const vsip\_cmview\_f*,\\*\hspace{1cm}const vsip\_cmview\_f*);\\
\end{tabular}
}
\vspace*{.25cm}\newline
\hspace*{1cm}\texttt{
\begin{tabular}[H]{l}
\multicolumn{1}{c}{\rmfamily \bfseries Mixed Depth Scalar - View Add Functions}\\ \hline
void vsip\_rscvadd\_d(\\*\hspace{1cm}vsip\_scalar\_d, const vsip\_cvview\_d*,\\*\hspace{1cm}const vsip\_cvview\_d*);\\
void vsip\_rscvadd\_f(\\*\hspace{1cm}vsip\_scalar\_f, const vsip\_cvview\_f*,\\*\hspace{1cm}const vsip\_cvview\_f*);\\
void vsip\_rscmadd\_d(\\*\hspace{1cm}vsip\_scalar\_d, const vsip\_cmview\_d*,\\*\hspace{1cm}const vsip\_cmview\_d*);\\
void vsip\_rscmadd\_f(\\*\hspace{1cm}vsip\_scalar\_f, const vsip\_cmview\_f*,\\*\hspace{1cm}const vsip\_cmview\_f*);\\
\end{tabular}
}
\clearpage
\hspace*{1cm}\texttt{
\begin{tabular}[H]{l}
\multicolumn{1}{c}{\rmfamily \bfseries Normal Scalar - View Add Functions}\\ \hline
void vsip\_svadd\_d(\\*\hspace{1cm}vsip\_scalar\_d, const vsip\_vview\_d*,\\*\hspace{1cm}const vsip\_vview\_d*);\\
void vsip\_svadd\_f(\\*\hspace{1cm}vsip\_scalar\_f, const vsip\_vview\_f*,\\*\hspace{1cm}const vsip\_vview\_f*);\\
void vsip\_smadd\_d(\\*\hspace{1cm}vsip\_scalar\_d, const vsip\_mview\_d*,\\*\hspace{1cm}const vsip\_mview\_d*);\\
void vsip\_smadd\_f(\\*\hspace{1cm}vsip\_scalar\_f, const vsip\_mview\_f*,\\*\hspace{1cm}const vsip\_mview\_f*);\\
void vsip\_csvadd\_d(\\*\hspace{1cm}vsip\_cscalar\_d, const vsip\_cvview\_d*,\\*\hspace{1cm}const vsip\_cvview\_d*);\\
void vsip\_csvadd\_f(\\*\hspace{1cm}vsip\_cscalar\_f, const vsip\_cvview\_f*,\\*\hspace{1cm}const vsip\_cvview\_f*);\\
void vsip\_csmadd\_d(\\*\hspace{1cm}vsip\_cscalar\_d, const vsip\_cmview\_d*,\\*\hspace{1cm}const vsip\_cmview\_d*);\\
void vsip\_csmadd\_f(\\*\hspace{1cm}vsip\_cscalar\_f, const vsip\_cmview\_f*,\\*\hspace{1cm}const vsip\_cmview\_f*);\\
void vsip\_svadd\_i(\\*\hspace{1cm}vsip\_scalar\_i, const vsip\_vview\_i*,\\*\hspace{1cm}const vsip\_vview\_i*);\\
void vsip\_svadd\_si(\\*\hspace{1cm}vsip\_scalar\_si, const vsip\_vview\_si*,\\*\hspace{1cm}const vsip\_vview\_si*);\\
void vsip\_svadd\_uc(\\*\hspace{1cm}vsip\_scalar\_uc, const vsip\_vview\_uc*,\\*\hspace{1cm}const vsip\_vview\_uc*);\\
void vsip\_svadd\_vi(\\*\hspace{1cm}vsip\_scalar\_vi, const vsip\_vview\_vi*,\\*\hspace{1cm}const vsip\_vview\_vi*);\\
\end{tabular}
}
\pyjvsiph
\newline\hspace*{.8cm}{\textbf{View Method}\\
\hspace*{1.1cm}Overloaded on plus operator.\\
\hspace*{1.1cm}\textbf{In Place: }\hspace{.2cm} yes\\
\hspace*{1.1cm}\textbf{Example: }\ttbf{a += b; a += 2}\\*
\hspace*{1.5cm}Elements of \ttbf{view a} replaced with result.\\
\hspace*{1.1cm}\textbf{Out of Place: }\hspace{.2cm} yes\\
\hspace*{1.1cm}\textbf{Example: }\ttbf{c = a + b; d = 2 + c}\\*
\hspace*{1.5cm}\ttbf{view c} and \ttbf{view d} created and filled with result of operation.\\
\apyfunc{yes}{out = add(in1,in2,out)}
\pyComment{\item{The \ttbf{add} function works much the same as the C VSIPL version except that a convenience pointer to the output view is returned. }
\item{This may be done in-place if \ttbf{in1==out} or \ttbf{in2==out}.}
\item{Argument \ttbf{in1} may be a scalar. For clues to what is allowed see C VSIPL function list.}}\afuncT{acos}{Inverse Cosine. An elementary math function.}{elementaryMath}
\\\cvsiplh
\begin{cfuncs}
vsip\_scalar\_f~vsip\_acos\_f(vsip\_scalar\_f);\\
vsip\_scalar\_d~vsip\_acos\_d(vsip\_scalar\_d);\\
void vsip\_macos\_d(const~vsip\_mview\_d*, const~vsip\_mview\_d*);\\
void vsip\_macos\_f(const~vsip\_mview\_f*, const~vsip\_mview\_f*);\\
void vsip\_vacos\_d(const~vsip\_vview\_d*, const~vsip\_vview\_d*);\\
void vsip\_vacos\_f(const~vsip\_vview\_f*, const~vsip\_vview\_f*);\\
\end{cfuncs}
\pyjvsiph
\viewmthd{yes}{yes}{yes}{inOut.acos}
\apyfunc{yes}{out = acos(in,out)}
\\\begin{minipage}{\textwidth}
\hspace*{.04\textwidth}\textbf{Comments}\\ 
\hspace*{.04\textwidth}\parbox{.95\textwidth}
{\vspace*{.1cm}
\begin{itemize}
\item{The \ttbf{acos} function works much the same as the C VSIPL version except that a convenience pointer to the output view is returned.}
\item{This may be done in-place if \ttbf{in==out}.}
\end{itemize}
}
\end{minipage}\afuncT{am}{Add and multiply. An element-wise function.}{ternaryOperations}
\\\cvsiplh
\newline \hspace*{.8cm} \vspace*{.1cm} \textbf{Available Functions }
\newline \hspace*{1.1cm} {
\ttfamily
\begin{tabular}[H]{l}
void vsip\_cvam\_d(const vsip\_cvview\_d*,const vsip\_cvview\_d*, \\*\hspace{.7cm}const vsip\_cvview\_d*, const vsip\_cvview\_d*)\\
void vsip\_cvam\_f(const vsip\_cvview\_f*,const vsip\_cvview\_f*, \\*\hspace{.7cm}const vsip\_cvview\_f*, const vsip\_cvview\_f*)\\
void vsip\_cvsam\_d(const vsip\_cvview\_d*,vsip\_cscalar\_d, \\*\hspace{.7cm}const vsip\_cvview\_d*, const vsip\_cvview\_d*)\\
void vsip\_cvsam\_f(const vsip\_cvview\_f*,vsip\_cscalar\_f, \\*\hspace{.7cm}const vsip\_cvview\_f*, const vsip\_cvview\_f*)\\
void vsip\_vam\_d(const vsip\_vview\_d*,\\*\hspace{.7cm}const vsip\_vview\_d*,\\*\hspace{.7cm}const vsip\_vview\_d*, const vsip\_vview\_d*)\\
void vsip\_vam\_f(const vsip\_vview\_f*,const vsip\_vview\_f*,\\*\hspace{.7cm}const vsip\_vview\_f*, const vsip\_vview\_f*)\\
void vsip\_vsam\_d(const vsip\_vview\_d*,\\*\hspace{.7cm}vsip\_scalar\_d,const vsip\_vview\_d*, const vsip\_vview\_d*)\\
void vsip\_vsam\_f(const vsip\_vview\_f*,\\*\hspace{.7cm}vsip\_scalar\_f,\\*\hspace{.7cm}const vsip\_vview\_f*, const vsip\_vview\_f*)\\
\end{tabular}
}
\pyComment{\item{The C VSIPL spec has separate man pages for add-multiply functions containing scalar arguments, and those containing only \ttbf{view} arguments.}}
\\\pyjvsiph
\viewmthd{No}{NA}{NA}{NA}
\apyfunc{yes}{\ttbf{out = am(in1,in2,in3,out)}}
\pyComment{\item{Argument \ttbf{in1} is always a \ttbf{view}, argument \ttbf{in2} is either a \ttbf{view} or a scalar and argument \ttbf{in3} is always a \ttbf{view}.}
\item{The \ttbf{am} function works much the same as the C VSIPL version except that a convenience pointer to the output \ttbf{view} is returned.}
\item{This may be done in-place if an input \ttbf{view} is the same as the output \ttbf{view}.}}\afuncT{arg}{Compute the radian value argument of complex elements in the interval $[-\pi,\pi]$. An Unary Operation.}{unaryOperations}
\\\cvsiplh
\newline \hspace*{.8cm} \vspace*{.1cm} \textbf{Available Functions }
\newline \hspace*{1.1cm} {
\ttfamily
\begin{tabular}[H]{l}
vsip\_scalar\_d vsip\_arg\_d(vsip\_cscalar\_d);\\
vsip\_scalar\_f vsip\_arg\_f(vsip\_cscalar\_f);\\
void vsip\_marg\_d(\\*\hspace{.5cm}const vsip\_cmview\_d*, const vsip\_mview\_d*);\\
void vsip\_marg\_f(\\*\hspace{.5cm}const vsip\_cmview\_f*, const vsip\_mview\_f*);\\
void vsip\_varg\_d(\\*\hspace{.5cm}const vsip\_cvview\_d*, const vsip\_vview\_d*);\\
void vsip\_varg\_f(\\*\hspace{.5cm}const vsip\_cvview\_f*, const vsip\_vview\_f*);\\
\end{tabular}
}
\\\pyjvsiph
\viewmthd{yes}{yes}{No}{out=in.arg}
\apyfunc{yes}{out = arg(in,out)}
\newline \hspace*{.8cm}\textbf{Comment}\\
\hspace*{.8cm}\parbox{11cm}{\vspace*{.2cm}
\begin{itemize}
\item{Since \ttbf{arg} takes a view of \emph{depth} complex and outputs to a view of \emph{depth} real of the same \emph{shape} and \emph{precision} as the input view the \ttbf{arg} method will create a view of the proper type and size and return it.}
\item{The \ttbf{arg} function works the same as the C VSIPL function except a convenience pointer is returned to the output view}
\item{For the function limited in-place functionality exists with replacement of the real or imaginary view of the input with the output. For instance \ilCode{out=arg(in,in.realview)} works fine.}
\end{itemize}}
\afuncT{asin}{Inverse Cosine. An elementary math function.}{elementaryMath}
\\\cvsiplh
\begin{cfuncs}
vsip\_scalar\_f vsip\_asin\_f(vsip\_scalar\_f a);\Bs\\
vsip\_scalar\_d vsip\_asin\_d(vsip\_scalar\_d a);\Bs\\
void vsip\_masin\_d(const~vsip\_mview\_d*, const~vsip\_mview\_d*);\Bs\\
void vsip\_masin\_f(const~vsip\_mview\_f*, const~vsip\_mview\_f*);\Bs\\
void vsip\_vasin\_d(const~vsip\_vview\_d*, const~vsip\_vview\_d*);\Bs\\
void vsip\_vasin\_f(const~vsip\_vview\_f*, const~vsip\_vview\_f*);\Bs\\
\end{cfuncs}
\pyjvsiph
\viewmthd{yes}{yes}{yes}{inOut.asin}
\apyfunc{yes}{out = asin(in,out)}
\pyComment{
\item{The \ttbf{asin} function works much the same as the C VSIPL version except that a convenience pointer to the output view is returned. This may be done in-place if \ttbf{in==out}.}}

\afunc{atan}{Computes the principal radian value,[-+/2, +/2] of the arctangent for each element of a \ttbf{view}. See elementary math functions table \ref{tab:elementaryMath}.}
\cvsiplh
\pyjvsiph\afuncT{atan2}{Arctangent of Two Arguments; An elementwise function. Computes the four quadrant radian value, $[-\pi,\pi]$, of the arctangent of the ratio of the corresponding elements of two input views.}{elementaryMath}
\\\cvsiplh
\hspace*{.8cm} \vspace*{.1cm} \textbf{Available Functions } \\
\hspace*{1cm}
\texttt{
\begin{tabular}[H]{l}
vsip\_scalar\_d vsip\_atan2\_d(vsip\_scalar\_d, vsip\_scalar\_d);\\
vsip\_scalar\_f vsip\_atan2\_f(vsip\_scalar\_f, vsip\_scalar\_f);\\
void vsip\_matan2\_d(const vsip\_mview\_d*, const vsip\_mview\_d*, const vsip\_mview\_d*);\\
void vsip\_matan2\_f(const vsip\_mview\_f*, const vsip\_mview\_f*, const vsip\_mview\_f*);\\
void vsip\_vatan2\_d(const vsip\_vview\_d*, const vsip\_vview\_d*, const vsip\_vview\_d*);\\
void vsip\_vatan2\_f(const vsip\_vview\_f*, const vsip\_vview\_f*, const vsip\_vview\_f*);\\
\end{tabular}
}
\\\pyjvsiph
\viewmthd{no}{NA}{NA}{NA}
\apyfunc{yes}{out = atan2(inOne,inTwo,out)}
\pyComment{\item{The \ttbf{atan2} function works much the same as the C VSIPL version except that a convenience pointer to the output view is returned. This may be done in-place if \ttbf{inOne==out} or \ttbf{inTwo==out}.}}
\afunc{blockadmit}{A Block Object Support Function. See table \ref{tab:blockSupport}}
\\\cvsiplh
\\\pyjvsiph
\pyjvComment{
\item{The \ilCode{blockadmit} function}
}
\afunc{blockbind}{A Block Object Support Function. See table \ref{tab:blockSupport}}
\\\cvsiplh
\\\pyjvsiph
\pyjvComment{
\item{The \ilCode{blockbind} function}
}
\afunc{blockcreate}{A Block Object Support Function. See table \ref{tab:blockSupport}}
\\\cvsiplh
\\\pyjvsiph
\pyjvComment{
\item{The \ilCode{blockcreate} function}
}
\afunc{blockdestroy}{A Block Object Support Function. See table \ref{tab:blockSupport}}
\\\cvsiplh
\\\pyjvsiph
\pyjvComment{
\item{The \ilCode{blockbind} function}
}
\afunc{blockfind}{A Block Object Support Function. See table \ref{tab:blockSupport}}
\\\cvsiplh
\\\pyjvsiph
\pyjvComment{
\item{The \ilCode{blockbind} function}
}
\afunc{blockrebind}{A Block Object Support Function. See table \ref{tab:blockSupport}}
\\\cvsiplh
\\\pyjvsiph
\pyjvComment{
\item{The \ilCode{blockbind} function}
}
\afunc{blockrelease}{A Block Object Support Function. See table \ref{tab:blockSupport}}
\\\cvsiplh
\\\pyjvsiph
\pyjvComment{
\item{The \ilCode{blockbind} function}
}
\afunc{complete}{A Block Object Support Function. See table \ref{tab:blockSupport}}
\\\cvsiplh
\\\pyjvsiph
\pyjvComment{
\item{The \ilCode{blockbind} function}
}
\afunc{cstorage}{A Block Object Support Function. See table \ref{tab:blockSupport}}
\\\cvsiplh
\\\pyjvsiph
\pyjvComment{
\item{The \ilCode{blockbind} function}
}
\afunc{ceil}{Ceiling. Not currently supported in \jv. An unary operation. See table \ref{tab:unaryOperations}}
\cvsiplh
\pyjvsiph\clearpage
{\large \textbf{\hypertarget{convFunc}{CONV Class}}\hspace*{\fill}\hyperlink{ConvCorrFunctions}{up}\vspace{.2cm}\\
\\\cvsiplh 
\\\pyjvsiph\clearpage
{\large \textbf{\hypertarget{corrFunc}{CORR Class}}}\vspace{.2cm}
\\\cvsiplh
\\\pyjvsiph
\afuncT{cos}{Cosine; An elementary math function.}{elementaryMath}
\\\cvsiplh
\newline \hspace*{.8cm} \vspace*{.1cm} \textbf{Available Functions }
\newline \hspace*{1.1cm} {
\ttfamily
\begin{tabular}[H]{l}
vsip\_scalar\_f vsip\_cos\_f(vsip\_scalar\_f a);\\
vsip\_scalar\_d vsip\_cos\_d(vsip\_scalar\_d a);\\
void vsip\_mcos\_d(const vsip\_mview\_d*, const vsip\_mview\_d*);\\
void vsip\_mcos\_f(const vsip\_mview\_f*, const vsip\_mview\_f*);\\
void vsip\_vcos\_d(const vsip\_vview\_d*, const vsip\_vview\_d*);\\
void vsip\_vcos\_f(const vsip\_vview\_f*, const vsip\_vview\_f*);\\
\end{tabular}
}
\\\pyjvsiph
\viewmthd{yes}{yes}{yes}{inOut.cos}
\apyfunc{yes}{out = cos(in,out)}
\pyComment{\item{The \ttbf{cos} function works much the same as the C VSIPL version except that a convenience pointer to the output view is returned. This may be done in-place if \ttbf{in==out}.}}
\afuncT{cosh}{Hyperbolic Cosine; An elementwise function}{elementaryMath}
\\\cvsiplh
\afh
\\\hspace*{.04\textwidth} {
\ttfamily
\begin{tabular}[H]{l}
vsip\_scalar\_f vsip\_cosh\_f(vsip\_scalar\_f a);\\
vsip\_scalar\_d vsip\_cosh\_d(vsip\_scalar\_d a);\\
void vsip\_mcosh\_d(const vsip\_mview\_d*, const vsip\_mview\_d*);\\
void vsip\_mcosh\_f(const vsip\_mview\_f*, const vsip\_mview\_f*);\\
void vsip\_vcosh\_d(const vsip\_vview\_d*, const vsip\_vview\_d*);\\
void vsip\_vcosh\_f(const vsip\_vview\_f*, const vsip\_vview\_f*);\\
\end{tabular}
}
\\\pyjvsiph
\viewmthd{yes}{yes}{yes}{inOut.cosh}
\apyfunc{yes}{out = cosh(in,out)}
\pyComment{
\item{The \ttbf{cosh} function works much the same as the C VSIPL version except that a convenience pointer to the output view is returned. This may be done in-place if \ttbf{in==out}.}}
\afunc{ceil}{Ceiling. Not currently supported in \jv. An unary operation. See table \ref{tab:unaryOperations}}
\cvsiplh
\pyjvsiph\afunc{conj}{Conjugate}
\cvsiplh
\pyjvsiph
\afuncT{cumsum}{Cumulative Sum}{unaryOperations}
\index{Cumulative Sum}
\\\cvsiplh
\\\pyjvsiph\afunc{copy}{Copy Data between two views. Some mixed types are supported so this method can be used to produce a copy of data of a new precision}{elementGenerationOperations}
\\\cvsiplh
\newline \hspace*{.8cm} \vspace*{.1cm} \textbf{Available Functions }
\newline \hspace*{1cm} {\ttfamily
\begin{tabular}[H]{l}
void vsip\_cmcopy\_d\_d(\\*
\hspace{1cm}const vsip\_cmview\_d*, const vsip\_cmview\_d*);\\
void vsip\_cmcopy\_d\_f(\\*
\hspace{1cm}const vsip\_cmview\_d*, const vsip\_cmview\_f*);\\
void vsip\_cmcopy\_f\_d(\\*
\hspace{1cm}const vsip\_cmview\_f*, const vsip\_cmview\_d*);\\
void vsip\_cmcopy\_f\_f(\\*
\hspace{1cm}const vsip\_cmview\_f*, const vsip\_cmview\_f*);\\
void vsip\_cvcopy\_d\_d\\*
\hspace{1cm}(const vsip\_cvview\_d*, const vsip\_cvview\_d*);\\
$\cdots$  \emph{etc.} \end{tabular}
}
\newline \hspace*{1cm}
\parbox{11cm}{There are many copy functions. To see all supported search the \ilCode{vsip.h} header file.\footnotemark}
\footnotetext{For instance \ttbf{grep copy\_ vsip.h} will list all available copy functions.}
\\\pyjvsiph
\viewmthd{yes}{yes}{no}{\parbox[t]{4cm}{out=in.copy\\out=in.copyrm\\out=in.copycm}}
\newline\hspace*{1cm}\parbox{11cm}{The \ttbf{copy} method creates a new view and data space that is the same shape, precision and depth as the input view and copies the data from the \ilCode{in} view to the \ilCode{out} view. The block in the \ilCode{out} view will be the exact size needed to hold the data and will be unit stride along the major direction of the \ilCode{in} view.\\The {\texttt{\bfseries{copycm}}} method is the same as the \ilCode{copy} method except the output view will always be row major independent of the input views major direction.\\The \ttbf{copyrm} method is the same as the \ilCode{copy} method except the output view will always be column major independent of the input views major direction.\\If the input view is a vector the three copy methods have identical results.}
\newline
\apyfunc{yes}{out = copy(in,out)}
\newline\hspace*{1cm}\parbox{11cm}{The \ttbf{copy} function works much the same as the C VSIPL version except that a convenience pointer to the output view is returned.}

\afunc{div}{Divide two \ttbf{view}s, a scalar and a \ttbf{view} or a \ttbf{view} and a scalar. A binary operation. See table \ref{tab:binaryOperations}.}
\cvsiplh
\newline
\hspace*{1cm}\parbox{.9\textwidth}{\textrm{There are many combinations of divide available in \jv. The specification provides the normal \ttbf{view} divides of complex-complex and real-real types; but also provides real-complex and complex-real divides as well as mixed scalar -\ttbf{view} and \ttbf{view}-scalar devides. Consequently the listed available functions are broken up into several tables below.}}\vspace{.1cm}
\newline
{\hspace*{.8cm} \vspace*{.1cm} \textbf{Available Functions } \newline}
\hspace*{1cm}
\texttt{
\begin{tabular}[H]{l}
\multicolumn{1}{c}{\rmfamily \bfseries Scalar Functions}\\ \hline
vsip\_cscalar\_d vsip\_cdiv\_d(\\*\hspace{1cm}vsip\_cscalar\_d, vsip\_cscalar\_d);\\
vsip\_cscalar\_d vsip\_crdiv\_d(\\*\hspace{1cm}vsip\_cscalar\_d, vsip\_scalar\_d);\\
vsip\_cscalar\_f vsip\_cdiv\_f(\\*\hspace{1cm}vsip\_cscalar\_f, vsip\_cscalar\_f);\\
vsip\_cscalar\_f vsip\_crdiv\_f(\\*\hspace{1cm}vsip\_cscalar\_f, vsip\_scalar\_f);\\
\end{tabular}
}
\clearpage
\hspace*{1cm}\texttt{
\begin{tabular}[H]{l}
\multicolumn{1}{c}{\rmfamily \bfseries Normal View Functions}\\ \hline
void vsip\_vdiv\_d(\\*\hspace{1cm}const vsip\_vview\_d*, const vsip\_vview\_d*,\\*\hspace{1cm}const vsip\_vview\_d*);\\
void vsip\_vdiv\_f(\\*\hspace{1cm}const vsip\_vview\_f*, const vsip\_vview\_f*,\\*\hspace{1cm}const vsip\_vview\_f*);\\
void vsip\_mdiv\_d(\\*\hspace{1cm}const vsip\_mview\_d*, const vsip\_mview\_d*,\\*\hspace{1cm}const vsip\_mview\_d*);\\
void vsip\_mdiv\_f(\\*\hspace{1cm}const vsip\_mview\_f*, const vsip\_mview\_f*,\\*\hspace{1cm}const vsip\_mview\_f*);\\
void vsip\_cvdiv\_d(\\*\hspace{1cm}const vsip\_cvview\_d*, const vsip\_cvview\_d*,\\*\hspace{1cm}const vsip\_cvview\_d*);\\
void vsip\_cvdiv\_f(\\*\hspace{1cm}const vsip\_cvview\_f*, const vsip\_cvview\_f*,\\*\hspace{1cm}const vsip\_cvview\_f*);\\
void vsip\_cmdiv\_d(\\*\hspace{1cm}const vsip\_cmview\_d*, const vsip\_cmview\_d*,\\*\hspace{1cm}const vsip\_cmview\_d*);\\
void vsip\_cmdiv\_f(\\*\hspace{1cm}const vsip\_cmview\_f*, const vsip\_cmview\_f*,\\*\hspace{1cm}const vsip\_cmview\_f*);\\
\end{tabular}
}
\clearpage
\hspace*{1cm}\texttt{
\begin{tabular}[H]{l}
\multicolumn{1}{c}{\rmfamily \bfseries Mixed Depth View Functions}\\ \hline
void vsip\_rcvdiv\_d(\\*\hspace{1cm}const vsip\_vview\_d*, const vsip\_cvview\_d*,\\*\hspace{1cm}const vsip\_cvview\_d*);\\
void vsip\_rcvdiv\_f(\\*\hspace{1cm}const vsip\_vview\_f*, const vsip\_cvview\_f*,\\*\hspace{1cm}const vsip\_cvview\_f*);\\
void vsip\_crvdiv\_d(\\*\hspace{1cm}const vsip\_cvview\_d*, const vsip\_vview\_d*,\\*\hspace{1cm}const vsip\_cvview\_d*);\\
void vsip\_crvdiv\_f(\\*\hspace{1cm}const vsip\_cvview\_f*, const vsip\_vview\_f*,\\*\hspace{1cm}const vsip\_cvview\_f*);\\
void vsip\_rcmdiv\_d(\\*\hspace{1cm}const vsip\_mview\_d*, const vsip\_cmview\_d*,\\*\hspace{1cm}const vsip\_cmview\_d*);\\
void vsip\_rcmdiv\_f(\\*\hspace{1cm}const vsip\_mview\_f*, const vsip\_cmview\_f*,\\*\hspace{1cm}const vsip\_cmview\_f*);\\
void vsip\_crmdiv\_d(\\*\hspace{1cm}const vsip\_cmview\_d*, const vsip\_mview\_d*,\\*\hspace{1cm}const vsip\_cmview\_d*);\\
void vsip\_crmdiv\_f(\\*\hspace{1cm}const vsip\_cmview\_f*, const vsip\_mview\_f*,\\*\hspace{1cm}const vsip\_cmview\_f*);\\
\end{tabular}
}
\newline\hspace*{1cm}\texttt{
\begin{tabular}[H]{l}
\multicolumn{1}{c}{\rmfamily \bfseries View Divide Scalar Functions}\\ \hline
void vsip\_vsdiv\_d(\\*\hspace{1cm}const vsip\_vview\_d*, vsip\_scalar\_d,\\*\hspace{1cm}const vsip\_vview\_d*);\\
void vsip\_vsdiv\_f(\\*\hspace{1cm}const vsip\_vview\_f*, vsip\_scalar\_f,\\*\hspace{1cm}const vsip\_vview\_f*);\\
void vsip\_cvrsdiv\_d(\\*\hspace{1cm}const vsip\_cvview\_d*, vsip\_scalar\_d,\\*\hspace{1cm}const vsip\_cvview\_d*);\\
void vsip\_cvrsdiv\_f(\\*\hspace{1cm}const vsip\_cvview\_f*, vsip\_scalar\_f,\\*\hspace{1cm}const vsip\_cvview\_f*);\\
void vsip\_msdiv\_d(\\*\hspace{1cm}const vsip\_mview\_d*, vsip\_scalar\_d,\\*\hspace{1cm}const vsip\_mview\_d*);\\
void vsip\_msdiv\_f(\\*\hspace{1cm}const vsip\_mview\_f*, vsip\_scalar\_f,\\*\hspace{1cm}const vsip\_mview\_f*);\\
void vsip\_cmrsdiv\_d(\\*\hspace{1cm}const vsip\_cmview\_d*, vsip\_scalar\_d,\\*\hspace{1cm}const vsip\_cmview\_d*);\\
void vsip\_cmrsdiv\_f(\\*\hspace{1cm}const vsip\_cmview\_f*, vsip\_scalar\_f,\\*\hspace{1cm}const vsip\_cmview\_f*);\\
\end{tabular}
}
\newline\hspace*{1cm}\texttt{
\begin{tabular}[H]{l}
\multicolumn{1}{c}{\rmfamily \bfseries Scalar Divide View Functions}\\ \hline
void vsip\_svdiv\_d(\\*\hspace{1cm}vsip\_scalar\_d, const vsip\_vview\_d*,\\*\hspace{1cm}const vsip\_vview\_d*);\\
void vsip\_svdiv\_f(\\*\hspace{1cm}vsip\_scalar\_f, const vsip\_vview\_f*,\\*\hspace{1cm}const vsip\_vview\_f*);\\
void vsip\_rscvdiv\_d(\\*\hspace{1cm}vsip\_scalar\_d, const vsip\_cvview\_d*,\\*\hspace{1cm}const vsip\_cvview\_d*);\\
void vsip\_rscvdiv\_f(\\*\hspace{1cm}vsip\_scalar\_f, const vsip\_cvview\_f*,\\*\hspace{1cm}const vsip\_cvview\_f*);\\
void vsip\_csvdiv\_d(\\*\hspace{1cm}vsip\_cscalar\_d, const vsip\_cvview\_d*,\\*\hspace{1cm}const vsip\_cvview\_d*);\\
void vsip\_csvdiv\_f(\\*\hspace{1cm}vsip\_cscalar\_f, const vsip\_cvview\_f*,\\*\hspace{1cm}const vsip\_cvview\_f*);\\
void vsip\_smdiv\_d(\\*\hspace{1cm}vsip\_scalar\_d, const vsip\_mview\_d*,\\*\hspace{1cm}const vsip\_mview\_d*);\\
void vsip\_smdiv\_f(\\*\hspace{1cm}vsip\_scalar\_f, const vsip\_mview\_f*,\\*\hspace{1cm}const vsip\_mview\_f*);\\
void vsip\_rscmdiv\_d(\\*\hspace{1cm}vsip\_scalar\_d, const vsip\_cmview\_d*,\\*\hspace{1cm}const vsip\_cmview\_d*);\\
void vsip\_rscmdiv\_f(\\*\hspace{1cm}vsip\_scalar\_f, const vsip\_cmview\_f*,\\*\hspace{1cm}const vsip\_cmview\_f*);\\
void vsip\_csmdiv\_d(\\*\hspace{1cm}vsip\_cscalar\_d, const vsip\_cmview\_d*,\\*\hspace{1cm}const vsip\_cmview\_d*);\\
void vsip\_csmdiv\_f(\\*\hspace{1cm}vsip\_cscalar\_f, const vsip\_cmview\_f*,\\*\hspace{1cm}const vsip\_cmview\_f*);\\
\end{tabular}
}
\pyjvsiph
\newline\hspace*{.8cm}{\textbf{View Method}\\
\hspace*{1.1cm}Overloaded on divide operator.\\
\hspace*{1.1cm}\textbf{In Place: }\hspace{.2cm} yes\\
\hspace*{1.1cm}\textbf{Example: }\ttbf{a /= b; a /= 2}\\*
\hspace*{1.5cm}Elements of \ttbf{view a} replaced with result.\\
\hspace*{1.1cm}\textbf{Out of Place: }\hspace{.2cm} yes\\
\hspace*{1.1cm}\textbf{Example: }\ttbf{c = a / b; d = 2 / c; e = c / 2}\\*
\hspace*{1.5cm}\parbox{.85\textwidth}{\ttbf{view c}, \ttbf{view d}, and \ttbf{view e}  created and filled with result of operation.}\\
\afuncT{euler}{Euler}{unaryOperations}
\\\cvsiplh
\afh
\\\hspace*{.04\textwidth} {
\ttfamily
}
\\\pyjvsiph\afuncT{exp}{Exponential; An elementwise function}{elementaryMath}
\\\cvsiplh
\\\pyjvsiph
\afunc{exp10}{Exponential Base 10; An elementwise function}
\\\cvsiplh
\\\pyjvsiph

\afuncT{log}{Natural logarithm; An element-wise function.}{elementaryMath}
\\\cvsiplh
\newline \hspace*{.8cm} \vspace*{.1cm} \textbf{Available Functions }
\newline \hspace*{1.1cm} {
\ttfamily
\begin{tabular}[H]{l}
\end{tabular}
}
\\\pyjvsiph
\viewmthd{yes}{yes}{yes}{inOut.sin}
\apyfunc{yes}{out = sin(in,out)}
\newline\hspace*{1.2cm}\parbox{10.8cm}{\vspace*{.1cm}The \ttbf{log} function works much the same as the C VSIPL version except that a convenience pointer to the output view is returned. This may be done in-place if \ttbf{in==out}.}
\afuncT{log10}{Compute the base ten logarithm; An element-wise function.}{elementaryMath}
\\\cvsiplh
\newline \hspace*{.8cm} \vspace*{.1cm} \textbf{Available Functions }
\newline \hspace*{1.1cm} {
\ttfamily
\begin{tabular}[H]{l}
vsip\_scalar\_d vsip\_log10\_d(vsip\_scalar\_d)\\
vsip\_scalar\_f vsip\_log10\_f(vsip\_scalar\_f)\\
void vsip\_mlog10\_d(\\*\hspace{1cm}const vsip\_mview\_d*, const vsip\_mview\_d*);\\
void vsip\_mlog10\_f(\\*\hspace{1cm}const vsip\_mview\_f*, const vsip\_mview\_f*);\\
void vsip\_vlog10\_d(\\*\hspace{1cm}const vsip\_vview\_d*, const vsip\_vview\_d*);\\
void vsip\_vlog10\_f(\\*\hspace{1cm}const vsip\_vview\_f*, const vsip\_vview\_f*);\\
\end{tabular}
}
\\\pyjvsiph
\viewmthd{yes}{yes}{yes}{inOut.sin}
\apyfunc{yes}{out = sin(in,out)}
\newline\hspace*{1.2cm}\parbox{10.8cm}{\vspace*{.1cm}The \ttbf{log10} function works much the same as the C VSIPL version except that a convenience pointer to the output view is returned. This may be done in-place if \ttbf{in==out}.}

\clearpage
{\large \textbf{\hypertarget{fftFunc}{FFT Class}}}\vspace{.2cm}\\
\hspace*{.3cm}
\parbox{0.85\textwidth}{Discrete Fourier Transforms. See FFT Functions table \ref{tab:fftFunctions}}
\cvsiplh 
\newline \hspace*{.8cm} \vspace*{.1cm} \textbf{Available Functions }
\newline \hspace*{.8cm} \vspace*{.1cm} \texttt{fft\_create}
\newline \hspace*{1.1cm} {
\ttfamily
\begin{tabular}[H]{l}\hline
\hline \multicolumn{1}{c}{\rmfamily \bfseries FFT Create Functions}\\ \hline
vsip\_fft\_d* vsip\_ccfftip\_create\_d(vsip\_length, vsip\_scalar\_d,\\*\hspace{.7cm}vsip\_fft\_dir, unsigned int, vsip\_alg\_hint);\\
vsip\_fft\_d* vsip\_ccfftop\_create\_d(vsip\_length, vsip\_scalar\_d,\\*\hspace{.7cm}vsip\_fft\_dir, unsigned int, vsip\_alg\_hint);\\
vsip\_fft\_d* vsip\_crfftop\_create\_d(vsip\_length,vsip\_scalar\_d,\\*\hspace{.7cm}unsigned int, vsip\_alg\_hint);\\
vsip\_fft\_d* vsip\_rcfftop\_create\_d(vsip\_length,vsip\_scalar\_d,\\*\hspace{.7cm}unsigned int, vsip\_alg\_hint);\\
vsip\_fft\_f* vsip\_ccfftip\_create\_f(vsip\_length,vsip\_scalar\_f,\\*\hspace{.7cm}vsip\_fft\_dir, unsigned int, vsip\_alg\_hint);\\
vsip\_fft\_f* vsip\_ccfftop\_create\_f(vsip\_length,vsip\_scalar\_f,\\*\hspace{.7cm}vsip\_fft\_dir, unsigned int, vsip\_alg\_hint);\\
vsip\_fft\_f* vsip\_crfftop\_create\_f(vsip\_length,vsip\_scalar\_f,\\*\hspace{.7cm}unsigned int, vsip\_alg\_hint);\\
vsip\_fft\_f* vsip\_rcfftop\_create\_f(vsip\_length,vsip\_scalar\_f,\\*\hspace{.7cm}unsigned int, vsip\_alg\_hint);\\
\hline \multicolumn{1}{c}{\rmfamily \bfseries Multiple FFT Create Functions}\\ \hline
vsip\_fftm\_d* vsip\_ccfftmip\_create\_d(vsip\_length, vsip\_length,\\*\hspace{.7cm}vsip\_scalar\_d, vsip\_fft\_dir, vsip\_major, unsigned int,\\*\hspace{.7cm}vsip\_alg\_hint);\\
vsip\_fftm\_d* vsip\_ccfftmop\_create\_d(vsip\_length, vsip\_length,\\*\hspace{.7cm}vsip\_scalar\_d, vsip\_fft\_dir, vsip\_major, unsigned int,\\*\hspace{.7cm}vsip\_alg\_hint);\\
vsip\_fftm\_d* vsip\_crfftmop\_create\_d(vsip\_length, vsip\_length,\\*\hspace{.7cm}vsip\_scalar\_d, vsip\_major, unsigned int, vsip\_alg\_hint);\\
vsip\_fftm\_d* vsip\_rcfftmop\_create\_d(vsip\_length, vsip\_length,\\*\hspace{.7cm}vsip\_scalar\_d, vsip\_major, unsigned int, vsip\_alg\_hint);\\
vsip\_fftm\_f* vsip\_ccfftmip\_create\_f(vsip\_length, vsip\_length,\\*\hspace{.7cm}vsip\_scalar\_f, vsip\_fft\_dir, vsip\_major, unsigned int,\\*\hspace{.7cm}vsip\_alg\_hint);\\
vsip\_fftm\_f* vsip\_ccfftmop\_create\_f(vsip\_length, vsip\_length,\\*\hspace{.7cm}vsip\_scalar\_f, vsip\_fft\_dir, vsip\_major, unsigned int,\\*\hspace{.7cm}vsip\_alg\_hint);\\
vsip\_fftm\_f* vsip\_crfftmop\_create\_f(vsip\_length, vsip\_length,\\*\hspace{.7cm}vsip\_scalar\_f, vsip\_major, unsigned int, vsip\_alg\_hint);\\
vsip\_fftm\_f* vsip\_rcfftmop\_create\_f(vsip\_length, vsip\_length,\\*\hspace{.7cm}vsip\_scalar\_f, vsip\_major, unsigned int, vsip\_alg\_hint);\\ \hline
\end{tabular}
}
\clearpage
\hspace*{.8cm} \vspace*{.1cm} \texttt{fft\_destroy}
\newline \hspace*{1.1cm} {
\ttfamily
\begin{tabular}[H]{l}\hline
\hline \multicolumn{1}{c}{\rmfamily \bfseries FFT Destroy Functions}\\ \hline
int vsip\_fft\_destroy\_d(vsip\_fft\_d*);\\
int vsip\_fft\_destroy\_f(vsip\_fft\_f*);\\
\hline \multicolumn{1}{c}{\rmfamily \bfseries Multiple FFT Destroy Functions}\\ \hline
int vsip\_fftm\_destroy\_d(vsip\_fftm\_d*);\\
int vsip\_fftm\_destroy\_f(vsip\_fftm\_f*);\\ \hline
\end{tabular}
}\vspace{.1cm}
\newline \hspace*{.8cm} \vspace*{.1cm} \texttt{fft}
\newline \hspace*{1.1cm} {
\ttfamily
\begin{tabular}[H]{l} \hline
\hline \multicolumn{1}{c}{\rmfamily \bfseries FFT Functions}\\ \hline
void vsip\_ccfftip\_d(const vsip\_fft\_d*,\\*\hspace{.7cm}const vsip\_cvview\_d*);\\
void vsip\_ccfftip\_f(const vsip\_fft\_f*,\\*\hspace{.7cm} const vsip\_cvview\_f*);\\
void vsip\_ccfftop\_d(const vsip\_fft\_d*,\\*\hspace{.7cm} const vsip\_cvview\_d*, const vsip\_cvview\_d*);\\
void vsip\_ccfftop\_f(const vsip\_fft\_f*,\\*\hspace{.7cm} const vsip\_cvview\_f*, const vsip\_cvview\_f*);\\
void vsip\_crfftop\_d(const vsip\_fft\_d*,\\*\hspace{.7cm} const vsip\_cvview\_d*, const vsip\_vview\_d*);\\
void vsip\_crfftop\_f(const vsip\_fft\_f*,\\*\hspace{.7cm} const vsip\_cvview\_f*, const vsip\_vview\_f*);\\
void vsip\_rcfftop\_d(const vsip\_fft\_d*,\\*\hspace{.7cm} const vsip\_vview\_d*, const vsip\_cvview\_d*);\\
void vsip\_rcfftop\_f(const vsip\_fft\_f*,\\*\hspace{.7cm} const vsip\_vview\_f*, const vsip\_cvview\_f*);\\ \hline \multicolumn{1}{c}{\rmfamily \bfseries Multiple FFT Functions}\\ \hline
void vsip\_ccfftmip\_d(const vsip\_fftm\_d*,\\*\hspace{.7cm} const vsip\_cmview\_d*);\\
void vsip\_ccfftmip\_f(const vsip\_fftm\_f*,\\*\hspace{.7cm} const vsip\_cmview\_f*);\\
void vsip\_ccfftmop\_d(const vsip\_fftm\_d*,\\*\hspace{.7cm} const vsip\_cmview\_d*, const vsip\_cmview\_d*);\\
void vsip\_ccfftmop\_f(const vsip\_fftm\_f*,\\*\hspace{.7cm} const vsip\_cmview\_f*, const vsip\_cmview\_f*);\\
void vsip\_crfftmop\_d(const vsip\_fftm\_d*,\\*\hspace{.7cm} const vsip\_cmview\_d*, const vsip\_mview\_d*);\\
void vsip\_crfftmop\_f(const vsip\_fftm\_f*,\\*\hspace{.7cm} const vsip\_cmview\_f*, const vsip\_mview\_f*);\\
void vsip\_rcfftmop\_d(const vsip\_fftm\_d*,\\*\hspace{.7cm} const vsip\_mview\_d*, const vsip\_cmview\_d*);\\
void vsip\_rcfftmop\_f(const vsip\_fftm\_f*,\\*\hspace{.7cm} const vsip\_mview\_f*, const vsip\_cmview\_f*);\\ \hline \end{tabular}
}
\clearpage
\hspace*{.8cm} \texttt{fft\_getattr}
\newline \hspace*{1.1cm} {
\ttfamily
\begin{tabular}[H]{l}\hline
\hline \multicolumn{1}{c}{\rmfamily \bfseries FFT Get Attributes Functions}\\ \hline
void vsip\_fft\_getattr\_d(const vsip\_fft\_d*, vsip\_fft\_attr\_d*);\\
void vsip\_fft\_getattr\_f(const vsip\_fft\_f*, vsip\_fft\_attr\_f*);\\
\hline \multicolumn{1}{c}{\rmfamily \bfseries Multiple FFT Get Attributes Functions}\\ \hline
void vsip\_fftm\_getattr\_d(const vsip\_fftm\_d*, vsip\_fftm\_attr\_d*);\\
void vsip\_fftm\_getattr\_f(const vsip\_fftm\_f*, vsip\_fftm\_attr\_f*);\\
\end{tabular}
}
\pyjvsiph
\newline\hspace*{.8cm}{\textbf{View Methods}\\
\hspace*{1cm}\parbox{10.5cm}{
\begin{itemize}
\item {Special \ttbf{view} methods exist.} 
\item {The scale factor is always one for \ttbf{view} methods.}
\item {For out of place the method will create and return the output \ttbf{view}.
\item {\ttbf{View} methods determine if the FFT is a multiple FFT or a vector FFT by the \ttbf{view} type}
}
\end{itemize}}\\
\hspace*{1.1cm}\textbf{In-Place: }\hspace{.2cm} yes\\
\hspace*{1.1cm}\textbf{Example: }\\
\hspace*{2.9cm}Forward transform of vector \ttbf{x} in-place\\*
\hspace*{3.5cm}\ttbf{x.fftip}\\*
\hspace*{2.9cm}For matrix FFT multiple use major attribute.\\*
\hspace*{3.5cm}\ttbf{x.ROW.fftip} \& \ttbf{x.COL.fftip}\\*
\hspace*{2.9cm}Inverse transform of vector \ttbf{x} in-place\\
\hspace*{3.5cm}\ttbf{x.ifftip}\\*
\hspace*{1.1cm}\textbf{Out-of-Place: }\hspace{.2cm} yes\\
\hspace*{1.1cm}\textbf{Example: }\\
\hspace*{2.9cm}Real to complex and complex to real FFT\\*
\hspace*{3.5cm}\ttbf{ y=x.rcfft}\\*
\hspace*{3.5cm}\ttbf{ z=y.crfft}\\*
\hspace*{2.9cm}Complex to complex transform of vector \ttbf{x} out-of-place\\
\hspace*{3.5cm}\ttbf{ y=x.fftop}\\
 \hspace*{2.9cm}Complex to complex inverse transform of vector \ttbf{x}\\*\hspace*{2.9cm}out-of-place\\
\hspace*{3.5cm}\ttbf{ y=x.ifftop}\\
 \hspace*{2.9cm}Complex to complex multiple transform of matrix \ttbf{x}\\*\hspace*{2.9cm}out-of-place by column\\
\hspace*{3.5cm}\ttbf{ y=x.COL.fftop}\\
 \hspace*{2.9cm}Complex to complex multiple transform of matrix \ttbf{x}\\*\hspace*{2.9cm}out-of-place by row\\
\hspace*{3.5cm}\ttbf{ y=x.ROW.fftop}\\
 \hspace*{2.9cm}Complex to complex multiple inverse transform of matrix \ttbf{x}\\*\hspace*{2.9cm}out-of-place by column\\
\hspace*{3.5cm}\ttbf{ y=x.COL.ifftop}\\
 \hspace*{2.9cm}Complex to complex multiple inverse transform of matrix \ttbf{x}\\*\hspace*{2.9cm}out-of-place by row\\
\hspace*{3.5cm}\ttbf{ y=x.ROW.ifftop}\\
\clearpage
\hspace*{.8cm}{\textbf{FFT Methods}\\
\hspace*{1.cm}\parbox{.85\textwidth}{To create an FFT object use \\*
\hspace*{1.cm} \ttbf{fftObj=FFT(t,*args)}\\*
Where \ttbf{args} is a tuple containing the create parameters for the FFT type selected, and \ttbf{t} is a string indicating the type of FFT to create.}\\
Note \ttbf{arg} will contain some or all of the following in the order listed}
\begin{itemize}
\item{\ttbf{M}\hspace*{.85cm} \parbox[t]{.8\textwidth}{Column Length}}
\item{\ttbf{N} \hspace*{.8cm} \parbox[t]{.8\textwidth}{Row Length for \ttbf{matrix} or Vector length for \ttbf{vector}}}
\item{\ttbf{scl} \hspace*{.5cm} \parbox[t]{.8\textwidth}{Scale Factor}}
\item{\ttbf{dir} \hspace*{.5cm} \parbox[t]{.8\textwidth}{Direction flag for FFT either VSIP\_FFT\_FWD or VSIP\_FFT\_INV}}
\item{\ttbf{major} \hspace*{.2cm} \parbox[t]{.8\textwidth}{For multiple FFT by row (VSIP\_ROW) or by column (VSIP\_COL)}}
\item{\ttbf{ntimes} \hspace*{.1cm} \parbox[t]{.8\textwidth}{Hint for how much the FFT object will be used. Zero indicates many times.}}
\item{\ttbf{hint} \hspace*{.45cm}\parbox[t]{.8\textwidth}{Algorithm hint to optimize for speed (VSIP\_ALG\_TIME), size (VSIP\_ALG\_SPACE), or accuracy (VSIP\_ALG\_NOISE)}}
\end{itemize}
\begin{table}[H]
\caption{FFT Types and Argument Strings}
\label{tab:fftTypesAndArugments}
\begin{center}
\begin{tabular}{|l l|}\hline
\multicolumn{2}{|c|}{\rmfamily \bfseries Discrete Fourier Transform Class}\\
'ccfftip\_f' & Complex-to-complex FFT single precision float in-place\\
\multicolumn{2}{c}{\ttbf{arg = }}\\
'ccfftop\_f' & Complex-to-complex FFT single precision float out-of-place\\
\multicolumn{2}{c}{\ttbf{arg = }}\\
'rcfftop\_f' & Real-to-complex FFT single precision float out-of-place\\
\multicolumn{2}{c}{\ttbf{arg = }}\\
'crfftop\_f'& Complex-to-real FFT single precision out-of-place\\
\multicolumn{2}{c}{\ttbf{arg = }}\\
'ccfftip\_d' & Complex-to-complex FFT double precision in-place\\
\multicolumn{2}{c}{\ttbf{arg = }}\\
'ccfftop\_d'& Complex-to-complex FFT double precision out-of-place\\
\multicolumn{2}{c}{\ttbf{arg = }}\\
'rcfftop\_d'& Real-to-complex multiple FFT single precision out-of-place\\
\multicolumn{2}{c}{\ttbf{arg = }}\\
'crfftop\_d'& Complex-to-real multiple FFT single precision out-of-place\\
\multicolumn{2}{c}{\ttbf{arg = }}\\
'ccfftmip\_f' & Complex-to-complex multiple FFT single precision in-place\\
\multicolumn{2}{c}{\ttbf{arg = }}\\
'ccfftmop\_f' & Complex-to-complex multiple FFT single precision out-of-place\\
\multicolumn{2}{c}{\ttbf{arg = }}\\
'rcfftmop\_f' & Real-to-complex multiple FFT single precision out-of-place\\
\multicolumn{2}{c}{\ttbf{arg = }}\\
'crfftmop\_f' & Complex-to-real multiple FFT single precision out-of-place\\
\multicolumn{2}{c}{\ttbf{arg = }}\\
'ccfftmip\_d' & Complex-to-complex multiple FFT double precision in-place\\
\multicolumn{2}{c}{\ttbf{arg = }}\\
'ccfftmop\_d' & Complex-to-complex multiple FFT double precision out-of-place\\
\multicolumn{2}{c}{\ttbf{arg = }}\\
'rcfftmop\_d' & Real-to-complex multiple FFT double precision out-of-place\\
\multicolumn{2}{c}{\ttbf{arg = }}\\
'crfftmop\_d' & Complex-to-real multiple FFT double precision out-of-place\\
\multicolumn{2}{c}{\ttbf{arg = }}\\
\hline\end{tabular}
\end{center}
\label{default}
\end{table}
\clearpage
{\large \textbf{\hypertarget{firFunc}{FIR Class}}}\vspace{.2cm}\\
\hspace*{.3cm}
\parbox{0.85\textwidth}{Finite Impulse Response Class. See filter functions table \ref{tab:filterFunctions}}
\cvsiplh 
\newline \hspace*{.8cm} \vspace*{.1cm} \textbf{Available Functions }
\newline \hspace*{.8cm} \vspace*{.1cm} \texttt{fir\_create}
\newline \hspace*{1.1cm} {
\ttfamily
\begin{tabular}[H]{l}
vsip\_rcfir\_d* vsip\_rcfir\_create\_d(\\*\hspace{.7cm}const vsip\_vview\_d*, vsip\_symmetry, vsip\_length,\\*\hspace{.7cm}vsip\_length, vsip\_obj\_state, unsigned, vsip\_alg\_hint);\\
vsip\_rcfir\_f* vsip\_rcfir\_create\_f(\\*\hspace{.7cm}const vsip\_vview\_f*, vsip\_symmetry, vsip\_length,\\*\hspace{.7cm}vsip\_length, vsip\_obj\_state, unsigned, vsip\_alg\_hint);\\
vsip\_cfir\_d* vsip\_cfir\_create\_d(\\*\hspace{.7cm}const vsip\_cvview\_d*, vsip\_symmetry, vsip\_length,\\*\hspace{.7cm}vsip\_length, vsip\_obj\_state, unsigned, vsip\_alg\_hint);\\
vsip\_cfir\_f* vsip\_cfir\_create\_f(\\*\hspace{.7cm}const vsip\_cvview\_f*, vsip\_symmetry, vsip\_length,\\*\hspace{.7cm}vsip\_length, vsip\_obj\_state, unsigned, vsip\_alg\_hint);\\
vsip\_fir\_d* vsip\_fir\_create\_d(\\*\hspace{.7cm}const vsip\_vview\_d*, vsip\_symmetry, vsip\_length,\\*\hspace{.7cm}vsip\_length, vsip\_obj\_state, unsigned, vsip\_alg\_hint);\\
vsip\_fir\_f* vsip\_fir\_create\_f(\\*\hspace{.7cm}const vsip\_vview\_f*, vsip\_symmetry, vsip\_length,\\*\hspace{.7cm}vsip\_length, vsip\_obj\_state, unsigned, vsip\_alg\_hint);\\
\end{tabular}
}
\newline \hspace*{.8cm} \vspace*{.1cm} \texttt{fir\_destroy}
\newline \hspace*{1.1cm} {
\ttfamily
\begin{tabular}[H]{l}
int vsip\_rcfir\_destroy\_d(vsip\_rcfir\_d*);\\
int vsip\_rcfir\_destroy\_f(vsip\_rcfir\_f*);\\
int vsip\_cfir\_destroy\_d(vsip\_cfir\_d*);\\
int vsip\_cfir\_destroy\_f(vsip\_cfir\_f*);\\
int vsip\_fir\_destroy\_d(vsip\_fir\_d*);\\
int vsip\_fir\_destroy\_f(vsip\_fir\_f*);\\
\end{tabular}
}\vspace{.1cm}
\newline \hspace*{.8cm} \vspace*{.1cm} \texttt{firflt}
\newline \hspace*{1.1cm} {
\ttfamily
\begin{tabular}[H]{l}
int vsip\_rcfirflt\_d(vsip\_rcfir\_d*, const vsip\_cvview\_d*,\\*\hspace{.7cm}const vsip\_cvview\_d*);\\
int vsip\_rcfirflt\_f(vsip\_rcfir\_f*, const vsip\_cvview\_f*,\\*\hspace{.7cm}const vsip\_cvview\_f*);\\
int vsip\_cfirflt\_d(vsip\_cfir\_d*, const vsip\_cvview\_d*,\\*\hspace{.7cm}const vsip\_cvview\_d*);\\
int vsip\_cfirflt\_f(vsip\_cfir\_f*, const vsip\_cvview\_f*,\\*\hspace{.7cm}const vsip\_cvview\_f*);\\
int vsip\_firflt\_d(vsip\_fir\_d*, const vsip\_vview\_d*,\\*\hspace{.7cm}const vsip\_vview\_d*);\\
int vsip\_firflt\_f(vsip\_fir\_f*, const vsip\_vview\_f*,\\*\hspace{.7cm}const vsip\_vview\_f*);\\
\end{tabular}
}
\clearpage
\hspace*{.8cm} \texttt{fir\_getattr}
\newline \hspace*{1.1cm} {
\ttfamily
\begin{tabular}[H]{l}
void vsip\_rcfir\_getattr\_d(const vsip\_rcfir\_d*,\\*\hspace{.7cm}vsip\_rcfir\_attr*);\\
void vsip\_rcfir\_getattr\_f(const vsip\_rcfir\_f*,\\*\hspace{.7cm}vsip\_rcfir\_attr*);\\
void vsip\_cfir\_getattr\_d(const vsip\_cfir\_d*,\\*\hspace{.7cm}vsip\_cfir\_attr*);\\
void vsip\_cfir\_getattr\_f(const vsip\_cfir\_f*,\\*\hspace{.7cm}vsip\_cfir\_attr*);\\
void vsip\_fir\_getattr\_d(const vsip\_fir\_d*,\\*\hspace{.7cm}vsip\_fir\_attr*);\\
void vsip\_fir\_getattr\_f(const vsip\_fir\_f*,\\*\hspace{.7cm}vsip\_fir\_attr*);\\
\end{tabular}
}\vspace{.1cm}
\newline\hspace*{.8cm} \texttt{fir\_reset}
\newline \hspace*{1.1cm} {
\ttfamily
\begin{tabular}[H]{l}
void vsip\_rcfir\_reset\_d(vsip\_rcfir\_d*)\\
void vsip\_rcfir\_reset\_f(vsip\_rcfir\_f*)\\
void vsip\_cfir\_reset\_d(vsip\_cfir\_d*)\\
void vsip\_cfir\_reset\_f(vsip\_cfir\_f*)\\
void vsip\_fir\_reset\_d(vsip\_fir\_d*)\\
void vsip\_fir\_reset\_f(vsip\_fir\_f*)\\
\end{tabular}\
}
\pyjvsiph
\afunc{floor}{For each element in the input \ttbf{view} round to the largest integral value not greater than the input. An unary operation. See table \ref{tab:unaryOperations}.}
\cvsiplh
\pyjvsiph
\pyjvComment{
\item{The \ilCode{floor} function is not supported in \jv at this time}
}
\afuncT{lud}{Lower-Upper Decomposition Class.}{generalSquareSolver}
\\\cvsiplh 
\\ \hspace*{.8cm} \vspace*{.1cm} \textbf{Available Functions }
%
\\ \hspace*{.9cm} {
\ttfamily\vspace{.3cm}
\begin{tabular}{|l|}
\multicolumn{1}{c}{\rmfamily \bfseries Create LU Object\vspace{.1cm}}\\ \hline
vsip\_lu\_d* vsip\_lud\_create\_d(vsip\_length);\\
vsip\_lu\_f* vsip\_lud\_create\_f(vsip\_length);\\
vsip\_clu\_d* vsip\_clud\_create\_d(vsip\_length);\\
vsip\_clu\_f* vsip\_clud\_create\_f(vsip\_length);\\
\hline\end{tabular}\\}
%
%\\ \hspace*{.8cm} \vspace*{.1cm} \texttt{lud\_destroy}
\hspace*{.9cm} {
\ttfamily\vspace{.3cm}
\begin{tabular}{|l|}
\multicolumn{1}{c}{\rmfamily \bfseries Destroy LU Object\vspace{.1cm}}\\ \hline
int vsip\_lud\_destroy\_d(vsip\_lu\_d*);\\
int vsip\_lud\_destroy\_f(vsip\_lu\_f*);\\
int vsip\_clud\_destroy\_d(vsip\_clu\_d*);\\
int vsip\_clud\_destroy\_f(vsip\_clu\_f*);\\
\hline\end{tabular}\\}
\hspace*{.9cm}{
\ttfamily\vspace{.3cm}
\begin{tabular}{|l|}
\multicolumn{1}{c}{\rmfamily \bfseries Calculate LU Decomposition\vspace{.1cm}}\\ \hline
int vsip\_lud\_d(vsip\_lu\_d*, const vsip\_mview\_d*);\\
int vsip\_lud\_f(vsip\_lu\_f*, const vsip\_mview\_f*);\\
int vsip\_clud\_d(vsip\_clu\_d*, const vsip\_cmview\_d*);\\
int vsip\_clud\_f(vsip\_clu\_f*, const vsip\_cmview\_f*);\\
\hline\end{tabular}\\}
%
\hspace*{.9cm}{
\ttfamily\vspace{.3cm}
\begin{tabular}{|l|}
\multicolumn{1}{c}{\rmfamily \bfseries Solve Using Calculated LU Decomposition\vspace{.3cm}}\\ \hline
int vsip\_lusol\_d(\\
\hspace*{3.cm}const vsip\_lu\_d*, vsip\_mat\_op, const vsip\_mview\_d*);\\
int vsip\_lusol\_f(\\
\hspace*{3.cm}const vsip\_lu\_f*, vsip\_mat\_op, const vsip\_mview\_f*);\\
int vsip\_clusol\_d(\\
\hspace*{3.cm}const vsip\_clu\_d*, vsip\_mat\_op, const vsip\_cmview\_d*);\\
int vsip\_clusol\_f(\\
\hspace*{3.cm}const vsip\_clu\_f*, vsip\_mat\_op, const vsip\_cmview\_f*);\\
\hline\end{tabular}\\}
%
\hspace*{.8cm}{
\ttfamily\vspace{.3cm}
\begin{tabular}{|l|}
\multicolumn{1}{c}{\rmfamily \bfseries Fill LU Attribute Structure\vspace{.1cm}}\\ \hline
void vsip\_lud\_getattr\_d(const vsip\_lu\_d*, vsip\_lu\_attr\_d*);\\
void vsip\_lud\_getattr\_f(const vsip\_lu\_f*, vsip\_lu\_attr\_f*);\\
void vsip\_clud\_getattr\_d(const vsip\_clu\_d*, vsip\_clu\_attr\_d*);\\
void vsip\_clud\_getattr\_f(const vsip\_clu\_f*, vsip\_clu\_attr\_f*);\\
\hline\end{tabular}\\}
\pyjvsiph
\\ \hspace*{.8cm}{\textbf{View Methods\vspace{.2cm}}\\
\hspace*{1.1cm}\parbox{.9\textwidth}{
\begin{itemize}\raggedright
\item {Three \ttbf{view} methods have been defined for LU Decompostion.}
\subitem{\ttbf{luSolve{(xb)}} - Calculate an in-place solution to $A\vec{x}=\vec{b}$ or $A X = B$.\Bs}
\subitem{\ttbf{luInv} - A method to obtain a matrix inverse using LU for computation.}
\subitem{\ttbf{lu} - A method to obtain a computed LU object from a matrix.\Bs}
\item{\Ts Methods \ttbf{lu} and \ttbf{luInv} are defined as properties.}
\item{LU decomposition will overwrite the input matrix so use a copy to preserve the calling view.}
\end{itemize}\vspace{2mm}}}\\
\hspace*{1.1cm}\textbf{Example: }\vspace*{.1cm}\\
\hspace*{1.9cm}\parbox{.85\textwidth}{\raggedright \Ts Given a square data matrix \ttbf{view} \ttbf{A} and a known 
matrix or (column) vector \ttbf{B} solve for unknown \ttbf{X} in $A X = B$.\\*
\hspace*{1cm}\ttbf{A.luSolve(B)} \\*
Done in-place. Matrix \ttbf{B} contains the answer \ttbf{X}}.\\
\hspace*{.8cm}{\textbf{LU Class Methods\vspace*{.2cm}}\\
\hspace*{1.cm}\parbox{.9\textwidth}{\raggedright To create an \ttbf{LU} object do\\
\hspace*{1.cm}\ttbf{luObj = LU(t,size)} \\
Where \ttbf{t} is a string indicating the type of \ttbf{LU} object to create and \ttbf{size} is the size of the square matrix the \ttbf{LU} object will decompose.\\
For class methods table we assume we have created an LU object we call \ttbf{luObj} and we have an input matrix \ttbf{view A} compliant with \ttbf{luObj} and some compliant \ttbf{view B} where we want to solve for the unknown \ttbf{X} in $A X = B $.\vspace{.2cm}}\\
\begin{table}
\caption{Flags and Types for LU Decomposition}
\begin{center}\begin{tabular}{|l|l|}
\multicolumn{2}{c}{\Ts\parbox[t]{.6\textwidth}{\center{\rmfamily \bfseries LU Decomposition Types}}}\Bs\\\hline
'lu\_d' & Real \ttbf{LU}; double precision \Bs\\\hline
'lu\_f' & Real \ttbf{LU}; float precision\Bs\\\hline
'clu\_d' & Complex \ttbf{LU}; double precision\Bs\\\hline
'clu\_f' & Complex \ttbf{LU}; float precision\Bs\\\hline
\end{tabular}
\begin{tabular}{|l|l|}
\multicolumn{2}{c}{\parbox[t]{.6\textwidth}{\center{\rmfamily \bfseries Matrix Operator Flags (\ttbf{op})}}}\Bs\\\hline
\Ts'NTRANS' or \ttbf{VSIP\_MAT\_NTRANS} & No Transpose operator\Bs\\\hline
   'TRANS' or \ttbf{VSIP\_MAT\_TRANS} & Transpose operator\Bs\\\hline
   'HERM' or \ttbf{VSIP\_MAT\_HERM} & Hermitian operator\Bs\\\hline
 \end{tabular}\end{center}\end{table}
\hspace*{1.cm}\parbox[t]{.9\textwidth}{\begin{tabular}{|l|l|}
\multicolumn{2}{c}{\parbox[t]{.8\textwidth}{\center{\rmfamily \bfseries LU Decomposition Methods\vspace{.2cm}}}\Bs} \\ \hline
\ttbf{luObj.lud(A)} & \parbox[t]{.6\textwidth}{\raggedright Decompose matrix \ttbf{A}. The decomposition is stored in the \ttbf{luObj} but \ttbf{view A} may be overwritten so use a copy if you want to preserve \ttbf{A}\Bs}\\\hline
\ttbf{luObj.solve(op,XB)} & \parbox[t]{.6\textwidth}{\raggedright Solve problem $\text{op}(A) X = B$ in-place where \ttbf{XB} is input/output \ttbf{view} and \ttbf{op} is matrix operator flag.\Bs}\\\hline
\ttbf{luObj.size} & \parbox[t]{.6\textwidth}{\raggedright Property. Returns integer length size of square matrix \ttbf{LU} object will decompose.\Bs}\\\hline
\ttbf{luObj.singular} & \parbox[t]{.6\textwidth}{\raggedright Property. Returns \ttbf{True} if singular; \ttbf{False} if inverse exists.\Bs}\\\hline
\ttbf{luObj.type} & \parbox[t]{.6\textwidth}{\raggedright Returns string indicating LU type.\Bs}\\\hline
\ttbf{luObj.vsip} & \parbox[t]{.6\textwidth}{\raggedright Returns C VSIPL LU instance.\vspace*{.1cm}\Bs}\\\hline
\end{tabular}\vspace*{.4cm}}
\hspace*{1.1cm}\textbf{Example: }\vspace*{.1cm}\\
\hspace*{1.9cm}\parbox{.85\textwidth}{\raggedright
\Ts Given a square data matrix \ttbf{view} \ttbf{A} and a known 
matrix \ttbf{B} solve for unknown \ttbf{X} in $A X = B$. \\*
\hspace*{1cm}\ttbf{asize=A.rowlength; t=A.type} \\*
\hspace*{1cm}\ttbf{luType=\{'mview\_d':'lu\_d','mview\_f':'lu\_f',}\\*
\hspace*{2.1cm}\ttbf{'cmview\_d':'clu\_d','cmview\_f':'clu\_f'\}} \\*
\hspace*{1cm}\ttbf{assert A.collength==asize,'Matrix must be square'}\\*
\hspace*{1cm}\ttbf{assert t in luType,'Matrix type not supported for LU decomposition'}\\*
\hspace*{1cm}\ttbf{luObj=LU(luType[t],asize)} \\*
\hspace*{1cm}\ttbf{luObj.lud(A)} \\*
\hspace*{1cm}\ttbf{luObj.solve('NTRANS',B)} \\*
\hspace*{1cm}
Done in-place. Matrix \ttbf{B} contains the answer \ttbf{X}.}
\begin{minted}{python}
asize=A.rowlength; t=A.type
luType={'mview_d':'lu_d','mview_f','lu_f','cmview_d':'clu_d','cmview_f':'clu_f'}
assert A.collength==asize,'Matrix must be square'
assert t in luType,'Matrix type not supported for LU decomposition'
luObj=LU(luType[t],asize)
luObj.lud(A)
luObj.solve('NTRANS',B)
\end{minted}

\afuncT{ma}{Multiply and add. An element-wise function.}{ternaryOperations}
\\\cvsiplh
\afh
{
\ttfamily
\\\hspace*{.04\textwidth}\begin{tabular}[H]{l}
void vsip\_cvma\_d(const vsip\_cvview\_d*, const vsip\_cvview\_d*,\\*\hspace{.7cm}const vsip\_cvview\_d*, const vsip\_cvview\_d*);\\
void vsip\_cvma\_f(const vsip\_cvview\_f*, const vsip\_cvview\_f*,\\*\hspace{.7cm}const vsip\_cvview\_f*, const vsip\_cvview\_f*);\\
void vsip\_cvsma\_d(const vsip\_cvview\_d*, vsip\_cscalar\_d,\\*\hspace{.7cm}const vsip\_cvview\_d*, const vsip\_cvview\_d*);\\
void vsip\_cvsma\_f(const vsip\_cvview\_f*, vsip\_cscalar\_f,\\*\hspace{.7cm}const vsip\_cvview\_f*, const vsip\_cvview\_f*);\\
void vsip\_vma\_d(const vsip\_vview\_d*, const vsip\_vview\_d*,\\*\hspace{.7cm}const vsip\_vview\_d*, const vsip\_vview\_d*);\\
void vsip\_vma\_f(const vsip\_vview\_f*, const vsip\_vview\_f*,\\*\hspace{.7cm}const vsip\_vview\_f*, const vsip\_vview\_f*);\\
void vsip\_vsma\_d(const vsip\_vview\_d*, vsip\_scalar\_d,\\*\hspace{.7cm}const vsip\_vview\_d*, const vsip\_vview\_d*);\\
void vsip\_vsma\_f(const vsip\_vview\_f*, vsip\_scalar\_f,\\*\hspace{.7cm}const vsip\_vview\_f*, const vsip\_vview\_f*);\\
void vsip\_cvsmsa\_d(const vsip\_cvview\_d*, vsip\_cscalar\_d,\\*\hspace{.7cm}vsip\_cscalar\_d, const vsip\_cvview\_d*);\\
void vsip\_cvsmsa\_f(const vsip\_cvview\_f*, vsip\_cscalar\_f,\\*\hspace{.7cm}vsip\_cscalar\_f, const vsip\_cvview\_f*);\\
void vsip\_vsmsa\_d(const vsip\_vview\_d*, vsip\_scalar\_d,\\*\hspace{.7cm}vsip\_scalar\_d, const vsip\_vview\_d*);\\
void vsip\_vsmsa\_f(const vsip\_vview\_f*, vsip\_scalar\_f,\\*\hspace{.7cm}vsip\_scalar\_f, const vsip\_vview\_f*);\\
\end{tabular}
}
\pyComment{\item{The C VSIPL spec has separate man pages for multiply-add functions containing scalar arguments, and those containing only \ttbf{view} arguments.}}
\\\pyjvsiph
\viewmthd{No}{NA}{NA}{NA}
\apyfunc{yes}{\ttbf{out = ma(in1,in2,in3,out)}}
\pyComment{\item{Argument \ttbf{in1} is always a \ttbf{view}, argument \ttbf{in2} is either a \ttbf{view} or a scalar and argument \ttbf{in3} is either a \ttbf{view} or a scalar.}
\item{The \ttbf{ma} function works much the same as the C VSIPL version except that a convenience pointer to the output \ttbf{view} is returned.}
\item{This may be done in-place if an input \ttbf{view} is the same as the output \ttbf{view}.}}\afunc{mag}{Arctangent of Two Arguments; An elementwise function}
\cvsiplh
\pyjvsiph\afuncT{magsq}{Arctangent of Two Arguments; An elementwise function}{unaryOperations}
\\\cvsiplh
\afh
\\\hspace*{.04\textwidth} {
\ttfamily
}
\\\pyjvsiph\afunc{meanval}{Arctangent of Two Arguments; An elementwise function}
\cvsiplh
\pyjvsiph
\afuncT{meansqval}{Returns the mean value of all the elements of a view.}{unaryOperations}
\\\cvsiplh
\\ \hspace*{.8cm} \vspace*{.1cm} \textbf{Available Functions }
\\ \hspace*{1.1cm} {
\ttfamily
\begin{tabular}[H]{l}
vsip\_scalar\_d vsip\_cmmeansqval\_d(const vsip\_cmview\_d*);\\
vsip\_scalar\_d vsip\_cvmeansqval\_d(const vsip\_cvview\_d*);\\
vsip\_scalar\_d vsip\_mmeansqval\_d(const vsip\_mview\_d*);\\
vsip\_scalar\_d vsip\_vmeansqval\_d(const vsip\_vview\_d*);\\
vsip\_scalar\_f vsip\_cmmeansqval\_f(const vsip\_cmview\_f*);\\
vsip\_scalar\_f vsip\_cvmeansqval\_f(const vsip\_cvview\_f*);\\
vsip\_scalar\_f vsip\_mmeansqval\_f(const vsip\_mview\_f*);\\
vsip\_scalar\_f vsip\_vmeansqval\_f(const vsip\_vview\_f*);\\
\end{tabular}
}
\\\pyjvsiph
\viewmthd{Yes}{Yes}{NA}{msq=in.meansqval}
\apyfunc{No}{NA}
\pyComment{\item{There seemed to be no reason to include this as a separate function for \pyjv}}\afunc{modulate}{Arctangent of Two Arguments; An elementwise function}
\cvsiplh
\pyjvsiph\afunc{msb}{Multiply and subtract. An element-wise function. See ternary functions table \ref{tab:ternaryOperations}.}
\cvsiplh
\newline \hspace*{.8cm} \vspace*{.1cm} \textbf{Available Functions }
\newline \hspace*{1.1cm} {
\ttfamily
\begin{tabular}[H]{l}
void vsip\_cvmsb\_d(const vsip\_cvview\_d*, const vsip\_cvview\_d*,\\*\hspace{.7cm}const vsip\_cvview\_d*, const vsip\_cvview\_d*);
void vsip\_cvmsb\_f(const vsip\_cvview\_f*, const vsip\_cvview\_f*,\\*\hspace{.7cm}const vsip\_cvview\_f*, const vsip\_cvview\_f*); 
void vsip\_vmsb\_d(const vsip\_vview\_d*, const vsip\_vview\_d*,\\*\hspace{.7cm}const vsip\_vview\_d*, const vsip\_vview\_d*); 
void vsip\_vmsb\_f(const vsip\_vview\_f*, const vsip\_vview\_f*,\\*\hspace{.7cm}const vsip\_vview\_f*, const vsip\_vview\_f*); 
\end{tabular}
}
\pyjvsiph
\viewmthd{No}{NA}{NA}{NA}
\apyfunc{yes}{out = msb(in1,in2,in3,out)}
\newline\hspace*{1.2cm}\parbox{10.8cm}{\vspace*{.1cm}The \ttbf{log} function works much the same as the C VSIPL version except that a convenience pointer to the output view is returned. This may be done in-place if one of the input views is the same as the output view.}
\afunc{neg}{Arctangent of Two Arguments; An elementwise function}
\cvsiplh
\pyjvsiph
\afunc{recip}{Arctangent of Two Arguments; An elementwise function}
\cvsiplh
\pyjvsiph\afunc{round}{Round to nearest integral value; An elementwise function. See unary operations table \ref{tab:unaryOperations}}
\cvsiplh
\pyjvsiph
\pyjvComment{
\item{The \ilCode{round} function is not supported in \jv at this time}
}
\afunc{rsqrt}{Arctangent of Two Arguments; An elementwise function}
\cvsiplh
\pyjvsiph
\afuncT{sbm}{Subtract and multiply. An element-wise function.}{ternaryOperations}
\\\cvsiplh
\afh
{
\ttfamily
\\\hspace*{.04\textwidth}\begin{tabular}[H]{l}
void vsip\_vsbm\_d(const vsip\_vview\_d*, const vsip\_vview\_d*,\\*\hspace{.7cm}const vsip\_vview\_d*, const vsip\_vview\_d*); 
void vsip\_vsbm\_f(const vsip\_vview\_f*, const vsip\_vview\_f*,\\*\hspace{.7cm}const vsip\_vview\_f*, const vsip\_vview\_f*); 
void vsip\_cvsbm\_d(const vsip\_cvview\_d*, const vsip\_cvview\_d*,\\*\hspace{.7cm}const vsip\_cvview\_d*, const vsip\_cvview\_d*); 
void vsip\_cvsbm\_f(const vsip\_cvview\_f*, const vsip\_cvview\_f*,\\*\hspace{.7cm}const vsip\_cvview\_f*, const vsip\_cvview\_f*); 
\end{tabular}
}
\\\pyjvsiph
\viewmthd{No}{NA}{NA}{NA}
\apyfunc{yes}{\ttbf{out = sbm(in1,in2,in3,out)}}
\pyComment{\item{Arguments \ttbf{in1}, \ttbf{in2} and \ttbf{in3} are always \ttbf{view}s. }
\item{The \ttbf{sbm} function works much the same as the C VSIPL version except that a convenience pointer to the output \ttbf{view} is returned.}
\item{This may be done in-place if an input \ttbf{view} is the same as the output \ttbf{view}.}}\afuncT{sin}{Sine; An element-wise function. Input \ttbf{view} elements are assumed to be in radians.}{elementaryMath}
\\\cvsiplh
\afh
\\\hspace*{.04\textwidth} {
\ttfamily
\begin{tabular}[H]{l}
vsip\_scalar\_f vsip\_sin\_f(vsip\_scalar\_f a);\\
vsip\_scalar\_d vsip\_sin\_d(vsip\_scalar\_d a);\\
void vsip\_msin\_d(\\*
\hspace{1cm}const vsip\_mview\_d*, const vsip\_mview\_d*);\\
void vsip\_msin\_f(\\*
\hspace{1cm}const vsip\_mview\_f*, const vsip\_mview\_f*);\\
void vsip\_vsin\_d(\\*
\hspace{1cm}const vsip\_vview\_d*, const vsip\_vview\_d*);\\
void vsip\_vsin\_f(\\*
\hspace{1cm}const vsip\_vview\_f*, const vsip\_vview\_f*);\\
\end{tabular}
}
\\\pyjvsiph
\viewmthd{yes}{yes}{yes}{inOut.sin}
\apyfunc{yes}{out = sin(in,out)}
\\\hspace*{.06\textwidth}\parbox{.9\textwidth}{\vspace*{.005\textheight}The \ttbf{sin} function works much the same as the C VSIPL version except that a convenience pointer to the output view is returned. This may be done in-place if \ttbf{in==out}.}
\afunc{sinh}{Hyperbolic Sine; An elementwise function. See elementary math functions table \ref{tab:elementaryMath}.}
\\\cvsiplh
\newline \hspace*{.8cm} \vspace*{.1cm} \textbf{Available Functions }
\newline \hspace*{1.1cm} {
\ttfamily
\begin{tabular}[H]{l}
vsip\_scalar\_f vsip\_sinh\_f(vsip\_scalar\_f a);\\
vsip\_scalar\_d vsip\_sinh\_d(vsip\_scalar\_d a);\\
void vsip\_msinh\_d(\\*
\hspace{1cm}const vsip\_mview\_d*, const vsip\_mview\_d*);\\
void vsip\_msinh\_f(\\*
\hspace{1cm}const vsip\_mview\_f*, const vsip\_mview\_f*);\\
void vsip\_vsinh\_d(\\*
\hspace{1cm}const vsip\_vview\_d*, const vsip\_vview\_d*);\\
void vsip\_vsinh\_f(\\*
\hspace{1cm}const vsip\_vview\_f*, const vsip\_vview\_f*);\\
\end{tabular}
}
\\\pyjvsiph
\viewmthd{yes}{yes}{yes}{inOut.sinh}
\apyfunc{yes}{out = sinh(in,out)}
\newline\hspace*{1.2cm}\parbox{10.8cm}{\vspace*{.1cm}The \ttbf{sinh} function works much the same as the C VSIPL version except that a convenience pointer to the output view is returned. This may be done in-place if \ttbf{in==out}.}
\afuncT{sqrt}{Square Root; An elementwise function.}{elementaryMath}
\\\cvsiplh
\afh
\\\hspace*{.04\textwidth} {
\ttfamily
\begin{tabular}[H]{l}
vsip\_scalar\_f vsip\_sqrt\_f(vsip\_scalar\_f a);\\
vsip\_scalar\_d vsip\_sqrt\_d(vsip\_scalar\_d a);\\
vsip\_cscalar\_d vsip\_csqrt\_d(vsip\_cscalar\_d);\\
vsip\_cscalar\_f vsip\_csqrt\_f(vsip\_cscalar\_f);\\
void vsip\_msqrt\_d(const vsip\_mview\_d*, const vsip\_mview\_d*);\\
void vsip\_msqrt\_f(const vsip\_mview\_f*, const vsip\_mview\_f*);\\
void vsip\_vsqrt\_d(const vsip\_vview\_d*, const vsip\_vview\_d*);\\
void vsip\_vsqrt\_f(const vsip\_vview\_f*, const vsip\_vview\_f*);\\
void vsip\_cmsqrt\_d(const vsip\_cmview\_d*, const vsip\_cmview\_d*);\\
void vsip\_cmsqrt\_f(const vsip\_cmview\_f*, const vsip\_cmview\_f*);\\
void vsip\_cvsqrt\_d(const vsip\_cvview\_d*, const vsip\_cvview\_d*);\\
void vsip\_cvsqrt\_f(const vsip\_cvview\_f*, const vsip\_cvview\_f*);\\
\end{tabular}
}
\\\pyjvsiph
\viewmthd{yes}{yes}{yes}{inOut.sqrt}
\apyfunc{yes}{out = sqrt(in,out)}
\pyComment{\item{The \ttbf{sqrt} function works much the same as the C VSIPL version except that a convenience pointer to the output view is returned. This may be done in-place if \ttbf{in==out}.}}

\afunc{sq}{Arctangent of Two Arguments; An elementwise function}
\cvsiplh
\pyjvsiph\afuncT{sumval}{Returns the sum of the the elements of a \ttbf{view}. Does not modify input.}{unaryOperations}
\\\cvsiplh
\afh
\\\hspace*{.04\textwidth} {
\ttfamily
\begin{tabular}[H]{l}
vsip\_cscalar\_d vsip\_cmsumval\_d(const vsip\_cmview\_d*);\\
vsip\_cscalar\_d vsip\_cvsumval\_d(const vsip\_cvview\_d*);\\
vsip\_cscalar\_f vsip\_cmsumval\_f(const vsip\_cmview\_f*);\\
vsip\_cscalar\_f vsip\_cvsumval\_f(const vsip\_cvview\_f*);\\
vsip\_scalar\_d vsip\_msumval\_d(const vsip\_mview\_d*);\\
vsip\_scalar\_d vsip\_vsumval\_d(const vsip\_vview\_d*);\\
vsip\_scalar\_f vsip\_msumval\_f(const vsip\_mview\_f*);\\
vsip\_scalar\_f vsip\_vsumval\_f(const vsip\_vview\_f*);\\
vsip\_scalar\_i vsip\_vsumval\_i(const vsip\_vview\_i*);\\
vsip\_scalar\_si vsip\_vsumval\_si(const vsip\_vview\_si*);\\
vsip\_scalar\_uc vsip\_vsumval\_uc(const vsip\_vview\_uc*);\\
vsip\_scalar\_vi vsip\_msumval\_bl(const vsip\_mview\_bl*);\\
vsip\_scalar\_vi vsip\_vsumval\_bl(const vsip\_vview\_bl*);\\
\end{tabular}
}
\\\pyjvsiph
\afunc{sumsqval}{Returns the sum of the squares of all the elements of a \ttbf{view}. Does not modify input. See table \ref{tab:unaryOperations}}
\\\cvsiplh
\newline \hspace*{.8cm} \vspace*{.1cm} \textbf{Available Functions }
\newline \hspace*{1.1cm} {
\ttfamily
\begin{tabular}[H]{l}
vsip\_scalar\_d vsip\_msumsqval\_d(const vsip\_mview\_d* );\\
vsip\_scalar\_d vsip\_vsumsqval\_d(const vsip\_vview\_d* );\\
vsip\_scalar\_f vsip\_msumsqval\_f(const vsip\_mview\_f* );\\
vsip\_scalar\_f vsip\_vsumsqval\_f(const vsip\_vview\_f* );\\
\end{tabular}
}
\\\pyjvsiph
\viewmthd{yes}{yes}{NA}{aValue=in.sumsqval}
\apyfunc{No}{}
\newline \hspace*{.8cm} \textbf{Comments}
\newline\hspace*{.9cm}\parbox{10.8cm}{\vspace*{.1cm}\begin{itemize}
\item{Since the \ttbf{sumsqval} function returns a scalar without modifying the \ttbf{view} there seemed little point in supporting this as a separate function call for \pyjv.}
\end{itemize}
}

\afunc{tan}{Tangent; An elementwise function}
\cvsiplh
\pyjvsiph
\afunc{tanh}{Hyperbolic Tangent; An elementwise function}
\cvsiplh
\pyjvsiph
