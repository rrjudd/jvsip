\clearpage
{\large \textbf{\hypertarget{firFunc}{FIR Class}}}\vspace{.2cm}\\
\hspace*{.3cm}
\parbox{0.85\textwidth}{Finite Impulse Response Class. See filter functions table \ref{tab:filterFunctions}}
\cvsiplh 
\newline \hspace*{.8cm} \vspace*{.1cm} \textbf{Available Functions }
\newline \hspace*{.8cm} \vspace*{.1cm} \texttt{fir\_create}
\newline \hspace*{1.1cm} {
\ttfamily
\begin{tabular}[H]{l}
vsip\_rcfir\_d* vsip\_rcfir\_create\_d(\\*\hspace{.7cm}const vsip\_vview\_d*, vsip\_symmetry, vsip\_length,\\*\hspace{.7cm}vsip\_length, vsip\_obj\_state, unsigned, vsip\_alg\_hint);\\
vsip\_rcfir\_f* vsip\_rcfir\_create\_f(\\*\hspace{.7cm}const vsip\_vview\_f*, vsip\_symmetry, vsip\_length,\\*\hspace{.7cm}vsip\_length, vsip\_obj\_state, unsigned, vsip\_alg\_hint);\\
vsip\_cfir\_d* vsip\_cfir\_create\_d(\\*\hspace{.7cm}const vsip\_cvview\_d*, vsip\_symmetry, vsip\_length,\\*\hspace{.7cm}vsip\_length, vsip\_obj\_state, unsigned, vsip\_alg\_hint);\\
vsip\_cfir\_f* vsip\_cfir\_create\_f(\\*\hspace{.7cm}const vsip\_cvview\_f*, vsip\_symmetry, vsip\_length,\\*\hspace{.7cm}vsip\_length, vsip\_obj\_state, unsigned, vsip\_alg\_hint);\\
vsip\_fir\_d* vsip\_fir\_create\_d(\\*\hspace{.7cm}const vsip\_vview\_d*, vsip\_symmetry, vsip\_length,\\*\hspace{.7cm}vsip\_length, vsip\_obj\_state, unsigned, vsip\_alg\_hint);\\
vsip\_fir\_f* vsip\_fir\_create\_f(\\*\hspace{.7cm}const vsip\_vview\_f*, vsip\_symmetry, vsip\_length,\\*\hspace{.7cm}vsip\_length, vsip\_obj\_state, unsigned, vsip\_alg\_hint);\\
\end{tabular}
}
\newline \hspace*{.8cm} \vspace*{.1cm} \texttt{fir\_destroy}
\newline \hspace*{1.1cm} {
\ttfamily
\begin{tabular}[H]{l}
int vsip\_rcfir\_destroy\_d(vsip\_rcfir\_d*);\\
int vsip\_rcfir\_destroy\_f(vsip\_rcfir\_f*);\\
int vsip\_cfir\_destroy\_d(vsip\_cfir\_d*);\\
int vsip\_cfir\_destroy\_f(vsip\_cfir\_f*);\\
int vsip\_fir\_destroy\_d(vsip\_fir\_d*);\\
int vsip\_fir\_destroy\_f(vsip\_fir\_f*);\\
\end{tabular}
}\vspace{.1cm}
\newline \hspace*{.8cm} \vspace*{.1cm} \texttt{firflt}
\newline \hspace*{1.1cm} {
\ttfamily
\begin{tabular}[H]{l}
int vsip\_rcfirflt\_d(vsip\_rcfir\_d*, const vsip\_cvview\_d*,\\*\hspace{.7cm}const vsip\_cvview\_d*);\\
int vsip\_rcfirflt\_f(vsip\_rcfir\_f*, const vsip\_cvview\_f*,\\*\hspace{.7cm}const vsip\_cvview\_f*);\\
int vsip\_cfirflt\_d(vsip\_cfir\_d*, const vsip\_cvview\_d*,\\*\hspace{.7cm}const vsip\_cvview\_d*);\\
int vsip\_cfirflt\_f(vsip\_cfir\_f*, const vsip\_cvview\_f*,\\*\hspace{.7cm}const vsip\_cvview\_f*);\\
int vsip\_firflt\_d(vsip\_fir\_d*, const vsip\_vview\_d*,\\*\hspace{.7cm}const vsip\_vview\_d*);\\
int vsip\_firflt\_f(vsip\_fir\_f*, const vsip\_vview\_f*,\\*\hspace{.7cm}const vsip\_vview\_f*);\\
\end{tabular}
}
\clearpage
\hspace*{.8cm} \texttt{fir\_getattr}
\newline \hspace*{1.1cm} {
\ttfamily
\begin{tabular}[H]{l}
void vsip\_rcfir\_getattr\_d(const vsip\_rcfir\_d*,\\*\hspace{.7cm}vsip\_rcfir\_attr*);\\
void vsip\_rcfir\_getattr\_f(const vsip\_rcfir\_f*,\\*\hspace{.7cm}vsip\_rcfir\_attr*);\\
void vsip\_cfir\_getattr\_d(const vsip\_cfir\_d*,\\*\hspace{.7cm}vsip\_cfir\_attr*);\\
void vsip\_cfir\_getattr\_f(const vsip\_cfir\_f*,\\*\hspace{.7cm}vsip\_cfir\_attr*);\\
void vsip\_fir\_getattr\_d(const vsip\_fir\_d*,\\*\hspace{.7cm}vsip\_fir\_attr*);\\
void vsip\_fir\_getattr\_f(const vsip\_fir\_f*,\\*\hspace{.7cm}vsip\_fir\_attr*);\\
\end{tabular}
}\vspace{.1cm}
\newline\hspace*{.8cm} \texttt{fir\_reset}
\newline \hspace*{1.1cm} {
\ttfamily
\begin{tabular}[H]{l}
void vsip\_rcfir\_reset\_d(vsip\_rcfir\_d*)\\
void vsip\_rcfir\_reset\_f(vsip\_rcfir\_f*)\\
void vsip\_cfir\_reset\_d(vsip\_cfir\_d*)\\
void vsip\_cfir\_reset\_f(vsip\_cfir\_f*)\\
void vsip\_fir\_reset\_d(vsip\_fir\_d*)\\
void vsip\_fir\_reset\_f(vsip\_fir\_f*)\\
\end{tabular}\
}
\pyjvsiph
\newline\hspace*{.8cm}{\textbf{View Methods\vspace{.2cm}}\\
\hspace*{1cm}\parbox{10.5cm}{
\begin{itemize}
\item {No \ttbf{view} methods have been defined.} 
\end{itemize}}\\
\hspace*{.8cm}{\textbf{FIR Class\vspace{.2cm}}\\
\hspace*{1.cm}\parbox{.9\textwidth}{To create an FIR object use \\*
\hspace*{1.cm} \ttbf{firObj=FIR(t,*args)}}\\
where \ttbf{args} is a tuple containing the create parameters for the FIR type selected, and \ttbf{t} is a string indicating the type of FIR to create.\vspace{.2cm}}\\
\hspace*{1.cm}\parbox{.9\textwidth}{Note \ttbf{args} will contain some or all of the following in the order listed. Each type string, is shown in the Finite Impulse Response Filter Types table below.\\
\begin{tabular}[t]{|l l|}\hline
\ttbf{filt} & \parbox[t]{.75\textwidth}{A vector \ttbf{view} of filter coefficients \vspace*{.1cm}}\\ \hline
\ttbf{sym} & \parbox[t]{.75\textwidth}{Symmetry of \ttbf{filt} kernel\vspace*{.1cm}} \\\hline
\ttbf{N} & \parbox[t]{.75\textwidth}{Length of input data vector \vspace*{.1cm}}\\\hline
\ttbf{D} & \parbox[t]{.75\textwidth}{Decimation factor\vspace*{.1cm}}\\\hline
\ttbf{state} & \parbox[t]{.75\textwidth}{ \vspace*{.1cm}}\\\hline
\ttbf{ntimes} & \parbox[t]{.75\textwidth}{Hint for how much the FIR object will be used. Zero indicates many times\vspace*{.1cm}} \\\hline
\ttbf{algHint} & \parbox[t]{.75\textwidth}{Algorithm hint to optimize for\\*speed (VSIP\_ALG\_TIME),\\*size (VSIP\_ALG\_SPACE),\\* or accuracy (VSIP\_ALG\_NOISE)\vspace*{.1cm}}\\
\hline \end{tabular}}
\newline
\hspace*{1.cm}\parbox[t]{.85\textwidth}{\begin{tabular}{|l l|}\hline
\multicolumn{2}{|c|}{\parbox[t]{.68\textwidth}{\center{\rmfamily \bfseries Finite Impulse Response Filter Types}\vspace{.2cm}}}\\ \hline \hline
'fir\_f' & \parbox[t]{.68\textwidth}{Real \ttbf{FIR}; float precision \vspace*{.1cm}}\\\hline
'cfir\_f' & \parbox[t]{.68\textwidth}{ Complex \ttbf{FIR}; float precision \vspace*{.1cm}}\\\hline
'rcfir\_f' & \parbox[t]{.68\textwidth}{ Complex \ttbf{FIR} with real \ttbf{kernel}; float precision \vspace*{.1cm}}\\\hline
'fir\_d' & \parbox[t]{.68\textwidth}{ Real \ttbf{FIR}; double precision \vspace*{.1cm}}\\\hline
'cfir\_d' & \parbox[t]{.68\textwidth}{Complex \ttbf{FIR}; double precision \vspace*{.1cm}}\\\hline
'rcfir\_d' & \parbox[t]{.68\textwidth}{Complex \ttbf{FIR} with real \ttbf{kernel}; double precision \vspace*{.1cm}}\\\hline
\hline\end{tabular}}
\clearpage\hspace*{.8cm}{\textbf{FFT Class Methods}\\
\hspace*{1.1cm} \parbox[t]{.88\textwidth}{Below we assume we have created an FFT object we call \ttbf{fftObj} and we have an input \ttbf{view x} compliant with \ttbf{fftObj} and if necessary a compliant output \ttbf{view y}.\vspace{.2cm}}
\newline\hspace*{1.2cm}\parbox[t]{.85\textwidth}{To calculate an in-place DFT we do\\*\hspace*{.5cm}\ttbf{fftObj.dft(x)}\\ To calculate an out-of-place DFT we do\\*\hspace*{.5cm} \ttbf{fftObj.dft(x,y)\vspace{.1cm}}\\
To get the FFT type (a string) we do\\*\hspace*{.5cm}\ttbf{t=fftObj.type}\vspace{.1cm}\\To get the argument list (a tuple) the FFT was created with we do\\*\hspace*{.5cm}\ttbf{arg=fftObj.arg}\vspace{.1cm}\\If we want to examine or use the C VSIPL FFT Object encapsulated inside the pyJvsip FFT object we do\\*\hspace*{.5cm}\ttbf{vsipObj=fftObj.vsip\\}}