\afuncT{am}{Add and multiply. An element-wise function.}{ternaryOperations}
\\\cvsiplh
\newline \hspace*{.8cm} \vspace*{.1cm} \textbf{Available Functions }
\newline \hspace*{1.1cm} {
\ttfamily
\begin{tabular}[H]{l}
void vsip\_cvam\_d(const vsip\_cvview\_d*,const vsip\_cvview\_d*, \\*\hspace{.7cm}const vsip\_cvview\_d*, const vsip\_cvview\_d*)\\
void vsip\_cvam\_f(const vsip\_cvview\_f*,const vsip\_cvview\_f*, \\*\hspace{.7cm}const vsip\_cvview\_f*, const vsip\_cvview\_f*)\\
void vsip\_cvsam\_d(const vsip\_cvview\_d*,vsip\_cscalar\_d, \\*\hspace{.7cm}const vsip\_cvview\_d*, const vsip\_cvview\_d*)\\
void vsip\_cvsam\_f(const vsip\_cvview\_f*,vsip\_cscalar\_f, \\*\hspace{.7cm}const vsip\_cvview\_f*, const vsip\_cvview\_f*)\\
void vsip\_vam\_d(const vsip\_vview\_d*,\\*\hspace{.7cm}const vsip\_vview\_d*,\\*\hspace{.7cm}const vsip\_vview\_d*, const vsip\_vview\_d*)\\
void vsip\_vam\_f(const vsip\_vview\_f*,const vsip\_vview\_f*,\\*\hspace{.7cm}const vsip\_vview\_f*, const vsip\_vview\_f*)\\
void vsip\_vsam\_d(const vsip\_vview\_d*,\\*\hspace{.7cm}vsip\_scalar\_d,const vsip\_vview\_d*, const vsip\_vview\_d*)\\
void vsip\_vsam\_f(const vsip\_vview\_f*,\\*\hspace{.7cm}vsip\_scalar\_f,\\*\hspace{.7cm}const vsip\_vview\_f*, const vsip\_vview\_f*)\\
\end{tabular}
}
\pyComment{\item{The C VSIPL spec has separate man pages for add-multiply functions containing scalar arguments, and those containing only \ttbf{view} arguments.}}
\\\pyjvsiph
\viewmthd{No}{NA}{NA}{NA}
\apyfunc{yes}{\ttbf{out = am(in1,in2,in3,out)}}
\pyComment{\item{Argument \ttbf{in1} is always a \ttbf{view}, argument \ttbf{in2} is either a \ttbf{view} or a scalar and argument \ttbf{in3} is always a \ttbf{view}.}
\item{The \ttbf{am} function works much the same as the C VSIPL version except that a convenience pointer to the output \ttbf{view} is returned.}
\item{This may be done in-place if an input \ttbf{view} is the same as the output \ttbf{view}.}}