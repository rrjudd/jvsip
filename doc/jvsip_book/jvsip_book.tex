\documentclass[10pt,oneside,a4paper]{book}
\usepackage[margin=2cm]{geometry}
%\usepackage{showframe}
\usepackage{graphicx}
\usepackage{parskip}
\usepackage{blindtext}
\usepackage[linktoc=all]{hyperref}
\setcounter{tocdepth}{3}
\usepackage{hyperref}
\hypersetup{colorlinks=true}
\usepackage{mathtools}
\usepackage[T1]{fontenc} 
\usepackage{lmodern}
\usepackage{listings} 
\usepackage{float}
\usepackage{makeidx}
\usepackage{minted}
\usepackage[hypcap]{caption}
\newcounter{cexctr} \setcounter{cexctr}{1}
\newcounter{pyjvexctr} \setcounter{pyjvexctr}{1}
\newcommand{\ttbf}[1]{{\ttfamily \bfseries #1}}
\newcommand{\ilCode}[1]{{\small {\ttfamily \bfseries #1}}}
% C Example heading
\newcommand{\cex}{{\begin{center} \bfseries{c VSIPL Example \arabic{cexctr}} \end{center}}}
%\newcommand{\pyjvex}{{\begin{center} \bfseries{pyJvsip Example \arabic{pyjvexctr}} \end{center}}}
%\newcommand*{\mhdr}{\fontfamily{pcr}\selectfont}
%*********JVSIP Function Commands
\newcommand{\afuncT}[3]{\clearpage{\Large \textbf{\hypertarget{{\ttfamily \bfseries#1Func}}{#1}}{\hspace*{\fill}\hypertarget{#1Func}{}\hyperlink{#3}{up}}} \vspace{.01\textheight}\\ \hspace*{.02\textwidth} \parbox{.97\textwidth}{#2 } \vspace{.005\textheight}}
%hlinkFunk used for hyperlink of table name with function page.
\newcommand{\hlnkFunc}[1]{\hyperlink{#1Func}{\texttt{#1}}}
\newcommand{\hlnkFuncT}[2]{\hyperlink{#1Func}{\texttt{#2}}}
%
\newcommand{\cvsiplh}{ \hspace*{.02\textwidth} {\large \textbf{C VSIPL}}\vspace{.005\textheight}}
%
\newcommand{\pyjvsiph}{\hspace*{.02\textwidth} {\large \textbf{pyJvsip}}\vspace*{.005\textheight}}
% Available Functions heading
\newcommand{\afh}{\\\hspace*{.04\textwidth} \vspace*{.005\textheight}\textbf{Available Functions }}
%
\newcommand{\pyjvComment}[1]{\newline \hspace*{.02\textwidth}\textbf{Comments}\\ \hspace*{.02\textwidth}\parbox{.95\textwidth}{ \vspace*{.005\textheight}\begin{itemize}{ #1}\end{itemize}}}
%
\newcommand{\pyComment}[1]{\\\begin{minipage}{\textwidth}\hspace*{.04\textwidth}\textbf{Comments}\\ \hspace*{.04\textwidth}\parbox{.95\textwidth}{\vspace*{.1cm}\begin{itemize}{ #1} \end{itemize}}\end{minipage}}
%
\newcommand{\viewmthd}[4]{\\\hspace*{.04\textwidth}\textbf{View Method}
\\ \hspace*{.06\textwidth}\textbf{Available: }#1\hspace{.25cm}\textbf{Property: }#2\hspace{.25cm}\textbf{In-Place: } #3 
\\ \hspace*{.06\textwidth}\textbf{Example:} {\ttfamily #4}}
%
%
\newcommand{\viewmthdu}[5]{\\\hspace*{.04\textwidth}\textbf{View Method}
\\ \hspace*{.06\textwidth}\textbf{Available: }#1\hspace{.25cm}\textbf{Property: }#2\hspace{.25cm}\textbf{In-Place: } #3 
\\ \hspace*{.06\textwidth}\textbf{Usage:} {\ttfamily #4}
\\ \hspace*{.07\textwidth}\textbf{Where:}
\\ \hspace*{.07\textwidth}\parbox{.91\textwidth}{\begin{itemize}{ #5} \end{itemize}}}
%
\newcommand{\vmthdh}{\hspace*{.04\textwidth} \textbf{View Method}\vspace*{.005\textheight}}
%
\newcommand{\apyfunc}[2]{\newline \hspace*{.04\textwidth}\textbf{Function} 
\\ \hspace*{.06\textwidth}\textbf{Available:} #1 
\\ \hspace*{.06\textwidth}\textbf{Example:} {\ttfamily #2}}
%
\newcommand{\aclassInstantiate}[3]{\\ \hspace*{.06\textwidth}{\textbf{#1 Instantiation}}
\\ \hspace*{.08\textwidth}\ttbf{#2}
\\ \hspace*{.08\textwidth}{#3}}
%
\newcommand{\aclassMethods}[1]{\\ \hspace*{.06\textwidth}{\textbf{#1 Methods}}}
%
\newcommand{\apyclass}[1]{\newline \hspace*{.04\textwidth}\textbf{Class} 
\\ \hspace*{.06\textwidth}\textbf{Available:} #1 
\\ \hspace*{.06\textwidth}\textbf{Example:} }

%use as \pyjv{}
\newcommand{\pyjv}{{\ttbf{pyJvsip}}}
%use as \jv{}
\newcommand{\jv}{\ttbf{JVSIP}}
%use as \cvl{}
\newcommand{\cvl}{\ttbf{C VSIPL}}
% from http://mirrors.rit.edu/CTAN/info/digests/ttn/ttn2n3.pdf
\newcommand\Ts{\rule{0pt}{2.6ex}}
\newcommand\Bs{\rule[-1.2ex]{0pt}{0pt}}
% function break
\newcommand{\fbrk}{\\*\hspace*{1cm}}
\DeclareMathOperator{\opConj}{opConj}
\DeclareMathOperator{\opTrans}{opTrans}
\DeclareMathOperator{\opHerm}{opHerm}
\DeclareMathOperator{\opM}{opM}
\DeclareMathOperator{\opA}{opA}
\DeclareMathOperator{\opB}{opB}
\makeindex
\begin{document}
\begin{titlepage}
\vspace*{3cm}
\begin{center} {\LARGE \bf JVSIP User Manual} \end{center}
\begin{center} Randall Judd \end{center}
\begin{center}  \today \end{center}
\begin{center} Draft \end{center}
\vfill
 \copyright 2014 Randall Judd, all rights reserved. 
{\flushleft {\small A non-exclusive, non-royalty bearing license is hereby granted to all persons to copy, modify, distribute and produce derivative works for any purpose, provided that this copyright notice and following disclaimer appear on all copies:}} 
{\flushleft {\footnotesize THIS LICENSE INCLUDES NO WARRANTIES, EXPRESSED OR IMPLIED, WHETHER ORAL OR WRITTEN, WITH RESPECT TO THE SOFTWARE OR OTHER MATERIAL INCLUDING, BUT NOT LIMITED TO, ANY IMPLIED WARRANTIES OF MERCHANTABILITY, OR FITNESS FOR A PARTICULAR PURPOSE, OR ARISING FROM A COURSE OF PERFORMANCE OR DEALING, OR FROM USAGE OR TRADE, OR OF NON-INFRINGEMENT OF ANY PATENTS OF THIRD PARTIES. THE INFORMATION IN THIS DOCUMENT SHOULD NOT BE CONSTRUED AS A COMMITMENT OF DEVELOPMENT BY ANY PARTY}} 
 .\end{titlepage}

\frontmatter
\chapter{Preface}
This book describes the functionality of the JVSIP implementation of the Vector/Signal/Image processing (VSIP) Library (VSIPL).  JVSIP includes all the functionality of the TASP Core Plus implementation (TVCPP) developed by the author as part of the the TASP (Tactical Advanced Signal Processing) COE (Common Operating Environment) effort.  

After I retired in 2006 I forked the TVCPP implementation to a new implementation I call JVSIP where the J is the first letter of my last name.  I wanted a mechanism to continue development and support of the VSIPL effort, and I wanted interested parties to be able to access my work. TASP is long gone and funding by traditional government channels seems to have dried up. To make my work available to the community I placed JVSIP on  \href{https://github.com/rrjudd/jvsip?}{{github}}. 

Despite a lack of traditional funding VSIPL trudges on under the guise of \href{http://www.omg.org/spec/VSIPL/}{{OMG}}. Also included on github are the development site for the \href{https://github.com/vsip/specs/tree/master/vsipl}{{OMG specification}}, and an open source implementation of \href{https://github.com/openvsip/openvsip}{{VSIPL++}}.  

Documentation is hard to do, generally lags behind implementations, and documents are always under development and seldom finished.  For this reason most documentation the author does is labeled as \emph{draft} even though I may not plan to get around to doing a \emph{non-draft} version.

Early in 2001 the author wrote a document describing the current functionality of the TASP VSIPL Core Plus implementation called \emph{TASP VSIPL Core Plus} which includes many examples and a fairly good overview of C VSIPL functionality. It was done in a hurry with little editing and was, of course, a draft.  The document needs updating and editing but it was done originally on a Sun workstation using Framemaker; a word processing package the author liked very much. Unfortunately it was not long before Framemaker effectively died (It is still out there as part of adobe but for my purposes it is dead), the Sun workstation was replaced by a PC and Microsoft Word became the only way to do things. So the original source of the \emph{TASP VSIPL Core Plus} book was basically lost and only the PDF document remains.  Updating the document without the original tools is difficult so was never done.

I have decided re-do the previous \emph{TASP VSIPL Core Plus} as the \emph{JVSIP User Manual}. The contents will look a lot like the previous book but will include some editing and a lot of new information including information on how to use pyJvsip.  Although this is a fresh, start since the source for the previous document is gone, the PDF content will be freely copied and pasted into this new document; many of the examples will be the same except updated and versions written in pyJvsip; and the author considers this document to be an update of the original. 

The new document will be done using LaTeX on a Mac with the MacTeX distribution supplying the tools and underlying environment.  LaTeX is not user friendly or wysiwyg and the author is not expert. But TeX is persistent and portable. All the necessary tools for doing a book are there and the source is all text so easily maintained. 

This book is not a copy of, nor a replacement for, the VSIPL specification.
\subsection*{VSIPL Forum - A short history}
In the early 90's signal processing boards by Mercury, Sky, CSPI, and others were becoming popular for use by Government  programs with compute heavy software.  Each company produced their own proprietary signal processing libraries for use on their boards. This caused concern of vendor lock-in because software would need to be rewritten every time a new board procurement was done.

The TASP group was at this time trying to do a bulk contract for signal processing boards similar to the Tactical Advanced Computer (TAC) contract.  In order to progress on the signal processing specification the TASP group was told they needed to specify a common operating environment for the boards. 

At about this time DARPA also saw a need for a signal processing library and awarded a contract to Hughes Research Laboratory to run a forum to produce an open signal processing specification for signal and image processing; and to write a reference implementation of said specification. 

TASP decided to participate in the forum as well as many other industry and university participants whose names may be found in the introductory pages to the original VSIPL document.

To make a long story short the DARPA VSIPL project and TASP have ended but the VSIPL forum has continued on in one form or another, currently with the OMG.  

\subsection*{Success Story?}
The title of this section is a question yet to be answered. Although VSIPL, as a specification, has been a minor success it may yet die for lack of support by the people who started the effort. There is only so much that volunteer efforts (such as JVSIP) can do and the cost/payback for companies developing VSIPL code is problematic without a big, paying, customer base.  Lacking any sort of funding or policy by DOD to support reference implementation development, specification development, or commercial vendor buy in by VSIPL requirements in procurement documents means that the effort may yet wither on the vine. 
\subsection*{Code History}
The original code basis for the C VSIP library implementation was a very early pre-alpha (incomplete) version of the VSIPL Reference library produced by Hughes Research Laboratory of Malibu, California in December of 1997. The original HRL release was template based using \href{https://www.gnu.org/software/m4/m4.html}{{m4}} as a code generator. The generated library was very slow and not really suitable for writing example codes of real world problems.  HRL was never successful in actually completing a complete reference implementation of VSIPL.

I was part of the TASP group and it was important to have an implementation suitable for writing real world applications.  In addition at the time I did not understand m4 and, I realize now, was only marginally competent at programing in C. I copied the generated C source files from the m4 base and modified them directly. The original was slow mostly because of the method of programing. They would start with a scalar function which would be called by a general element wise function which would be called by the actual function. This was very confusing to me so I just flattened everything out so the actual function call did all the work. This produced an enormous speed-up in example codes which was what I needed.

Over time many changes were made to the TASP implementation to add performance, and to keep up with the changing VSIPL specification. I learned a lot about C programing as time went on; and some of what I learned made it into the library. Eventually the TASP implementation became a de facto reference implementation.

I suspect HRL was never successful because of a lack of funding. For a company like HRL to produce a library as extensive as VSIPL would be an expensive proposition. For various reasons I found myself with good funding and not much direction for a couple of years. The VSIPL library was similar to some codes I had wanted to write anyway; and I had just completed a masters degree at UW with signal processing as my main study. So, with no tasking from above and freedom to set my own agenda, I worked like crazy for about a year and eventually had a code base extensive enough, and well tested enough, that folks could use it.

We are approaching 20 years since the original code base and little if any of the original HRL code remains in the library.  The original mostly contained support functions and simple element wise operations. Changes to the specification caused many changes to the support functions, and (as previously mentioned) the element wise functions provided by the HRL library were so slow as to be unusable for demonstration purposes. Most of what remains are the odd copyright statement. Except for (perhaps) function prototypes (defined by the spec) I would be surprised if any of the underlying code is from the original.


\tableofcontents
\listoffigures
\listoftables
\mainmatter
\chapter{Introduction To JVSIP}
\section*{Introduction}
If you are new to VSIPL or find you are too confused by the various acronyms or some of the terms here are unfamiliar read the Preface chapter above. It contains information about the origins and meaning of JVSIP and VSIPL.

First the big picture.

The JVSIP distribution contains a C signal processing library implementing (most of) the c VSIPL specification. It also contains a python vsip module encapsulating the C library.  Once the {\bf{vsip}} module was done a new module called {\bf{vsiputils}} was done to provide function overloading and to reduce the name space.  One of the main purposes of the {\bf{vsiputils}} module was to help me to learn python programing; but a lot of work was done there and the module still survives although it may go away in the future.  Eventually I got around to defining python classes and created the {\bf{pyJvsip}} module.  In this document we mainly treat the python interface defined in the {\bf{pyJvsip}} module but you should be aware other modules exist.

The distribution is available on \href{https://github.com/rrjudd/jvsip?}{{github}}. The distribution only contains source code. You will need a C compiler (supporting C89) to make the C Library. You will need the same C compiler, a python distribution (2.7), and a free open source package called \href{http://www.swig.org}{SWIG} to help encapsulate C code into python modules. The C library and the Python modules are independent except the same C source code is used for both.

Chapter one of this book will contain information and examples for a quick start for readers who want to get started programing.  Chapter two is mostly a reference chapter containing C and pyJvsip functions and usage information.  Following chapter delve deeper and cover the more complicated functions for signal processing and linear algebra.

\section*{C VSIP versus pyJvsip}
In this section I will make some comments about the difference between programing with c and the C VSIPL library and programing with pyJvsip in the python environment. 

The VSIPL library has support mechanisms for blocks and views. The pyJvsip module has support for blocks and views. The block and view in pyJvsip are instantiations of class definitions. The block and view in c VSIPL are opaque structures created with c VSIPL support functions defined for that purpose. This means a pyJvsip view is not the same as a VSIPL view even though I may write about them as if they are the same object. In general VSIPL objects (LUD, FFT, matrix view, etc.) created with create functions are contained inside a pyJvsip object as an instance variable.

The VSIPL library has a requirement for initialization and finalization.  PyJvsip is written on top of the c VSIPL implementation so we still need to initialize it and finalize it. However for pyJvsip I have abstracted that away so that when a pyJvsip object is created the initialization of the object checks to see if C VSIPL has been initialized and will call \ilCode{vsip\_init} if it needs to. There is a special class object which keeps track of pyJvsip objects and when no pyJvsip objects are left then it calls \ilCode{vsip\_finalize}. So pyJvsip has no explicit initialization/finalization other than the required python import statement.  

To avoid memory leaks there is a requirement for destruction of allocated objects after they are no longer needed in C VSIPL. For deallocation of VSIPL objects contained within a pyJvsip object; when a pyJvsip object has no reference left the python garbage collector will call the delete method.  This will destroy any c VSIPL objects that have been allocated for use with the pyJvsip object. So pyJvsip has no explicit destroy functions.

\subsection*{Polymorphism}
The encapsulation of c VSIPL using SWIG adds type information to the VSIPL python objects. Using this information, and information added to pyJvsip objects, as keys for python dictionaries allows VSIPL to become polymorphic. Most functions and methods in pyJvsip determine the underlying functionality using type information extracted from the calling object. Not every combination will necessarily work.  Someplace under the covers everything must be covered.  However it is generally possible to program in pyJvsip in a manner so that once the initial type has been chosen the rest of the code is generic even to the point of covering both real and complex. 

\section*{Example}
We do a simple example; see figure 1. 

%\subsection*{Example {\arabic{cexctr}}:  Add two vectors}
\begin{figure}[t]
\caption{Add Two Vectors}
\setlength{\parskip}{.25cm}
\begin{minipage}[t][5cm][t]{.5\textwidth}
{\begin{center} \bfseries{c VSIPL} \end{center}}\setlength{\parskip}{.25cm}
\inputminted[linenos=true,resetmargins=true,xleftmargin=.75cm,fontfamily=tt, fontsize= \tiny]{c}{./c_examples/example1.c}
\end{minipage}
\begin{minipage}[t][5cm][t]{.5\textwidth}
{\begin{center} \bfseries{pyJvsip} \end{center}}\setlength{\parskip}{.25cm}
\inputminted[linenos=true,resetmargins=true,xleftmargin=.75cm,fontfamily=tt,fontsize=\tiny]{python}{./pyJvsip_examples/example1a.py}\setlength{\parskip}{.25cm}
{\begin{center} \bfseries{Polymorphism with pyJvsip} \end{center}}\setlength{\parskip}{.25cm}
\inputminted[linenos=true,resetmargins=true,xleftmargin=.75cm,fontfamily=tt,fontsize=\tiny]{python}{./pyJvsip_examples/example1b.py}
\end{minipage}
\end{figure}
\chapter{Functions}
\section*{Introduction}
In this chapter I give basic usage information for the functions included in the JVSIP implementation of the C VSIPL specification and also related information for the \pyjv{} python module.  I only include functions implemented in \jv{}; either in the C VSIPL library or the \pyjv{} module. Functions covered in the C specification not (currently) supported in \ttbf{JVSIP} are not covered in this manual.

Usage information may also be found by reading the C VSIPL specification, either the old one included with the JVSIP distribution or the newer one developed by the HPEC working group of the OMG.  I currently recommend sticking with the old one included with the JVSIP distribution.  There is a lot of information about C VSIPL in the specification so C VSIPL information in this document will not be extensive; and since \pyjv{}	 has no specification document I will spend more time covering the \pyjv{} methodology.

I try and include information on the \pyjv{} methods and functions collocated with the corresponding C VSIPL information.  Reading the pyJvsip.py module file is also encouraged.  \ttbf{PyJvsip} includes some functionality not (directly) part of C VSIPL.  I will try and highlight these special cases.  

For python information the python help mechanism has also been supported somewhat; but keeping that information correct, up-to-date, and available for every function is a work in progress. 

Keep in mind this chapters main purpose is as a go-to reference for proper incantations when writing code. Except for the introductory sections it is probably not something you will want to read.

In order to have some reasonable ordering of the functions the alphabetical listing is based upon a root function name, not the actual vsip function. For instance the second function in the list is the \ilCode{add} function. There are several \ilCode{add} functions in the Core profile. All of them are placed together under \ilCode{add}.

When a C VSIPL function requires a special object it needs support functions to create the object, and destroy it, and perhaps query it for its attributes. For instance to do a discrete Fourier transform one needs a function to create an FFT object, a function to do the actual FFT using the FFT object, and a function to destroy the FFT object when it is no longer needed. The author calls functions which are designed to work together to do a single job function sets. Function sets are placed together under a single heading. For instance all the functions involved with doing an FFT are placed under the FFT heading.

As discussed in chapter one python supports polymorphism, and object oriented programing. A \pyjv{} object is an instantiation of a python class definition. The python object will contain a C VSIPL object as an instance variable as well as other information needed by \pyjv. For this reason the python garbage collector will destroy C VSIPL objects when no reference to the \pyjv{} object exists.

Because of the true object oriented nature of \pyjv{} there are methods defined for every class which accomplish most of the functionality of C VSIPL. \ttbf{PyJvsip} also defines many functions which operate on the \pyjv{} objects. Frequently you can use either a method or a function. This information is reflected in the JVSIP function list.

No attempt is made to be exhaustive in the function descriptions. Those interested in more detail are directed to the VSIPL specification document included with the \jv{} distribution. In addition various examples included in this document will provide more detail on the use of some of the more complicated functions.
%
\begin{table}[H]
\caption{VSIPL 1.3 API Chapters}
\label{tab:vsiplAPI}
\begin{center}
\begin{tabular}{l}
VSIPL INTRODUCTION\\
SUMMARY OF VSIPL TYPES\\
SUPPORT FUNCTIONS\\
SCALAR FUNCTIONS\\
RANDOM NUMBER GENERATION\\
VECTOR \& ELEMENTWISE OPERATIONS\\
SIGNAL PROCESSING FUNCTIONS\\
LINEAR ALGEBRA FUNCTIONS\\
IMPLEMENTATION DEPENDENT INPUT AND OUTPUT\\
VSIPL Addendum\\
\end{tabular}
\end{center}
\label{default}
\end{table}% %table of
%
\section*{VSIP Types}
This section covers the enumerated types and special structures. These are declared in the public header file \ilCode{vsip.h}.
%
\section*{JVSIP Function List}\addcontentsline{toc}{section}{JVSIP Function List}
The following pages are a list of available functions in JVSIP. The top part of each page will include a section of available C functions (basically extracted from \ilCode{vsip.h}). Since the VSIPL specification is the primary source of information for C VSIPL not much more is included.\\
%
Following the list of available functions is information on how (and if) the function is supported by \pyjv. The \pyjv{} section of the function page is more extensive than the C information. Basically there is a line indicating if it is available (as a 	tbf{view} method), if it is a property, and if it is in-place. Then there is a line indicating if it is available as a \pyjv{} function. Finally there is a comment section with additional information.\\
%
Note that comments may follow the C VSIPL and/or pyJvsip section and may also follow both indicating the comments pertain to the entire page and not just C or Python.\\
%
\subsection*{PyJvsip Methods}\addcontentsline{toc}{subsection}{PyJvsip Methods}
We note that saying the method is a property means you call it without even an empty argument list. For instance if \ilCode{a} is a \pyjv{} \ttbf{view} then \ilCode{a.cos} will replace the values in \ilCode{a} with there cosine. Since it is a property we DON'T say \ilCode{a.cos()}. Frequently, but not always, view methods are done in-place and that is also indicated. If it is not done in-place then the method will construct an appropriate output \ttbf{view} and return it filled out with the appropriate values.\\
%
An example of a \ttbf{view} method that is not done in place is \ilCode{copy}. For instance \ilCode{b=a.copy} will produce a copy of \ilCode{a} in an appropriate view. Note the new \ttbf{b view} will be the same precision, shape and depth as the calling \ttbf{view} but the \ttbf block will be of an exact size and the stride information will be the minimum stride required for the \ttbf{view}. Additional information on copy is available on its function page.\\
%
This is the type of information included on the function pages. Since there seem to be many exceptions we won't provide a lot of rules; and instead refer to the function page.\\
%
\subsection*{PyJvsip Functions}\addcontentsline{toc}{subsection}{PyJvsip Functions}
In the \pyjv{} \ttbf{Function} section we provide information on function calls. For \pyjv{} python function calls correspond closely with their C counterpart except that the shape, depth and precision are determined by the argument types used in the call and not by the actual name as is used by \ttbf{C VSIPL}.

Not all C functions have a corresponding \pyjv{} function call. In particular most functions that return a value will be handled using a \ttbf{view method} with no need for a function.
 
\afuncT{acos}{Inverse Cosine. An elementary math function.}{elementaryMath}
\\\cvsiplh
\begin{cfuncs}
vsip\_scalar\_f~vsip\_acos\_f(vsip\_scalar\_f);\\
vsip\_scalar\_d~vsip\_acos\_d(vsip\_scalar\_d);\\
void vsip\_macos\_d(const~vsip\_mview\_d*, const~vsip\_mview\_d*);\\
void vsip\_macos\_f(const~vsip\_mview\_f*, const~vsip\_mview\_f*);\\
void vsip\_vacos\_d(const~vsip\_vview\_d*, const~vsip\_vview\_d*);\\
void vsip\_vacos\_f(const~vsip\_vview\_f*, const~vsip\_vview\_f*);\\
\end{cfuncs}
\pyjvsiph
\viewmthd{yes}{yes}{yes}{inOut.acos}
\apyfunc{yes}{out = acos(in,out)}
\\\begin{minipage}{\textwidth}
\hspace*{.04\textwidth}\textbf{Comments}\\ 
\hspace*{.04\textwidth}\parbox{.95\textwidth}
{\vspace*{.1cm}
\begin{itemize}
\item{The \ttbf{acos} function works much the same as the C VSIPL version except that a convenience pointer to the output view is returned.}
\item{This may be done in-place if \ttbf{in==out}.}
\end{itemize}
}
\end{minipage}
\afunc{add}{Compute the sum of a scalar and a \ttbf{view} or two \ttbf{view}s. A binary operation. See table \ref{tab:binaryOperations}.}
\cvsiplh
\vspace{.25cm}\newline
\hspace*{1cm}\texttt{
\begin{tabular}[H]{l}
\multicolumn{1}{c}{\rmfamily \bfseries Scalar Add Functions}\\ \hline
vsip\_cscalar\_d vsip\_cadd\_d(\\*\hspace{1cm}vsip\_cscalar\_d, vsip\_cscalar\_d);\\
vsip\_cscalar\_d vsip\_rcadd\_d(\\*\hspace{1cm}vsip\_scalar\_d, vsip\_cscalar\_d);\\
vsip\_cscalar\_f vsip\_cadd\_f(\\*\hspace{1cm}vsip\_cscalar\_f, vsip\_cscalar\_f);\\
vsip\_cscalar\_f vsip\_rcadd\_f(\\*\hspace{1cm}vsip\_scalar\_f, vsip\_cscalar\_f);\\
\end{tabular}
}
\clearpage
\hspace*{1cm}\texttt{
\begin{tabular}[H]{l}
\multicolumn{1}{c}{\rmfamily \bfseries Normal View - View Add Functions}\\ \hline
void vsip\_vadd\_d(\\*\hspace{1cm}const vsip\_vview\_d*, const vsip\_vview\_d*,\\*\hspace{1cm}const vsip\_vview\_d*);\\
void vsip\_vadd\_f(\\*\hspace{1cm}const vsip\_vview\_f*, const vsip\_vview\_f*,\\*\hspace{1cm}const vsip\_vview\_f*);\\
void vsip\_cvadd\_d(\\*\hspace{1cm}const vsip\_cvview\_d*, const vsip\_cvview\_d*,\\*\hspace{1cm}const vsip\_cvview\_d*);\\
void vsip\_cvadd\_f(\\*\hspace{1cm}const vsip\_cvview\_f*, const vsip\_cvview\_f*,\\*\hspace{1cm}const vsip\_cvview\_f*);\\
void vsip\_madd\_d(\\*\hspace{1cm}const vsip\_mview\_d*, const vsip\_mview\_d*,\\*\hspace{1cm}const vsip\_mview\_d*);\\
void vsip\_madd\_f(\\*\hspace{1cm}const vsip\_mview\_f*, const vsip\_mview\_f*,\\*\hspace{1cm}const vsip\_mview\_f*);\\
void vsip\_cmadd\_d(\\*\hspace{1cm}const vsip\_cmview\_d*, const vsip\_cmview\_d*,\\*\hspace{1cm}const vsip\_cmview\_d*);\\
void vsip\_cmadd\_f(\\*\hspace{1cm}const vsip\_cmview\_f*, const vsip\_cmview\_f*,\\*\hspace{1cm}const vsip\_cmview\_f*);\\
void vsip\_vadd\_i(\\*\hspace{1cm}const vsip\_vview\_i*, const vsip\_vview\_i*,\\*\hspace{1cm}const vsip\_vview\_i*);\\
void vsip\_madd\_i(\\*\hspace{1cm}const vsip\_mview\_i*, const vsip\_mview\_i*,\\*\hspace{1cm}const vsip\_mview\_i*);\\
void vsip\_vadd\_si(\\*\hspace{1cm}const vsip\_mview\_si*, const vsip\_mview\_si*,\\*\hspace{1cm}const vsip\_mview\_si*);\\
void vsip\_madd\_si(\\*\hspace{1cm}const vsip\_mview\_si*, const vsip\_mview\_si*,\\*\hspace{1cm}const vsip\_mview\_si*);\\
void vsip\_vadd\_uc(\\*\hspace{1cm}const vsip\_vview\_uc*, const vsip\_vview\_uc*,\\*\hspace{1cm}const vsip\_vview\_uc*);\\
void vsip\_vadd\_vi(\\*\hspace{1cm}const vsip\_vview\_vi*, const vsip\_vview\_vi*,\\*\hspace{1cm}const vsip\_vview\_vi*);\\
\end{tabular}
}
%
\clearpage
\hspace*{1cm}\texttt{
\begin{tabular}[H]{l}
\multicolumn{1}{c}{\rmfamily \bfseries Mixed Depth View - View Add Functions}\\ \hline
void vsip\_rcvadd\_d(\\*\hspace{1cm}const vsip\_vview\_d*, const vsip\_cvview\_d*,\\*\hspace{1cm}const vsip\_cvview\_d*);\\
void vsip\_rcvadd\_f(\\*\hspace{1cm}const vsip\_vview\_f*, const vsip\_cvview\_f*,\\*\hspace{1cm}const vsip\_cvview\_f*);\\
void vsip\_rcmadd\_d(\\*\hspace{1cm}const vsip\_mview\_d*, const vsip\_cmview\_d*,\\*\hspace{1cm}const vsip\_cmview\_d*);\\
void vsip\_rcmadd\_f(\\*\hspace{1cm}const vsip\_mview\_f*, const vsip\_cmview\_f*,\\*\hspace{1cm}const vsip\_cmview\_f*);\\
\end{tabular}
}
\vspace*{.25cm}\newline
\hspace*{1cm}\texttt{
\begin{tabular}[H]{l}
\multicolumn{1}{c}{\rmfamily \bfseries Mixed Depth Scalar - View Add Functions}\\ \hline
void vsip\_rscvadd\_d(\\*\hspace{1cm}vsip\_scalar\_d, const vsip\_cvview\_d*,\\*\hspace{1cm}const vsip\_cvview\_d*);\\
void vsip\_rscvadd\_f(\\*\hspace{1cm}vsip\_scalar\_f, const vsip\_cvview\_f*,\\*\hspace{1cm}const vsip\_cvview\_f*);\\
void vsip\_rscmadd\_d(\\*\hspace{1cm}vsip\_scalar\_d, const vsip\_cmview\_d*,\\*\hspace{1cm}const vsip\_cmview\_d*);\\
void vsip\_rscmadd\_f(\\*\hspace{1cm}vsip\_scalar\_f, const vsip\_cmview\_f*,\\*\hspace{1cm}const vsip\_cmview\_f*);\\
\end{tabular}
}
\clearpage
\hspace*{1cm}\texttt{
\begin{tabular}[H]{l}
\multicolumn{1}{c}{\rmfamily \bfseries Normal Scalar - View Add Functions}\\ \hline
void vsip\_svadd\_d(\\*\hspace{1cm}vsip\_scalar\_d, const vsip\_vview\_d*,\\*\hspace{1cm}const vsip\_vview\_d*);\\
void vsip\_svadd\_f(\\*\hspace{1cm}vsip\_scalar\_f, const vsip\_vview\_f*,\\*\hspace{1cm}const vsip\_vview\_f*);\\
void vsip\_smadd\_d(\\*\hspace{1cm}vsip\_scalar\_d, const vsip\_mview\_d*,\\*\hspace{1cm}const vsip\_mview\_d*);\\
void vsip\_smadd\_f(\\*\hspace{1cm}vsip\_scalar\_f, const vsip\_mview\_f*,\\*\hspace{1cm}const vsip\_mview\_f*);\\
void vsip\_csvadd\_d(\\*\hspace{1cm}vsip\_cscalar\_d, const vsip\_cvview\_d*,\\*\hspace{1cm}const vsip\_cvview\_d*);\\
void vsip\_csvadd\_f(\\*\hspace{1cm}vsip\_cscalar\_f, const vsip\_cvview\_f*,\\*\hspace{1cm}const vsip\_cvview\_f*);\\
void vsip\_csmadd\_d(\\*\hspace{1cm}vsip\_cscalar\_d, const vsip\_cmview\_d*,\\*\hspace{1cm}const vsip\_cmview\_d*);\\
void vsip\_csmadd\_f(\\*\hspace{1cm}vsip\_cscalar\_f, const vsip\_cmview\_f*,\\*\hspace{1cm}const vsip\_cmview\_f*);\\
void vsip\_svadd\_i(\\*\hspace{1cm}vsip\_scalar\_i, const vsip\_vview\_i*,\\*\hspace{1cm}const vsip\_vview\_i*);\\
void vsip\_svadd\_si(\\*\hspace{1cm}vsip\_scalar\_si, const vsip\_vview\_si*,\\*\hspace{1cm}const vsip\_vview\_si*);\\
void vsip\_svadd\_uc(\\*\hspace{1cm}vsip\_scalar\_uc, const vsip\_vview\_uc*,\\*\hspace{1cm}const vsip\_vview\_uc*);\\
void vsip\_svadd\_vi(\\*\hspace{1cm}vsip\_scalar\_vi, const vsip\_vview\_vi*,\\*\hspace{1cm}const vsip\_vview\_vi*);\\
\end{tabular}
}
\pyjvsiph
\newline\hspace*{.8cm}{\textbf{View Method}\\
\hspace*{1.1cm}Overloaded on plus operator.\\
\hspace*{1.1cm}\textbf{In Place: }\hspace{.2cm} yes\\
\hspace*{1.1cm}\textbf{Example: }\ttbf{a += b; a += 2}\\*
\hspace*{1.5cm}Elements of \ttbf{view a} replaced with result.\\
\hspace*{1.1cm}\textbf{Out of Place: }\hspace{.2cm} yes\\
\hspace*{1.1cm}\textbf{Example: }\ttbf{c = a + b; d = 2 + c}\\*
\hspace*{1.5cm}\ttbf{view c} and \ttbf{view d} created and filled with result of operation.\\
\apyfunc{yes}{out = add(in1,in2,out)}
\pyComment{\item{The \ttbf{add} function works much the same as the C VSIPL version except that a convenience pointer to the output view is returned. }
\item{This may be done in-place if \ttbf{in1==out} or \ttbf{in2==out}.}
\item{Argument \ttbf{in1} may be a scalar. For clues to what is allowed see C VSIPL function list.}}
\afuncT{alldestroy}{Free \ttbf{view} and associated \ttbf{block}}{ViewObjects}
\index{alldestroy}
\\\cvsiplh
\begin{cfuncs}
void vsip\_cmalldestroy\_d(vsip\_cmview\_d*);\\
void vsip\_cmalldestroy\_f(vsip\_cmview\_f*);\\
void vsip\_ctalldestroy\_d(vsip\_ctview\_d*);\\
void vsip\_ctalldestroy\_f(vsip\_ctview\_f*);\\
void vsip\_cvalldestroy\_d(vsip\_cvview\_d*);\\
void vsip\_cvalldestroy\_f(vsip\_cvview\_f*);\\
void vsip\_malldestroy\_bl(vsip\_mview\_bl*);\\
void vsip\_malldestroy\_d(vsip\_mview\_d*);\\
void vsip\_malldestroy\_f(vsip\_mview\_f*);\\
void vsip\_malldestroy\_i(vsip\_mview\_i*);\\
void vsip\_malldestroy\_si(vsip\_mview\_si*);\\
void vsip\_malldestroy\_uc(vsip\_mview\_uc*);\\
void vsip\_talldestroy\_d(vsip\_tview\_d*);\\
void vsip\_talldestroy\_f(vsip\_tview\_f*);\\
void vsip\_talldestroy\_i(vsip\_tview\_i*);\\
void vsip\_talldestroy\_si(vsip\_tview\_si*);\\
void vsip\_talldestroy\_uc(vsip\_tview\_uc*);\\
void vsip\_valldestroy\_bl(vsip\_vview\_bl*);\\
void vsip\_valldestroy\_d(vsip\_vview\_d*);\\
void vsip\_valldestroy\_f(vsip\_vview\_f*);\\
void vsip\_valldestroy\_i(vsip\_vview\_i*);\\
void vsip\_valldestroy\_mi(vsip\_vview\_mi*);\\
void vsip\_valldestroy\_si(vsip\_vview\_si*);\\
void vsip\_valldestroy\_uc(vsip\_vview\_uc*);\\
void vsip\_valldestroy\_vi(vsip\_vview\_vi*);\\
\end{cfuncs}
\pyjvsiph\\
\hspace*{1cm}For pyJvsip deletion is handled by the python garbage collector.

afunc{alltrue}{Returns true if all the elements of a vector/matrix of type \_bl are true.}
\\\cvsiplh
\\\pyjvsiph

\afuncT{am}{Add and multiply. An element-wise function.}{ternaryOperations}
\\\cvsiplh
\newline \hspace*{.8cm} \vspace*{.1cm} \textbf{Available Functions }
\newline \hspace*{1.1cm} {
\ttfamily
\begin{tabular}[H]{l}
void vsip\_cvam\_d(const vsip\_cvview\_d*,const vsip\_cvview\_d*, \\*\hspace{.7cm}const vsip\_cvview\_d*, const vsip\_cvview\_d*)\\
void vsip\_cvam\_f(const vsip\_cvview\_f*,const vsip\_cvview\_f*, \\*\hspace{.7cm}const vsip\_cvview\_f*, const vsip\_cvview\_f*)\\
void vsip\_cvsam\_d(const vsip\_cvview\_d*,vsip\_cscalar\_d, \\*\hspace{.7cm}const vsip\_cvview\_d*, const vsip\_cvview\_d*)\\
void vsip\_cvsam\_f(const vsip\_cvview\_f*,vsip\_cscalar\_f, \\*\hspace{.7cm}const vsip\_cvview\_f*, const vsip\_cvview\_f*)\\
void vsip\_vam\_d(const vsip\_vview\_d*,\\*\hspace{.7cm}const vsip\_vview\_d*,\\*\hspace{.7cm}const vsip\_vview\_d*, const vsip\_vview\_d*)\\
void vsip\_vam\_f(const vsip\_vview\_f*,const vsip\_vview\_f*,\\*\hspace{.7cm}const vsip\_vview\_f*, const vsip\_vview\_f*)\\
void vsip\_vsam\_d(const vsip\_vview\_d*,\\*\hspace{.7cm}vsip\_scalar\_d,const vsip\_vview\_d*, const vsip\_vview\_d*)\\
void vsip\_vsam\_f(const vsip\_vview\_f*,\\*\hspace{.7cm}vsip\_scalar\_f,\\*\hspace{.7cm}const vsip\_vview\_f*, const vsip\_vview\_f*)\\
\end{tabular}
}
\pyComment{\item{The C VSIPL spec has separate man pages for add-multiply functions containing scalar arguments, and those containing only \ttbf{view} arguments.}}
\\\pyjvsiph
\viewmthd{No}{NA}{NA}{NA}
\apyfunc{yes}{\ttbf{out = am(in1,in2,in3,out)}}
\pyComment{\item{Argument \ttbf{in1} is always a \ttbf{view}, argument \ttbf{in2} is either a \ttbf{view} or a scalar and argument \ttbf{in3} is always a \ttbf{view}.}
\item{The \ttbf{am} function works much the same as the C VSIPL version except that a convenience pointer to the output \ttbf{view} is returned.}
\item{This may be done in-place if an input \ttbf{view} is the same as the output \ttbf{view}.}}
\afunc{and}{Boolean or bitwise "AND" operation for integer and boolean views.}
\\\cvsiplh
\\\pyjvsiph
\afuncT{anytrue}{Returns true if one or more elements of a \ttbf{view} of type \ttbf{\_bl} are true.}{logicalOperations}
\\\cvsiplh
\afh
{
\ttfamily
\\\hspace*{.04\textwidth}\begin{tabular}[H]{l}
vsip\_scalar\_bl vsip\_manytrue\_bl(const vsip\_mview\_bl*);\\
vsip\_scalar\_bl vsip\_vanytrue\_bl(const vsip\_vview\_bl*);\\ 
\end{tabular}
}
\\\pyjvsiph
\viewmthd{Yes}{Yes}{NA}{bl=in.anytrue}
\apyfunc{No}{NA}
\pyComment{
\item{Return value \ttbf{bl} is returned as a python \ttbf{True} or \ttbf{False} for \pyjv}
\item{The \ttbf{view} method is defined as a property}
\item{There seemed to be no reason to include this as a separate function for \pyjv}
}
\afuncT{arg}{Compute the radian value argument of complex elements in the interval $[-\pi,\pi]$. An Unary Operation.}{unaryOperations}
\\\cvsiplh
\newline \hspace*{.8cm} \vspace*{.1cm} \textbf{Available Functions }
\newline \hspace*{1.1cm} {
\ttfamily
\begin{tabular}[H]{l}
vsip\_scalar\_d vsip\_arg\_d(vsip\_cscalar\_d);\\
vsip\_scalar\_f vsip\_arg\_f(vsip\_cscalar\_f);\\
void vsip\_marg\_d(\\*\hspace{.5cm}const vsip\_cmview\_d*, const vsip\_mview\_d*);\\
void vsip\_marg\_f(\\*\hspace{.5cm}const vsip\_cmview\_f*, const vsip\_mview\_f*);\\
void vsip\_varg\_d(\\*\hspace{.5cm}const vsip\_cvview\_d*, const vsip\_vview\_d*);\\
void vsip\_varg\_f(\\*\hspace{.5cm}const vsip\_cvview\_f*, const vsip\_vview\_f*);\\
\end{tabular}
}
\\\pyjvsiph
\viewmthd{yes}{yes}{No}{out=in.arg}
\apyfunc{yes}{out = arg(in,out)}
\newline \hspace*{.8cm}\textbf{Comment}\\
\hspace*{.8cm}\parbox{11cm}{\vspace*{.2cm}
\begin{itemize}
\item{Since \ttbf{arg} takes a view of \emph{depth} complex and outputs to a view of \emph{depth} real of the same \emph{shape} and \emph{precision} as the input view the \ttbf{arg} method will create a view of the proper type and size and return it.}
\item{The \ttbf{arg} function works the same as the C VSIPL function except a convenience pointer is returned to the output view}
\item{For the function limited in-place functionality exists with replacement of the real or imaginary view of the input with the output. For instance \ilCode{out=arg(in,in.realview)} works fine.}
\end{itemize}}

\afuncT{asin}{Inverse Cosine. An elementary math function.}{elementaryMath}
\\\cvsiplh
\begin{cfuncs}
vsip\_scalar\_f vsip\_asin\_f(vsip\_scalar\_f a);\Bs\\
vsip\_scalar\_d vsip\_asin\_d(vsip\_scalar\_d a);\Bs\\
void vsip\_masin\_d(const~vsip\_mview\_d*, const~vsip\_mview\_d*);\Bs\\
void vsip\_masin\_f(const~vsip\_mview\_f*, const~vsip\_mview\_f*);\Bs\\
void vsip\_vasin\_d(const~vsip\_vview\_d*, const~vsip\_vview\_d*);\Bs\\
void vsip\_vasin\_f(const~vsip\_vview\_f*, const~vsip\_vview\_f*);\Bs\\
\end{cfuncs}
\pyjvsiph
\viewmthd{yes}{yes}{yes}{inOut.asin}
\apyfunc{yes}{out = asin(in,out)}
\pyComment{
\item{The \ttbf{asin} function works much the same as the C VSIPL version except that a convenience pointer to the output view is returned. This may be done in-place if \ttbf{in==out}.}}

\afuncT{atan2}{Arctangent of Two Arguments; An elementwise function. Computes the four quadrant radian value, $[-\pi,\pi]$, of the arctangent of the ratio of the corresponding elements of two input views.}{elementaryMath}
\\\cvsiplh
\hspace*{.8cm} \vspace*{.1cm} \textbf{Available Functions } \\
\hspace*{1cm}
\texttt{
\begin{tabular}[H]{l}
vsip\_scalar\_d vsip\_atan2\_d(vsip\_scalar\_d, vsip\_scalar\_d);\\
vsip\_scalar\_f vsip\_atan2\_f(vsip\_scalar\_f, vsip\_scalar\_f);\\
void vsip\_matan2\_d(const vsip\_mview\_d*, const vsip\_mview\_d*, const vsip\_mview\_d*);\\
void vsip\_matan2\_f(const vsip\_mview\_f*, const vsip\_mview\_f*, const vsip\_mview\_f*);\\
void vsip\_vatan2\_d(const vsip\_vview\_d*, const vsip\_vview\_d*, const vsip\_vview\_d*);\\
void vsip\_vatan2\_f(const vsip\_vview\_f*, const vsip\_vview\_f*, const vsip\_vview\_f*);\\
\end{tabular}
}
\\\pyjvsiph
\viewmthd{no}{NA}{NA}{NA}
\apyfunc{yes}{out = atan2(inOne,inTwo,out)}
\pyComment{\item{The \ttbf{atan2} function works much the same as the C VSIPL version except that a convenience pointer to the output view is returned. This may be done in-place if \ttbf{inOne==out} or \ttbf{inTwo==out}.}}
\afunc{atan}{Computes the principal radian value,[-+/2, +/2] of the arctangent for each element of a \ttbf{view}. See elementary math functions table \ref{tab:elementaryMath}.}
\cvsiplh
\pyjvsiph
\clearpage
\hypertarget{blockFunc}{\large \textbf{Block Function Set}}\vspace{.2cm}\\
\\\cvsiplh 
\newline \hspace*{.8cm} \vspace*{.1cm} \textbf{Available Functions }
%
\newline \hspace*{0.5cm} {
\texttt{
\begin{tabular}[H]{l}
\multicolumn{1}{c}{\rmfamily \bfseries Create Block Object\vspace{.1cm}}\\\hline
vsip\_block\_bl* vsip\_blockcreate\_bl(\\*\hspace{1cm}size\_t, vsip\_memory\_hint);\\
vsip\_block\_d* vsip\_blockcreate\_d(\\*\hspace{1cm}size\_t, vsip\_memory\_hint);\\
vsip\_block\_f* vsip\_blockcreate\_f(\\*\hspace{1cm}size\_t, vsip\_memory\_hint);\\
vsip\_block\_i* vsip\_blockcreate\_i(\\*\hspace{1cm}size\_t, vsip\_memory\_hint);\\
vsip\_block\_mi* vsip\_blockcreate\_mi(\\*\hspace{1cm}size\_t, vsip\_memory\_hint);\\
vsip\_block\_si* vsip\_blockcreate\_si(\\*\hspace{1cm}size\_t, vsip\_memory\_hint);\\
vsip\_block\_uc* vsip\_blockcreate\_uc(\\*\hspace{1cm}size\_t, vsip\_memory\_hint);\\
vsip\_block\_vi* vsip\_blockcreate\_vi(\\*\hspace{1cm}size\_t, vsip\_memory\_hint);\\
vsip\_cblock\_d* vsip\_cblockcreate\_d(\\*\hspace{1cm}size\_t, vsip\_memory\_hint);\\
vsip\_cblock\_f* vsip\_cblockcreate\_f(\\*\hspace{1cm}size\_t, vsip\_memory\_hint);\vspace{.2cm}\\ 
\hline\hline\\
\multicolumn{1}{c}{\rmfamily \bfseries Free Block Object\vspace{.1cm}}\\\hline
void vsip\_blockdestroy\_bl(vsip\_block\_bl*);\\
void vsip\_blockdestroy\_d(vsip\_block\_d*);\\
void vsip\_blockdestroy\_f(vsip\_block\_f*);\\
void vsip\_blockdestroy\_i(vsip\_block\_i*);\\
void vsip\_blockdestroy\_mi(vsip\_block\_mi*);\\
void vsip\_blockdestroy\_si(vsip\_block\_si*);\\
void vsip\_blockdestroy\_uc(vsip\_block\_uc*);\\
void vsip\_blockdestroy\_vi(vsip\_block\_vi *);\\
void vsip\_cblockdestroy\_d(vsip\_cblock\_d*);\\
void vsip\_cblockdestroy\_f(vsip\_cblock\_f*);\\
\end{tabular}
}}
\\\pyjvsiph

\afunc{ceil}{Ceiling. Not currently supported in \jv. An unary operation. See table \ref{tab:unaryOperations}}
\cvsiplh
\pyjvsiph
\afuncT{chold}{Cholesky Decomposition Class.}{symmetricPositiveDefiniteSolvers}
\\\cvsiplh 
\\ \hspace*{.8cm} \vspace*{.1cm} \textbf{Available Functions }
%
%\\ \hspace*{.8cm} \vspace*{.1cm} \texttt{chold\_create}
\\ \hspace*{1.cm} {
\ttfamily\vspace{.3cm}
\begin{tabular}[H]{|l|}
\multicolumn{1}{c}{\rmfamily \bfseries Create LU Object\vspace{.1cm}}\\ \hline
vsip\_chol\_d* vsip\_chold\_create\_d(vsip\_length);\\
vsip\_chol\_f* vsip\_chold\_create\_f(vsip\_length);\\
vsip\_cchol\_d* vsip\_cchold\_create\_d(vsip\_length);\\
vsip\_cchol\_f* vsip\_cchold\_create\_f(vsip\_length);\\
\hline\end{tabular}\\}
%
%\\ \hspace*{.8cm} \vspace*{.1cm} \texttt{chold\_destroy}
\\ \hspace*{1.cm} {
\ttfamily\vspace{.3cm}
\begin{tabular}[H]{|l|}
\multicolumn{1}{c}{\rmfamily \bfseries Destroy LU Object\vspace{.1cm}}\\ \hline
int vsip\_chold\_destroy\_d(vsip\_chol\_d*);\\
int vsip\_chold\_destroy\_f(vsip\_chol\_f*);\\
int vsip\_cchold\_destroy\_d(vsip\_cchol\_d*);\\
int vsip\_cchold\_destroy\_f(vsip\_cchol\_f*);\\
\hline\end{tabular}\\}
%
%\\ \hspace*{.8cm} \vspace*{.1cm} \texttt{chold}
\\ \hspace*{1.cm}{
\ttfamily\vspace{.3cm}
\begin{tabular}[H]{|l|}
\multicolumn{1}{c}{\rmfamily \bfseries Calculate LU Decomposition\vspace{.1cm}}\\ \hline
int vsip\_chold\_d(vsip\_chol\_d*, const vsip\_mview\_d*);\\
int vsip\_chold\_f(vsip\_chol\_f*, const vsip\_mview\_f*);\\
int vsip\_cchold\_d(vsip\_cchol\_d*, const vsip\_cmview\_d*);\\
int vsip\_cchold\_f(vsip\_cchol\_f*, const vsip\_cmview\_f*);\\
\hline\end{tabular}\\}
%
%\\ \hspace*{.8cm} \vspace*{.1cm} \texttt{cholsol}\\
\\ \hspace*{1.cm}{
\ttfamily\vspace{.3cm}
\begin{tabular}[H]{|l|}
\multicolumn{1}{c}{\rmfamily \bfseries Solve Using Calculated LU Decomposition\vspace{.1cm}}\\ \hline
int vsip\_cholsol\_d(const vsip\_chol\_d*, vsip\_mat\_op, const vsip\_mview\_d*);\\
int vsip\_cholsol\_f(const vsip\_chol\_f*, vsip\_mat\_op, const vsip\_mview\_f*);\\
int vsip\_ccholsol\_d(const vsip\_cchol\_d*, vsip\_mat\_op, const vsip\_cmview\_d*);\\
int vsip\_ccholsol\_f(const vsip\_cchol\_f*, vsip\_mat\_op, const vsip\_cmview\_f*);\\
\hline\end{tabular}\\}
%
%\\ \hspace*{.8cm} \vspace*{.1cm} \texttt{chold\_getattr}
\\ \hspace*{1.cm}{
\ttfamily\vspace{.3cm}
\begin{tabular}[H]{|l|}
\multicolumn{1}{c}{\rmfamily \bfseries Fill LU Attribute Structure\vspace{.1cm}}\\ \hline
void vsip\_chold\_getattr\_d(const vsip\_chol\_d*, vsip\_chol\_attr\_d*);\\
void vsip\_chold\_getattr\_f(const vsip\_chol\_f*, vsip\_chol\_attr\_f*);\\
void vsip\_cchold\_getattr\_d(const vsip\_cchol\_d*, vsip\_cchol\_attr\_d*);\\
void vsip\_cchold\_getattr\_f(const vsip\_cchol\_f*, vsip\_cchol\_attr\_f*);\\
\hline\end{tabular}\\}
%
\clearpage\pyjvsiph
\\\hspace*{.8cm}{\textbf{View Methods\vspace{.2cm}}\\
\hspace*{1cm}\parbox{10.5cm}{
\begin{itemize}
\item {A \ttbf{view} method has been defined for the kernel \ttbf{view}. The kernel is treated as non-symmetric so the entire kernel is assumed.\footnotemark[1]}
\item {A variable argument list is supported.} 
\subitem{The first required argument is the input data \ttbf{view}.}
\subitem {\parbox[t]{.74\textwidth}{The second optional argument is the decimation factor. It defaults to one.}\vspace*{.1cm}}
\item{Other parameters are either set to there default value, or are calculated from included parameters.\vspace{.2cm}}
\end{itemize}}\\
\hspace*{1.cm}\textbf{In-Place: }\hspace{.2cm} no\\
\hspace*{1.1cm}\textbf{Out-Of-Place: }\hspace{.2cm} yes\\
\hspace*{1.1cm}\textbf{Example: }\vspace*{.1cm}\\
\hspace*{1.cm}\parbox[t]{.85\textwidth}{
\begin{tabular}[t]{|l l|}
\multicolumn{2}{c}{\parbox[t]{.68\textwidth}{\center{\rmfamily \bfseries Finite Impulse Response Argument List\vspace{.2cm}}}}\\ \hline
\ttbf{filt} & \parbox[t]{.74\textwidth}{A vector \ttbf{view} of filter coefficients.\\*Required argument \vspace*{.1cm}}\\ \hline
\ttbf{sym} & \parbox[t]{.74\textwidth}{Symmetry of \ttbf{filt} kernel. \\* Required argument\vspace*{.1cm}} \\\hline
\ttbf{N} & \parbox[t]{.74\textwidth}{Length of input data vector. \\* Required argument\vspace*{.1cm}}\\\hline
\ttbf{D} & \parbox[t]{.74\textwidth}{Decimation factor.\\* Required argument\vspace*{.1cm}}\\\hline
\ttbf{state} & \parbox[t]{.74\textwidth}{Flag to indicate if the filter state is to be saved.\\*
\ttbf{VSIP\_STATE\_SAVE} or \ttbf{VSIP\_STATE\_NO\_SAVE}\\* Argument is supported but defaults to not saving. \\* Instead of \ttbf{VSIP} flags you may use the strings \ttbf{'YES'} or \ttbf{'NO'}.\vspace*{.1cm}}\\ \hline
\ttbf{ntimes} & \parbox[t]{.74\textwidth}{Hint for how much the LUD object will be used. Zero indicates many times.\\*For \jv{} this argument is only supported at the interface level and defaults to zero.\vspace*{.1cm}} \\\hline
\ttbf{algHint} & \parbox[t]{.74\textwidth}{Algorithm hint to optimize for\\*speed (\ttbf{VSIP\_ALG\_TIME}),\\*size (\ttbf{VSIP\_ALG\_SPACE}),\\* or accuracy (\ttbf{VSIP\_ALG\_NOISE}).\\*For \jv{} this argument is only supported at the interface level and defaults to time.\vspace*{.1cm}}\\
\hline \end{tabular}\\
\begin{tabular}{|l l|}
\multicolumn{2}{c}{\parbox[t]{.68\textwidth}{\center{\rmfamily \bfseries Finite Impulse Response Filter Types\vspace{.2cm}}}}\\ \hline
'fir\_f' & \parbox[t]{.68\textwidth}{Real \ttbf{LUD}; float precision \vspace*{.1cm}}\\\hline
'cfir\_f' & \parbox[t]{.68\textwidth}{ Complex \ttbf{LUD}; float precision \vspace*{.1cm}}\\\hline
'rcfir\_f' & \parbox[t]{.68\textwidth}{ Complex \ttbf{LUD} with real \ttbf{kernel}; float precision \vspace*{.1cm}}\\\hline
'fir\_d' & \parbox[t]{.68\textwidth}{ Real \ttbf{LUD}; double precision \vspace*{.1cm}}\\\hline
'cfir\_d' & \parbox[t]{.68\textwidth}{Complex \ttbf{LUD}; double precision \vspace*{.1cm}}\\\hline
'rcfir\_d' & \parbox[t]{.68\textwidth}{Complex \ttbf{LUD} with real \ttbf{kernel}; double precision \vspace*{.1cm}}\\\hline
\end{tabular}\vspace*{.4cm}}\\
\clearpage
%
\hspace*{.8cm}{\textbf{LUD Class Methods}\\
\hspace*{1.1cm} \parbox[t]{.88\textwidth}{For class methods table we assume we have created an LUD object we call \ttbf{firObj} and we have an input \ttbf{view x} compliant with \ttbf{firObj} and a compliant output \ttbf{view y}.\vspace{.2cm}}
\\
\hspace*{1.cm}\parbox[t]{.85\textwidth}{\begin{tabular}{|l l|}
\multicolumn{2}{c}{\parbox[t]{.58\textwidth}{\center{\rmfamily \bfseries Finite Impulse Response Filter Methods\vspace{.2cm}}}}\\ \hline
\ttbf{firObj.flt(x,y)} & \parbox[t]{.58\textwidth}{Filter the data \ttbf{x} and place the results in \ttbf{y}}\\\hline
\ttbf{firObj.decimation} & \parbox[t]{.58\textwidth}{ Returns integer decimation factor. \vspace*{.1cm}}\\\hline
\ttbf{firObj.length} & \parbox[t]{.58\textwidth}{ Returns integer length for \ttbf{x}. \vspace*{.1cm}}\\\hline
\ttbf{firObj.lengthOut} & \parbox[t]{.58\textwidth}{ Returns integer of valid data points in \ttbf{y} \vspace*{.1cm}}\\\hline
\ttbf{firObj.reset} & \parbox[t]{.58\textwidth}{Resets LUD filter to it's initial state. \vspace*{.1cm}}\\\hline
\ttbf{firObj.state} & \parbox[t]{.58\textwidth}{Returns \ttbf{True} if filter state is saved, otherwise returns \ttbf{False}.\vspace*{.1cm}}\\\hline
\ttbf{firObj.type} & \parbox[t]{.58\textwidth}{Returns string indicating filter type.\vspace*{.1cm}}\\\hline
\ttbf{firObj.vsip} & \parbox[t]{.58\textwidth}{Returns C VSIPL filter instance.\vspace*{.1cm}}\\\hline
\end{tabular}\vspace*{.4cm}}\\
\hspace*{.7cm} \parbox[t]{.91\textwidth}{
\begin{itemize}
\item{Methods \ttbf{decimation}, \ttbf{length}, \ttbf{state}, \ttbf{type} and \ttbf{vsip} are set when the LUD instance is created and do not change after create}
\item{Method \ttbf{lengthOut}\footnotemark[2] is calculated during the execution of method \ttbf{flt(x,y)} and is useful if state is saved and the filter object is used multiple times on a long piece of data.}
\item{Method \ttbf{reset} is used if state is saved and the filter is used multiple times on a long data set and then \emph{reset}\footnotemark[3] to it's initial state for use on multiple long data sets.}
\end{itemize}}
\footnotetext[1]{This does not preclude symmetric kernels. You just need the entire kernel.}
\footnotetext[2]{See C VSIPL specification for more information on length of output data.}
\footnotetext[3]{See signal processing text on overlap-add and overlap-save filtering.}

\afunc{clip}{Clip}
\\\cvsiplh
\\\pyjvsiph
\afuncT{cmplx}{Create a complex view from two real views, one of which represents the real part and the other representing the imaginary part.}{elementGenerationOperations}
\\\cvsiplh
\\\pyjvsiph
%
\viewmthd{No}{NA}{NA}{NA}
%
\apyfunc{No}{NA}
%
\pyComment{
\item{No comments}
}
\afunc{conj}{Conjugate}
\cvsiplh
\pyjvsiph
\clearpage
{\large \textbf{\hypertarget{convFunc}{CONV Class}}\hspace*{\fill}\hyperlink{ConvCorrFunctions}{up}\vspace{.2cm}\\
\\\cvsiplh 
\\\pyjvsiph
\afunc{copy}{Copy Data between two views. Some mixed types are supported so this method can be used to produce a copy of data of a new precision}{elementGenerationOperations}
\\\cvsiplh
\newline \hspace*{.8cm} \vspace*{.1cm} \textbf{Available Functions }
\newline \hspace*{1cm} {\ttfamily
\begin{tabular}[H]{l}
void vsip\_cmcopy\_d\_d(\\*
\hspace{1cm}const vsip\_cmview\_d*, const vsip\_cmview\_d*);\\
void vsip\_cmcopy\_d\_f(\\*
\hspace{1cm}const vsip\_cmview\_d*, const vsip\_cmview\_f*);\\
void vsip\_cmcopy\_f\_d(\\*
\hspace{1cm}const vsip\_cmview\_f*, const vsip\_cmview\_d*);\\
void vsip\_cmcopy\_f\_f(\\*
\hspace{1cm}const vsip\_cmview\_f*, const vsip\_cmview\_f*);\\
void vsip\_cvcopy\_d\_d\\*
\hspace{1cm}(const vsip\_cvview\_d*, const vsip\_cvview\_d*);\\
$\cdots$  \emph{etc.} \end{tabular}
}
\newline \hspace*{1cm}
\parbox{11cm}{There are many copy functions. To see all supported search the \ilCode{vsip.h} header file.\footnotemark}
\footnotetext{For instance \ttbf{grep copy\_ vsip.h} will list all available copy functions.}
\\\pyjvsiph
\viewmthd{yes}{yes}{no}{\parbox[t]{4cm}{out=in.copy\\out=in.copyrm\\out=in.copycm}}
\newline\hspace*{1cm}\parbox{11cm}{The \ttbf{copy} method creates a new view and data space that is the same shape, precision and depth as the input view and copies the data from the \ilCode{in} view to the \ilCode{out} view. The block in the \ilCode{out} view will be the exact size needed to hold the data and will be unit stride along the major direction of the \ilCode{in} view.\\The {\texttt{\bfseries{copycm}}} method is the same as the \ilCode{copy} method except the output view will always be row major independent of the input views major direction.\\The \ttbf{copyrm} method is the same as the \ilCode{copy} method except the output view will always be column major independent of the input views major direction.\\If the input view is a vector the three copy methods have identical results.}
\newline
\apyfunc{yes}{out = copy(in,out)}
\newline\hspace*{1cm}\parbox{11cm}{The \ttbf{copy} function works much the same as the C VSIPL version except that a convenience pointer to the output view is returned.}

\clearpage
{\large \textbf{\hypertarget{copyfrom}{copyfrom\_user}}\\
{ \ref{tab:elementGenerationOperations}.}
\\\cvsiplh
\\\pyjvsiph

\afuncT{copyto\_user}{Copy data out of a view into user memory.}{elementGenerationOperations}
\\\cvsiplh
\\\begin{cfuncs}
void vsip\_cmcopyto\_user\_d (const vsip\_cmview\_d*, vsip\_major, vsip\_scalar\_d*~const, vsip\_scalar\_d*~const);\\
void vsip\_cmcopyto\_user\_f (const vsip\_cmview\_f*, vsip\_major, vsip\_scalar\_f*~const, vsip\_scalar\_f*~const);\\
void vsip\_cvcopyto\_user\_d (const vsip\_cvview\_d*, vsip\_scalar\_d*~const, vsip\_scalar\_d*~const);\\
void vsip\_cvcopyto\_user\_f (const vsip\_cvview\_f*, vsip\_scalar\_f*~const, vsip\_scalar\_f*~const);\\
void vsip\_mcopyto\_user\_d (const vsip\_mview\_d*, vsip\_major, vsip\_scalar\_d*~const);\\
void vsip\_mcopyto\_user\_f (const vsip\_mview\_f*, vsip\_major, vsip\_scalar\_f*~const);\\
void vsip\_vcopyto\_user\_d (const vsip\_vview\_d*, vsip\_scalar\_d*~const);\\
void vsip\_vcopyto\_user\_f (const vsip\_vview\_f*, vsip\_scalar\_f*~const);\\
void vsip\_vcopyto\_user\_i (const vsip\_vview\_i*, vsip\_scalar\_i*~const);\\
void vsip\_vcopyto\_user\_li (const vsip\_vview\_li*, vsip\_scalar\_li *~const);\\
void vsip\_vcopyto\_user\_si (const vsip\_vview\_si*, vsip\_scalar\_si*~const);\\
void vsip\_vcopyto\_user\_vi (const vsip\_vview\_vi*, vsip\_scalar\_vi*~const);\\
\end{cfuncs}
\pyjvsiph
%
\viewmthd{No}{NA}{NA}{NA}
%
\apyfunc{No}{NA}
%
\pyComment{
\item{No comments}
}

\clearpage
{\large \textbf{\hypertarget{corrFunc}{CORR Class}}}\vspace{.2cm}
\\\cvsiplh
\\\pyjvsiph
\afuncT{cos}{Cosine; An elementary math function.}{elementaryMath}
\\\cvsiplh
\newline \hspace*{.8cm} \vspace*{.1cm} \textbf{Available Functions }
\newline \hspace*{1.1cm} {
\ttfamily
\begin{tabular}[H]{l}
vsip\_scalar\_f vsip\_cos\_f(vsip\_scalar\_f a);\\
vsip\_scalar\_d vsip\_cos\_d(vsip\_scalar\_d a);\\
void vsip\_mcos\_d(const vsip\_mview\_d*, const vsip\_mview\_d*);\\
void vsip\_mcos\_f(const vsip\_mview\_f*, const vsip\_mview\_f*);\\
void vsip\_vcos\_d(const vsip\_vview\_d*, const vsip\_vview\_d*);\\
void vsip\_vcos\_f(const vsip\_vview\_f*, const vsip\_vview\_f*);\\
\end{tabular}
}
\\\pyjvsiph
\viewmthd{yes}{yes}{yes}{inOut.cos}
\apyfunc{yes}{out = cos(in,out)}
\pyComment{\item{The \ttbf{cos} function works much the same as the C VSIPL version except that a convenience pointer to the output view is returned. This may be done in-place if \ttbf{in==out}.}}

\afuncT{cosh}{Hyperbolic Cosine; An elementwise function}{elementaryMath}
\\\cvsiplh
\afh
\\\hspace*{.04\textwidth} {
\ttfamily
\begin{tabular}[H]{l}
vsip\_scalar\_f vsip\_cosh\_f(vsip\_scalar\_f a);\\
vsip\_scalar\_d vsip\_cosh\_d(vsip\_scalar\_d a);\\
void vsip\_mcosh\_d(const vsip\_mview\_d*, const vsip\_mview\_d*);\\
void vsip\_mcosh\_f(const vsip\_mview\_f*, const vsip\_mview\_f*);\\
void vsip\_vcosh\_d(const vsip\_vview\_d*, const vsip\_vview\_d*);\\
void vsip\_vcosh\_f(const vsip\_vview\_f*, const vsip\_vview\_f*);\\
\end{tabular}
}
\\\pyjvsiph
\viewmthd{yes}{yes}{yes}{inOut.cosh}
\apyfunc{yes}{out = cosh(in,out)}
\pyComment{
\item{The \ttbf{cosh} function works much the same as the C VSIPL version except that a convenience pointer to the output view is returned. This may be done in-place if \ttbf{in==out}.}}

\input{covsolFunc}
\afuncT{cumsum}{Cumulative Sum}{unaryOperations}
\index{Cumulative Sum}
\\\cvsiplh
\\\pyjvsiph
\hspace*{1cm}\texttt{
\begin{tabular}[H]{l}
vsip\_block\_bl* vsip\_mdestroy\_bl(vsip\_mview\_bl*);\\
vsip\_block\_bl* vsip\_vdestroy\_bl(vsip\_vview\_bl*);\\
vsip\_block\_d* vsip\_tdestroy\_d(vsip\_tview\_d*);\\
vsip\_block\_d* vsip\_mdestroy\_d(vsip\_mview\_d*);\\
vsip\_block\_d* vsip\_vdestroy\_d(vsip\_vview\_d*);\\
vsip\_block\_f* vsip\_tdestroy\_f(vsip\_tview\_f*);\\
vsip\_block\_f* vsip\_mdestroy\_f(vsip\_mview\_f*);\\
vsip\_block\_f* vsip\_vdestroy\_f(vsip\_vview\_f*);\\
vsip\_block\_i* vsip\_tdestroy\_i(vsip\_tview\_i*);\\
vsip\_block\_i* vsip\_mdestroy\_i(vsip\_mview\_i*);\\
vsip\_block\_i* vsip\_vdestroy\_i(vsip\_vview\_i*);\\
vsip\_block\_mi* vsip\_vdestroy\_mi(vsip\_vview\_mi*);\\
vsip\_block\_si* vsip\_tdestroy\_si(vsip\_tview\_si*);\\
vsip\_block\_si* vsip\_mdestroy\_si(vsip\_mview\_si*);\\
vsip\_block\_si* vsip\_vdestroy\_si(vsip\_vview\_si*);\\
vsip\_block\_uc* vsip\_tdestroy\_uc(vsip\_tview\_uc*);\\
vsip\_block\_uc* vsip\_mdestroy\_uc(vsip\_mview\_uc*);\\
vsip\_block\_uc* vsip\_vdestroy\_uc(vsip\_vview\_uc*);\\
vsip\_block\_vi* vsip\_vdestroy\_vi(vsip\_vview\_vi*);\\
vsip\_cblock\_d* vsip\_ctdestroy\_d(vsip\_ctview\_d*);\\
vsip\_cblock\_d* vsip\_cmdestroy\_d(vsip\_cmview\_d*);\\
vsip\_cblock\_d* vsip\_cvdestroy\_d(vsip\_cvview\_d*);\\
vsip\_cblock\_f* vsip\_ctdestroy\_f(vsip\_ctview\_f*);\\
vsip\_cblock\_f* vsip\_cmdestroy\_f(vsip\_cmview\_f*);\\
vsip\_cblock\_f* vsip\_cvdestroy\_f(vsip\_cvview\_f*);\\
\end{tabular}
}
\newline\hspace*{1cm}\texttt{
\begin{tabular}[H]{l}
int vsip\_chold\_destroy\_d(vsip\_chol\_d*);\\
int vsip\_chold\_destroy\_f(vsip\_chol\_f*);\\
int vsip\_cchold\_destroy\_d(vsip\_cchol\_d*);\\
int vsip\_cchold\_destroy\_f(vsip\_cchol\_f*);\\
\end{tabular}
}
\newline\hspace*{1cm}\texttt{
\begin{tabular}[H]{l}
int vsip\_corr1d\_destroy\_d(vsip\_corr1d\_d*);\\
int vsip\_corr1d\_destroy\_f(vsip\_corr1d\_f*);\\
int vsip\_ccorr1d\_destroy\_d(vsip\_ccorr1d\_d*);\\
int vsip\_ccorr1d\_destroy\_f(vsip\_ccorr1d\_f *cor);\\
\end{tabular}
}
\newline\hspace*{1cm}\texttt{
\begin{tabular}[H]{l}
int vsip\_fir\_destroy\_d(vsip\_fir\_d*);\\
int vsip\_fir\_destroy\_f(vsip\_fir\_f*);\\
int vsip\_rcfir\_destroy\_d(vsip\_rcfir\_d*);\\
int vsip\_rcfir\_destroy\_f(vsip\_rcfir\_f*);\\
int vsip\_cfir\_destroy\_d(vsip\_cfir\_d*);\\
int vsip\_cfir\_destroy\_f(vsip\_cfir\_f*);\\
\end{tabular}
}
\newline\hspace*{1cm}\texttt{
\begin{tabular}[H]{l}
\end{tabular}
}
\newline\hspace*{1cm}\texttt{
\begin{tabular}[H]{l}
int vsip\_lud\_destroy\_d(vsip\_lu\_d*);\\
int vsip\_lud\_destroy\_f(vsip\_lu\_f*);\\
int vsip\_clud\_destroy\_d(vsip\_clu\_d*);\\
int vsip\_clud\_destroy\_f(vsip\_clu\_f*);\\
\end{tabular}
}
\newline\hspace*{1cm}\texttt{
\begin{tabular}[H]{l}
int vsip\_conv1d\_destroy\_d(vsip\_conv1d\_d*);\\
int vsip\_conv1d\_destroy\_f(vsip\_conv1d\_f*);\\
\end{tabular}
}
\newline\hspace*{1cm}\texttt{
\begin{tabular}[H]{l}
int vsip\_qrd\_destroy\_d(vsip\_qr\_d*);\\
int vsip\_qrd\_destroy\_f(vsip\_qr\_f*);\\
int vsip\_cqrd\_destroy\_d(vsip\_cqr\_d*);\\
int vsip\_cqrd\_destroy\_f(vsip\_cqr\_f*);\\
\end{tabular}
}
\newline\hspace*{1cm}\texttt{
\begin{tabular}[H]{l}
int vsip\_fft\_destroy\_d(vsip\_fft\_d*);\\
int vsip\_fft\_destroy\_f(vsip\_fft\_f*);\\
\end{tabular}
}
\newline\hspace*{1cm}\texttt{
\begin{tabular}[H]{l}
int vsip\_fftm\_destroy\_d(vsip\_fftm\_d*);\\
int vsip\_fftm\_destroy\_f(vsip\_fftm\_f*);\\
\end{tabular}
}
\newline\hspace*{1cm}\texttt{
\begin{tabular}[H]{l}
int vsip\_randdestroy(vsip\_randstate *) ;\\
\end{tabular}
}
\newline\hspace*{1cm}\texttt{
\begin{tabular}[H]{l}
void vsip\_blockdestroy\_bl(vsip\_block\_bl*);\\
void vsip\_blockdestroy\_d(vsip\_block\_d*);\\
void vsip\_blockdestroy\_f(vsip\_block\_f*);\\
void vsip\_blockdestroy\_i(vsip\_block\_i*);\\
void vsip\_blockdestroy\_mi(vsip\_block\_mi*);\\
void vsip\_blockdestroy\_si(vsip\_block\_si*);\\
void vsip\_blockdestroy\_uc(vsip\_block\_uc*);\\
void vsip\_blockdestroy\_vi(vsip\_block\_vi *);\\
void vsip\_cblockdestroy\_d(vsip\_cblock\_d*);\\
void vsip\_cblockdestroy\_f(vsip\_cblock\_f*);\\
\end{tabular}
}
\newline\hspace*{1cm}\texttt{
\begin{tabular}[H]{l}
void vsip\_cmalldestroy\_d(vsip\_cmview\_d*);\\
void vsip\_cmalldestroy\_f(vsip\_cmview\_f*);\\
void vsip\_ctalldestroy\_d(vsip\_ctview\_d*);\\
void vsip\_ctalldestroy\_f(vsip\_ctview\_f*);\\
void vsip\_cvalldestroy\_d(vsip\_cvview\_d*);\\
void vsip\_cvalldestroy\_f(vsip\_cvview\_f*);\\
void vsip\_malldestroy\_bl(vsip\_mview\_bl*);\\
void vsip\_malldestroy\_d(vsip\_mview\_d*);\\
void vsip\_malldestroy\_f(vsip\_mview\_f*);\\
void vsip\_malldestroy\_i(vsip\_mview\_i*);\\
void vsip\_malldestroy\_si(vsip\_mview\_si*);\\
void vsip\_malldestroy\_uc(vsip\_mview\_uc*);\\
void vsip\_talldestroy\_d(vsip\_tview\_d*);\\
void vsip\_talldestroy\_f(vsip\_tview\_f*);\\
void vsip\_talldestroy\_i(vsip\_tview\_i*);\\
void vsip\_talldestroy\_si(vsip\_tview\_si*);\\
void vsip\_talldestroy\_uc(vsip\_tview\_uc*);\\
void vsip\_valldestroy\_bl(vsip\_vview\_bl*);\\
void vsip\_valldestroy\_d(vsip\_vview\_d*);\\
void vsip\_valldestroy\_f(vsip\_vview\_f*);\\
void vsip\_valldestroy\_i(vsip\_vview\_i*);\\
void vsip\_valldestroy\_mi(vsip\_vview\_mi*);\\
void vsip\_valldestroy\_si(vsip\_vview\_si*);\\
void vsip\_valldestroy\_uc(vsip\_vview\_uc*);\\
void vsip\_valldestroy\_vi(vsip\_vview\_vi*);\\
\end{tabular}
}
\newline\hspace*{1cm}\texttt{
\begin{tabular}[H]{l}
void vsip\_permute\_destroy(vsip\_permute*);\\
\end{tabular}
}
\newline\hspace*{1cm}\texttt{
\begin{tabular}[H]{l}
void vsip\_spline\_destroy\_d(vsip\_spline\_d *);\\
void vsip\_spline\_destroy\_f(vsip\_spline\_f *);\\
\end{tabular}
}
\newline\hspace*{1cm}\texttt{
\begin{tabular}[H]{l}
int vsip\_svd\_destroy\_f(vsip\_sv\_f* );\\
int vsip\_csvd\_destroy\_f(vsip\_csv\_f* svd);\\
int vsip\_svd\_destroy\_d(vsip\_sv\_d* );\\
int vsip\_csvd\_destroy\_d(vsip\_csv\_d* svd);\\
\end{tabular}
}

\afunc{div}{Divide two \ttbf{view}s, a scalar and a \ttbf{view} or a \ttbf{view} and a scalar. A binary operation. See table \ref{tab:binaryOperations}.}
\cvsiplh
\newline
\hspace*{1cm}\parbox{.9\textwidth}{\textrm{There are many combinations of divide available in \jv. The specification provides the normal \ttbf{view} divides of complex-complex and real-real types; but also provides real-complex and complex-real divides as well as mixed scalar -\ttbf{view} and \ttbf{view}-scalar devides. Consequently the listed available functions are broken up into several tables below.}}\vspace{.1cm}
\newline
{\hspace*{.8cm} \vspace*{.1cm} \textbf{Available Functions } \newline}
\hspace*{1cm}
\texttt{
\begin{tabular}[H]{l}
\multicolumn{1}{c}{\rmfamily \bfseries Scalar Functions}\\ \hline
vsip\_cscalar\_d vsip\_cdiv\_d(\\*\hspace{1cm}vsip\_cscalar\_d, vsip\_cscalar\_d);\\
vsip\_cscalar\_d vsip\_crdiv\_d(\\*\hspace{1cm}vsip\_cscalar\_d, vsip\_scalar\_d);\\
vsip\_cscalar\_f vsip\_cdiv\_f(\\*\hspace{1cm}vsip\_cscalar\_f, vsip\_cscalar\_f);\\
vsip\_cscalar\_f vsip\_crdiv\_f(\\*\hspace{1cm}vsip\_cscalar\_f, vsip\_scalar\_f);\\
\end{tabular}
}
\clearpage
\hspace*{1cm}\texttt{
\begin{tabular}[H]{l}
\multicolumn{1}{c}{\rmfamily \bfseries Normal View Functions}\\ \hline
void vsip\_vdiv\_d(\\*\hspace{1cm}const vsip\_vview\_d*, const vsip\_vview\_d*,\\*\hspace{1cm}const vsip\_vview\_d*);\\
void vsip\_vdiv\_f(\\*\hspace{1cm}const vsip\_vview\_f*, const vsip\_vview\_f*,\\*\hspace{1cm}const vsip\_vview\_f*);\\
void vsip\_mdiv\_d(\\*\hspace{1cm}const vsip\_mview\_d*, const vsip\_mview\_d*,\\*\hspace{1cm}const vsip\_mview\_d*);\\
void vsip\_mdiv\_f(\\*\hspace{1cm}const vsip\_mview\_f*, const vsip\_mview\_f*,\\*\hspace{1cm}const vsip\_mview\_f*);\\
void vsip\_cvdiv\_d(\\*\hspace{1cm}const vsip\_cvview\_d*, const vsip\_cvview\_d*,\\*\hspace{1cm}const vsip\_cvview\_d*);\\
void vsip\_cvdiv\_f(\\*\hspace{1cm}const vsip\_cvview\_f*, const vsip\_cvview\_f*,\\*\hspace{1cm}const vsip\_cvview\_f*);\\
void vsip\_cmdiv\_d(\\*\hspace{1cm}const vsip\_cmview\_d*, const vsip\_cmview\_d*,\\*\hspace{1cm}const vsip\_cmview\_d*);\\
void vsip\_cmdiv\_f(\\*\hspace{1cm}const vsip\_cmview\_f*, const vsip\_cmview\_f*,\\*\hspace{1cm}const vsip\_cmview\_f*);\\
\end{tabular}
}
\clearpage
\hspace*{1cm}\texttt{
\begin{tabular}[H]{l}
\multicolumn{1}{c}{\rmfamily \bfseries Mixed Depth View Functions}\\ \hline
void vsip\_rcvdiv\_d(\\*\hspace{1cm}const vsip\_vview\_d*, const vsip\_cvview\_d*,\\*\hspace{1cm}const vsip\_cvview\_d*);\\
void vsip\_rcvdiv\_f(\\*\hspace{1cm}const vsip\_vview\_f*, const vsip\_cvview\_f*,\\*\hspace{1cm}const vsip\_cvview\_f*);\\
void vsip\_crvdiv\_d(\\*\hspace{1cm}const vsip\_cvview\_d*, const vsip\_vview\_d*,\\*\hspace{1cm}const vsip\_cvview\_d*);\\
void vsip\_crvdiv\_f(\\*\hspace{1cm}const vsip\_cvview\_f*, const vsip\_vview\_f*,\\*\hspace{1cm}const vsip\_cvview\_f*);\\
void vsip\_rcmdiv\_d(\\*\hspace{1cm}const vsip\_mview\_d*, const vsip\_cmview\_d*,\\*\hspace{1cm}const vsip\_cmview\_d*);\\
void vsip\_rcmdiv\_f(\\*\hspace{1cm}const vsip\_mview\_f*, const vsip\_cmview\_f*,\\*\hspace{1cm}const vsip\_cmview\_f*);\\
void vsip\_crmdiv\_d(\\*\hspace{1cm}const vsip\_cmview\_d*, const vsip\_mview\_d*,\\*\hspace{1cm}const vsip\_cmview\_d*);\\
void vsip\_crmdiv\_f(\\*\hspace{1cm}const vsip\_cmview\_f*, const vsip\_mview\_f*,\\*\hspace{1cm}const vsip\_cmview\_f*);\\
\end{tabular}
}
\newline\hspace*{1cm}\texttt{
\begin{tabular}[H]{l}
\multicolumn{1}{c}{\rmfamily \bfseries View Divide Scalar Functions}\\ \hline
void vsip\_vsdiv\_d(\\*\hspace{1cm}const vsip\_vview\_d*, vsip\_scalar\_d,\\*\hspace{1cm}const vsip\_vview\_d*);\\
void vsip\_vsdiv\_f(\\*\hspace{1cm}const vsip\_vview\_f*, vsip\_scalar\_f,\\*\hspace{1cm}const vsip\_vview\_f*);\\
void vsip\_cvrsdiv\_d(\\*\hspace{1cm}const vsip\_cvview\_d*, vsip\_scalar\_d,\\*\hspace{1cm}const vsip\_cvview\_d*);\\
void vsip\_cvrsdiv\_f(\\*\hspace{1cm}const vsip\_cvview\_f*, vsip\_scalar\_f,\\*\hspace{1cm}const vsip\_cvview\_f*);\\
void vsip\_msdiv\_d(\\*\hspace{1cm}const vsip\_mview\_d*, vsip\_scalar\_d,\\*\hspace{1cm}const vsip\_mview\_d*);\\
void vsip\_msdiv\_f(\\*\hspace{1cm}const vsip\_mview\_f*, vsip\_scalar\_f,\\*\hspace{1cm}const vsip\_mview\_f*);\\
void vsip\_cmrsdiv\_d(\\*\hspace{1cm}const vsip\_cmview\_d*, vsip\_scalar\_d,\\*\hspace{1cm}const vsip\_cmview\_d*);\\
void vsip\_cmrsdiv\_f(\\*\hspace{1cm}const vsip\_cmview\_f*, vsip\_scalar\_f,\\*\hspace{1cm}const vsip\_cmview\_f*);\\
\end{tabular}
}
\newline\hspace*{1cm}\texttt{
\begin{tabular}[H]{l}
\multicolumn{1}{c}{\rmfamily \bfseries Scalar Divide View Functions}\\ \hline
void vsip\_svdiv\_d(\\*\hspace{1cm}vsip\_scalar\_d, const vsip\_vview\_d*,\\*\hspace{1cm}const vsip\_vview\_d*);\\
void vsip\_svdiv\_f(\\*\hspace{1cm}vsip\_scalar\_f, const vsip\_vview\_f*,\\*\hspace{1cm}const vsip\_vview\_f*);\\
void vsip\_rscvdiv\_d(\\*\hspace{1cm}vsip\_scalar\_d, const vsip\_cvview\_d*,\\*\hspace{1cm}const vsip\_cvview\_d*);\\
void vsip\_rscvdiv\_f(\\*\hspace{1cm}vsip\_scalar\_f, const vsip\_cvview\_f*,\\*\hspace{1cm}const vsip\_cvview\_f*);\\
void vsip\_csvdiv\_d(\\*\hspace{1cm}vsip\_cscalar\_d, const vsip\_cvview\_d*,\\*\hspace{1cm}const vsip\_cvview\_d*);\\
void vsip\_csvdiv\_f(\\*\hspace{1cm}vsip\_cscalar\_f, const vsip\_cvview\_f*,\\*\hspace{1cm}const vsip\_cvview\_f*);\\
void vsip\_smdiv\_d(\\*\hspace{1cm}vsip\_scalar\_d, const vsip\_mview\_d*,\\*\hspace{1cm}const vsip\_mview\_d*);\\
void vsip\_smdiv\_f(\\*\hspace{1cm}vsip\_scalar\_f, const vsip\_mview\_f*,\\*\hspace{1cm}const vsip\_mview\_f*);\\
void vsip\_rscmdiv\_d(\\*\hspace{1cm}vsip\_scalar\_d, const vsip\_cmview\_d*,\\*\hspace{1cm}const vsip\_cmview\_d*);\\
void vsip\_rscmdiv\_f(\\*\hspace{1cm}vsip\_scalar\_f, const vsip\_cmview\_f*,\\*\hspace{1cm}const vsip\_cmview\_f*);\\
void vsip\_csmdiv\_d(\\*\hspace{1cm}vsip\_cscalar\_d, const vsip\_cmview\_d*,\\*\hspace{1cm}const vsip\_cmview\_d*);\\
void vsip\_csmdiv\_f(\\*\hspace{1cm}vsip\_cscalar\_f, const vsip\_cmview\_f*,\\*\hspace{1cm}const vsip\_cmview\_f*);\\
\end{tabular}
}
\pyjvsiph
\newline\hspace*{.8cm}{\textbf{View Method}\\
\hspace*{1.1cm}Overloaded on divide operator.\\
\hspace*{1.1cm}\textbf{In Place: }\hspace{.2cm} yes\\
\hspace*{1.1cm}\textbf{Example: }\ttbf{a /= b; a /= 2}\\*
\hspace*{1.5cm}Elements of \ttbf{view a} replaced with result.\\
\hspace*{1.1cm}\textbf{Out of Place: }\hspace{.2cm} yes\\
\hspace*{1.1cm}\textbf{Example: }\ttbf{c = a / b; d = 2 / c; e = c / 2}\\*
\hspace*{1.5cm}\parbox{.85\textwidth}{\ttbf{view c}, \ttbf{view d}, and \ttbf{view e}  created and filled with result of operation.}\\
\afuncT{dot}{Vector Dot Product }{matrixOperations}
\\\cvsiplh
\\ \hspace*{.8cm} \vspace*{.1cm} \textbf{Available Functions }
\\ \hspace*{0.03\textwidth} {
\ttfamily
\begin{tabular}[H]{l}
vsip\_cscalar\_d vsip\_cvdot\_d(\\*\hspace{.6cm}const vsip\_cvview\_d*, const vsip\_cvview\_d*);\\
vsip\_cscalar\_f vsip\_cvdot\_f(\\*\hspace{.6cm}const vsip\_cvview\_f*, const vsip\_cvview\_f*);\\
vsip\_scalar\_d vsip\_vdot\_d(\\*\hspace{.6cm}const vsip\_vview\_d*, const vsip\_vview\_d*);\\
vsip\_scalar\_f vsip\_vdot\_f(\\*\hspace{.6cm}const vsip\_vview\_f*, const vsip\_vview\_f*);\\
\end{tabular}
}
\\\pyjvsiph

\afuncT{euler}{Euler}{unaryOperations}
\\\cvsiplh
\afh
\\\hspace*{.04\textwidth} {
\ttfamily
}
\\\pyjvsiph
\afunc{exp10}{Exponential Base 10; An elementwise function}
\\\cvsiplh
\\\pyjvsiph

\afuncT{exp}{Exponential; An elementwise function}{elementaryMath}
\\\cvsiplh
\\\pyjvsiph

\afuncT{expoavg}{Exponential Average. Computes an exponential weighted average, by element, of two \ttbf{view}s.}{binaryOperations}
\\\cvsiplh
\afh
{\ttfamily
\\\hspace*{.04\textwidth}\begin{tabular}[H]{l}
void vsip\_cmexpoavg\_d(\\*\hspace*{1cm}vsip\_scalar\_d, const vsip\_cmview\_d*, const vsip\_cmview\_d*);\Bs\\
void vsip\_cmexpoavg\_f(\\*\hspace*{1cm}vsip\_scalar\_f, const vsip\_cmview\_f*, const vsip\_cmview\_f*);\Bs\\
void vsip\_cvexpoavg\_d(\\*\hspace*{1cm}vsip\_scalar\_d, const vsip\_cvview\_d*, const vsip\_cvview\_d*);\Bs\\
void vsip\_cvexpoavg\_f(\\*\hspace*{1cm}vsip\_scalar\_f, const vsip\_cvview\_f*, const vsip\_cvview\_f*);\Bs\\
void vsip\_mexpoavg\_d(\\*\hspace*{1cm}vsip\_scalar\_d, const vsip\_mview\_d*, const vsip\_mview\_d*);\Bs\\
void vsip\_mexpoavg\_f(\\*\hspace*{1cm}vsip\_scalar\_f, const vsip\_mview\_f*, const vsip\_mview\_f*);\Bs\\
void vsip\_vexpoavg\_d(\\*\hspace*{1cm}vsip\_scalar\_d, const vsip\_vview\_d*, const vsip\_vview\_d*);\Bs\\
void vsip\_vexpoavg\_f(\\*\hspace*{1cm}vsip\_scalar\_f, const vsip\_vview\_f*, const vsip\_vview\_f*);\Bs\\
\end{tabular}
}
\\\pyjvsiph
\clearpage
{\large \textbf{\hypertarget{fftFunc}{FFT Class}}}\vspace{.2cm}\\
\hspace*{.3cm}
\parbox{0.85\textwidth}{Discrete Fourier Transforms. See FFT Functions table \ref{tab:fftFunctions}}
\cvsiplh 
\newline \hspace*{.8cm} \vspace*{.1cm} \textbf{Available Functions }
\newline \hspace*{.8cm} \vspace*{.1cm} \texttt{fft\_create}
\newline \hspace*{1.1cm} {
\ttfamily
\begin{tabular}[H]{l}\hline
\hline \multicolumn{1}{c}{\rmfamily \bfseries FFT Create Functions}\\ \hline
vsip\_fft\_d* vsip\_ccfftip\_create\_d(vsip\_length, vsip\_scalar\_d,\\*\hspace{.7cm}vsip\_fft\_dir, unsigned int, vsip\_alg\_hint);\\
vsip\_fft\_d* vsip\_ccfftop\_create\_d(vsip\_length, vsip\_scalar\_d,\\*\hspace{.7cm}vsip\_fft\_dir, unsigned int, vsip\_alg\_hint);\\
vsip\_fft\_d* vsip\_crfftop\_create\_d(vsip\_length,vsip\_scalar\_d,\\*\hspace{.7cm}unsigned int, vsip\_alg\_hint);\\
vsip\_fft\_d* vsip\_rcfftop\_create\_d(vsip\_length,vsip\_scalar\_d,\\*\hspace{.7cm}unsigned int, vsip\_alg\_hint);\\
vsip\_fft\_f* vsip\_ccfftip\_create\_f(vsip\_length,vsip\_scalar\_f,\\*\hspace{.7cm}vsip\_fft\_dir, unsigned int, vsip\_alg\_hint);\\
vsip\_fft\_f* vsip\_ccfftop\_create\_f(vsip\_length,vsip\_scalar\_f,\\*\hspace{.7cm}vsip\_fft\_dir, unsigned int, vsip\_alg\_hint);\\
vsip\_fft\_f* vsip\_crfftop\_create\_f(vsip\_length,vsip\_scalar\_f,\\*\hspace{.7cm}unsigned int, vsip\_alg\_hint);\\
vsip\_fft\_f* vsip\_rcfftop\_create\_f(vsip\_length,vsip\_scalar\_f,\\*\hspace{.7cm}unsigned int, vsip\_alg\_hint);\\
\hline \multicolumn{1}{c}{\rmfamily \bfseries Multiple FFT Create Functions}\\ \hline
vsip\_fftm\_d* vsip\_ccfftmip\_create\_d(vsip\_length, vsip\_length,\\*\hspace{.7cm}vsip\_scalar\_d, vsip\_fft\_dir, vsip\_major, unsigned int,\\*\hspace{.7cm}vsip\_alg\_hint);\\
vsip\_fftm\_d* vsip\_ccfftmop\_create\_d(vsip\_length, vsip\_length,\\*\hspace{.7cm}vsip\_scalar\_d, vsip\_fft\_dir, vsip\_major, unsigned int,\\*\hspace{.7cm}vsip\_alg\_hint);\\
vsip\_fftm\_d* vsip\_crfftmop\_create\_d(vsip\_length, vsip\_length,\\*\hspace{.7cm}vsip\_scalar\_d, vsip\_major, unsigned int, vsip\_alg\_hint);\\
vsip\_fftm\_d* vsip\_rcfftmop\_create\_d(vsip\_length, vsip\_length,\\*\hspace{.7cm}vsip\_scalar\_d, vsip\_major, unsigned int, vsip\_alg\_hint);\\
vsip\_fftm\_f* vsip\_ccfftmip\_create\_f(vsip\_length, vsip\_length,\\*\hspace{.7cm}vsip\_scalar\_f, vsip\_fft\_dir, vsip\_major, unsigned int,\\*\hspace{.7cm}vsip\_alg\_hint);\\
vsip\_fftm\_f* vsip\_ccfftmop\_create\_f(vsip\_length, vsip\_length,\\*\hspace{.7cm}vsip\_scalar\_f, vsip\_fft\_dir, vsip\_major, unsigned int,\\*\hspace{.7cm}vsip\_alg\_hint);\\
vsip\_fftm\_f* vsip\_crfftmop\_create\_f(vsip\_length, vsip\_length,\\*\hspace{.7cm}vsip\_scalar\_f, vsip\_major, unsigned int, vsip\_alg\_hint);\\
vsip\_fftm\_f* vsip\_rcfftmop\_create\_f(vsip\_length, vsip\_length,\\*\hspace{.7cm}vsip\_scalar\_f, vsip\_major, unsigned int, vsip\_alg\_hint);\\ \hline
\end{tabular}
}
\clearpage
\hspace*{.8cm} \vspace*{.1cm} \texttt{fft\_destroy}
\newline \hspace*{1.1cm} {
\ttfamily
\begin{tabular}[H]{l}\hline
\hline \multicolumn{1}{c}{\rmfamily \bfseries FFT Destroy Functions}\\ \hline
int vsip\_fft\_destroy\_d(vsip\_fft\_d*);\\
int vsip\_fft\_destroy\_f(vsip\_fft\_f*);\\
\hline \multicolumn{1}{c}{\rmfamily \bfseries Multiple FFT Destroy Functions}\\ \hline
int vsip\_fftm\_destroy\_d(vsip\_fftm\_d*);\\
int vsip\_fftm\_destroy\_f(vsip\_fftm\_f*);\\ \hline
\end{tabular}
}\vspace{.1cm}
\newline \hspace*{.8cm} \vspace*{.1cm} \texttt{fft}
\newline \hspace*{1.1cm} {
\ttfamily
\begin{tabular}[H]{l} \hline
\hline \multicolumn{1}{c}{\rmfamily \bfseries FFT Functions}\\ \hline
void vsip\_ccfftip\_d(const vsip\_fft\_d*,\\*\hspace{.7cm}const vsip\_cvview\_d*);\\
void vsip\_ccfftip\_f(const vsip\_fft\_f*,\\*\hspace{.7cm} const vsip\_cvview\_f*);\\
void vsip\_ccfftop\_d(const vsip\_fft\_d*,\\*\hspace{.7cm} const vsip\_cvview\_d*, const vsip\_cvview\_d*);\\
void vsip\_ccfftop\_f(const vsip\_fft\_f*,\\*\hspace{.7cm} const vsip\_cvview\_f*, const vsip\_cvview\_f*);\\
void vsip\_crfftop\_d(const vsip\_fft\_d*,\\*\hspace{.7cm} const vsip\_cvview\_d*, const vsip\_vview\_d*);\\
void vsip\_crfftop\_f(const vsip\_fft\_f*,\\*\hspace{.7cm} const vsip\_cvview\_f*, const vsip\_vview\_f*);\\
void vsip\_rcfftop\_d(const vsip\_fft\_d*,\\*\hspace{.7cm} const vsip\_vview\_d*, const vsip\_cvview\_d*);\\
void vsip\_rcfftop\_f(const vsip\_fft\_f*,\\*\hspace{.7cm} const vsip\_vview\_f*, const vsip\_cvview\_f*);\\ \hline \multicolumn{1}{c}{\rmfamily \bfseries Multiple FFT Functions}\\ \hline
void vsip\_ccfftmip\_d(const vsip\_fftm\_d*,\\*\hspace{.7cm} const vsip\_cmview\_d*);\\
void vsip\_ccfftmip\_f(const vsip\_fftm\_f*,\\*\hspace{.7cm} const vsip\_cmview\_f*);\\
void vsip\_ccfftmop\_d(const vsip\_fftm\_d*,\\*\hspace{.7cm} const vsip\_cmview\_d*, const vsip\_cmview\_d*);\\
void vsip\_ccfftmop\_f(const vsip\_fftm\_f*,\\*\hspace{.7cm} const vsip\_cmview\_f*, const vsip\_cmview\_f*);\\
void vsip\_crfftmop\_d(const vsip\_fftm\_d*,\\*\hspace{.7cm} const vsip\_cmview\_d*, const vsip\_mview\_d*);\\
void vsip\_crfftmop\_f(const vsip\_fftm\_f*,\\*\hspace{.7cm} const vsip\_cmview\_f*, const vsip\_mview\_f*);\\
void vsip\_rcfftmop\_d(const vsip\_fftm\_d*,\\*\hspace{.7cm} const vsip\_mview\_d*, const vsip\_cmview\_d*);\\
void vsip\_rcfftmop\_f(const vsip\_fftm\_f*,\\*\hspace{.7cm} const vsip\_mview\_f*, const vsip\_cmview\_f*);\\ \hline \end{tabular}
}
\clearpage
\hspace*{.8cm} \texttt{fft\_getattr}
\newline \hspace*{1.1cm} {
\ttfamily
\begin{tabular}[H]{l}\hline
\hline \multicolumn{1}{c}{\rmfamily \bfseries FFT Get Attributes Functions}\\ \hline
void vsip\_fft\_getattr\_d(const vsip\_fft\_d*, vsip\_fft\_attr\_d*);\\
void vsip\_fft\_getattr\_f(const vsip\_fft\_f*, vsip\_fft\_attr\_f*);\\
\hline \multicolumn{1}{c}{\rmfamily \bfseries Multiple FFT Get Attributes Functions}\\ \hline
void vsip\_fftm\_getattr\_d(const vsip\_fftm\_d*, vsip\_fftm\_attr\_d*);\\
void vsip\_fftm\_getattr\_f(const vsip\_fftm\_f*, vsip\_fftm\_attr\_f*);\\
\end{tabular}
}
\pyjvsiph
\newline\hspace*{.8cm}{\textbf{View Methods}\\
\hspace*{1cm}\parbox{10.5cm}{
\begin{itemize}
\item {Special \ttbf{view} methods exist.} 
\item {The scale factor is always one for \ttbf{view} methods.}
\item {For out of place the method will create and return the output \ttbf{view}.
\item {\ttbf{View} methods determine if the FFT is a multiple FFT or a vector FFT by the \ttbf{view} type}
}
\end{itemize}}\\
\hspace*{1.1cm}\textbf{In-Place: }\hspace{.2cm} yes\\
\hspace*{1.1cm}\textbf{Example: }\\
\hspace*{2.9cm}Forward transform of vector \ttbf{x} in-place\\*
\hspace*{3.5cm}\ttbf{x.fftip}\\*
\hspace*{2.9cm}For matrix FFT multiple use major attribute.\\*
\hspace*{3.5cm}\ttbf{x.ROW.fftip} \& \ttbf{x.COL.fftip}\\*
\hspace*{2.9cm}Inverse transform of vector \ttbf{x} in-place\\
\hspace*{3.5cm}\ttbf{x.ifftip}\\*
\hspace*{1.1cm}\textbf{Out-of-Place: }\hspace{.2cm} yes\\
\hspace*{1.1cm}\textbf{Example: }\\
\hspace*{2.9cm}Real to complex and complex to real FFT\\*
\hspace*{3.5cm}\ttbf{ y=x.rcfft}\\*
\hspace*{3.5cm}\ttbf{ z=y.crfft}\\*
\hspace*{2.9cm}Complex to complex transform of vector \ttbf{x} out-of-place\\
\hspace*{3.5cm}\ttbf{ y=x.fftop}\\
 \hspace*{2.9cm}Complex to complex inverse transform of vector \ttbf{x}\\*\hspace*{2.9cm}out-of-place\\
\hspace*{3.5cm}\ttbf{ y=x.ifftop}\\
 \hspace*{2.9cm}Complex to complex multiple transform of matrix \ttbf{x}\\*\hspace*{2.9cm}out-of-place by column\\
\hspace*{3.5cm}\ttbf{ y=x.COL.fftop}\\
 \hspace*{2.9cm}Complex to complex multiple transform of matrix \ttbf{x}\\*\hspace*{2.9cm}out-of-place by row\\
\hspace*{3.5cm}\ttbf{ y=x.ROW.fftop}\\
 \hspace*{2.9cm}Complex to complex multiple inverse transform of matrix \ttbf{x}\\*\hspace*{2.9cm}out-of-place by column\\
\hspace*{3.5cm}\ttbf{ y=x.COL.ifftop}\\
 \hspace*{2.9cm}Complex to complex multiple inverse transform of matrix \ttbf{x}\\*\hspace*{2.9cm}out-of-place by row\\
\hspace*{3.5cm}\ttbf{ y=x.ROW.ifftop}\\
\clearpage
\hspace*{.8cm}{\textbf{FFT Methods}\\
\hspace*{1.cm}\parbox{.85\textwidth}{To create an FFT object use \\*
\hspace*{1.cm} \ttbf{fftObj=FFT(t,*args)}\\*
Where \ttbf{args} is a tuple containing the create parameters for the FFT type selected, and \ttbf{t} is a string indicating the type of FFT to create.}\\
Note \ttbf{arg} will contain some or all of the following in the order listed}
\begin{itemize}
\item{\ttbf{M}\hspace*{.85cm} \parbox[t]{.8\textwidth}{Column Length}}
\item{\ttbf{N} \hspace*{.8cm} \parbox[t]{.8\textwidth}{Row Length for \ttbf{matrix} or Vector length for \ttbf{vector}}}
\item{\ttbf{scl} \hspace*{.5cm} \parbox[t]{.8\textwidth}{Scale Factor}}
\item{\ttbf{dir} \hspace*{.5cm} \parbox[t]{.8\textwidth}{Direction flag for FFT either VSIP\_FFT\_FWD or VSIP\_FFT\_INV}}
\item{\ttbf{major} \hspace*{.2cm} \parbox[t]{.8\textwidth}{For multiple FFT by row (VSIP\_ROW) or by column (VSIP\_COL)}}
\item{\ttbf{ntimes} \hspace*{.1cm} \parbox[t]{.8\textwidth}{Hint for how much the FFT object will be used. Zero indicates many times.}}
\item{\ttbf{hint} \hspace*{.45cm}\parbox[t]{.8\textwidth}{Algorithm hint to optimize for speed (VSIP\_ALG\_TIME), size (VSIP\_ALG\_SPACE), or accuracy (VSIP\_ALG\_NOISE)}}
\end{itemize}
\begin{table}[H]
\caption{FFT Types and Argument Strings}
\label{tab:fftTypesAndArugments}
\begin{center}
\begin{tabular}{|l l|}\hline
\multicolumn{2}{|c|}{\rmfamily \bfseries Discrete Fourier Transform Class}\\
'ccfftip\_f' & Complex-to-complex FFT single precision float in-place\\
\multicolumn{2}{c}{\ttbf{arg = }}\\
'ccfftop\_f' & Complex-to-complex FFT single precision float out-of-place\\
\multicolumn{2}{c}{\ttbf{arg = }}\\
'rcfftop\_f' & Real-to-complex FFT single precision float out-of-place\\
\multicolumn{2}{c}{\ttbf{arg = }}\\
'crfftop\_f'& Complex-to-real FFT single precision out-of-place\\
\multicolumn{2}{c}{\ttbf{arg = }}\\
'ccfftip\_d' & Complex-to-complex FFT double precision in-place\\
\multicolumn{2}{c}{\ttbf{arg = }}\\
'ccfftop\_d'& Complex-to-complex FFT double precision out-of-place\\
\multicolumn{2}{c}{\ttbf{arg = }}\\
'rcfftop\_d'& Real-to-complex multiple FFT single precision out-of-place\\
\multicolumn{2}{c}{\ttbf{arg = }}\\
'crfftop\_d'& Complex-to-real multiple FFT single precision out-of-place\\
\multicolumn{2}{c}{\ttbf{arg = }}\\
'ccfftmip\_f' & Complex-to-complex multiple FFT single precision in-place\\
\multicolumn{2}{c}{\ttbf{arg = }}\\
'ccfftmop\_f' & Complex-to-complex multiple FFT single precision out-of-place\\
\multicolumn{2}{c}{\ttbf{arg = }}\\
'rcfftmop\_f' & Real-to-complex multiple FFT single precision out-of-place\\
\multicolumn{2}{c}{\ttbf{arg = }}\\
'crfftmop\_f' & Complex-to-real multiple FFT single precision out-of-place\\
\multicolumn{2}{c}{\ttbf{arg = }}\\
'ccfftmip\_d' & Complex-to-complex multiple FFT double precision in-place\\
\multicolumn{2}{c}{\ttbf{arg = }}\\
'ccfftmop\_d' & Complex-to-complex multiple FFT double precision out-of-place\\
\multicolumn{2}{c}{\ttbf{arg = }}\\
'rcfftmop\_d' & Real-to-complex multiple FFT double precision out-of-place\\
\multicolumn{2}{c}{\ttbf{arg = }}\\
'crfftmop\_d' & Complex-to-real multiple FFT double precision out-of-place\\
\multicolumn{2}{c}{\ttbf{arg = }}\\
\hline\end{tabular}
\end{center}
\label{default}
\end{table}

\afuncT{fill}{Fill a \ttbf{view} with a value.}{elementGenerationOperations}
\\\cvsiplh
\\ \hspace*{.8cm} \vspace*{.1cm} \textbf{Available Functions }
\\ \hspace*{1cm} {\ttfamily
\begin{tabular}[H]{l}
void vsip\_cmfill\_d(vsip\_cscalar\_d, const vsip\_cmview\_d*);\\
void vsip\_cmfill\_f(vsip\_cscalar\_f, const vsip\_cmview\_f*);\\
void vsip\_cvfill\_d(vsip\_cscalar\_d, const vsip\_cvview\_d*);\\
void vsip\_cvfill\_f(vsip\_cscalar\_f, const vsip\_cvview\_f*);\\
void vsip\_mfill\_d(vsip\_scalar\_d, const vsip\_mview\_d*);\\
void vsip\_mfill\_f(vsip\_scalar\_f, const vsip\_mview\_f*);\\
void vsip\_mfill\_i(vsip\_scalar\_i, const vsip\_mview\_i*);\\
void vsip\_mfill\_si(vsip\_scalar\_si, const vsip\_mview\_si*);\\
void vsip\_mfill\_uc(vsip\_scalar\_uc, const vsip\_mview\_uc*);\\
void vsip\_vfill\_d(vsip\_scalar\_d, const vsip\_vview\_d*);\\
void vsip\_vfill\_f(vsip\_scalar\_f, const vsip\_vview\_f*);\\
void vsip\_vfill\_i(vsip\_scalar\_i, const vsip\_vview\_i*);\\
void vsip\_vfill\_si(vsip\_scalar\_si, const vsip\_vview\_si*);\\
void vsip\_vfill\_uc(vsip\_scalar\_uc, const vsip\_vview\_uc*);\\
void vsip\_vfill\_vi(vsip\_scalar\_vi, const vsip\_vview\_vi*);\\
\end{tabular}}
\\\pyjvsiph
%
\viewmthd{Yes}{No}{Yes}{aView.fill(aScalarValue)}
%
\apyfunc{No}{NA}
%
\pyComment{
\item{Using slicing operator overloaded on equal.}
\subitem {\ttbf{For instance:} \ilCode{aView[:]=aScalarValue} or \ilCode{aView[3:4:3]=aScalarValue}.}
\subitem \ttbf{Caution:} \ilCode{aView=aScalarValue} will turn \ilCode{aView} into \ilCode{aScalarValue} releasing the \ttbf{View}.}
}

\afunc{finalize}{ \ref{tab:initSupport}.}
\\\cvsiplh
\\\pyjvsiph

\clearpage
{\large \textbf{\hypertarget{firFunc}{FIR Class}}}\vspace{.2cm}\\
\hspace*{.3cm}
\parbox{0.85\textwidth}{Finite Impulse Response Class. See filter functions table \ref{tab:filterFunctions}}
\cvsiplh 
\newline \hspace*{.8cm} \vspace*{.1cm} \textbf{Available Functions }
\newline \hspace*{.8cm} \vspace*{.1cm} \texttt{fir\_create}
\newline \hspace*{1.1cm} {
\ttfamily
\begin{tabular}[H]{l}
vsip\_rcfir\_d* vsip\_rcfir\_create\_d(\\*\hspace{.7cm}const vsip\_vview\_d*, vsip\_symmetry, vsip\_length,\\*\hspace{.7cm}vsip\_length, vsip\_obj\_state, unsigned, vsip\_alg\_hint);\\
vsip\_rcfir\_f* vsip\_rcfir\_create\_f(\\*\hspace{.7cm}const vsip\_vview\_f*, vsip\_symmetry, vsip\_length,\\*\hspace{.7cm}vsip\_length, vsip\_obj\_state, unsigned, vsip\_alg\_hint);\\
vsip\_cfir\_d* vsip\_cfir\_create\_d(\\*\hspace{.7cm}const vsip\_cvview\_d*, vsip\_symmetry, vsip\_length,\\*\hspace{.7cm}vsip\_length, vsip\_obj\_state, unsigned, vsip\_alg\_hint);\\
vsip\_cfir\_f* vsip\_cfir\_create\_f(\\*\hspace{.7cm}const vsip\_cvview\_f*, vsip\_symmetry, vsip\_length,\\*\hspace{.7cm}vsip\_length, vsip\_obj\_state, unsigned, vsip\_alg\_hint);\\
vsip\_fir\_d* vsip\_fir\_create\_d(\\*\hspace{.7cm}const vsip\_vview\_d*, vsip\_symmetry, vsip\_length,\\*\hspace{.7cm}vsip\_length, vsip\_obj\_state, unsigned, vsip\_alg\_hint);\\
vsip\_fir\_f* vsip\_fir\_create\_f(\\*\hspace{.7cm}const vsip\_vview\_f*, vsip\_symmetry, vsip\_length,\\*\hspace{.7cm}vsip\_length, vsip\_obj\_state, unsigned, vsip\_alg\_hint);\\
\end{tabular}
}
\newline \hspace*{.8cm} \vspace*{.1cm} \texttt{fir\_destroy}
\newline \hspace*{1.1cm} {
\ttfamily
\begin{tabular}[H]{l}
int vsip\_rcfir\_destroy\_d(vsip\_rcfir\_d*);\\
int vsip\_rcfir\_destroy\_f(vsip\_rcfir\_f*);\\
int vsip\_cfir\_destroy\_d(vsip\_cfir\_d*);\\
int vsip\_cfir\_destroy\_f(vsip\_cfir\_f*);\\
int vsip\_fir\_destroy\_d(vsip\_fir\_d*);\\
int vsip\_fir\_destroy\_f(vsip\_fir\_f*);\\
\end{tabular}
}\vspace{.1cm}
\newline \hspace*{.8cm} \vspace*{.1cm} \texttt{firflt}
\newline \hspace*{1.1cm} {
\ttfamily
\begin{tabular}[H]{l}
int vsip\_rcfirflt\_d(vsip\_rcfir\_d*, const vsip\_cvview\_d*,\\*\hspace{.7cm}const vsip\_cvview\_d*);\\
int vsip\_rcfirflt\_f(vsip\_rcfir\_f*, const vsip\_cvview\_f*,\\*\hspace{.7cm}const vsip\_cvview\_f*);\\
int vsip\_cfirflt\_d(vsip\_cfir\_d*, const vsip\_cvview\_d*,\\*\hspace{.7cm}const vsip\_cvview\_d*);\\
int vsip\_cfirflt\_f(vsip\_cfir\_f*, const vsip\_cvview\_f*,\\*\hspace{.7cm}const vsip\_cvview\_f*);\\
int vsip\_firflt\_d(vsip\_fir\_d*, const vsip\_vview\_d*,\\*\hspace{.7cm}const vsip\_vview\_d*);\\
int vsip\_firflt\_f(vsip\_fir\_f*, const vsip\_vview\_f*,\\*\hspace{.7cm}const vsip\_vview\_f*);\\
\end{tabular}
}
\clearpage
\hspace*{.8cm} \texttt{fir\_getattr}
\newline \hspace*{1.1cm} {
\ttfamily
\begin{tabular}[H]{l}
void vsip\_rcfir\_getattr\_d(const vsip\_rcfir\_d*,\\*\hspace{.7cm}vsip\_rcfir\_attr*);\\
void vsip\_rcfir\_getattr\_f(const vsip\_rcfir\_f*,\\*\hspace{.7cm}vsip\_rcfir\_attr*);\\
void vsip\_cfir\_getattr\_d(const vsip\_cfir\_d*,\\*\hspace{.7cm}vsip\_cfir\_attr*);\\
void vsip\_cfir\_getattr\_f(const vsip\_cfir\_f*,\\*\hspace{.7cm}vsip\_cfir\_attr*);\\
void vsip\_fir\_getattr\_d(const vsip\_fir\_d*,\\*\hspace{.7cm}vsip\_fir\_attr*);\\
void vsip\_fir\_getattr\_f(const vsip\_fir\_f*,\\*\hspace{.7cm}vsip\_fir\_attr*);\\
\end{tabular}
}\vspace{.1cm}
\newline\hspace*{.8cm} \texttt{fir\_reset}
\newline \hspace*{1.1cm} {
\ttfamily
\begin{tabular}[H]{l}
void vsip\_rcfir\_reset\_d(vsip\_rcfir\_d*)\\
void vsip\_rcfir\_reset\_f(vsip\_rcfir\_f*)\\
void vsip\_cfir\_reset\_d(vsip\_cfir\_d*)\\
void vsip\_cfir\_reset\_f(vsip\_cfir\_f*)\\
void vsip\_fir\_reset\_d(vsip\_fir\_d*)\\
void vsip\_fir\_reset\_f(vsip\_fir\_f*)\\
\end{tabular}\
}
\pyjvsiph
\afuncT{first}{First}{selectionOperations}
\\\cvsiplh
\\\pyjvsiph
\afunc{floor}{For each element in the input \ttbf{view} round to the largest integral value not greater than the input. An unary operation. See table \ref{tab:unaryOperations}.}
\cvsiplh
\pyjvsiph
\pyjvComment{
\item{The \ilCode{floor} function is not supported in \jv at this time}
}
\afuncT{freqswap}{Swaps halves of a vector, or quadrants of a matrix, to remap zero frequencies from the origin to
the middle.}{miscSigProcFunctions}
\\\cvsiplh
\\\pyjvsiph


\afuncT{gather}{Gather indexed data from an input \ttbf{view} and copy it elementwise into an output \ttbf{view}.}{elementGenerationOperations}
\\\cvsiplh
\\\pyjvsiph
%
\viewmthd{No}{NA}{NA}{NA}
%
\apyfunc{No}{NA}
%
\pyComment{
\item{No comments}
}
\afuncT{gemp}{General matrix product }{matrixOperations}
\\\cvsiplh
\\ \hspace*{.8cm} \vspace*{.1cm} \textbf{Available Functions }
\\ \hspace*{1.1cm} {
\ttfamily
\begin{tabular}[H]{l}
void vsip\_cgemp\_d(vsip\_cscalar\_d,\\*\hspace{.6cm}const vsip\_cmview\_d *, vsip\_mat\_op, const vsip\_cmview\_d *,\\*\hspace{.6cm}vsip\_mat\_op,vsip\_cscalar\_d, const vsip\_cmview\_d *);\\
void vsip\_cgemp\_f(vsip\_cscalar\_f,\\*\hspace{.6cm}const vsip\_cmview\_f *, vsip\_mat\_op, const vsip\_cmview\_f *,\\*\hspace{.6cm}vsip\_mat\_op, vsip\_cscalar\_f, const vsip\_cmview\_f *);\\
void vsip\_gemp\_d(vsip\_scalar\_d,\\*\hspace{.6cm}const vsip\_mview\_d *, vsip\_mat\_op, const vsip\_mview\_d *,\\*\hspace{.6cm}vsip\_mat\_op, vsip\_scalar\_d, const vsip\_mview\_d *);\\
void vsip\_gemp\_f(vsip\_scalar\_f,\\*\hspace{.6cm}const vsip\_mview\_f *, vsip\_mat\_op, const vsip\_mview\_f *,\\*\hspace{.6cm}vsip\_mat\_op, vsip\_scalar\_f, const vsip\_mview\_f *);\\
\end{tabular}
}
\\\pyjvsiph
\viewmthd{No}{No}{No}{}
\apyfunc{yes}{out = gemp(alpha, a, op\_a, b, op\_b, beta, c)}
\pyComment{
\item{For matrix \ttbf{view} a and matrix \ttbf{view} b the general matrix product does $c \leftarrow \alpha \cdot op\_a(a) * op\_b(b) + \beta \cdot c$ where $ * $ indicates a matrix product and $\cdot$ indicates an elementwise multiply.}
\item{The \ttbf{gemp} function works much the same as the C VSIPL version except that a 
convenience pointer to the output view is returned.}
}

\afuncT{gems}{General Matrix Sum. For scalar $\alpha$ and $\beta$ and Matrix $A$ and $C$ and a matrix operator flag $\opM$ do the operation $C \leftarrow \alpha \cdot \opM_{a}(A) + \beta \cdot C$. }{matrixOperations}
\\\cvsiplh
\\ \hspace*{.8cm} \vspace*{.1cm} \textbf{Available Functions }
\\ \hspace*{0.03\textwidth} {
\ttfamily
\begin{tabular}[H]{l}
void vsip\_cgems\_d(\\*\hspace{.6cm}vsip\_cscalar\_d, const vsip\_cmview\_d *,\\*\hspace{.6cm}vsip\_mat\_op, vsip\_cscalar\_d, const vsip\_cmview\_d *);\\
void vsip\_cgems\_f(\\*\hspace{.6cm}vsip\_cscalar\_f, const vsip\_cmview\_f *,\\*\hspace{.6cm}vsip\_mat\_op, vsip\_cscalar\_f, const vsip\_cmview\_f *);\\
void vsip\_gems\_d(\\*\hspace{.6cm}vsip\_scalar\_d, const vsip\_mview\_d *,\\*\hspace{.6cm}vsip\_mat\_op, vsip\_scalar\_d, const vsip\_mview\_d *);\\
void vsip\_gems\_f(\\*\hspace{.6cm}vsip\_scalar\_f, const vsip\_mview\_f *,\\*\hspace{.6cm}vsip\_mat\_op, vsip\_scalar\_f, const vsip\_mview\_f *);\\
\end{tabular}
}
\\\pyjvsiph
\viewmthd{No}{No}{No}{}
\apyfunc{yes}{out = gems(alpha,A,opM,beta,C)}
\pyComment{
\item{The \ttbf{gems} function works much the same as the C VSIPL version except that a 
convenience pointer to the output view is returned. This must be done out-of-place.}
\item{Matrix operator flag $\opM_a$ may be entered as strings 'NTRANS', 'TRANS', 'HERM', or 'CONJ'}
}
\afuncT{herm}{Matrix Hermitian. This function must be done out-of-place unless the input
matrix is square.}{matrixOperations}
\\\cvsiplh
\\\hspace*{.06\textwidth}\parbox{.94\textwidth}{The output matrix must be created of the 
proper size and attributes to accomodate the transpose of the input data.\Bs}
\afh
\\\hspace*{.03\textwidth} {
\ttfamily
\begin{tabular}[H]{l}
void vsip\_cmherm\_f(\\*\hspace{.5cm}const vsip\_cmview\_f*, const vsip\_cmview\_f*);\Bs\\
void vsip\_cmherm\_d(\\*\hspace{.5cm}const vsip\_cmview\_d*, const vsip\_cmview\_d*);\Bs\\
\end{tabular}
}
\\ \hspace*{1.1cm} {
\ttfamily
\begin{tabular}[H]{l}
\end{tabular}
}
\\\pyjvsiph
\viewmthd{yes}{yes}{No}{out=in.herm}
\apyfunc{yes}{out = herm(in,out)}
\pyComment{
\item {The \ttbf{herm} method creates a compact row major matrix of the proper
type to store the output of the transpose and returns it.  The method is defined as a 
property since no arguments are required and is always done out-of-place}
\item{For the \ttbf{herm} method if the calling view is real then the transpose is returned.}
\item{The \ttbf{herm} function works much the same as the C VSIPL version except that a 
convenience pointer to the output view is returned. This must be done out-of-place unless
the matrix is square.}
}
\afunc{histo}{Calculate histogram. \ref{tab:unaryOperations}}
\\\cvsiplh
\\\pyjvsiph


\afuncT{hypot}{Computes the square root of the sum of squares, by element, of two input \ttbf{view}s.}{binaryOperations}
\\\cvsiplh
\afh
{\ttfamily
\\\hspace*{.04\textwidth}\begin{tabular}[H]{l}
vsip\_scalar\_d vsip\_hypot\_d(vsip\_scalar\_d,vsip\_scalar\_d);\Bs\\
vsip\_scalar\_f vsip\_hypot\_f(vsip\_scalar\_f,vsip\_scalar\_f);\Bs\\
void vsip\_mhypot\_d(\\*\hspace*{1cm}const vsip\_mview\_d*, const vsip\_mview\_d*, const vsip\_mview\_d*);\Bs\\
void vsip\_mhypot\_f(\\*\hspace*{1cm}const vsip\_mview\_f*, const vsip\_mview\_f*, const vsip\_mview\_f*);\Bs\\
void vsip\_vhypot\_d(\\*\hspace*{1cm}const vsip\_vview\_d*, const vsip\_vview\_d*, const vsip\_vview\_d*);\Bs\\
void vsip\_vhypot\_f(\\*\hspace*{1cm}const vsip\_vview\_f*, const vsip\_vview\_f*, const vsip\_vview\_f*);\Bs\\
\end{tabular}
}
\\\pyjvsiph
\afuncT{imag}{Return a new real \ttbf{view} of the imaginary part of a complex \ttbf{view}.}{elementGenerationOperations}
\\\cvsiplh
\\\pyjvsiph
%
\viewmthd{No}{NA}{NA}{NA}
%
\apyfunc{No}{NA}
%
\pyComment{
\item{No comments}
}
\afunc{indexbool}{Index a Boolean}
\\\cvsiplh
\\\pyjvsiph
\afuncT{init} {Initialize \cvl{}. This is hidden (not necessary) for \pyjv{} code.}{initSupport}
\\\cvsiplh
\\\pyjvsiph

\afunc{invclip}{Inverse Clip}
\\\cvsiplh
\\\pyjvsiph
\afuncT{jdot}{Complex Vector Conjugate Dot Product}{matrixOperations}
\\ \hspace*{.8cm}For complex vector views $a$ and $b$
\begin{equation*}
\alpha = \sum \limits_{i=0}^{N-1} a_i \cdot \opConj(b_i)
\end{equation*}
\\\cvsiplh
\\ \hspace*{.8cm} \vspace*{.1cm} \textbf{Available Functions }
\\ \hspace*{0.03\textwidth} {
\ttfamily
\begin{tabular}[H]{l}
vsip\_cscalar\_d vsip\_cvjdot\_d(\\*\hspace{.6cm}const vsip\_cvview\_d*, const vsip\_cvview\_d*);\\
vsip\_cscalar\_f vsip\_cvjdot\_f(\\*\hspace{.6cm}const vsip\_cvview\_f*, const vsip\_cvview\_f*);\\
\end{tabular}
}
\\\pyjvsiph
\viewmthd{yes}{No}{NA}{$\alpha$=a.jdot(b)}
\apyfunc{No}{}
\pyComment{\item {For C VSPL only complex vectors are supported. For \pyjv{} \ttbf{jdot} has been extended to support mixed depth vectors.}
\item{Precision of input vectors must agree.}
}
\afuncT{jmul}{Computes the product of a complex \ttbf{view} with the conjugate of a second complex \ttbf{view}, by element. In C VSIPL specification this is called (in overloaded format) \ilCode{vsip\_c\emph{s}jmul\_\emph{p}} because it only works with complex views of shape vector or matrix.  For pyJvsip view methods it has been extended to work with any float views by assuming zero imaginary entries for a real view.}{binaryOperations}
\\\cvsiplh
\begin{cfuncs}
vsip\_cscalar\_d vsip\_cjmul\_d(vsip\_cscalar\_d, vsip\_cscalar\_d);\Bs\\
vsip\_cscalar\_f vsip\_cjmul\_f(vsip\_cscalar\_f, vsip\_cscalar\_f);\Bs\\void vsip\_cmjmul\_d(const~vsip\_cmview\_d*, const~vsip\_cmview\_d*, const~vsip\_cmview\_d*);\Bs\\
void vsip\_cmjmul\_f(const~vsip\_cmview\_f*, const~vsip\_cmview\_f*, const~vsip\_cmview\_f*);\Bs\\
void vsip\_cvjmul\_d(const~vsip\_cvview\_d*, const~vsip\_cvview\_d*, const~vsip\_cvview\_d*);\Bs\\
void vsip\_cvjmul\_f(const~vsip\_cvview\_f*, const~vsip\_cvview\_f*, const~vsip\_cvview\_f*);\Bs\\
\end{cfuncs}
}
\pyjvsiph
\afuncT{kron}{Kronecker Product}{matrixOperations}
\\\cvsiplh
\\ \hspace*{.8cm} \vspace*{.1cm} \textbf{Available Functions }
\\ \hspace*{1.1cm} {
\ttfamily
\begin{tabular}[H]{l}
void vsip\_cmkron\_d(vsip\_cscalar\_d, const vsip\_cmview\_d *,\\*\hspace{.6cm}const vsip\_cmview\_d *, const vsip\_cmview\_d *);\\
void vsip\_cmkron\_f(vsip\_cscalar\_f, const vsip\_cmview\_f *,\\*\hspace{.6cm}const vsip\_cmview\_f *, const vsip\_cmview\_f *);\\
void vsip\_cvkron\_d(vsip\_cscalar\_d, const vsip\_cvview\_d *,\\*\hspace{.6cm}const vsip\_cvview\_d *, const vsip\_cmview\_d *);\\
void vsip\_cvkron\_f(vsip\_cscalar\_f, const vsip\_cvview\_f *,\\*\hspace{.6cm}const vsip\_cvview\_f *, const vsip\_cmview\_f *);\\
void vsip\_mkron\_d(vsip\_scalar\_d, const vsip\_mview\_d *,\\*\hspace{.6cm}const vsip\_mview\_d *, const vsip\_mview\_d *);\\
void vsip\_mkron\_f(vsip\_scalar\_f, const vsip\_mview\_f *,\\*\hspace{.6cm}const vsip\_mview\_f *, const vsip\_mview\_f *);\\
void vsip\_vkron\_d(vsip\_scalar\_d, const vsip\_vview\_d *,\\*\hspace{.6cm}const vsip\_vview\_d *, const vsip\_mview\_d *);\\
void vsip\_vkron\_f(vsip\_scalar\_f, const vsip\_vview\_f *,\\*\hspace{.6cm}const vsip\_vview\_f *, const vsip\_mview\_f *);\\
\end{tabular}
}
\\\pyjvsiph
\viewmthd{No}{No}{No}{}
\apyfunc{yes}{out = kron(alpha,inOne,inTwo,out)}
\pyComment{
\item{The \ttbf{kron} function works much the same as the C VSIPL version except that a 
convenience pointer to the output view is returned. This must be done out-of-place.}
}
afunc{leq}{Computes the boolean comparison of “equal,” by element, of two views.}
\\\cvsiplh
\\\pyjvsiph
\afuncT{lge}{Computes the boolean comparison of “greater than or equal,” by element, of two views.}{logicalOperations}
\\\cvsiplh
\\\pyjvsiph
\afuncT{lgt}{Logical Greater Than}{logicalOperations}
\\\cvsiplh
\\\pyjvsiph
\afuncT{linear}{Returns the mean value of all the elements of a view.}{unaryOperations}
\\\cvsiplh
\newline \hspace*{.8cm} \vspace*{.1cm} \textbf{Available Functions }
\newline \hspace*{1.1cm} {
\ttfamily
\begin{tabular}[H]{l}
void vsip\_minterp\_linear\_f ( const vsip\_vview\_f *, const vsip\_mview\_f *, vsip\_major, const vsip\_vview\_f *, const vsip\_mview\_f * ) ;\\
void vsip\_vinterp\_linear\_f ( const vsip\_vview\_f *, const vsip\_vview\_f *, const vsip\_vview\_f *, const vsip\_vview\_f * ) ;\\
void vsip\_minterp\_linear\_d ( const vsip\_vview\_d *, const vsip\_mview\_d *, vsip\_major, const vsip\_vview\_d *, const vsip\_mview\_d * ) ;\\
void vsip\_vinterp\_linear\_d( const vsip\_vview\_d *, const vsip\_vview\_d *, const vsip\_vview\_d *, const vsip\_vview\_d *);\\
\end{tabular}
}\\
\\\pyjvsiph
\viewmthd{Yes}{Yes}{NA}{msq=in.meansqval}
\apyfunc{No}{NA}
\pyComment{\item{There seemed to be no reason to include this as a separate function for \pyjv}}
\afuncT{lle}{Logical less than or equal}{logicalOperations}
\\\cvsiplh
\\\pyjvsiph
\afuncT{llsqsol}{Solve linear least squares problem.}{specialLinearSystemSolvers}
\\\cvsiplh
\\ \hspace*{.8cm} \vspace*{.1cm} \textbf{Available Functions }
\\ \hspace*{1.1cm} {
\ttfamily
\begin{tabular}[H]{l}
\end{tabular}
}
\\\pyjvsiph

\afunc{llt}{Logical less than}
\\\cvsiplh
\\\pyjvsiph
\afuncT{lne}{Logical not equal}{logicalOperations}
\\\cvsiplh
\afh
{
\ttfamily
\\\hspace*{.04\textwidth}\begin{tabular}[H]{l}
void vsip\_mlne\_d(\\*\hspace{1cm}const vsip\_mview\_d*, const vsip\_mview\_d*, const vsip\_mview\_bl*);\\
void vsip\_mlne\_f(\\*\hspace{1cm}const vsip\_mview\_f*, const vsip\_mview\_f*, const vsip\_mview\_bl*);\\
void vsip\_svlne\_f(\\*\hspace{1cm}vsip\_scalar\_f, const vsip\_vview\_f*, const vsip\_vview\_bl*);\\
void vsip\_svlne\_d(\\*\hspace{1cm}vsip\_scalar\_d, const vsip\_vview\_d*, const vsip\_vview\_bl*);\\
void vsip\_vlne\_d(\\*\hspace{1cm}const vsip\_vview\_d*, const vsip\_vview\_d*, const vsip\_vview\_bl*);\\
void vsip\_vlne\_f(\\*\hspace{1cm}const vsip\_vview\_f*, const vsip\_vview\_f*, const vsip\_vview\_bl*);\\
void vsip\_vlne\_i(\\*\hspace{1cm}const vsip\_vview\_i*, const vsip\_vview\_i*, const vsip\_vview\_bl*);\\
void vsip\_vlne\_si(\\*\hspace{1cm}const vsip\_vview\_si*, const vsip\_vview\_si*, const vsip\_vview\_bl*);\\
void vsip\_vlne\_uc(\\*\hspace{1cm}const vsip\_vview\_uc*, const vsip\_vview\_uc*, const vsip\_vview\_bl*);\\
\end{tabular}
}
\\\pyjvsiph
\viewmthd{Yes}{No}{No}{out=in.lne(arg)}
\apyfunc{Yes}{out = lne(in1,in2,out)}
\pyComment{
\item{For return value \ttbf{out} precision is of type \ttbf{\_bl} and shape is compliant with input \ttbf{view}s.}
\item{For method check is \ttbf{in != arg} and for function check is \ttbf{in1 != in2}.}
\item{For functions input \ttbf{in1} may be scalar for precisions of type \ttbf{\_f} and \ttbf{\_d} with vector shape. Other options are defined but have not been done for \jv{} at this time. The function will complain if comparisons are not available.}
\item{For methods if input \ttbf{arg} is a scalar it is converted to a constant vector with one element, stride of zero, and compliant length before the comparison. For this reason the \ttbf{view} method will work for some cases the function does not.}
}
\afuncT{log10}{Compute the base ten logarithm; An element-wise function.}{elementaryMath}
\\\cvsiplh
\newline \hspace*{.8cm} \vspace*{.1cm} \textbf{Available Functions }
\newline \hspace*{1.1cm} {
\ttfamily
\begin{tabular}[H]{l}
vsip\_scalar\_d vsip\_log10\_d(vsip\_scalar\_d)\\
vsip\_scalar\_f vsip\_log10\_f(vsip\_scalar\_f)\\
void vsip\_mlog10\_d(\\*\hspace{1cm}const vsip\_mview\_d*, const vsip\_mview\_d*);\\
void vsip\_mlog10\_f(\\*\hspace{1cm}const vsip\_mview\_f*, const vsip\_mview\_f*);\\
void vsip\_vlog10\_d(\\*\hspace{1cm}const vsip\_vview\_d*, const vsip\_vview\_d*);\\
void vsip\_vlog10\_f(\\*\hspace{1cm}const vsip\_vview\_f*, const vsip\_vview\_f*);\\
\end{tabular}
}
\\\pyjvsiph
\viewmthd{yes}{yes}{yes}{inOut.sin}
\apyfunc{yes}{out = sin(in,out)}
\newline\hspace*{1.2cm}\parbox{10.8cm}{\vspace*{.1cm}The \ttbf{log10} function works much the same as the C VSIPL version except that a convenience pointer to the output view is returned. This may be done in-place if \ttbf{in==out}.}

\afuncT{log}{Natural logarithm; An element-wise function.}{elementaryMath}
\\\cvsiplh
\newline \hspace*{.8cm} \vspace*{.1cm} \textbf{Available Functions }
\newline \hspace*{1.1cm} {
\ttfamily
\begin{tabular}[H]{l}
\end{tabular}
}
\\\pyjvsiph
\viewmthd{yes}{yes}{yes}{inOut.sin}
\apyfunc{yes}{out = sin(in,out)}
\newline\hspace*{1.2cm}\parbox{10.8cm}{\vspace*{.1cm}The \ttbf{log} function works much the same as the C VSIPL version except that a convenience pointer to the output view is returned. This may be done in-place if \ttbf{in==out}.}

\afuncT{lud}{Lower-Upper Decomposition Class.}{generalSquareSolver}
\\\cvsiplh 
\\ \hspace*{.8cm} \vspace*{.1cm} \textbf{Available Functions }
%
\\ \hspace*{.9cm} {
\ttfamily\vspace{.3cm}
\begin{tabular}{|l|}
\multicolumn{1}{c}{\rmfamily \bfseries Create LU Object\vspace{.1cm}}\\ \hline
vsip\_lu\_d* vsip\_lud\_create\_d(vsip\_length);\\
vsip\_lu\_f* vsip\_lud\_create\_f(vsip\_length);\\
vsip\_clu\_d* vsip\_clud\_create\_d(vsip\_length);\\
vsip\_clu\_f* vsip\_clud\_create\_f(vsip\_length);\\
\hline\end{tabular}\\}
%
%\\ \hspace*{.8cm} \vspace*{.1cm} \texttt{lud\_destroy}
\hspace*{.9cm} {
\ttfamily\vspace{.3cm}
\begin{tabular}{|l|}
\multicolumn{1}{c}{\rmfamily \bfseries Destroy LU Object\vspace{.1cm}}\\ \hline
int vsip\_lud\_destroy\_d(vsip\_lu\_d*);\\
int vsip\_lud\_destroy\_f(vsip\_lu\_f*);\\
int vsip\_clud\_destroy\_d(vsip\_clu\_d*);\\
int vsip\_clud\_destroy\_f(vsip\_clu\_f*);\\
\hline\end{tabular}\\}
\hspace*{.9cm}{
\ttfamily\vspace{.3cm}
\begin{tabular}{|l|}
\multicolumn{1}{c}{\rmfamily \bfseries Calculate LU Decomposition\vspace{.1cm}}\\ \hline
int vsip\_lud\_d(vsip\_lu\_d*, const vsip\_mview\_d*);\\
int vsip\_lud\_f(vsip\_lu\_f*, const vsip\_mview\_f*);\\
int vsip\_clud\_d(vsip\_clu\_d*, const vsip\_cmview\_d*);\\
int vsip\_clud\_f(vsip\_clu\_f*, const vsip\_cmview\_f*);\\
\hline\end{tabular}\\}
%
\hspace*{.9cm}{
\ttfamily\vspace{.3cm}
\begin{tabular}{|l|}
\multicolumn{1}{c}{\rmfamily \bfseries Solve Using Calculated LU Decomposition\vspace{.3cm}}\\ \hline
int vsip\_lusol\_d(\\
\hspace*{3.cm}const vsip\_lu\_d*, vsip\_mat\_op, const vsip\_mview\_d*);\\
int vsip\_lusol\_f(\\
\hspace*{3.cm}const vsip\_lu\_f*, vsip\_mat\_op, const vsip\_mview\_f*);\\
int vsip\_clusol\_d(\\
\hspace*{3.cm}const vsip\_clu\_d*, vsip\_mat\_op, const vsip\_cmview\_d*);\\
int vsip\_clusol\_f(\\
\hspace*{3.cm}const vsip\_clu\_f*, vsip\_mat\_op, const vsip\_cmview\_f*);\\
\hline\end{tabular}\\}
%
\hspace*{.8cm}{
\ttfamily\vspace{.3cm}
\begin{tabular}{|l|}
\multicolumn{1}{c}{\rmfamily \bfseries Fill LU Attribute Structure\vspace{.1cm}}\\ \hline
void vsip\_lud\_getattr\_d(const vsip\_lu\_d*, vsip\_lu\_attr\_d*);\\
void vsip\_lud\_getattr\_f(const vsip\_lu\_f*, vsip\_lu\_attr\_f*);\\
void vsip\_clud\_getattr\_d(const vsip\_clu\_d*, vsip\_clu\_attr\_d*);\\
void vsip\_clud\_getattr\_f(const vsip\_clu\_f*, vsip\_clu\_attr\_f*);\\
\hline\end{tabular}\\}
\pyjvsiph
\\ \hspace*{.8cm}{\textbf{View Methods\vspace{.2cm}}\\
\hspace*{1.1cm}\parbox{.9\textwidth}{
\begin{itemize}\raggedright
\item {Three \ttbf{view} methods have been defined for LU Decompostion.}
\subitem{\ttbf{luSolve{(xb)}} - Calculate an in-place solution to $A\vec{x}=\vec{b}$ or $A X = B$.\Bs}
\subitem{\ttbf{luInv} - A method to obtain a matrix inverse using LU for computation.}
\subitem{\ttbf{lu} - A method to obtain a computed LU object from a matrix.\Bs}
\item{\Ts Methods \ttbf{lu} and \ttbf{luInv} are defined as properties.}
\item{LU decomposition will overwrite the input matrix so use a copy to preserve the calling view.}
\end{itemize}\vspace{2mm}}}\\
\hspace*{1.1cm}\textbf{Example: }\vspace*{.1cm}\\
\hspace*{1.9cm}\parbox{.85\textwidth}{\raggedright \Ts Given a square data matrix \ttbf{view} \ttbf{A} and a known 
matrix or (column) vector \ttbf{B} solve for unknown \ttbf{X} in $A X = B$.\\*
\hspace*{1cm}\ttbf{A.luSolve(B)} \\*
Done in-place. Matrix \ttbf{B} contains the answer \ttbf{X}}.\\
\hspace*{.8cm}{\textbf{LU Class Methods\vspace*{.2cm}}\\
\hspace*{1.cm}\parbox{.9\textwidth}{\raggedright To create an \ttbf{LU} object do\\
\hspace*{1.cm}\ttbf{luObj = LU(t,size)} \\
Where \ttbf{t} is a string indicating the type of \ttbf{LU} object to create and \ttbf{size} is the size of the square matrix the \ttbf{LU} object will decompose.\\
For class methods table we assume we have created an LU object we call \ttbf{luObj} and we have an input matrix \ttbf{view A} compliant with \ttbf{luObj} and some compliant \ttbf{view B} where we want to solve for the unknown \ttbf{X} in $A X = B $.\vspace{.2cm}}\\
\begin{table}
\caption{Flags and Types for LU Decomposition}
\begin{center}\begin{tabular}{|l|l|}
\multicolumn{2}{c}{\Ts\parbox[t]{.6\textwidth}{\center{\rmfamily \bfseries LU Decomposition Types}}}\Bs\\\hline
'lu\_d' & Real \ttbf{LU}; double precision \Bs\\\hline
'lu\_f' & Real \ttbf{LU}; float precision\Bs\\\hline
'clu\_d' & Complex \ttbf{LU}; double precision\Bs\\\hline
'clu\_f' & Complex \ttbf{LU}; float precision\Bs\\\hline
\end{tabular}
\begin{tabular}{|l|l|}
\multicolumn{2}{c}{\parbox[t]{.6\textwidth}{\center{\rmfamily \bfseries Matrix Operator Flags (\ttbf{op})}}}\Bs\\\hline
\Ts'NTRANS' or \ttbf{VSIP\_MAT\_NTRANS} & No Transpose operator\Bs\\\hline
   'TRANS' or \ttbf{VSIP\_MAT\_TRANS} & Transpose operator\Bs\\\hline
   'HERM' or \ttbf{VSIP\_MAT\_HERM} & Hermitian operator\Bs\\\hline
 \end{tabular}\end{center}\end{table}
\hspace*{1.cm}\parbox[t]{.9\textwidth}{\begin{tabular}{|l|l|}
\multicolumn{2}{c}{\parbox[t]{.8\textwidth}{\center{\rmfamily \bfseries LU Decomposition Methods\vspace{.2cm}}}\Bs} \\ \hline
\ttbf{luObj.lud(A)} & \parbox[t]{.6\textwidth}{\raggedright Decompose matrix \ttbf{A}. The decomposition is stored in the \ttbf{luObj} but \ttbf{view A} may be overwritten so use a copy if you want to preserve \ttbf{A}\Bs}\\\hline
\ttbf{luObj.solve(op,XB)} & \parbox[t]{.6\textwidth}{\raggedright Solve problem $\text{op}(A) X = B$ in-place where \ttbf{XB} is input/output \ttbf{view} and \ttbf{op} is matrix operator flag.\Bs}\\\hline
\ttbf{luObj.size} & \parbox[t]{.6\textwidth}{\raggedright Property. Returns integer length size of square matrix \ttbf{LU} object will decompose.\Bs}\\\hline
\ttbf{luObj.singular} & \parbox[t]{.6\textwidth}{\raggedright Property. Returns \ttbf{True} if singular; \ttbf{False} if inverse exists.\Bs}\\\hline
\ttbf{luObj.type} & \parbox[t]{.6\textwidth}{\raggedright Returns string indicating LU type.\Bs}\\\hline
\ttbf{luObj.vsip} & \parbox[t]{.6\textwidth}{\raggedright Returns C VSIPL LU instance.\vspace*{.1cm}\Bs}\\\hline
\end{tabular}\vspace*{.4cm}}
\hspace*{1.1cm}\textbf{Example: }\vspace*{.1cm}\\
\hspace*{1.9cm}\parbox{.85\textwidth}{\raggedright
\Ts Given a square data matrix \ttbf{view} \ttbf{A} and a known 
matrix \ttbf{B} solve for unknown \ttbf{X} in $A X = B$. \\*
\hspace*{1cm}\ttbf{asize=A.rowlength; t=A.type} \\*
\hspace*{1cm}\ttbf{luType=\{'mview\_d':'lu\_d','mview\_f':'lu\_f',}\\*
\hspace*{2.1cm}\ttbf{'cmview\_d':'clu\_d','cmview\_f':'clu\_f'\}} \\*
\hspace*{1cm}\ttbf{assert A.collength==asize,'Matrix must be square'}\\*
\hspace*{1cm}\ttbf{assert t in luType,'Matrix type not supported for LU decomposition'}\\*
\hspace*{1cm}\ttbf{luObj=LU(luType[t],asize)} \\*
\hspace*{1cm}\ttbf{luObj.lud(A)} \\*
\hspace*{1cm}\ttbf{luObj.solve('NTRANS',B)} \\*
\hspace*{1cm}
Done in-place. Matrix \ttbf{B} contains the answer \ttbf{X}.}
\begin{minted}{python}
asize=A.rowlength; t=A.type
luType={'mview_d':'lu_d','mview_f','lu_f','cmview_d':'clu_d','cmview_f':'clu_f'}
assert A.collength==asize,'Matrix must be square'
assert t in luType,'Matrix type not supported for LU decomposition'
luObj=LU(luType[t],asize)
luObj.lud(A)
luObj.solve('NTRANS',B)
\end{minted}

\afuncT{ma}{Multiply and add. An element-wise function.}{ternaryOperations}
\\\cvsiplh
\afh
{
\ttfamily
\\\hspace*{.04\textwidth}\begin{tabular}[H]{l}
void vsip\_cvma\_d(const vsip\_cvview\_d*, const vsip\_cvview\_d*,\\*\hspace{.7cm}const vsip\_cvview\_d*, const vsip\_cvview\_d*);\\
void vsip\_cvma\_f(const vsip\_cvview\_f*, const vsip\_cvview\_f*,\\*\hspace{.7cm}const vsip\_cvview\_f*, const vsip\_cvview\_f*);\\
void vsip\_cvsma\_d(const vsip\_cvview\_d*, vsip\_cscalar\_d,\\*\hspace{.7cm}const vsip\_cvview\_d*, const vsip\_cvview\_d*);\\
void vsip\_cvsma\_f(const vsip\_cvview\_f*, vsip\_cscalar\_f,\\*\hspace{.7cm}const vsip\_cvview\_f*, const vsip\_cvview\_f*);\\
void vsip\_vma\_d(const vsip\_vview\_d*, const vsip\_vview\_d*,\\*\hspace{.7cm}const vsip\_vview\_d*, const vsip\_vview\_d*);\\
void vsip\_vma\_f(const vsip\_vview\_f*, const vsip\_vview\_f*,\\*\hspace{.7cm}const vsip\_vview\_f*, const vsip\_vview\_f*);\\
void vsip\_vsma\_d(const vsip\_vview\_d*, vsip\_scalar\_d,\\*\hspace{.7cm}const vsip\_vview\_d*, const vsip\_vview\_d*);\\
void vsip\_vsma\_f(const vsip\_vview\_f*, vsip\_scalar\_f,\\*\hspace{.7cm}const vsip\_vview\_f*, const vsip\_vview\_f*);\\
void vsip\_cvsmsa\_d(const vsip\_cvview\_d*, vsip\_cscalar\_d,\\*\hspace{.7cm}vsip\_cscalar\_d, const vsip\_cvview\_d*);\\
void vsip\_cvsmsa\_f(const vsip\_cvview\_f*, vsip\_cscalar\_f,\\*\hspace{.7cm}vsip\_cscalar\_f, const vsip\_cvview\_f*);\\
void vsip\_vsmsa\_d(const vsip\_vview\_d*, vsip\_scalar\_d,\\*\hspace{.7cm}vsip\_scalar\_d, const vsip\_vview\_d*);\\
void vsip\_vsmsa\_f(const vsip\_vview\_f*, vsip\_scalar\_f,\\*\hspace{.7cm}vsip\_scalar\_f, const vsip\_vview\_f*);\\
\end{tabular}
}
\pyComment{\item{The C VSIPL spec has separate man pages for multiply-add functions containing scalar arguments, and those containing only \ttbf{view} arguments.}}
\\\pyjvsiph
\viewmthd{No}{NA}{NA}{NA}
\apyfunc{yes}{\ttbf{out = ma(in1,in2,in3,out)}}
\pyComment{\item{Argument \ttbf{in1} is always a \ttbf{view}, argument \ttbf{in2} is either a \ttbf{view} or a scalar and argument \ttbf{in3} is either a \ttbf{view} or a scalar.}
\item{The \ttbf{ma} function works much the same as the C VSIPL version except that a convenience pointer to the output \ttbf{view} is returned.}
\item{This may be done in-place if an input \ttbf{view} is the same as the output \ttbf{view}.}}
\afunc{mag}{Arctangent of Two Arguments; An elementwise function}
\cvsiplh
\pyjvsiph
\afuncT{magsq}{Arctangent of Two Arguments; An elementwise function}{unaryOperations}
\\\cvsiplh
\afh
\\\hspace*{.04\textwidth} {
\ttfamily
}
\\\pyjvsiph
\afuncT{max}{Maximum}{selectionOperations}
\\\cvsiplh
\afh
{
\ttfamily
\\\hspace*{.04\textwidth}\begin{tabular}[H]{l}
\end{tabular}
}
\\\pyjvsiph
 afunc{maxmg}{Maximum Magnitude}
\\\cvsiplh
\\\pyjvsiph
\afuncT{cmaxmgsq}{Maximum Magnitude of the Square}{selectionOperations}
\\\cvsiplh
\afh
{
\ttfamily
\\\hspace*{.04\textwidth}\begin{tabular*}{.92\textwidth}[H]{l}
void vsip\_mcmaxmgsq\_d(\\*\hspace*{1cm}const vsip\_cmview\_d*, const vsip\_cmview\_d*, const vsip\_mview\_d*);\\
void vsip\_mcmaxmgsq\_f(\\*\hspace*{1cm}const vsip\_cmview\_f*, const vsip\_cmview\_f*, const vsip\_mview\_f*);\\
void vsip\_vcmaxmgsq\_d(\\*\hspace*{1cm}const vsip\_cvview\_d*, const vsip\_cvview\_d*, const vsip\_vview\_d*);\\
void vsip\_vcmaxmgsq\_f(\\*\hspace*{1cm}const vsip\_cvview\_f*, const vsip\_cvview\_f*, const vsip\_vview\_f*);\\
\end{tabular*}
}
\\\pyjvsiph
%
\viewmthd{No}{NA}{NA}{NA}
%
\apyfunc{No}{NA}
%
\pyComment{
\item{No comments}
}
\afuncT{cmaxmgsqval}{Maximum Magnitude Square Value}{selectionOperations}
\\\cvsiplh
\afh
{
\ttfamily
\\\hspace*{.04\textwidth}\begin{tabular}[H]{l}
vsip\_scalar\_d vsip\_mcmaxmgsqval\_d(const vsip\_cmview\_d*, vsip\_scalar\_mi*);\\
vsip\_scalar\_d vsip\_vcmaxmgsqval\_d(const vsip\_cvview\_d*, vsip\_index*);\\
vsip\_scalar\_f vsip\_mcmaxmgsqval\_f(const vsip\_cmview\_f*, vsip\_scalar\_mi*);\\
vsip\_scalar\_f vsip\_vcmaxmgsqval\_f(const vsip\_cvview\_f*, vsip\_index *);\\
\end{tabular}
}
\\\pyjvsiph
%
\viewmthd{No}{NA}{NA}{NA}
%
\apyfunc{No}{NA}
%
\pyComment{
\item{No comments}
}

\afuncT{maxmgval}{Maximum Magnitude Value}{selectionOperations}
\\\cvsiplh
\afh
{
\ttfamily
\\\hspace*{.04\textwidth}\begin{tabular}[H]{l}
vsip\_scalar\_d vsip\_mmaxmgval\_d(const vsip\_mview\_d*, vsip\_scalar\_mi*);\\
vsip\_scalar\_d vsip\_vmaxmgval\_d(const vsip\_vview\_d*, vsip\_index *);\\
vsip\_scalar\_f vsip\_mmaxmgval\_f(const vsip\_mview\_f*, vsip\_scalar\_mi*);\\
vsip\_scalar\_f vsip\_vmaxmgval\_f(const vsip\_vview\_f*, vsip\_index *);\\
\end{tabular}
}
\\\pyjvsiph
%
\viewmthd{No}{NA}{NA}{NA}
%
\apyfunc{No}{NA}
%
\pyComment{
\item{No comments}
}
 afunc{maxval}{Maximum Value}
\\\cvsiplh
\\\pyjvsiph
\afuncT{meansqval}{Returns the mean value of all the elements of a view.}{unaryOperations}
\\\cvsiplh
\\ \hspace*{.8cm} \vspace*{.1cm} \textbf{Available Functions }
\\ \hspace*{1.1cm} {
\ttfamily
\begin{tabular}[H]{l}
vsip\_scalar\_d vsip\_cmmeansqval\_d(const vsip\_cmview\_d*);\\
vsip\_scalar\_d vsip\_cvmeansqval\_d(const vsip\_cvview\_d*);\\
vsip\_scalar\_d vsip\_mmeansqval\_d(const vsip\_mview\_d*);\\
vsip\_scalar\_d vsip\_vmeansqval\_d(const vsip\_vview\_d*);\\
vsip\_scalar\_f vsip\_cmmeansqval\_f(const vsip\_cmview\_f*);\\
vsip\_scalar\_f vsip\_cvmeansqval\_f(const vsip\_cvview\_f*);\\
vsip\_scalar\_f vsip\_mmeansqval\_f(const vsip\_mview\_f*);\\
vsip\_scalar\_f vsip\_vmeansqval\_f(const vsip\_vview\_f*);\\
\end{tabular}
}
\\\pyjvsiph
\viewmthd{Yes}{Yes}{NA}{msq=in.meansqval}
\apyfunc{No}{NA}
\pyComment{\item{There seemed to be no reason to include this as a separate function for \pyjv}}
\afunc{meanval}{Arctangent of Two Arguments; An elementwise function}
\cvsiplh
\pyjvsiph
\afuncT{min}{Minimum}{selectionOperations}
\\\cvsiplh
\afh
{
\ttfamily
\\\hspace*{.04\textwidth}
\begin{tabular}{l}
vsip\_scalar\_d vsip\_min\_d(vsip\_scalar\_d,vsip\_scalar\_d);\\
vsip\_scalar\_f vsip\_min\_f(vsip\_scalar\_f,vsip\_scalar\_f);\\
void vsip\_mmin\_d(const vsip\_mview\_d*, const vsip\_mview\_d*, const vsip\_mview\_d*);\\
void vsip\_mmin\_f(const vsip\_mview\_f*, const vsip\_mview\_f*, const vsip\_mview\_f*);\\
void vsip\_vmin\_d(const vsip\_vview\_d*, const vsip\_vview\_d*, const vsip\_vview\_d*);\\
void vsip\_vmin\_f(const vsip\_vview\_f*, const vsip\_vview\_f*, const vsip\_vview\_f*);\\
\end{tabular}
}
\\\pyjvsiph
%
\viewmthd{No}{NA}{NA}{NA}
%
\apyfunc{No}{NA}
%
\pyComment{
\item{No comments}
}
\afuncT{minmg}{Minimum Magnitude}{selectionOperations}
\\\cvsiplh
\afh
{
\ttfamily
\\\hspace*{.04\textwidth}\begin{tabular}[H]{l}
void vsip\_mminmg\_d(const vsip\_mview\_d*, const vsip\_mview\_d*, const vsip\_mview\_d*);\\
void vsip\_mminmg\_f(const vsip\_mview\_f*, const vsip\_mview\_f*, const vsip\_mview\_f*);\\
void vsip\_vminmg\_d(const vsip\_vview\_d*, const vsip\_vview\_d*, const vsip\_vview\_d*);\\
void vsip\_vminmg\_f(const vsip\_vview\_f*, const vsip\_vview\_f*, const vsip\_vview\_f*);\\
\end{tabular}
}
\\\pyjvsiph
%
\viewmthd{No}{NA}{NA}{NA}
%
\apyfunc{No}{NA}
%
\pyComment{
\item{No comments}
}
\afuncT{cminmgsq}{Complex Minimum Magnitude Square}{selectionOperations}
\\\cvsiplh
\afh
{
\ttfamily
\\\hspace*{.04\textwidth}\begin{tabular}{l}
void vsip\_mcminmgsq\_d(\fbrk{}const vsip\_cmview\_d*, const vsip\_cmview\_d*, const vsip\_mview\_d*);\\
void vsip\_mcminmgsq\_f(\fbrk{}const vsip\_cmview\_f*, const vsip\_cmview\_f*, const vsip\_mview\_f*);\\
void vsip\_vcminmgsq\_d(\fbrk{}const vsip\_cvview\_d*, const vsip\_cvview\_d*, const vsip\_vview\_d*);\\
void vsip\_vcminmgsq\_f(\fbrk{}const vsip\_cvview\_f*, const vsip\_cvview\_f*, const vsip\_vview\_f*);\\
\end{tabular}
}
\\\pyjvsiph
%
\viewmthd{No}{NA}{NA}{NA}
%
\apyfunc{No}{NA}
%
\pyComment{
\item{No comments}
}
\afuncT{cminmgsqval}{Minimum Magnitude Square Value}{selectionOperations}
\\\cvsiplh
\afh
{
\ttfamily
\\\hspace*{.04\textwidth}\begin{tabular}[H]{l}
vsip\_scalar\_d vsip\_mcminmgsqval\_d(const vsip\_cmview\_d*, vsip\_scalar\_mi*);\\
vsip\_scalar\_d vsip\_vcminmgsqval\_d(const vsip\_cvview\_d*, vsip\_index*);\\
vsip\_scalar\_f vsip\_mcminmgsqval\_f(const vsip\_cmview\_f*, vsip\_scalar\_mi*);\\
vsip\_scalar\_f vsip\_vcminmgsqval\_f(const vsip\_cvview\_f*, vsip\_index *);\\
\end{tabular}
}
\\\pyjvsiph
%
\viewmthd{No}{NA}{NA}{NA}
%
\apyfunc{No}{NA}
%
\pyComment{
\item{No comments}
}
\afuncT{minmgval}{Minimum Magnitude Value}{selectionOperations}
\\\cvsiplh
\\\pyjvsiph
\afuncT{minval}{Minimum Value}{selectionOperations}
\\\cvsiplh
\\\pyjvsiph
\afunc{modulate}{Arctangent of Two Arguments; An elementwise function}
\cvsiplh
\pyjvsiph
\afunc{msb}{Multiply and subtract. An element-wise function. See ternary functions table \ref{tab:ternaryOperations}.}
\cvsiplh
\newline \hspace*{.8cm} \vspace*{.1cm} \textbf{Available Functions }
\newline \hspace*{1.1cm} {
\ttfamily
\begin{tabular}[H]{l}
void vsip\_cvmsb\_d(const vsip\_cvview\_d*, const vsip\_cvview\_d*,\\*\hspace{.7cm}const vsip\_cvview\_d*, const vsip\_cvview\_d*);
void vsip\_cvmsb\_f(const vsip\_cvview\_f*, const vsip\_cvview\_f*,\\*\hspace{.7cm}const vsip\_cvview\_f*, const vsip\_cvview\_f*); 
void vsip\_vmsb\_d(const vsip\_vview\_d*, const vsip\_vview\_d*,\\*\hspace{.7cm}const vsip\_vview\_d*, const vsip\_vview\_d*); 
void vsip\_vmsb\_f(const vsip\_vview\_f*, const vsip\_vview\_f*,\\*\hspace{.7cm}const vsip\_vview\_f*, const vsip\_vview\_f*); 
\end{tabular}
}
\pyjvsiph
\viewmthd{No}{NA}{NA}{NA}
\apyfunc{yes}{out = msb(in1,in2,in3,out)}
\newline\hspace*{1.2cm}\parbox{10.8cm}{\vspace*{.1cm}The \ttbf{log} function works much the same as the C VSIPL version except that a convenience pointer to the output view is returned. This may be done in-place if one of the input views is the same as the output view.}
\afunc{mul}{computes the difference of a scalar and a \ttbf{view} or between two \ttbf{view}s. A binary operation. See table \ref{tab:binaryOperations}.}
\\\cvsiplh
\\\pyjvsiph
\afuncT{nearest}{Nearest Interpolation}{interpolation}
\\\cvsiplh
\newline \hspace*{.8cm} \vspace*{.1cm} \textbf{Available Functions }
\newline \hspace*{1.1cm} {
\ttfamily
\begin{tabular}[H]{l}
void vsip\_minterp\_nearest\_f ( const vsip\_vview\_f *, const vsip\_mview\_f *, vsip\_major, const vsip\_vview\_f *, const vsip\_mview\_f * ) ;\\
void vsip\_vinterp\_nearest\_f ( const vsip\_vview\_f *, const vsip\_vview\_f *, const vsip\_vview\_f *, const vsip\_vview\_f * ) ;\\
void vsip\_vinterp\_nearest\_d( const vsip\_vview\_d *, const vsip\_vview\_d *, const vsip\_vview\_d *, const vsip\_vview\_d *);\\
void vsip\_minterp\_nearest\_d ( const vsip\_vview\_d*, const vsip\_mview\_d*, vsip\_major, const vsip\_vview\_d*, const vsip\_mview\_d* ) ;\\
\end{tabular}
}\\
\\\pyjvsiph
\viewmthd{Yes}{Yes}{NA}{msq=in.meansqval}
\apyfunc{No}{NA}
\pyComment{\item{There seemed to be no reason to include this as a separate function for \pyjv}}
\afunc{neg}{Arctangent of Two Arguments; An elementwise function}
\cvsiplh
\pyjvsiph
\afuncT{not}{Boolean or bitwise "NOT" operation for integer and boolean views.}{bitwiseOperations}
\\\cvsiplh
\afh
{
\ttfamily
\\\hspace*{.04\textwidth}\begin{tabular}[H]{l}
\end{tabular}
}
\\\pyjvsiph
\afuncT{or}{Boolean or bitwise "OR" operation for integer and boolean views.}{bitwiseOperations}
\\\cvsiplh
\\\pyjvsiph
\afuncT{outer}{Vector outer product.}{matrixOperations}
\\\cvsiplh
\begin{cfuncs}
void vsip\_cvouter\_d(vsip\_cscalar\_d, const vsip\_cvview\_d *,\\*\hspace{.6cm}const vsip\_cvview\_d *, const vsip\_cmview\_d *);\\
void vsip\_cvouter\_f(vsip\_cscalar\_f, const vsip\_cvview\_f *,\\*\hspace{.6cm}const vsip\_cvview\_f *, const vsip\_cmview\_f *);\\
void vsip\_vouter\_d(vsip\_scalar\_d, const vsip\_vview\_d*,\\*\hspace{.6cm}const vsip\_vview\_d*, const vsip\_mview\_d*);\\
void vsip\_vouter\_f(vsip\_scalar\_f, const vsip\_vview\_f*,\\*\hspace{.6cm}const vsip\_vview\_f*, const vsip\_mview\_f*);\\
\end{cfuncs}
\pyjvsiph
\viewmthd{yes}{no}{no}{out=inOne.outer(inTwo) or inOne.outer(aScalar,inTwo)}
\apyfunc{yes}{out = outer(aScalar,inOne,inTwo,out)}
\begin{comments}
\item{The \ttbf{inOne} will be the first (column) vector and \ttbf{inTwo} will be the second (row) vector.}
\item{The element \ttbf{aScalar} is a constant multiplier times the resultant outer product. For the view method if no constant is supplied it is assumed to be one}
\item{The \ttbf{outer} function works much the same as the C VSIPL version except that a convenience pointer to the output view is returned.}
\item{The \ttbf{view} method will create an output view for you.}
\end{comments}
\afuncT{permute}{Permute a view}{permute}
\\\cvsiplh
\newline \hspace*{.8cm} \vspace*{.1cm} \textbf{Available Functions }
\newline \hspace*{1.1cm} {
\ttfamily
\begin{tabular}[H]{l}
void vsip\_mpermute\_d ( const vsip\_mview\_d*, const vsip\_permute*, const vsip\_mview\_d* ) ;\\
void vsip\_mpermute\_f ( const vsip\_mview\_f*, const vsip\_permute*, const vsip\_mview\_f* ) ;\\
void vsip\_mpermute\_once\_d ( const vsip\_mview\_d*, vsip\_major, const vsip\_vview\_vi*, const vsip\_mview\_d* ) ;\\
void vsip\_mpermute\_once\_f ( const vsip\_mview\_f*, vsip\_major, const vsip\_vview\_vi*, const vsip\_mview\_f* ) ;\\
void vsip\_cmpermute\_once\_d ( const vsip\_cmview\_d*, vsip\_major, const vsip\_vview\_vi*, const vsip\_cmview\_d* ) ;\\
void vsip\_cmpermute\_once\_f ( const vsip\_cmview\_f*, vsip\_major, const vsip\_vview\_vi*, const vsip\_cmview\_f* ) ;\\
void vsip\_permute\_destroy ( vsip\_permute* ) ;\\
vsip\_permute* vsip\_mpermute\_create\_d ( vsip\_length, vsip\_length, vsip\_major ) ;\\
vsip\_permute* vsip\_mpermute\_create\_f ( vsip\_length, vsip\_length, vsip\_major ) ;\\
vsip\_permute* vsip\_permute\_init ( vsip\_permute*, const vsip\_vview\_vi* ) ;\\
\end{tabular}
}\\
\\\pyjvsiph
\viewmthd{Yes}{Yes}{NA}{msq=in.meansqval}
\apyfunc{No}{NA}
\pyComment{\item{There seemed to be no reason to include this as a separate function for \pyjv}}
\afuncT{polar}{Polar}{elementGenerationOperations}
\\\cvsiplh
\\\pyjvsiph

\afuncT{prod3}{Special matrix product for 3 by 3 \ttbf{views}}{matrixOperations}
\\\cvsiplh
\\ \hspace*{.8cm} \vspace*{.1cm} \textbf{Available Functions }
\\ \hspace*{.03\textwidth} {
\ttfamily
\begin{tabular}{l}
void vsip\_cmprod3\_d(\\*\hspace{.6cm}const vsip\_cmview\_d*, const vsip\_cmview\_d*, const vsip\_cmview\_d*);\Bs\\
void vsip\_cmprod3\_f(\\*\hspace{.6cm}const vsip\_cmview\_f*, const vsip\_cmview\_f*, const vsip\_cmview\_f*);\Bs\\
void vsip\_cmvprod3\_d(\\*\hspace{.6cm}const vsip\_cmview\_d*, const vsip\_cvview\_d*, const vsip\_cvview\_d*);\Bs\\
void vsip\_cmvprod3\_f(\\*\hspace{.6cm}const vsip\_cmview\_f*, const vsip\_cvview\_f*, const vsip\_cvview\_f*);\Bs\\
void vsip\_mprod3\_d(\\*\hspace{.6cm}const vsip\_mview\_d*, const vsip\_mview\_d*, const vsip\_mview\_d*);\Bs\\
void vsip\_mprod3\_f(\\*\hspace{.6cm}const vsip\_mview\_f*, const vsip\_mview\_f*, const vsip\_mview\_f*);\Bs\\
void vsip\_mvprod3\_d(\\*\hspace{.6cm}const vsip\_mview\_d*, const vsip\_vview\_d*, const vsip\_vview\_d*);\Bs\\
void vsip\_mvprod3\_f(\\*\hspace{.6cm}const vsip\_mview\_f*, const vsip\_vview\_f*, const vsip\_vview\_f*);\Bs\\
\end{tabular}
}
\\\pyjvsiph
\viewmthd{yes}{No}{No}{out=inOne.prod(inTwo)}
\apyfunc{yes}{out = prod3(inOne,inTwo,out)}
\pyComment{
\item{The \ttbf{prod3} function works much the same as the C VSIPL version except that a
 convenience pointer to the output view is returned. This may not be done in-place.}
\item{There is no special \ttbf{prod3} method. The \ttbf{prod} method will select and 
use the \ttbf{prod3} C VSIPL routine if the conditions
exist to support it.}
}
\afuncT{prod4}{Special matrix product for 4 by 4 \ttbf{views}}{matrixOperations}
\\\cvsiplh
\\ \hspace*{.8cm} \vspace*{.1cm} \textbf{Available Functions }
\\ \hspace*{.03\textwidth} {
\ttfamily
\begin{tabular*}{.92\textwidth}[H]{l}
void vsip\_cmprod4\_d(\\*\hspace{.6cm}const vsip\_cmview\_d*, const vsip\_cmview\_d*, const vsip\_cmview\_d*);\Bs\\
void vsip\_cmprod4\_f(\\*\hspace{.6cm}const vsip\_cmview\_f*, const vsip\_cmview\_f*, const vsip\_cmview\_f*);\Bs\\
void vsip\_cmvprod4\_d(\\*\hspace{.6cm}const vsip\_cmview\_d*, const vsip\_cvview\_d*, const vsip\_cvview\_d*);\Bs\\
void vsip\_cmvprod4\_f(\\*\hspace{.6cm}const vsip\_cmview\_f*, const vsip\_cvview\_f*, const vsip\_cvview\_f*);\Bs\\
void vsip\_mprod4\_d(\\*\hspace{.6cm}const vsip\_mview\_d*, const vsip\_mview\_d*, const vsip\_mview\_d*);\Bs\\
void vsip\_mprod4\_f(\\*\hspace{.6cm}const vsip\_mview\_f*, const vsip\_mview\_f*, const vsip\_mview\_f*);\Bs\\
void vsip\_mvprod4\_d(\\*\hspace{.6cm}const vsip\_mview\_d*, const vsip\_vview\_d*, const vsip\_vview\_d*);\Bs\\
void vsip\_mvprod4\_f(\\*\hspace{.6cm}const vsip\_mview\_f*, const vsip\_vview\_f*, const vsip\_vview\_f*);\Bs\\
\end{tabular*}
}
\\\pyjvsiph
\viewmthd{yes}{No}{No}{out=inOne.prod(inTwo)}
\apyfunc{yes}{out = prod4(inOne,inTwo,out)}
\pyComment{
\item{The \ttbf{prod4} function works much the same as the C VSIPL version except that a
 convenience pointer to the output view is returned. This may not be done in-place.}
\item{There is no special \ttbf{prod4} method. The \ttbf{prod} method will select and 
use the \ttbf{prod4} C VSIPL routine if the conditions
exist to support it.}
}
\afuncT{prod}{Matrix product.}{matrixOperations}
\\\cvsiplh
\\ \hspace*{.8cm} \vspace*{.1cm} \textbf{Available Functions }
\\ \hspace*{0.03\textwidth} {
\ttfamily
\begin{tabular}[H]{l}
void vsip\_cmprod\_d(\\*\hspace{.6cm}const vsip\_cmview\_d*, const vsip\_cmview\_d*, const vsip\_cmview\_d*);\\ 
void vsip\_cmprod\_f(\\*\hspace{.6cm}const vsip\_cmview\_f*, const vsip\_cmview\_f*, const vsip\_cmview\_f*);\\ 
void vsip\_cmvprod\_d(\\*\hspace{.6cm}const vsip\_cmview\_d*, const vsip\_cvview\_d*, const vsip\_cvview\_d*);\\ 
void vsip\_cmvprod\_f(\\*\hspace{.6cm}const vsip\_cmview\_f*, const vsip\_cvview\_f*, const vsip\_cvview\_f*);\\ 
void vsip\_cvmprod\_d(\\*\hspace{.6cm}const vsip\_cvview\_d*, const vsip\_cmview\_d*, const vsip\_cvview\_d*);\\ 
void vsip\_cvmprod\_f(\\*\hspace{.6cm}const vsip\_cvview\_f*, const vsip\_cmview\_f*, const vsip\_cvview\_f*);\\ 
void vsip\_mprod\_d(\\*\hspace{.6cm}const vsip\_mview\_d*, const vsip\_mview\_d*, const vsip\_mview\_d*);\\ 
void vsip\_mprod\_f(\\*\hspace{.6cm}const vsip\_mview\_f*, const vsip\_mview\_f*, const vsip\_mview\_f*);\\ 
void vsip\_mvprod\_d(\\*\hspace{.6cm}const vsip\_mview\_d*, const vsip\_vview\_d*, const vsip\_vview\_d*);\\ 
void vsip\_mvprod\_f(\\*\hspace{.6cm}const vsip\_mview\_f*, const vsip\_vview\_f*, const vsip\_vview\_f*);\\ 
void vsip\_vmprod\_d(\\*\hspace{.6cm}const vsip\_vview\_d*, const vsip\_mview\_d*, const vsip\_vview\_d*);\\ 
void vsip\_vmprod\_f(\\*\hspace{.6cm}const vsip\_vview\_f*, const vsip\_mview\_f*, const vsip\_vview\_f*);\\ 
\end{tabular}
}
\\\pyjvsiph
\viewmthd{yes}{No}{No}{out=inOne.prod(inTwo)}
\apyfunc{yes}{out = prod(inOne,inTwo,out)}
\pyComment{
\item{The \ttbf{prod} function works much the same as the C VSIPL version except that a
 convenience pointer to the output view is returned. This may not be done in-place.}
}

\afuncT{prodh}{Matrix hermitian product.}{matrixOperations}
\\\cvsiplh
\\ \hspace*{.8cm} \vspace*{.1cm} \textbf{Available Functions }
\\ \hspace*{0.03\textwidth} {
\ttfamily
\begin{tabular}[H]{l}
void vsip\_cmprodh\_d(\\*\hspace{.6cm}
    const vsip\_cmview\_d*, const vsip\_cmview\_d*, const vsip\_cmview\_d*);\Bs\\
void vsip\_cmprodh\_f(\\*\hspace{.6cm}
    const vsip\_cmview\_f*, const vsip\_cmview\_f*, const vsip\_cmview\_f*);\Bs\\
\end{tabular}
}
\\\pyjvsiph
\viewmthd{yes}{No}{No}{out=inOne.prodh(inTwo)}
\apyfunc{yes}{out = prodh(inOne,inTwo,out)}
\pyComment{
\item{The \ttbf{prodh} function works much the same as the C VSIPL version except that a convenience pointer to the output view is returned. This may not be done in-place.}
\item{The \ttbf{prodh} method creates and returns a new \ttbf{view} with the result.}
}

\afuncT{prodj}{Matrix conjugate product.}{matrixOperations}
\\\cvsiplh
\\ \hspace*{.8cm} \vspace*{.1cm} \textbf{Available Functions }
\\ \hspace*{0.03\textwidth} {
\ttfamily
\begin{tabular}[H]{l}
void vsip\_cmprodj\_d(\\*\hspace{.6cm}
    const vsip\_cmview\_d*, const vsip\_cmview\_d*, const vsip\_cmview\_d*);\Bs\\
void vsip\_cmprodj\_f(\\*\hspace{.6cm}
    const vsip\_cmview\_f*, const vsip\_cmview\_f*, const vsip\_cmview\_f*);\Bs\\
\end{tabular}
}
\\\pyjvsiph
\viewmthd{yes}{No}{No}{out=inOne.prodj(inTwo)}
\apyfunc{yes}{out = prodj(inOne,inTwo,out)}
\pyComment{
\item{The \ttbf{prodj} function works much the same as the C VSIPL version except that a convenience pointer to the output view is returned. This may not be done in-place.}
\item{The \ttbf{prodj} method creates and returns a new \ttbf{view} with the result.}
}
\afuncT{prodt}{Matrix transpose product.}{matrixOperations}
\\\cvsiplh
\\ \hspace*{.8cm} \vspace*{.1cm} \textbf{Available Functions }
\\ \hspace*{0.03\textwidth} {
\ttfamily
\begin{tabular}[H]{l}
void vsip\_cmprodt\_d(\\*\hspace{.6cm}
    const vsip\_cmview\_d*, const vsip\_cmview\_d*, const vsip\_cmview\_d*);\\
void vsip\_cmprodt\_f(\\*\hspace{.6cm}
    const vsip\_cmview\_f*, const vsip\_cmview\_f*, const vsip\_cmview\_f*);\\
void vsip\_mprodt\_d(\\*\hspace{.6cm}
    const vsip\_mview\_d*, const vsip\_mview\_d*, const vsip\_mview\_d*);\\
void vsip\_mprodt\_f(\\*\hspace{.6cm}
    const vsip\_mview\_f*, const vsip\_mview\_f*, const vsip\_mview\_f*);\\
\end{tabular}
}
\\\pyjvsiph
\viewmthd{yes}{No}{No}{out=inOne.prodt(inTwo)}
\apyfunc{yes}{out = prodt(inOne,inTwo,out)}
\pyComment{
\item{The \ttbf{prodt} function works much the same as the C VSIPL version except that a convenience pointer to the output view is returned. This may not be done in-place.}
\item{The \ttbf{prodt} method creates and returns a new \ttbf{view} with the result.}
}
\afuncT{qrd}{QR Decomposition Class}{overDeterminedSolver}
\\\cvsiplh 
\\ \hspace*{.8cm} \vspace*{.1cm} \textbf{Available Functions }
%
%
\\ \hspace*{1.cm}{
\ttfamily\vspace{.3cm}
\begin{tabular}[H]{|l|}
\multicolumn{1}{c}{\rmfamily \bfseries Calculate QR Decomposition\vspace{.1cm}}\\ \hline\Ts
int vsip\_qrd\_d(vsip\_qr\_d*, const vsip\_mview\_d*);\Bs\\
int vsip\_qrd\_f(vsip\_qr\_f*, const vsip\_mview\_f*);\Bs\\
int vsip\_cqrd\_d(vsip\_cqr\_d*, const vsip\_cmview\_d*);\Bs\\
int vsip\_cqrd\_f(vsip\_cqr\_f*, const vsip\_cmview\_f*);\Bs\\
\hline\end{tabular}\\}
%
\hspace*{1.cm} {
\ttfamily\vspace{.3cm}
\begin{tabular}[H]{|l|}
\multicolumn{1}{c}{\rmfamily \bfseries Create QRD Object\vspace{.1cm}}\\ \hline\Ts
vsip\_qr\_d* vsip\_qrd\_create\_d(vsip\_length);\Bs\\
vsip\_qr\_f* vsip\_qrd\_create\_f(vsip\_length);\Bs\\
vsip\_cqr\_d* vsip\_cqrd\_create\_d(vsip\_length);\Bs\\
vsip\_cqr\_f* vsip\_cqrd\_create\_f(vsip\_length);\Bs\\
\hline\end{tabular}\\}
%
\\ \hspace*{1.cm} {
\ttfamily\vspace{.3cm}
\begin{tabular}[H]{|l|}
\multicolumn{1}{c}{\rmfamily \bfseries Destroy QRD Object\vspace{.1cm}}\\ \hline\Ts
int vsip\_qrd\_destroy\_d(vsip\_qr\_d*);\Bs\\
int vsip\_qrd\_destroy\_f(vsip\_qr\_f*);\Bs\\
int vsip\_cqrd\_destroy\_d(vsip\_cqr\_d*);\Bs\\
int vsip\_cqrd\_destroy\_f(vsip\_cqr\_f*);\Bs\\
\hline\end{tabular}\\}
%
\\ \hspace*{1.cm}{
\ttfamily\vspace{.3cm}
\begin{tabular}[H]{|l|}
\multicolumn{1}{c}{\rmfamily \bfseries Fill QRD Attribute Structure\vspace{.1cm}}\\ \hline\Ts
void vsip\_qrd\_getattr\_d(const vsip\_qr\_d*, vsip\_qr\_attr\_d*);\Bs\\
void vsip\_qrd\_getattr\_f(const vsip\_qr\_f*, vsip\_qr\_attr\_f*);\Bs\\
void vsip\_cqrd\_getattr\_d(const vsip\_cqr\_d*, vsip\_cqr\_attr\_d*);\Bs\\
void vsip\_cqrd\_getattr\_f(const vsip\_cqr\_f*, vsip\_cqr\_attr\_f*);\Bs\\
\hline\end{tabular}\\}
%
\\ \hspace*{1.cm}{
\ttfamily\vspace{.3cm}
\begin{tabular}[H]{|l|}
\multicolumn{1}{c}{\rmfamily \bfseries Product with Q from QR Decomposition\vspace{.1cm}}\\ \hline\Ts
int vsip\_qrdprodq\_d(const vsip\_qr\_d*, vsip\_mat\_op, vsip\_mat\_side,\\*\hspace*{1cm}const vsip\_mview\_d*);\Bs\\
int vsip\_qrdprodq\_f(const vsip\_qr\_f*, vsip\_mat\_op, vsip\_mat\_side,\\*\hspace*{1cm}const vsip\_mview\_f*);\Bs\\
int vsip\_cqrdprodq\_d(const vsip\_cqr\_d*, vsip\_mat\_op, vsip\_mat\_side,\\*\hspace*{1cm}const vsip\_cmview\_d*);\Bs\\
int vsip\_cqrdprodq\_f(const vsip\_cqr\_f*, vsip\_mat\_op, vsip\_mat\_side,\\*\hspace*{1cm}const vsip\_cmview\_f*);\Bs\\
\hline\end{tabular}\Bs\\}
%
%
\\ \hspace*{1.cm}{
\ttfamily\vspace{.3cm}
\begin{tabular}[H]{|l|}
\multicolumn{1}{c}{\rmfamily \bfseries Solve Linear System Based on R from QR Decomposition\vspace{.1cm}}\\ \hline\Ts
int vsip\_qrdsolr\_d(const vsip\_qr\_d *, vsip\_mat\_op opR, vsip\_scalar\_d,\\*\hspace*{1cm}const vsip\_mview\_d *);\Bs\\
int vsip\_qrdsolr\_f(const vsip\_qr\_f *, vsip\_mat\_op opR, vsip\_scalar\_f,\\*\hspace*{1cm}const vsip\_mview\_f *);\Bs\\
int vsip\_cqrdsolr\_d(const vsip\_cqr\_d *, vsip\_mat\_op opR, vsip\_cscalar\_d,\\*\hspace*{1cm}const vsip\_cmview\_d *);\Bs\\
int vsip\_cqrdsolr\_f(const vsip\_cqr\_f *, vsip\_mat\_op opR, vsip\_cscalar\_f,\\*\hspace*{1cm}const vsip\_cmview\_f *);\Bs\\
\hline\end{tabular}\Bs\\}
%
\\ \hspace*{1.cm}{
\ttfamily\vspace{.3cm}
\begin{tabular}[H]{|l|}
\multicolumn{1}{c}{\rmfamily \bfseries Use QRD to Solve a Covariance or LLSQ System\vspace{.1cm}}\\ \hline\Ts
int vsip\_qrsol\_d(const vsip\_qr\_d*, vsip\_qrd\_prob,\\*\hspace*{1cm}const vsip\_mview\_d*);\Bs\\
int vsip\_qrsol\_f(const vsip\_qr\_f*, vsip\_qrd\_prob,\\*\hspace*{1cm}const vsip\_mview\_f*);\Bs\\
int vsip\_cqrsol\_d(const vsip\_cqr\_d*, vsip\_qrd\\\*\hspace*{1cm}, const vsip\_cmview\_d*);\Bs\\
int vsip\_cqrsol\_f(const vsip\_cqr\_f*, vsip\_qrd\_prob,\\*\hspace*{1cm}const vsip\_cmview\_f*);\Bs\\
\hline\end{tabular}\Bs\\}
%
\\\pyjvsiph
%
\\ \hspace*{.8cm}{\textbf{View Methods\vspace{.2cm}}\\
\hspace*{1.1cm}\parbox{.9\textwidth}{
\begin{itemize}
\item {Two \ttbf{view} methods have been defined for QR Decompostion.}
\subitem{\ttbf{qrd} - A method to obtain matrices \ttbf{Q} and \ttbf{R} from a matrix.\Bs}
\subitem{\ttbf{qr} - A method to obtain a computed QR object from a matrix.\Bs}
\item{Vector \ttbf{view}s are treated as a matrix with a single column.\Bs}
\item{\Ts Both \ttbf{view} methods are defined as properties.}
\end{itemize}\vspace{2mm}}}\\
%%
\hspace*{1.1cm}\textbf{Example: }\vspace*{.1cm}\\
\hspace*{1.9cm}\parbox{.85\textwidth}{\Ts Assume matrix \ttbf{view} \ttbf{A} has been created and has data in it.}\\
\hspace*{1.9cm}\parbox{.8\textwidth}{\vspace{.3cm}\hspace*{1cm}\ttbf{qrObj=A.qr} \\*
will create a full \ttbf{QR} using flag \ttbf{VSIP\_QRD\_SAVEQ} and decompose the matrix.}\\
\hspace*{1.9cm}\parbox{.8\textwidth}{\vspace{.3cm}\hspace*{1cm}\ttbf{Q,R=A.qrd} \\*
will create matrix \ttbf{Q} and matrix \ttbf{R} using \ttbf{QR} and \ttbf{A} such that \ttbf{A = Q.prod(R)}}\\
%
\hspace*{.8cm}\textbf{QR Class\vspace{.2cm}}\\
\hspace*{1.1cm}\parbox{.9\textwidth}{To create an QR object use \\*
\hspace*{1.cm} \ttbf{qrObj=QR(t,m,n,qSave)}\\*
where:
\begin{itemize}
\item \ttbf{t} is a string indicating the QR type
\item \ttbf{m} is the column length
\item \ttbf{n} is the row length
\item \ttbf{qSave} is a flag indicating how much of the Q matrix is to be saved.
\end{itemize} \vspace{.2cm}}\\
%%
\begin{table}
\caption{Flags and Types for QR Decomposition}
\begin{center}\begin{tabular}{|l l|}
\multicolumn{2}{c}{\Ts\parbox[t]{.6\textwidth}{\center{\rmfamily \bfseries QR Decomposition Types}}}\Bs\\\hline
'qr\_d' & Real \ttbf{QR}; double precision \Bs\\\hline
'qr\_f' & Real \ttbf{QR}; float precision\Bs\\\hline
'cqr\_d' & Complex \ttbf{QR}; double precision\Bs\\\hline
'cqr\_f' & Complex \ttbf{QR}; float precision\Bs\\\hline
%%%
\multicolumn{2}{c}{\parbox[t]{.6\textwidth}{\center{\rmfamily \bfseries Q Save Flags}}}\Bs\\\hline
\Ts'NOSAVEQ' or \ttbf{VSIP\_QRD\_NOSAVEQ} & Do not save Q matrix\Bs\\\hline
'SAVEQ' or \ttbf{VSIP\_QRD\_SAVEQ} & Save full Q matrix\Bs\\\hline
'SAVEQ1' or \ttbf{VSIP\_QRD\_SAVEQ1} & Save skinny Q matrix\Bs\\\hline
%%%
\multicolumn{2}{c}{\parbox[t]{.6\textwidth}{\center{\rmfamily \bfseries QR Solve Problem Flags (\ttbf{prob})}}}\Bs\\\hline
\Ts'COV' or \ttbf{VSIP\_COV} & Solve Covariance Problem\Bs\\\hline
'LLS' or \ttbf{VSIP\_LLS} & Solve Linear Least Square Problem\Bs\\\hline
%%%
\multicolumn{2}{c}{\parbox[t]{.6\textwidth}{\center{\rmfamily \bfseries Matrix Operator Flags (\ttbf{op})}}}\Bs\\\hline
\Ts'NTRANS' or \ttbf{VSIP\_MAT\_NTRANS} & No Transpose operator\Bs\\\hline
   'TRANS' or \ttbf{VSIP\_MAT\_TRANS} & Transpose operator\Bs\\\hline
   'HERM' or \ttbf{VSIP\_MAT\_HERM} & Hermitian operator\Bs\\\hline
%%%
\multicolumn{2}{c}{\parbox[t]{.6\textwidth}{\center{\rmfamily \bfseries Side Flag}}}\Bs\\\hline
\Ts'LSIDE' or \ttbf{VSIP\_MAT\_LSIDE} & \ttbf{Q} matrix on the left side\Bs\\\hline
   'RSIDE' or \ttbf{VSIP\_MAT\_RSIDE} & \ttbf{Q} matrix on the right side\Bs\\\hline
 \end{tabular}\end{center}\end{table}
%%
\hspace*{1.1cm}{\textbf{QR Class Methods}\\
\hspace*{1.1cm} \parbox[t]{.9\textwidth}{For class methods table we assume we have created an QR object we call \ttbf{qrObj} and we have an input \ttbf{view A} compliant with \ttbf{firObj} and a compliant output \ttbf{view y}.\vspace{.2cm}}
\\
\hspace*{1.cm}\parbox[t]{.85\textwidth}{\begin{tabular}{|l l|}
\multicolumn{2}{c}{\parbox[t]{.58\textwidth}{\center{\rmfamily \bfseries QR Decomposition Methods\vspace{.2cm}}}}\Bs\\ \hline
%
\Ts\ttbf{qrObj.decompose(A)} & 
\parbox[t]{.58\textwidth}{Decompose \ttbf{A}. Results are in QR object}\Bs\\\hline
%
\ttbf{qrObj.args} & 
\parbox[t]{.58\textwidth}{Property. Return tuple consisting of input argument \ttbf{(m,n,qSave)} list for creation of QR}\Bs\\\hline
%
\ttbf{qrObj.qSize} & 
\parbox[t]{.58\textwidth}{Returns size of Q matrix represented by QR decomposition.}\Bs\\\hline
%
\ttbf{qrObj.size} & 
\parbox[t]{.58\textwidth}{Property. Returns size of input matrix to \ttbf{decompose} method \ttbf{(m,n)} \vspace*{.1cm}}\Bs\\\hline
%
\ttbf{qrObj.type} & 
\parbox[t]{.58\textwidth}{Property. Returns string indicating QR type.\vspace*{.1cm}}\Bs\\\hline
%
\ttbf{qrObj.vsip} & 
\parbox[t]{.58\textwidth}{Property. Returns C VSIPL QR instance.\vspace*{.1cm}}\\\hline
%
\ttbf{qrObj.solve(prob,XB)} & \parbox[t]{.58\textwidth}{Solve}\\\hline
%
\ttbf{qrObj.solveR(op,alpha,XB)} & \parbox[t]{.58\textwidth}{Solve using \ttbf{R} matrix from QR decompostion.\vspace*{.1cm}}\\\hline
%
\ttbf{qrObj.prodQ(op,side,XB)} &
 \parbox[t]{.58\textwidth}{Matrix product of \ttbf{Q} with \ttbf{X}.}\\\hline
%
\end{tabular}\vspace*{.4cm}}\\
\hspace*{.7cm} \parbox[t]{.91\textwidth}{
\begin{itemize}
\item{View \ttbf{XB} is an input-output view. Since the output may be of a different size than the input care must be used to use an input view with a block large enough to handle the expected output. Note for the solvers one would consider \ttbf{B} as the input and \ttbf{X} as the output and for the product vice versa.}
\item{There is no method defined by the \ttbf{VSIPL} specification to get the \ttbf{Q} or \ttbf{R} matrices explicitly. This information is held in the \ttbf{QR} object. For the \ttbf{qrd view} method these are calculated using the \ttbf{QR prodQ} method.}
\item{Constant \ttbf{alpha} in the \ttbf{solveR} method is a multiplier on input \ttbf{B}.}
\item{\ttbf{QR} is a bit busy. Reading the C VSIPL specification should help in understanding the constants and Flags.}
\end{itemize}}
\afuncT{ramp}{Given start and increment values fill a vector with incremental values.}{elementGenerationOperations}
\\\cvsiplh
\\\pyjvsiph
\viewmthd{yes}{no}{yes}{a.ramp(start,increment)}
\apyfunc{no}{NA}
\pyComment{
\item{Since \ttbf{ramp} is always done in place there is no advantage to also defining a function.}
}

\afuncT{rand}{Random Number Function Set.}{Random}
\\\cvsiplh
\begin{table}[h]
\centering
\begin{tabular}[H]{|l|}
\multicolumn{1}{c}{\hypertarget{randomNumbers}{Random Number Function Set}}\\
\hline
\multicolumn{1}{|c|}{Create and Destroy}\\
\hline
vsip\_randstate *vsip\_randcreate(\\*\hspace*{.8cm}vsip\_index, vsip\_index, vsip\_index, vsip\_rng);\\
int vsip\_randdestroy(vsip\_randstate *);\\
\hline
\hline
\multicolumn{1}{|c|}{Scalar Random Numbers}\\
\hline
vsip\_scalar\_d vsip\_randn\_d(vsip\_randstate*);\\
vsip\_scalar\_d vsip\_randu\_d(vsip\_randstate*);\\
vsip\_scalar\_f vsip\_randn\_f(vsip\_randstate*);\\
vsip\_scalar\_f vsip\_randu\_f(vsip\_randstate*);\\
vsip\_cscalar\_d vsip\_crandn\_d(vsip\_randstate*);\\
vsip\_cscalar\_d vsip\_crandu\_d(vsip\_randstate*);\\
vsip\_cscalar\_f vsip\_crandn\_f(vsip\_randstate*);\\
vsip\_cscalar\_f vsip\_crandu\_f(vsip\_randstate*);\\
\hline
\hline
\multicolumn{1}{|c|}{Normal Random Numbers On \ttbf{view}s}\\
\hline
void vsip\_vrandn\_d(vsip\_randstate*, const vsip\_vview\_d*);\\
void vsip\_vrandn\_f(vsip\_randstate*, const vsip\_vview\_f*);\\
void vsip\_mrandn\_d(vsip\_randstate*, const vsip\_mview\_d*);\\
void vsip\_mrandn\_f(vsip\_randstate*, const vsip\_mview\_f*);\\
void vsip\_cvrandn\_d(vsip\_randstate *, const vsip\_cvview\_d*);\\
void vsip\_cvrandn\_f(vsip\_randstate *, const vsip\_cvview\_f*);\\
void vsip\_cmrandn\_d(vsip\_randstate*, const vsip\_cmview\_d*);\\
void vsip\_cmrandn\_f(vsip\_randstate*, const vsip\_cmview\_f*);\\
\hline
\hline
\multicolumn{1}{|c|}{Uniform Random Numbers On \ttbf{view}s}\\
\hline
void vsip\_vrandu\_d(vsip\_randstate*, const vsip\_vview\_d*);\\
void vsip\_vrandu\_f(vsip\_randstate*, const vsip\_vview\_f*);\\
void vsip\_mrandu\_d(vsip\_randstate*, const vsip\_mview\_d*);\\
void vsip\_mrandu\_f(vsip\_randstate*, const vsip\_mview\_f*);\\
void vsip\_cvrandu\_d(vsip\_randstate *, const vsip\_cvview\_d*);\\
void vsip\_cvrandu\_f(vsip\_randstate *, const vsip\_cvview\_f*);\\
void vsip\_cmrandu\_d(vsip\_randstate*, const vsip\_cmview\_d*);\\
void vsip\_cmrandu\_f(vsip\_randstate*, const vsip\_cmview\_f*);\\
\hline
\end{tabular}
\end{table}
\\\pyjvsiph

\afuncT{real}{ Return a new real \ttbf{view} of the real part of a complex \ttbf{view}.}{elementGenerationOperations}
\\\cvsiplh
\\\pyjvsiph
%
\viewmthd{No}{NA}{NA}{NA}
%
\apyfunc{No}{NA}
%
\pyComment{
\item{No comments}
}
\afunc{recip}{Arctangent of Two Arguments; An elementwise function}
\cvsiplh
\pyjvsiph
\afuncT{rect}{Rect}{elementGenerationOperations}
\\\cvsiplh
\\\pyjvsiph
%
\viewmthd{No}{NA}{NA}{NA}
%
\apyfunc{No}{NA}
%
\pyComment{
\item{No comments}
}
\afunc{round}{Round to nearest integral value; An elementwise function. See unary operations table \ref{tab:unaryOperations}}
\cvsiplh
\pyjvsiph
\pyjvComment{
\item{The \ilCode{round} function is not supported in \jv at this time}
}

\afunc{rsqrt}{Arctangent of Two Arguments; An elementwise function}
\cvsiplh
\pyjvsiph
\afuncT{sbm}{Subtract and multiply. An element-wise function.}{ternaryOperations}
\\\cvsiplh
\afh
{
\ttfamily
\\\hspace*{.04\textwidth}\begin{tabular}[H]{l}
void vsip\_vsbm\_d(const vsip\_vview\_d*, const vsip\_vview\_d*,\\*\hspace{.7cm}const vsip\_vview\_d*, const vsip\_vview\_d*); 
void vsip\_vsbm\_f(const vsip\_vview\_f*, const vsip\_vview\_f*,\\*\hspace{.7cm}const vsip\_vview\_f*, const vsip\_vview\_f*); 
void vsip\_cvsbm\_d(const vsip\_cvview\_d*, const vsip\_cvview\_d*,\\*\hspace{.7cm}const vsip\_cvview\_d*, const vsip\_cvview\_d*); 
void vsip\_cvsbm\_f(const vsip\_cvview\_f*, const vsip\_cvview\_f*,\\*\hspace{.7cm}const vsip\_cvview\_f*, const vsip\_cvview\_f*); 
\end{tabular}
}
\\\pyjvsiph
\viewmthd{No}{NA}{NA}{NA}
\apyfunc{yes}{\ttbf{out = sbm(in1,in2,in3,out)}}
\pyComment{\item{Arguments \ttbf{in1}, \ttbf{in2} and \ttbf{in3} are always \ttbf{view}s. }
\item{The \ttbf{sbm} function works much the same as the C VSIPL version except that a convenience pointer to the output \ttbf{view} is returned.}
\item{This may be done in-place if an input \ttbf{view} is the same as the output \ttbf{view}.}}
\afuncT{scatter}{Scatter data from a view into another (indexed) view} {elementGenerationOperations}
\\\cvsiplh
\\\pyjvsiph
%
\viewmthd{No}{NA}{NA}{NA}
%
\apyfunc{No}{NA}
%
\pyComment{
\item{No comments}
}
\afuncT{sin}{Sine; An element-wise function. Input \ttbf{view} elements are assumed to be in radians.}{elementaryMath}
\\\cvsiplh
\afh
\\\hspace*{.04\textwidth} {
\ttfamily
\begin{tabular}[H]{l}
vsip\_scalar\_f vsip\_sin\_f(vsip\_scalar\_f a);\\
vsip\_scalar\_d vsip\_sin\_d(vsip\_scalar\_d a);\\
void vsip\_msin\_d(\\*
\hspace{1cm}const vsip\_mview\_d*, const vsip\_mview\_d*);\\
void vsip\_msin\_f(\\*
\hspace{1cm}const vsip\_mview\_f*, const vsip\_mview\_f*);\\
void vsip\_vsin\_d(\\*
\hspace{1cm}const vsip\_vview\_d*, const vsip\_vview\_d*);\\
void vsip\_vsin\_f(\\*
\hspace{1cm}const vsip\_vview\_f*, const vsip\_vview\_f*);\\
\end{tabular}
}
\\\pyjvsiph
\viewmthd{yes}{yes}{yes}{inOut.sin}
\apyfunc{yes}{out = sin(in,out)}
\\\hspace*{.06\textwidth}\parbox{.9\textwidth}{\vspace*{.005\textheight}The \ttbf{sin} function works much the same as the C VSIPL version except that a convenience pointer to the output view is returned. This may be done in-place if \ttbf{in==out}.}

\afunc{sinh}{Hyperbolic Sine; An elementwise function. See elementary math functions table \ref{tab:elementaryMath}.}
\\\cvsiplh
\newline \hspace*{.8cm} \vspace*{.1cm} \textbf{Available Functions }
\newline \hspace*{1.1cm} {
\ttfamily
\begin{tabular}[H]{l}
vsip\_scalar\_f vsip\_sinh\_f(vsip\_scalar\_f a);\\
vsip\_scalar\_d vsip\_sinh\_d(vsip\_scalar\_d a);\\
void vsip\_msinh\_d(\\*
\hspace{1cm}const vsip\_mview\_d*, const vsip\_mview\_d*);\\
void vsip\_msinh\_f(\\*
\hspace{1cm}const vsip\_mview\_f*, const vsip\_mview\_f*);\\
void vsip\_vsinh\_d(\\*
\hspace{1cm}const vsip\_vview\_d*, const vsip\_vview\_d*);\\
void vsip\_vsinh\_f(\\*
\hspace{1cm}const vsip\_vview\_f*, const vsip\_vview\_f*);\\
\end{tabular}
}
\\\pyjvsiph
\viewmthd{yes}{yes}{yes}{inOut.sinh}
\apyfunc{yes}{out = sinh(in,out)}
\newline\hspace*{1.2cm}\parbox{10.8cm}{\vspace*{.1cm}The \ttbf{sinh} function works much the same as the C VSIPL version except that a convenience pointer to the output view is returned. This may be done in-place if \ttbf{in==out}.}

\afuncT{sortip}{Sort a vector in-place}{sort}
\\\cvsiplh
\newline \hspace*{.8cm} \vspace*{.1cm} \textbf{Available Functions }
\newline \hspace*{1.1cm} {
\ttfamily
\begin{tabular}[H]{l}
void vsip\_vsortip\_d(const vsip\_vview\_d*, vsip\_sort\_mode,\\*\hspace*{1cm}vsip\_sort\_dir, vsip\_bool, const vsip\_vview\_vi*);\\
void vsip\_vsortip\_f(const vsip\_vview\_f*, vsip\_sort\_mode,\\*\hspace*{1cm}vsip\_sort\_dir, vsip\_bool, const vsip\_vview\_vi*);\\
void vsip\_vsortip\_vi(const vsip\_vview\_vi*, vsip\_sort\_mode,\\*\hspace*{1cm}vsip\_sort\_dir, vsip\_bool, const vsip\_vview\_vi*);\\
void vsip\_vsortip\_i(const vsip\_vview\_i*, vsip\_sort\_mode,\\*\hspace*{1cm}vsip\_sort\_dir, vsip\_bool, const vsip\_vview\_vi*);\\
\end{tabular}
}\\
\\\pyjvsiph
\viewmthd{NA}{NA}{NA}{sortip}
\apyfunc{NA}{NA}
\pyComment{\item{No Coment}}
\afuncT{spline}{Interpolation using cubic splines.}{interpolation}
\\\cvsiplh
\afh
\\\hspace*{.04\textwidth} {
\ttfamily
\begin{tabular}[H]{|l|}
\multicolumn{1}{c}{
\Ts\rmfamily \bfseries Create Spline Object\vspace{.1cm}}\Bs\\ \hline
vsip\_spline\_d* vsip\_spline\_create\_d(vsip\_length);\Bs\\
vsip\_spline\_f* vsip\_spline\_create\_f(vsip\_length);\Bs\\
\hline\multicolumn{1}{c}{\Ts\rmfamily \bfseries Destroy Spline Object\vspace{.1cm}}\Bs\\ \hline
void vsip\_spline\_destroy\_d(vsip\_spline\_d*);\Bs\\
void vsip\_spline\_destroy\_f(vsip\_spline\_f*);\Bs\\
\hline\multicolumn{1}{c}{\Ts \rmfamily \bfseries Vector Splines\vspace{.1cm}}\Bs\\ \hline
void vsip\_vinterp\_spline\_d(const vsip\_vview\_d*, const vsip\_vview\_d*,
\fbrk{}vsip\_spline\_d*, const vsip\_vview\_d*, const vsip\_vview\_d*);\Bs\\
void vsip\_vinterp\_spline\_f(const vsip\_vview\_f*, const vsip\_vview\_f*,
\fbrk{}vsip\_spline\_f*, const vsip\_vview\_f*, const vsip\_vview\_f* ) ;\Bs\\
\hline\multicolumn{1}{c}{\Ts \rmfamily \bfseries Matrix Splines by Row or by Column\vspace{.1cm}}\Bs\\ \hline
void vsip\_minterp\_spline\_d(\fbrk{}const vsip\_vview\_d*, const vsip\_mview\_d*,
vsip\_spline\_d*,\fbrk{}vsip\_major,const vsip\_vview\_d*, const vsip\_mview\_d*);\Bs\\
void vsip\_minterp\_spline\_f(\fbrk{}const vsip\_vview\_f*, const vsip\_mview\_f*,
vsip\_spline\_f*,\fbrk{}vsip\_major, const vsip\_vview\_f*, const vsip\_mview\_f*);\Bs\\\hline
\end{tabular}
}\vspace*{3mm}
\\\pyjvsiph
\viewmthd{No}{No}{NA}{NA}
\apyclass{Yes}
%Example
\inputminted[linenos=true,resetmargins=true,xleftmargin=.12\textwidth,fontfamily=tt,fontsize=\small]{python}{./pyJvsip_examples/eXspline.py}
\hspace*{.08\textwidth}{\rmfamily For result see figure \ref{fig:SplineExample}.}\\
\begin{minipage}[c]{\textwidth}\centering\includegraphics[width=0.8\textwidth]{./pyJvsip_examples/eXspline}\captionof{figure}{Spline Example}
\label{fig:SplineExample}\end{minipage}
%Instantiation
\aclassInstantiate{Spline}{aSplineObj=Spline(aType,maxLength)}{{Where:
\\ \hspace*{.10\textwidth} \ttbf{aType} is the string indicating the type of spline object.
\\ \hspace*{.10\textwidth} \ttbf{maxLength} is the maximum number of expected output nodes.}}
%Methods
\aclassMethods{Spline}
\pyComment{
\item{No comment}}
\afunc{sq}{Arctangent of Two Arguments; An elementwise function}
\cvsiplh
\pyjvsiph
\afuncT{sqrt}{Square Root; An elementwise function.}{elementaryMath}
\\\cvsiplh
\afh
\\\hspace*{.04\textwidth} {
\ttfamily
\begin{tabular}[H]{l}
vsip\_scalar\_f vsip\_sqrt\_f(vsip\_scalar\_f a);\\
vsip\_scalar\_d vsip\_sqrt\_d(vsip\_scalar\_d a);\\
vsip\_cscalar\_d vsip\_csqrt\_d(vsip\_cscalar\_d);\\
vsip\_cscalar\_f vsip\_csqrt\_f(vsip\_cscalar\_f);\\
void vsip\_msqrt\_d(const vsip\_mview\_d*, const vsip\_mview\_d*);\\
void vsip\_msqrt\_f(const vsip\_mview\_f*, const vsip\_mview\_f*);\\
void vsip\_vsqrt\_d(const vsip\_vview\_d*, const vsip\_vview\_d*);\\
void vsip\_vsqrt\_f(const vsip\_vview\_f*, const vsip\_vview\_f*);\\
void vsip\_cmsqrt\_d(const vsip\_cmview\_d*, const vsip\_cmview\_d*);\\
void vsip\_cmsqrt\_f(const vsip\_cmview\_f*, const vsip\_cmview\_f*);\\
void vsip\_cvsqrt\_d(const vsip\_cvview\_d*, const vsip\_cvview\_d*);\\
void vsip\_cvsqrt\_f(const vsip\_cvview\_f*, const vsip\_cvview\_f*);\\
\end{tabular}
}
\\\pyjvsiph
\viewmthd{yes}{yes}{yes}{inOut.sqrt}
\apyfunc{yes}{out = sqrt(in,out)}
\pyComment{\item{The \ttbf{sqrt} function works much the same as the C VSIPL version except that a convenience pointer to the output view is returned. This may be done in-place if \ttbf{in==out}.}}

\afuncT{sub}{Compute the difference of a scalar and a \ttbf{view} or between two \ttbf{view}s.}{binaryOperations}
\\\cvsiplh
\\\hspace*{.04\textwidth}\parbox{.93\textwidth}{
\textrm{There are many combinations of subtract functions available in \jv{}. The specification provides the normal \ttbf{view} subtractions of complex-complex and real-real types;\Bs but also provides real-complex multiplies as well as mixed \ttbf{scalar-view} multiplies. Consequently the listed available functions are broken up into several tables below.}
}\vspace{.005\textheight}
\afh
{
\ttfamily
\\\hspace*{.04\textwidth}\begin{tabular}[H]{l}
\multicolumn{1}{c}{\Ts\rmfamily \bfseries Scalar Subtraction Functions}\\ \hline
vsip\_cscalar\_d vsip\_crsub\_d(vsip\_cscalar\_d, vsip\_scalar\_d);\Bs\\
vsip\_cscalar\_d vsip\_csub\_d(vsip\_cscalar\_d, vsip\_cscalar\_d);\Bs\\
vsip\_cscalar\_d vsip\_rcsub\_d(vsip\_scalar\_d, vsip\_cscalar\_d);\Bs\\
vsip\_cscalar\_f vsip\_crsub\_f(vsip\_cscalar\_f, vsip\_scalar\_f);\Bs\\
vsip\_cscalar\_f vsip\_csub\_f(vsip\_cscalar\_f, vsip\_cscalar\_f);\Bs\\
vsip\_cscalar\_f vsip\_rcsub\_f(vsip\_scalar\_f, vsip\_cscalar\_f);\Bs\\
\end{tabular}
\\\hspace*{.04\textwidth}\begin{tabular}[H]{l}
\multicolumn{1}{c}{\Ts\rmfamily \bfseries Scalar-\ttbf{view} Subtraction Functions}\\ \hline
void vsip\_csmsub\_d(\\*\hspace*{1cm}vsip\_cscalar\_d, const vsip\_cmview\_d*, const vsip\_cmview\_d*);\Bs\\
void vsip\_csmsub\_f(\\*\hspace*{1cm}vsip\_cscalar\_f, const vsip\_cmview\_f*, const vsip\_cmview\_f*);\Bs\\
void vsip\_csvsub\_d(\\*\hspace*{1cm}vsip\_cscalar\_d, const vsip\_cvview\_d*, const vsip\_cvview\_d*);\Bs\\
void vsip\_csvsub\_f(\\*\hspace*{1cm}vsip\_cscalar\_f, const vsip\_cvview\_f*, const vsip\_cvview\_f*);\Bs\\
void vsip\_smsub\_d(\\*\hspace*{1cm}vsip\_scalar\_d, const vsip\_mview\_d*, const vsip\_mview\_d*);\Bs\\
void vsip\_smsub\_f(\\*\hspace*{1cm}vsip\_scalar\_f, const vsip\_mview\_f*, const vsip\_mview\_f*);\Bs\\
void vsip\_smsub\_i(\\*\hspace*{1cm}vsip\_scalar\_i, const vsip\_mview\_i*, const vsip\_mview\_i*);\Bs\\
void vsip\_smsub\_si(\\*\hspace*{1cm}vsip\_scalar\_si, const vsip\_mview\_si*, const vsip\_mview\_si*);\Bs\\
void vsip\_smsub\_si(\\*\hspace*{1cm}vsip\_scalar\_si, const vsip\_mview\_si*, const vsip\_mview\_si*);\Bs\\
void vsip\_svsub\_d(\\*\hspace*{1cm}vsip\_scalar\_d, const vsip\_vview\_d*, const vsip\_vview\_d*);\Bs\\
void vsip\_svsub\_f(\\*\hspace*{1cm}vsip\_scalar\_f, const vsip\_vview\_f*, const vsip\_vview\_f*);\Bs\\
void vsip\_svsub\_i(\\*\hspace*{1cm}vsip\_scalar\_i, const vsip\_vview\_i*, const vsip\_vview\_i*);\Bs\\
void vsip\_svsub\_si(\\*\hspace*{1cm}vsip\_scalar\_si, const vsip\_vview\_si*, const vsip\_vview\_si*);\Bs\\
void vsip\_svsub\_uc(\\*\hspace*{1cm}vsip\_scalar\_uc, const vsip\_vview\_uc*, const vsip\_vview\_uc*);\Bs\\
void vsip\_svsub\_vi(\\*\hspace*{1cm}vsip\_scalar\_vi, const vsip\_vview\_vi*, const vsip\_vview\_vi*);\Bs\\
\end{tabular}
\\\hspace*{.04\textwidth}\begin{tabular}[H]{l}
\multicolumn{1}{c}{\Ts\rmfamily \bfseries Mixed Depth scalar-\ttbf{view} Subtraction Functions}\\ \hline
void vsip\_rscmsub\_d(\\*\hspace*{1cm}vsip\_scalar\_d, const vsip\_cmview\_d*, const vsip\_cmview\_d*);\Bs\\
void vsip\_rscmsub\_f(\\*\hspace*{1cm}vsip\_scalar\_f, const vsip\_cmview\_f*, const vsip\_cmview\_f*);\Bs\\
void vsip\_rscvsub\_d(\\*\hspace*{1cm}vsip\_scalar\_d, const vsip\_cvview\_d*, const vsip\_cvview\_d*);\Bs\\
void vsip\_rscvsub\_f(\\*\hspace*{1cm}vsip\_scalar\_f, const vsip\_cvview\_f*, const vsip\_cvview\_f*);\Bs\\
\end{tabular}
\\\hspace*{.04\textwidth}\begin{tabular}[H]{l}
\multicolumn{1}{c}{\Ts\rmfamily \bfseries \ttbf{View}-\ttbf{view} Subtraction Functions}\\ \hline
void vsip\_vsub\_d(\\*\hspace*{1cm}const vsip\_vview\_d*, const vsip\_vview\_d*, const vsip\_vview\_d*);\Bs\Bs\\
void vsip\_vsub\_f(\\*\hspace*{1cm}const vsip\_vview\_f*, const vsip\_vview\_f*, const vsip\_vview\_f*);\Bs\\
void vsip\_cvsub\_d(\\*\hspace*{1cm}const vsip\_cvview\_d*, const vsip\_cvview\_d*, const vsip\_cvview\_d*);\Bs\\
void vsip\_cvsub\_f(\\*\hspace*{1cm}const vsip\_cvview\_f*, const vsip\_cvview\_f*, const vsip\_cvview\_f*);\Bs\\
void vsip\_vsub\_i(\\*\hspace*{1cm}const vsip\_vview\_i*, const vsip\_vview\_i*, const vsip\_vview\_i*);\Bs\\
void vsip\_vsub\_si(\\*\hspace*{1cm}const vsip\_vview\_si*, const vsip\_vview\_si*, const vsip\_vview\_si*);\Bs\\
void vsip\_vsub\_uc(\\*\hspace*{1cm}const vsip\_vview\_uc*, const vsip\_vview\_uc*, const vsip\_vview\_uc*);\Bs\\
void vsip\_msub\_d(\\*\hspace*{1cm}const vsip\_mview\_d*, const vsip\_mview\_d*, const vsip\_mview\_d*);\Bs\\
void vsip\_msub\_f(\\*\hspace*{1cm}const vsip\_mview\_f*, const vsip\_mview\_f*, const vsip\_mview\_f*);\Bs\\
void vsip\_cmsub\_d(\\*\hspace*{1cm}const vsip\_cmview\_d*, const vsip\_cmview\_d*, const vsip\_cmview\_d*);\Bs\\
void vsip\_cmsub\_f(\\*\hspace*{1cm}const vsip\_cmview\_f*, const vsip\_cmview\_f*, const vsip\_cmview\_f*);\Bs\\
void vsip\_msub\_i(\\*\hspace*{1cm}const vsip\_mview\_i*, const vsip\_mview\_i*, const vsip\_mview\_i*);\Bs\\
void vsip\_msub\_si(\\*\hspace*{1cm}const vsip\_mview\_si*, const vsip\_mview\_si*, const vsip\_mview\_si*);\Bs\\
\end{tabular}
\\\hspace*{.04\textwidth}\begin{tabular}[H]{l}
\multicolumn{1}{c}{\Ts\rmfamily \bfseries Mixed Depth \ttbf{view}-\ttbf{view} Subtraction Functions}\\ \hline
void vsip\_crmsub\_d(\\*\hspace*{1cm}const vsip\_cmview\_d*, const vsip\_mview\_d*, const vsip\_cmview\_d*);\Bs\\
void vsip\_crmsub\_f(\\*\hspace*{1cm}const vsip\_cmview\_f*, const vsip\_mview\_f*, const vsip\_cmview\_f*);\Bs\\
void vsip\_rcmsub\_d(\\*\hspace*{1cm}const vsip\_mview\_d*, const vsip\_cmview\_d*, const vsip\_cmview\_d*);\Bs\\
void vsip\_rcmsub\_f(\\*\hspace*{1cm}const vsip\_mview\_f*, const vsip\_cmview\_f*, const vsip\_cmview\_f*);\Bs\\
void vsip\_rcvsub\_d(\\*\hspace*{1cm}const vsip\_vview\_d*, const vsip\_cvview\_d*, const vsip\_cvview\_d*);\Bs\\
void vsip\_rcvsub\_f(\\*\hspace*{1cm}const vsip\_vview\_f*, const vsip\_cvview\_f*, const vsip\_cvview\_f*);\Bs\\
void vsip\_crvsub\_d(\\*\hspace*{1cm}const vsip\_cvview\_d*, const vsip\_vview\_d*, const vsip\_cvview\_d*);\Bs\\
void vsip\_crvsub\_f(\\*\hspace*{1cm}const vsip\_cvview\_f*, const vsip\_vview\_f*, const vsip\_cvview\_f*);\Bs\\
\end{tabular}\\
}
\pyjvsiph
\\\vmthdh
\hspace*{.06\textwidth}Overloaded on minus operator.\\
\hspace*{.06\textwidth}\textbf{In Place: }\hspace{2mm} yes\\
\hspace*{.08\textwidth}\textbf{Example: }\ttbf{a -= b;\Bs a -= 2}\\*
\hspace*{.1\textwidth}Elements of \ttbf{view a} replaced with result.\\
\hspace*{.06\textwidth}\textbf{Out of Place: }\hspace{.2cm} yes\\
\hspace*{.08\textwidth}\textbf{Example: }\ttbf{c = a - b;\Bs d = 2 - c}\\*
\hspace*{.1\textwidth}\ttbf{view c} and \ttbf{view d} created and filled with result of operation.
\apyfunc{yes}{out = sub(in1,in2,out)}
\pyComment{
\item{For the function arguments \ttbf{in1} may be either a view or a scalar. Argument \ttbf{in2} and \ttbf{out} must be a view for \pyjv.}
\item{For the overloaded minus operator \pyjv{} has been augmented so that if $\alpha$ is a scalar and $a$ is a view then either $y=a-\alpha$ or $y=\alpha - v$ will work as expected. The first case is actually done with the \ttbf{add} function from \cvl.}
\item{The \ttbf{sub} function works much the same as the C VSIPL version except that a convenience pointer to the output view is returned. See the \cvl{} specification for clues on how the function works.}
\item{This may be done in-place if \ttbf{in1==out} or \ttbf{in2==out}.}
}
\afuncT{submatrix}{Create indexed sub-matrix. The method \ttbf{submatrix} is done out-of-place by \ttbf{view}s of type matrix. It will either create a new child matrix by first dropping an indexed row and column from the parent matrix, or by copying selected rows and columns from a parent matrix to the child matrix. A new block space and matrix is created for the output by the method.}{}
\\\cvsiplh
\\\hspace*{.06\textwidth}Not supported in C VSIPL. 
\\\pyjvsiph
\viewmthdu{yes}{No}{No}{out=in.submatrix(cols,rows)}{
\item{\ttbf{cols} is an integer (index) indicating the column to drop or a \ttbf{view} of type \ttbf{vview\_vi} indicating the columns to keep}
\item{\ttbf{out} is a new output matrix \ttbf{view} created by the method.}
\item{\ttbf{in} is a \ttbf{view} of shape matrix.}
\item{\ttbf{rows} is an integer (index) indicating the row to drop or a \ttbf{view} of type \ttbf{vview\_vi} indicating the rows to keep.}
\item{Mixing an index vector with an integer index is not supported.}
}
\apyfunc{No}{NA}
\afunc{sumsqval}{Returns the sum of the squares of all the elements of a \ttbf{view}. Does not modify input. See table \ref{tab:unaryOperations}}
\\\cvsiplh
\newline \hspace*{.8cm} \vspace*{.1cm} \textbf{Available Functions }
\newline \hspace*{1.1cm} {
\ttfamily
\begin{tabular}[H]{l}
vsip\_scalar\_d vsip\_msumsqval\_d(const vsip\_mview\_d* );\\
vsip\_scalar\_d vsip\_vsumsqval\_d(const vsip\_vview\_d* );\\
vsip\_scalar\_f vsip\_msumsqval\_f(const vsip\_mview\_f* );\\
vsip\_scalar\_f vsip\_vsumsqval\_f(const vsip\_vview\_f* );\\
\end{tabular}
}
\\\pyjvsiph
\viewmthd{yes}{yes}{NA}{aValue=in.sumsqval}
\apyfunc{No}{}
\newline \hspace*{.8cm} \textbf{Comments}
\newline\hspace*{.9cm}\parbox{10.8cm}{\vspace*{.1cm}\begin{itemize}
\item{Since the \ttbf{sumsqval} function returns a scalar without modifying the \ttbf{view} there seemed little point in supporting this as a separate function call for \pyjv.}
\end{itemize}
}

\afuncT{sumval}{Returns the sum of the the elements of a \ttbf{view}. Does not modify input.}{unaryOperations}
\\\cvsiplh
\afh
\\\hspace*{.04\textwidth} {
\ttfamily
\begin{tabular}[H]{l}
vsip\_cscalar\_d vsip\_cmsumval\_d(const vsip\_cmview\_d*);\\
vsip\_cscalar\_d vsip\_cvsumval\_d(const vsip\_cvview\_d*);\\
vsip\_cscalar\_f vsip\_cmsumval\_f(const vsip\_cmview\_f*);\\
vsip\_cscalar\_f vsip\_cvsumval\_f(const vsip\_cvview\_f*);\\
vsip\_scalar\_d vsip\_msumval\_d(const vsip\_mview\_d*);\\
vsip\_scalar\_d vsip\_vsumval\_d(const vsip\_vview\_d*);\\
vsip\_scalar\_f vsip\_msumval\_f(const vsip\_mview\_f*);\\
vsip\_scalar\_f vsip\_vsumval\_f(const vsip\_vview\_f*);\\
vsip\_scalar\_i vsip\_vsumval\_i(const vsip\_vview\_i*);\\
vsip\_scalar\_si vsip\_vsumval\_si(const vsip\_vview\_si*);\\
vsip\_scalar\_uc vsip\_vsumval\_uc(const vsip\_vview\_uc*);\\
vsip\_scalar\_vi vsip\_msumval\_bl(const vsip\_mview\_bl*);\\
vsip\_scalar\_vi vsip\_vsumval\_bl(const vsip\_vview\_bl*);\\
\end{tabular}
}
\\\pyjvsiph
\afuncT{svd}{Singular Value Decomposition Class}{singularValueDecompostion}
\\\cvsiplh 
\\ \hspace*{.8cm} \vspace*{.1cm} \textbf{Available Functions }
%
%\\ \hspace*{.8cm} \vspace*{.1cm} \texttt{svd\_create}
\\ \hspace*{.03\textwidth} {
\ttfamily\vspace{.3cm}
\begin{tabular}[H]{|l|}
\multicolumn{1}{c}{\rmfamily \bfseries Create SVD Object\vspace{.1cm}}\\ \hline \Ts
vsip\_sv\_d* vsip\_svd\_create\_d(vsip\_length, vsip\_length,\\*\hspace*{1cm}vsip\_svd\_uv , vsip\_svd\_uv);\Bs\\
vsip\_sv\_f* vsip\_svd\_create\_f(vsip\_length, vsip\_length,\\*\hspace*{1cm}vsip\_svd\_uv , vsip\_svd\_uv);\Bs\\
vsip\_csv\_d* vsip\_csvd\_create\_d(vsip\_length, vsip\_length,\\*\hspace*{1cm}vsip\_svd\_uv , vsip\_svd\_uv);\Bs\\
vsip\_csv\_f* vsip\_csvd\_create\_f(vsip\_length, vsip\_length,\\*\hspace*{1cm}vsip\_svd\_uv , vsip\_svd\_uv);\Bs\\
\hline\end{tabular}\\}
%
%\\ \hspace*{.8cm} \vspace*{.1cm} \texttt{svd\_destroy}
\\ \hspace*{.03\textwidth} {
\ttfamily\vspace{.3cm}
\begin{tabular}[H]{|l|}
\multicolumn{1}{c}{\rmfamily \bfseries Destroy SVD Object\vspace{.1cm}}\\ \hline\Ts
int vsip\_svd\_destroy\_d(vsip\_sv\_d*);\Bs\\
int vsip\_svd\_destroy\_f(vsip\_sv\_f*);\Bs\\
int vsip\_csvd\_destroy\_d(vsip\_csv\_d*);\Bs\\
int vsip\_csvd\_destroy\_f(vsip\_csv\_f*);\Bs\\
\hline\end{tabular}\\}
%
%\\ \hspace*{.8cm} \vspace*{.1cm} \texttt{svd}
\\ \hspace*{.03\textwidth}{
\ttfamily\vspace{.3cm}
\begin{tabular}[H]{|l|}
\multicolumn{1}{c}{\rmfamily \bfseries Compute SVD\vspace{.1cm}}\\ \hline \Ts
int vsip\_svd\_d(vsip\_sv\_d*, const vsip\_mview\_d*, vsip\_vview\_d*);\Bs\\
int vsip\_svd\_f(vsip\_sv\_f*, const vsip\_mview\_f*, vsip\_vview\_f*);\Bs\\
int vsip\_csvd\_d(vsip\_csv\_d*, const vsip\_cmview\_d*, vsip\_vview\_d*);\Bs\\
int vsip\_csvd\_f(vsip\_csv\_f*, const vsip\_cmview\_f*, vsip\_vview\_f*);\Bs\\
\hline\end{tabular}\\}
%
%\\ \hspace*{.8cm} \vspace*{.1cm} \texttt{svd\_getattr}
\\ \hspace*{.03\textwidth}{
\ttfamily\vspace{.3cm}
\begin{tabular}[H]{|l|}
\multicolumn{1}{c}{\rmfamily \bfseries Fill SVD Attribute Structure\vspace{.1cm}}\\ \hline \Ts
void vsip\_svd\_getattr\_d(const vsip\_sv\_d*,  vsip\_sv\_attr\_d*);\Bs\\
void vsip\_svd\_getattr\_f(const vsip\_sv\_f*,  vsip\_sv\_attr\_f*);\Bs\\
void vsip\_csvd\_getattr\_d(const vsip\_csv\_d*,  vsip\_csv\_attr\_d*);\Bs\\
void vsip\_csvd\_getattr\_f(const vsip\_csv\_f*,  vsip\_csv\_attr\_f*);\Bs\\
\hline\end{tabular}\\}
%
\\ \hspace*{0.4cm} {
\ttfamily\vspace{.3cm}
\begin{tabular}[H]{|l|}
\multicolumn{1}{c}{\rmfamily \bfseries Product with U from SV Decomposition\vspace{.1cm}}\\ \hline\Ts
\Ts int vsip\_svdprodu\_f(const vsip\_sv\_f*,vsip\_mat\_op, vsip\_mat\_side,\\*\hspace*{1cm}const vsip\_mview\_f*);\Bs\\
int vsip\_csvdprodu\_f(const vsip\_csv\_f*, vsip\_mat\_op, vsip\_mat\_side,\\*\hspace*{1cm}const vsip\_cmview\_f*);\Bs\\
int vsip\_svdprodu\_d(const vsip\_sv\_d*, vsip\_mat\_op, vsip\_mat\_side,\\*\hspace*{1cm}const vsip\_mview\_d*);\Bs\\
int vsip\_svdprodu\_f(const vsip\_sv\_f*, vsip\_mat\_op, vsip\_mat\_side,\\*\hspace*{1cm}const vsip\_mview\_f*);\Bs\\
int vsip\_csvdprodu\_d(const vsip\_csv\_d*, vsip\_mat\_op, vsip\_mat\_side,\\*\hspace*{1cm}const vsip\_cmview\_d*);\Bs\\
int vsip\_csvdprodu\_f(const vsip\_csv\_f*, vsip\_mat\_op, vsip\_mat\_side,\\*\hspace*{1cm}const vsip\_cmview\_f*);\Bs\\
\hline\end{tabular}\\}
%
\\ \hspace*{0.4cm} {
\ttfamily\vspace{.3cm}
\begin{tabular}[H]{|l|}
\multicolumn{1}{c}{\rmfamily \bfseries Product with U from SV Decomposition\vspace{.1cm}}\\ \hline \Ts
int vsip\_svdprodv\_f(const vsip\_sv\_f*,vsip\_mat\_op, vsip\_mat\_side,\\*\hspace*{1cm}const vsip\_mview\_f*);\Bs\\
int vsip\_csvdprodv\_f(const vsip\_csv\_f*, vsip\_mat\_op, vsip\_mat\_side,\\*\hspace*{1cm}const vsip\_cmview\_f*);\Bs\\
int vsip\_svdprodv\_d(const vsip\_sv\_d*, vsip\_mat\_op, vsip\_mat\_side,\\*\hspace*{1cm}const vsip\_mview\_d*);\Bs\\
int vsip\_svdprodv\_f(const vsip\_sv\_f*, vsip\_mat\_op, vsip\_mat\_side,\\*\hspace*{1cm}const vsip\_mview\_f*);\Bs\\
int vsip\_csvdprodv\_d(const vsip\_csv\_d*, vsip\_mat\_op, vsip\_mat\_side,\\*\hspace*{1cm}const vsip\_cmview\_d*);\Bs\\
int vsip\_csvdprodv\_f(const vsip\_csv\_f*, vsip\_mat\_op, vsip\_mat\_side,\\*\hspace*{1cm}const vsip\_cmview\_f*);\Bs\\
\hline\end{tabular}\\}
%
\\ \hspace*{.03\textwidth} {
\ttfamily\vspace{.3cm}
\begin{tabular}[H]{|l|}
\multicolumn{1}{c}{\rmfamily \bfseries Return with U from SV Decomposition\vspace{.1cm}}\\ \hline\Ts
int vsip\_svdmatu\_f(const vsip\_sv\_f*, vsip\_scalar\_vi, \\*\hspace*{1cm}vsip\_scalar\_vi, const vsip\_mview\_f*);\Bs\\
int vsip\_csvdmatu\_f(const vsip\_csv\_f*, vsip\_scalar\_vi, \\*\hspace*{1cm}vsip\_scalar\_vi, const vsip\_cmview\_f*);\Bs\\
int vsip\_svdmatu\_d(const vsip\_sv\_d*, vsip\_scalar\_vi, \\*\hspace*{1cm}vsip\_scalar\_vi, const vsip\_mview\_d*);\Bs\\
int vsip\_csvdmatu\_d(const vsip\_csv\_d*, vsip\_scalar\_vi, \\*\hspace*{1cm}vsip\_scalar\_vi, const vsip\_cmview\_d*);\Bs\\
\hline\end{tabular}\\}
%
\\ \hspace*{.03\textwidth} {
\ttfamily\vspace{.3cm}
\begin{tabular}[H]{|l|}
\multicolumn{1}{c}{\rmfamily \bfseries Return with V from SV Decomposition\vspace{.1cm}}\\ \hline\Ts
int vsip\_svdmatv\_f(const vsip\_sv\_f*, vsip\_scalar\_vi, \\*\hspace*{1cm}vsip\_scalar\_vi, const vsip\_mview\_f*);\Bs\\
int vsip\_csvdmatv\_f(const vsip\_csv\_f*, vsip\_scalar\_vi, \\*\hspace*{1cm}vsip\_scalar\_vi, const vsip\_cmview\_f*);\Bs\\
int vsip\_svdmatv\_d(const vsip\_sv\_d*, vsip\_scalar\_vi, \\*\hspace*{1cm}vsip\_scalar\_vi, const vsip\_mview\_d*);\Bs\\
int vsip\_csvdmatv\_d(const vsip\_csv\_d*, vsip\_scalar\_vi, \\*\hspace*{1cm}vsip\_scalar\_vi, const vsip\_cmview\_d*);\Bs\\
\hline\end{tabular}\\}
%
\\\pyjvsiph
% table of flags and types
\begin{table}
\caption{Flags and Types for SV Decomposition}
\label{tab:svdFlagsAndType}
\begin{center}\begin{tabular}{|l l|}
\multicolumn{2}{c}{\Ts\parbox[t]{.6\textwidth}{\center{\rmfamily \bfseries SV Decomposition Types}}\vspace*{1mm}}\Bs\\\hline
'sv\_d' & Real \ttbf{SV}; double precision \Bs\\\hline
'sv\_f' & Real \ttbf{SV}; float precision\Bs\\\hline
'csv\_d' & Complex \ttbf{SV}; double precision\Bs\\\hline
'csv\_f' & Complex \ttbf{SV}; float precision\Bs\\\hline
%%%
\multicolumn{2}{c}{\parbox[t]{.6\textwidth}{\center{\rmfamily \bfseries Flags \ttbf{opU} and \ttbf{opV} specifying computing options for matrices \ttbf{U} and \ttbf{V}}}\vspace*{1mm}}\Bs\\\hline
\Ts'NOS' or \ttbf{VSIP\_SVD\_UVNOS} & Do not compute.\Bs\\\hline
'FULL' or \ttbf{VSIP\_SVD\_UVFULL} & Compute full matrix including null space.\Bs\\\hline
'PART' or \ttbf{VSIP\_SVD\_UVPART} & Compute matrix without the null space.\Bs\\\hline
%%%
\multicolumn{2}{c}{\parbox[t]{.6\textwidth}{\center{\rmfamily \bfseries Matrix Operator Flags (\ttbf{op})}}\vspace*{1mm}}\Bs\\\hline
\Ts'NTRANS' or \ttbf{VSIP\_MAT\_NTRANS} & No Transpose operator\Bs\\\hline
   'TRANS' or \ttbf{VSIP\_MAT\_TRANS} & Transpose operator\Bs\\\hline
   'HERM' or \ttbf{VSIP\_MAT\_HERM} & Hermitian operator\Bs\\\hline
%%%
\multicolumn{2}{c}{\parbox[t]{.6\textwidth}{\center{\rmfamily \bfseries Side Flag for \ttbf{matu} and \ttbf{matv}}}\vspace*{1mm}}\Bs\\\hline
\Ts'LSIDE' or \ttbf{VSIP\_MAT\_LSIDE} & \ttbf{U} or \ttbf{V} Matrix on the left side\Bs\\\hline
   'RSIDE' or \ttbf{VSIP\_MAT\_RSIDE} & \ttbf{U} or \ttbf{V} Matrix on the right side\Bs\\\hline
 \end{tabular}\end{center}\end{table}
%%
\\\hspace*{.8cm}{\textbf{View Methods\vspace{.2cm}}\\
\hspace*{.04\textwidth}\parbox{.93\textwidth}{
\begin{itemize}
\item{For singular value decomposition the input matrix is overwritten and (basically) owned by the SVD. To retain the calling \ttbf{view} make a copy.}
\item{Five view methods have been defined for the SVD.}
\subitem{\ttbf{sv}\hspace{6.5mm}\parbox[t]{.7\textwidth}{A property to obtain a vector of singular values for the calling \ttbf{view}}}
\subitem{\ttbf{svd }\hspace{2mm}\parbox[t]{.7\textwidth}{A property to obtain full \ttbf{U}, \ttbf{s} and \ttbf{V} \ttbf{view}s for a calling \ttbf{view}}}
\subitem{\ttbf{svdP }\parbox[t]{.7\textwidth}{A property to obtain \ttbf{U}, \ttbf{s} and \ttbf{V} \ttbf{view}s for a calling \ttbf{view} without the null space (\ttbf{SV} with the 'PART' option).}}
\subitem{\ttbf{svdU }\parbox[t]{.7\textwidth}{A property to obtain full \ttbf{U} and \ttbf{s} \ttbf{view}s for a calling \ttbf{view}. \ttbf{V} is not calculated.}}
\subitem{\ttbf{svdV }\parbox[t]{.7\textwidth}{A property to obtain \ttbf{s} and full \ttbf{V} \ttbf{view}s for a calling \ttbf{view}. \ttbf{U} is not calculated.}}
\end{itemize}\vspace*{2mm}}}\\
%EXAMPLE
\hspace*{1.1cm}\textbf{Example: }\vspace*{.1cm}\\
\hspace*{1.9cm}\parbox{.85\textwidth}{\Ts Assume matrix \ttbf{view} \ttbf{A} has been created and has data in it.}\\
\hspace*{1.9cm}\parbox{.8\textwidth}{\vspace{.3cm}\hspace*{1cm}\ttbf{s=A.sv} \\*
will calculate a vector of singular vales for \ttbf{A}. To retain \ttbf{A} use \ttbf{s=A.copy.sv}\vspace*{3mm}}\\
%CLASS METHODS
\hspace*{.8cm}{\textbf{SV Class Methods\vspace*{2mm}}\\
\hspace*{.03\textwidth}\parbox{.9\textwidth}{To create an \ttbf{SV} object do\\
\hspace*{.03\textwidth}\ttbf{svObj = SV(t,m,n,opU,opV)} \\
Where \ttbf{t} is a string indicating the type of \ttbf{SV} object to create, \ttbf{m} and \ttbf{n} indicate the size (\ttbf{m,n}) of the input matrix to decompose, and \ttbf{opU} and  \ttbf{opV} are flags specifying computing options for matrices \ttbf{U} and \ttbf{V} respectively. See table \ref{tab:svdFlagsAndType}.\\
For convenience if the target of method \ttbf{svd} is matrix \ttbf{view A} then you may create a compatible \ttbf{SV} object using\\
\hspace*{.03\textwidth}\ttbf{svObj=SV(A.type,A.size,opU,opV)}\\
For the class methods table we assume we have created an \ttbf{SV} object we call \ttbf{svObj} and we have an input matrix \ttbf{view A} compliant with \ttbf{svObj}.\vspace{.2cm}}\\
\hspace*{.03\textwidth}\parbox{.9\textwidth}{For methods \ttbf{prodU} and \ttbf{prodV} the input/output matrix \ttbf{XB} may have a different size on output. Both views are in the same data space and will overlap (this is an in-place operation). To ensure matrix \ttbf{XB} is sized properly make sure it is large enough to hold all the data for the largest matrix and place the other matrix in a subview with origin at \ttbf{0,0} for both input and output. Note that \pyjv{} has functions \ttbf{sizeIn} and \ttbf{sizeOut} to help in calculating sizes given the somewhat confusing number of options for matrix operators and matrix sides available in matrix product functions.}\\
\hspace*{.03\textwidth}\parbox[t]{.94\textwidth}{\begin{tabular}{|l|l|}
\multicolumn{2}{c}{\parbox[t]{.58\textwidth}{\center{\rmfamily \bfseries SV Decomposition Methods}\vspace*{2mm}}}\Bs\\ \hline
%
\Ts\ttbf{svObj.svd(A)} & 
\parbox[t]{.58\textwidth}{Decompose \ttbf{A}. Results are in SV object.\vspace*{1mm}}\Bs\\\hline
%
\ttbf{svObj.sizeU} & 
\parbox[t]{.58\textwidth}{Property. Returns size of U matrix created by \ttbf{svd} method.\vspace*{2mm}}\Bs\\\hline
%
\ttbf{svObj.sizeV} & 
\parbox[t]{.58\textwidth}{Property. Returns size of V matrix created by \ttbf{svd} method.\vspace*{2mm}}\Bs\\\hline
%
\ttbf{svObj.size} & 
\parbox[t]{.58\textwidth}{Property. Returns size of expected input matrix to \ttbf{svd} method \ttbf{(m,n)}\vspace{1mm}}\Bs\\\hline
%
\ttbf{svObj.type} & 
\parbox[t]{.58\textwidth}{Property. Returns string indicating SV type.\vspace*{2mm}}\Bs\\\hline
%
\ttbf{svObj.vsip} & 
\parbox[t]{.58\textwidth}{Property. Returns C VSIPL SV instance.\vspace*{2mm}}\Bs\\\hline
%
\ttbf{svObj.s} &
 \parbox[t]{.58\textwidth}{Property. Returns vector of singular values.\vspace*{1mm}}\Bs\\\hline
%
\ttbf{svObj.matU(low,high,U)} &
 \parbox[t]{.58\textwidth}{Method to get matrix \ttbf{U} from \ttbf{SV} object.  Arguments \ttbf{low} and \ttbf{high} are indices of the first and last column to be returned. Argument \ttbf{U} is the supplied output matrix of proper size. For convenience one may do \ttbf{U = svObj.matU(low,high)} to have the method create the U matrix for you. For convenience one may do \ttbf{U = svObj.matU()} if the entire available \ttbf{U} matrix is desired.\vspace{2mm}}\Bs\\\hline
%
\ttbf{svObj.matV(low,high,V)} &
 \parbox[t]{.58\textwidth}{Method to get matrix \ttbf{V} from \ttbf{SV} object. The \ttbf{matV} method has same optional argument lists as \ttbf{matU}.\vspace*{2mm}}\Bs\\\hline
%
\ttbf{svObj.prodU(op,side,XB)} &
 \parbox[t]{.58\textwidth}{Matrix product of \ttbf{U} with \ttbf{XB}. Input/Output in \ttbf{XB}.\vspace*{1mm}}\Bs\\\hline
%
\ttbf{svObj.prodV(op,side,XB)} &
 \parbox[t]{.58\textwidth}{Matrix product of \ttbf{V} with \ttbf{XB}. Input/Output in \ttbf{XB}.\vspace*{1mm}}\Bs\\\hline
%
\end{tabular}\vspace*{.4cm}}\\

\afunc{swap}{ \ref{tab:manipulationOperations}.}
\\\cvsiplh
\\\pyjvsiph

\afunc{tan}{Tangent; An elementwise function}
\cvsiplh
\pyjvsiph

\afunc{tanh}{Hyperbolic Tangent; An elementwise function}
\cvsiplh
\pyjvsiph

\afunc{toepsol}{Solve Toeplitz system. \ref{tab:specialLinearSystemSolvers}.}
\\\cvsiplh
\newline \hspace*{.8cm} \vspace*{.1cm} \textbf{Available Functions }
\newline \hspace*{1.1cm} {
\ttfamily
\begin{tabular}[H]{l}
\end{tabular}
}
\\\pyjvsiph

\afuncT{trans}{Matrix transpose. This function is done out of place unless the input Matrix
is square and the input and output matrix resolve to the same object; otherwise the function
moves data from an input matrix into an output matrix.}{matrixOperations}
\\\cvsiplh
\\\hspace*{.06\textwidth}\parbox{.94\textwidth}{\raggedright The output matrix must be created of 
the proper size and attributes to accomodate the transpose of the input data.\Bs}
\begin{cfuncs}
void vsip\_mtrans\_f(const~vsip\_mview\_f*, const~vsip\_mview\_f*);\Bs\\
void vsip\_cmtrans\_f(const~vsip\_cmview\_f*, const~vsip\_cmview\_f*);\Bs\\
void vsip\_mtrans\_d(const~vsip\_mview\_d*, const~vsip\_mview\_d*);\Bs\\
void vsip\_cmtrans\_d(const~vsip\_cmview\_d*, const~vsip\_cmview\_d*);\Bs\\
\end{cfuncs}
\pyjvsiph
\viewmthd{yes}{yes}{No}{out=in.trans}
\apyfunc{yes}{out = trans(in,out)}
\pyComment{
\item {The \ttbf{trans} method creates a compact row major matrix of the proper
type to store the output of the transpose and returns it.  The method is defined as a 
property since no arguments are required and is always done out-of-place}
\item{The \ttbf{trans} function works much the same as the C VSIPL version except that a 
convenience pointer to the output view is returned. This must be done out-of-place unless
the matrix is square.}
}

\clearpage
User Block Function Set\hspace*{\fill}\hyperlink{userBlockFunctionSet}{(up)}\\
There are a group of functions to create user blocks. User blocks are block objects associated with user memory. User memory is any memory allocated by methods not associated with VSIPL. These blocks have functions associated with them to allow input and output of data from VSIPL without exposing the VSIPL implementations architecture to the user.\\
User blocks have no counterpart in \pyjv.
\\\cvsiplh\\
\texttt{
\begin{tabular}[H]{l}
\multicolumn{1}{c}{\rmfamily \bfseries  Admit Block Object\vspace{.1cm}}\\ \hline
int vsip\_blockadmit\_bl(vsip\_block\_bl*, vsip\_scalar\_bl);\\
int vsip\_blockadmit\_d(vsip\_block\_d*, vsip\_scalar\_bl);\\
int vsip\_blockadmit\_f(vsip\_block\_f*, vsip\_scalar\_bl);\\
int vsip\_blockadmit\_i(vsip\_block\_i*, vsip\_scalar\_bl);\\
int vsip\_blockadmit\_mi(vsip\_block\_mi*, vsip\_scalar\_bl);\\
int vsip\_blockadmit\_si(vsip\_block\_si*, vsip\_scalar\_bl);\\
int vsip\_blockadmit\_uc(vsip\_block\_uc*, vsip\_scalar\_bl);\\
int vsip\_blockadmit\_vi(vsip\_block\_vi*, vsip\_scalar\_bl);\\
int vsip\_cblockadmit\_d(vsip\_cblock\_d*, vsip\_scalar\_bl);\\
int vsip\_cblockadmit\_f(vsip\_cblock\_f*, vsip\_scalar\_bl);\\
\end{tabular}\\
\begin{tabular}[H]{l}
\multicolumn{1}{c}{\rmfamily \bfseries Bind User Memory To Block Object\vspace{.1cm}}\\ \hline
vsip\_block\_bl* vsip\_blockbind\_bl(\\*\hspace{.7cm}
vsip\_scalar\_bl * const, size\_t, vsip\_memory\_hint);\\
vsip\_block\_d* vsip\_blockbind\_d(\\*\hspace{.7cm}
vsip\_scalar\_d * const, size\_t, vsip\_memory\_hint);\\
vsip\_block\_f* vsip\_blockbind\_f(\\*\hspace{.7cm}vsip\_scalar\_f * const, size\_t, vsip\_memory\_hint);\\
vsip\_block\_i* vsip\_blockbind\_i(\\*\hspace{.7cm}vsip\_scalar\_i * const, size\_t, vsip\_memory\_hint);\\
vsip\_block\_mi* vsip\_blockbind\_mi(\\*\hspace{.7cm}vsip\_scalar\_vi * const, size\_t, vsip\_memory\_hint);\\
vsip\_block\_si* vsip\_blockbind\_si(\\*\hspace{.7cm}vsip\_scalar\_si * const, size\_t, vsip\_memory\_hint);\\
vsip\_block\_uc* vsip\_blockbind\_uc(\\*\hspace{.7cm}vsip\_scalar\_uc * const, size\_t, vsip\_memory\_hint);\\
vsip\_block\_vi* vsip\_blockbind\_vi(\\*\hspace{.7cm}vsip\_scalar\_vi * const, size\_t, vsip\_memory\_hint);\\
vsip\_cblock\_d* vsip\_cblockbind\_d(\\*\hspace{.7cm}vsip\_scalar\_d* const, vsip\_scalar\_d* const,\\*\hspace{.7cm}size\_t, vsip\_memory\_hint);\\
vsip\_cblock\_f* vsip\_cblockbind\_f(\\*\hspace{.7cm}vsip\_scalar\_f* const, vsip\_scalar\_f* const,\\*\hspace{.7cm}size\_t, vsip\_memory\_hint);\\
\end{tabular}\\
\begin{tabular}[H]{l}
\multicolumn{1}{c}{\rmfamily \bfseries Release Block Object\vspace{.1cm}}\\ \hline
void vsip\_cblockrelease\_d(vsip\_cblock\_d*,\\*\hspace{.7cm}vsip\_scalar\_bl, vsip\_scalar\_d**, vsip\_scalar\_d**);\\
void vsip\_cblockrelease\_f(vsip\_cblock\_f*,\\*\hspace{.7cm}vsip\_scalar\_bl,vsip\_scalar\_f**, vsip\_scalar\_f**);\\
vsip\_scalar\_bl* vsip\_blockrelease\_bl(\\*\hspace{.7cm}vsip\_block\_bl*, vsip\_scalar\_bl);\\
vsip\_scalar\_d* vsip\_blockrelease\_d(\\*\hspace{.7cm}vsip\_block\_d*, vsip\_scalar\_bl);\\
vsip\_scalar\_f* vsip\_blockrelease\_f(\\*\hspace{.7cm}vsip\_block\_f*, vsip\_scalar\_bl);\\
vsip\_scalar\_i* vsip\_blockrelease\_i(\\*\hspace{.7cm}vsip\_block\_i*, vsip\_scalar\_bl);\\
vsip\_scalar\_si* vsip\_blockrelease\_si(\\*\hspace{.7cm}vsip\_block\_si*, vsip\_scalar\_bl);\\
vsip\_scalar\_uc* vsip\_blockrelease\_uc(\\*\hspace{.7cm}vsip\_block\_uc*, vsip\_scalar\_bl);\\
vsip\_scalar\_vi* vsip\_blockrelease\_mi(\\*\hspace{.7cm}vsip\_block\_mi*, vsip\_scalar\_bl);\\
vsip\_scalar\_vi* vsip\_blockrelease\_vi(\\*\hspace{.7cm}vsip\_block\_vi*, vsip\_scalar\_bl);\\
\end{tabular}\\
\begin{tabular}[H]{l}
\multicolumn{1}{c}{\rmfamily \bfseries Rebind New User Memory To Block Object\vspace{.1cm}}\\ \hline
void vsip\_cblockrebind\_d(vsip\_cblock\_d*,\\*\hspace{.7cm}vsip\_scalar\_d* const, vsip\_scalar\_d* const,\\*\hspace{.7cm}vsip\_scalar\_d**, vsip\_scalar\_d**);\\
void vsip\_cblockrebind\_f(\\*\hspace{.7cm}vsip\_cblock\_f*, vsip\_scalar\_f* const,\\*\hspace{.7cm}vsip\_scalar\_f* const, vsip\_scalar\_f**, vsip\_scalar\_f**);\\
vsip\_scalar\_bl* vsip\_blockrebind\_bl(\\*\hspace{.7cm}vsip\_block\_bl*, vsip\_scalar\_bl* const);\\
vsip\_scalar\_d* vsip\_blockrebind\_d(\\*\hspace{.7cm}vsip\_block\_d*, vsip\_scalar\_d* const);\\
vsip\_scalar\_f* vsip\_blockrebind\_f(\\*\hspace{.7cm}vsip\_block\_f*, vsip\_scalar\_f* const);\\
vsip\_scalar\_i* vsip\_blockrebind\_i(\\*\hspace{.7cm}vsip\_block\_i*, vsip\_scalar\_i* const);\\
vsip\_scalar\_si* vsip\_blockrebind\_si(\\*\hspace{.7cm}vsip\_block\_si*, vsip\_scalar\_si* const);\\
vsip\_scalar\_uc* vsip\_blockrebind\_uc(\\*\hspace{.7cm}vsip\_block\_uc*, vsip\_scalar\_uc* const);\\
vsip\_scalar\_vi* vsip\_blockrebind\_mi(\\*\hspace{.7cm}vsip\_block\_mi*, vsip\_scalar\_vi* const);\\
vsip\_scalar\_vi* vsip\_blockrebind\_vi(\\*\hspace{.7cm}vsip\_block\_vi*, vsip\_scalar\_vi* const);\\\\
\end{tabular}\\
\begin{tabular}[H]{l}
\multicolumn{1}{c}{\rmfamily \bfseries Return Pointer To User Memory.\vspace{.1cm}}\\ \hline
void vsip\_cblockfind\_d(const vsip\_cblock\_d*,\\*\hspace{.7cm}vsip\_scalar\_d**, vsip\_scalar\_d**);\\
void vsip\_cblockfind\_f(const vsip\_cblock\_f*,\\*\hspace{.7cm}vsip\_scalar\_f**, vsip\_scalar\_f**);\\
vsip\_scalar\_bl* vsip\_blockfind\_bl(const vsip\_block\_bl*);\\
vsip\_scalar\_d* vsip\_blockfind\_d(const vsip\_block\_d*);\\
vsip\_scalar\_f* vsip\_blockfind\_f(const vsip\_block\_f*);\\
vsip\_scalar\_i* vsip\_blockfind\_i(const vsip\_block\_i*);\\
vsip\_scalar\_si* vsip\_blockfind\_si(const vsip\_block\_si*);\\
vsip\_scalar\_uc* vsip\_blockfind\_uc(const vsip\_block\_uc*);\\
vsip\_scalar\_vi* vsip\_blockfind\_mi(const vsip\_block\_mi*);\\
vsip\_scalar\_vi* vsip\_blockfind\_vi(const vsip\_block\_vi*);\\
\end{tabular}
}

\afuncT{vmmul}{Vector-Matrix elementwise multiply by row or column.}{binaryOperations}
\\\cvsiplh
\afh
{\ttfamily
\\\hspace*{.04\textwidth}\begin{tabular}[H]{l}
void vsip\_cvmmul\_d(const vsip\_cvview\_d*,\\*\hspace*{1cm}const vsip\_cmview\_d*, vsip\_major, const vsip\_cmview\_d*);\Bs\\
void vsip\_cvmmul\_f(const vsip\_cvview\_f*,\\*\hspace*{1cm}const vsip\_cmview\_f*, vsip\_major, const vsip\_cmview\_f*);\Bs\\
void vsip\_vmmul\_d(const vsip\_vview\_d*,\\*\hspace*{1cm}const vsip\_mview\_d*, vsip\_major, const vsip\_mview\_d*);\Bs\\
void vsip\_vmmul\_f(const vsip\_vview\_f*,\\*\hspace*{1cm}const vsip\_mview\_f*, vsip\_major, const vsip\_mview\_f*);\Bs\\
\end{tabular}
}
\\\pyjvsiph
\afunc{Window}{Window or Data Taper functions \ref{tab:unaryOperations}}
\\\cvsiplh
\\\pyjvsiph


 afunc{invclip}{{Boolean or bitwise "Exclusive OR" operation for integer and boolean views.}
\\\cvsiplh
\\\pyjvsiph

\chapter{Introduction to JVSIP Programming}
\section*{Introduction}
\section*{Support Functions}
\subsection*{Block Creation}
\subsection*{Vector Creation}
\subsection*{Other methods of view creation and view modification}
\subsection*{Viewing the Real and Imaginary portions of a Complex Vector}
\section*{VSIPL Input and Output Methods}

%\chapter{Introduction to VSIPL Matrices}
\section*{Introduction}
\section*{Matrix Fundamentals}
\section*{Simple Matrix Manipulations}

\chapter{Introduction to Vector Index Views, Boolean views, Gather, Scatter and Indexbool}
\section*{Introduction}\addcontentsline{toc}{section}{Introduction}

\chapter{Signal Processing Functionality}
\section*{Introduction}
\section*{Window}
VSIPL provides functions to create Blackman, Chebyshev, Hanning and Kaiser windows. Unlike most functions in C VSIPL the window creation routines do not use an already created vector and fill it. Instead they actually create a block, allocate data for the block, create a unit stride full length vector on the block, fill the vector with the window coefficients, and then return the pointer to the vector view. The return value will be \ilCode{NULL} on an allocation failure.

For pyJvsip the windows are defined as a method on a view so the functionality, from a user perspective, is to create a vector of a certain type and length and then fill the vector with a window.  Size information are taken from the calling view. Under the covers the C VSIPL window functions are used so a copy is actually taking place to meet the requirements of pyJvsip.

\newpage
\cex
\inputminted[linenos=true,resetmargins=true]{c}{./c_examples/example18.c}
\newpage
\section*{Convolution, Correlation and FIR Filtering}
\section*{Fourier Transforms}

\chapter{Linear Algebra}
\addcontentsline{toc}{section}{Introduction}
\section*{Introduction}
VSIPL specifies support for standard matrix operations such as matrix products,
methods to solve the standard matrix equation $A \vec{x} = \vec{b}$, and methods to solve least squares problems. VSIPL hides the decomposition of matrices in objects. So in addition to standard matrix products, special functions for doing matrix products with decomposition matrices are provided.

We note that although vectors are treated as column vectors in equations, VSIPL vector views have only one stride and so the action of the vector within the function is defined only by the function definition.

In general all matrix views passed into a function are defined as type const. This means that the area of the block mapped by the view does not change inside of the function call. For some of the defined in place operations where the input and output are defined by the same view the input matrix size may be different than that required by the output data. For these cases the strides of the input view define where the output data is placed. The first element of the output data replaces the first element of the input data. The author recommends defining a view of the output data space for convenience. For a couple of cases the output data space may be bigger than the input data space. Defining an output data view will ensure that the strides of the input view and the size of the block are sufficient to hold the output data. 
\addcontentsline{toc}{section}{Simple Matrix-Matrix and Vector-Matrix Operations}
\section*{Simple Matrix-Matrix and Vector-Matrix Operations}
\section*{Simple Solvers}\addcontentsline{toc}{section}{Simple Solvers}
\section*{LU Decomposition}\addcontentsline{toc}{section}{LU Decomposition}
\section*{Cholesky Decompostion}\addcontentsline{toc}{section}{Cholesky Decompostion}
\section*{QR Decompostion}\addcontentsline{toc}{section}{QR Decompostion}
\section*{Singular Value Decomposition}\addcontentsline{toc}{section}{Singular Value Decomposition}


\printindex
\end{document}