\clearpage
\section*{Depth, Shape, Precision; VSIPL Naming}\addcontentsline{toc}{section}{Depth, Shape, Precision; VSIPL Naming}
In order to understand \cvl one needs to understand something about the convention used when naming functions, types, structures, scalars, etc. in \cvl. This also will help one understand some of the reasons behind the \ttbf{block} and \ttbf{view} structures in \cvl. I try to maintain the same conventions in this document and extend them to cover \pyjv type strings.
%
\subsection*{Depth}\addcontentsline{toc}{subsection}{Depth}
A scalar element has a \ttbf{depth}.  For VSIPL this is pretty simple.  It is either complex or real. I suppose in the future it is possible other scalar depths could be defined. For instance a scalar defining a pixel in an image might have red, green, blue components. 

Note that \ttbf{depth} is an attribute of a \ttbf{block}. 
%
\subsection*{Precision}\addcontentsline{toc}{subsection}{Precision}
Precision indicates how accurate the numbers are. In C this would be indicated by \ilCode{float}, \ilCode{double}, \ilCode{int}, etc. \jv only supports standard ANSI C89 precisions but the naming conventions for the VSIP specification allow for just about any precision to be be declared if an implementation wants to support it.

Note that \ttbf{precision} is an attribute of a \ttbf{block}
%
\subsection*{Shape}\addcontentsline{toc}{subsection}{Shape}
A \ttbf{view} defines the \ttbf{shape} of a VSIPL object. A \ttbf{block} is basically an abstract notion of memory storage. It has a \ttbf{depth} and a \ttbf{precision} and provides to a view a linear array of scalar elements.  How the elements are defined on the underlying memory of the compute device is implementation dependent. The \ttbf{view} then places a shape on the block allowing one to access the data as a vector, matrix, or tensor. 

So the \ttbf{shape} is an attribute of the \ttbf{view}. Views are basically index sets.
%
\subsection*{Function Naming for \cvl}\addcontentsline{toc}{subsection}{Function Naming for \cvl}
\subsubsection*{Depth Affix}\addcontentsline{toc}{subsubsection}{Depth Affix}
Generally the prefix \ttbf{c} is used to indicate complex and the prefix \ttbf{r} is used to indicate real. For real the precision is frequently understood with no \ttbf{r} except in some cases where both real and complex are needed. For instance \ilCode{vsip\_vadd\_f} is for real vectors and \ilCode{vsip\_cvadd\_f} is for complex vectors. The function \ttbf{add} has been defined to allow for adding a real vector to a complex vector resulting in \ilCode{vsip\_rcvadd\_f}.

We also have an \ttbf{mi} depth type which goes in the precision place-holder. This is a matrix index type the scalar of which is defined as a structure in the \cvl specification (similar to the way complex is defined).
%
\subsubsection*{Precision Affix}\addcontentsline{toc}{subsubsection}{Precision Affix}
There are many precisions available for use in VSIPL. The ones used in \jv are contained in table \ref{tab:jvsipPrecisions}.  

We note that the matrix index has elements that are the same precision as the vector index. The matrix index comes in the precision place when naming but it is much like the complex type and actually indicates a \ttbf{depth}.

The type \ttbf{ue32} is used in the definition of \ttbf{VSIPL} random numbers. There are no blocks defined for it, only a scalar.  For JVSIP this is defined to be an \ilCode{unsigned int} which normally is 32 bits long; but I don't think this is required by the C89 specification. So in general the declaration of this type (in \ttbf{vsip.h}) will be implementation dependent.
\begin{table}[H]
\caption{Precision Affix in JVSIP}
\label{tab:jvsipPrecisions}
\begin{center}
\begin{tabular}{|l|l|l}
Precision & Affix  & Comment\\ \hline
float & \ttbf{f} & standard c \ilCode{float}\\
double & \ttbf{d} & standard c \ilCode{double}\\
int & \ttbf{i} & standard c signed \ilCode{int}\\
short & \ttbf{si} & standard c signed \ilCode{short int}\\
unsigned char & \ttbf{uc} & standard c \ilCode{unsigned char} \\
implementation dependent & \ttbf{vi} & Vector Index. For \jv \\* & &\ilCode{unsigned long int} \\
\cvl defined & \ttbf{mi} &  Matrix Index\\* & & Actually a \ttbf{depth}\\
exactly 32 bit unsigned & \ttbf{ue32} & For \jv \ilCode{unsigned int} \\
\end{tabular}
\end{center}
\label{default}
\end{table}%

%
\subsubsection*{Shape Affix}\addcontentsline{toc}{subsubsection}{Shape Affix}
Shapes in \cv library are basically indicated by an \ttbf{s} for a scalar, a \ttbf{v} for a vector, a \ttbf{m} for a matrix and a \ttbf{t} for a tensor.

\subsubsection*{Comments on Naming in C VSIPL specification}\addcontentsline{toc}{subsubsection}{Comments on Naming in C VSIPL specification}
In the \cvl specification we have characters in italic font for d (depth), \\*s (shape), p (any precision), f (any float), i (any integer), etc. 

These special characters indicate an overloaded specification telling the implementor what general types may be defined for a function in an implementation.   I don't use these character types in this document because I am talking about an actual implementation; not a specification.  The characters I use indicate what is actually implemented.