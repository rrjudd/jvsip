\afuncT{log10}{Compute the base ten logarithm; An element-wise function.}{elementaryMath}
\\\cvsiplh
\afh
\\\hspace*{.04\textwidth} {
\ttfamily
\begin{tabular}[H]{l}
vsip\_scalar\_d vsip\_log10\_d(vsip\_scalar\_d)\\
vsip\_scalar\_f vsip\_log10\_f(vsip\_scalar\_f)\\
void vsip\_mlog10\_d(\\*\hspace{1cm}const vsip\_mview\_d*, const vsip\_mview\_d*);\\
void vsip\_mlog10\_f(\\*\hspace{1cm}const vsip\_mview\_f*, const vsip\_mview\_f*);\\
void vsip\_vlog10\_d(\\*\hspace{1cm}const vsip\_vview\_d*, const vsip\_vview\_d*);\\
void vsip\_vlog10\_f(\\*\hspace{1cm}const vsip\_vview\_f*, const vsip\_vview\_f*);\\
\end{tabular}
}
\\\pyjvsiph
\viewmthd{yes}{yes}{yes}{inOut.log10}
\apyfunc{yes}{out = log10(in,out)}
\\ \hspace*{1.2cm}\parbox{10.8cm}{\vspace*{.1cm}The \ttbf{log10} function works much the same as the C VSIPL version except that a convenience pointer to the output view is returned. This may be done in-place if \ttbf{in==out}.}
