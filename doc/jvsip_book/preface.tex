\chapter{Preface}
This book describes the functionality of the JVSIP implementation of the Vector/Signal/Image processing (VSIP) Library (VSIPL).  JVSIP includes all the functionality of the TASP Core Plus implementation (TVCPP) developed by the author as part of the the TASP (Tactical Advanced Signal Processing) COE (Common Operating Environment) effort.  

After I retired in 2006 I forked the TVCPP implementation to a new implementation I call JVSIP where the J is the first letter of my last name.  I wanted a mechanism to continue development and support of the VSIPL effort, and I wanted interested parties to be able to access my work. TASP is long gone and funding by traditional government channels seems to have dried up. To make my work available to the community I placed JVSIP on  \href{https://github.com/rrjudd/jvsip?}{{github}}. 

Despite a lack of traditional funding VSIPL trudges on under the guise of \href{http://www.omg.org/spec/VSIPL/}{{OMG}}. Also included on github are the development site for the \href{https://github.com/vsip/specs/tree/master/vsipl}{{OMG specification}}, and an open source implementation of \href{https://github.com/openvsip/openvsip}{{VSIPL++}}.  

Documentation is hard to do, generally lags behind implementations, and documents are always under development and seldom finished.  For this reason most documentation the author does is labeled as \emph{draft} even though I may not plan to get around to doing a \emph{non-draft} version.

Early in 2001 the author wrote a document describing the current functionality of the TASP VSIPL Core Plus implementation called \emph{TASP VSIPL Core Plus} which includes many examples and a fairly good overview of C VSIPL functionality. It was done in a hurry with little editing and was, of course, a draft.  The document needs updating and editing but it was done originally on a Sun workstation using Framemaker; a word processing package the author liked very much. Unfortunately it was not long before Framemaker effectively died (It is still out there as part of adobe but for my purposes it is dead), the Sun workstation was replaced by a PC and Microsoft Word became the only way to do things. So the original source of the \emph{TASP VSIPL Core Plus} book was basically lost and only the PDF document remains.  Updating the document without the original tools is difficult so was never done.

I have decided re-do the previous \emph{TASP VSIPL Core Plus} as the \emph{JVSIP User Manual}. The contents will look a lot like the previous book but will include some editing and a lot of new information including information on how to use pyJvsip.  Although this is a fresh, start since the source for the previous document is gone, the PDF content will be freely copied and pasted into this new document; many of the examples will be the same except updated and versions written in pyJvsip; and the author considers this document to be an update of the original. 

The new document will be done using LaTeX on a Mac with the MacTeX distribution supplying the tools and underlying environment.  LaTeX is not user friendly or wysiwyg and the author is not expert. But TeX is persistent and portable. All the necessary tools for doing a book are there and the source is all text so easily maintained. 

This book is not a copy of, nor a replacement for, the VSIPL specification.
\subsection*{VSIPL Forum - A short history}
In the early 90's signal processing boards by Mercury, Sky, CSPI, and others were becoming popular for use by Government  programs with compute heavy software.  Each company produced their own proprietary signal processing libraries for use on their boards. This caused concern of vendor lock-in because software would need to be rewritten every time a new board procurement was done.

The TASP group was at this time trying to do a bulk contract for signal processing boards similar to the Tactical Advanced Computer (TAC) contract.  In order to progress on the signal processing specification the TASP group was told they needed to specify a common operating environment for the boards. 

At about this time DARPA also saw a need for a signal processing library and awarded a contract to Hughes Research Laboratory to run a forum to produce an open signal processing specification for signal and image processing; and to write a reference implementation of said specification. 

TASP decided to participate in the forum as well as many other industry and university participants whose names may be found in the introductory pages to the original VSIPL document.

To make a long story short the DARPA VSIPL project and TASP have ended but the VSIPL forum has continued on in one form or another, currently with the OMG.  

\subsection*{Success Story?}
The title of this section is a question yet to be answered. Although VSIPL, as a specification, has been a minor success it may yet die for lack of support by the people who started the effort. There is only so much that volunteer efforts (such as JVSIP) can do and the cost/payback for companies developing VSIPL code is problematic without a big, paying, customer base.  Lacking any sort of funding or policy by DOD to support reference implementation development, specification development, or commercial vendor buy in by VSIPL requirements in procurement documents means that the effort may yet wither on the vine. 
\subsection*{Code History}
The original code basis for the C VSIP library implementation was a very early pre-alpha (incomplete) version of the VSIPL Reference library produced by Hughes Research Laboratory of Malibu, California in December of 1997. The original HRL release was template based using \href{https://www.gnu.org/software/m4/m4.html}{{m4}} as a code generator. The generated library was very slow and not really suitable for writing example codes of real world problems.  HRL was never successful in actually completing a complete reference implementation of VSIPL.

I was part of the TASP group and it was important to have an implementation suitable for writing real world applications.  In addition at the time I did not understand m4 and, I realize now, was only marginally competent at programing in C. I copied the generated C source files from the m4 base and modified them directly. The original was slow mostly because of the method of programing. They would start with a scalar function which would be called by a general element wise function which would be called by the actual function. This was very confusing to me so I just flattened everything out so the actual function call did all the work. This produced an enormous speed-up in example codes which was what I needed.

Over time many changes were made to the TASP implementation to add performance, and to keep up with the changing VSIPL specification. I learned a lot about C programing as time went on; and some of what I learned made it into the library. Eventually the TASP implementation became a de facto reference implementation.

I suspect HRL was never successful because of a lack of funding. For a company like HRL to produce a library as extensive as VSIPL would be an expensive proposition. For various reasons I found myself with good funding and not much direction for a couple of years. The VSIPL library was similar to some codes I had wanted to write anyway; and I had just completed a masters degree at UW with signal processing as my main study. So, with no tasking from above and freedom to set my own agenda, I worked like crazy for about a year and eventually had a code base extensive enough, and well tested enough, that folks could use it.

We are approaching 20 years since the original code base and little if any of the original HRL code remains in the library.  The original mostly contained support functions and simple element wise operations. Changes to the specification caused many changes to the support functions, and (as previously mentioned) the element wise functions provided by the HRL library were so slow as to be unusable for demonstration purposes. Most of what remains are the odd copyright statement. Except for (perhaps) function prototypes (defined by the spec) I would be surprised if any of the underlying code is from the original.

