\chapter{Introduction To JVSIP}
\section*{Introduction}\addcontentsline{toc}{section}{Introduction}
If you are new to \ttbf{VSIPL} and find you are confused by the various acronyms, or you find some of the terms here are unfamiliar, try reading the Preface chapter above. It contains information about the origins and meaning of \jv{} and \ttbf{VSIPL} and may reduce the confusion factor for the person new to \ttbf{VSIPL}.

First the big picture.

The \jv{} distribution contains a C signal processing library implementing (most of) the \ttbf{C VSIPL} specification. \jv{} also contains a python vsip module encapsulating the \ttbf{C VSIP} library.  Once the {\ttbf{vsip}} module was done a new module called {\ttbf{vsiputils}} was done to provide function overloading and to reduce the name space.  One of the main purposes of the {\ttbf{vsiputils}} module was to help me to learn python programing; but a lot of work was done there and the module still survives although it may go away in the future.  Eventually I got around to defining python classes and created the \pyjv{} module.  In this document we mainly treat the python interface defined in the \pyjv{} module but you should be aware other modules exist.

The distribution is available on \href{https://github.com/rrjudd/jvsip?}{{github}}. The distribution only contains source code. You will need a C compiler (supporting C89) to make the C Library. You will need the same C compiler, a python distribution (2.7), and a free open source package called \href{http://www.swig.org}{SWIG} to help encapsulate C code into python modules. The C library and the Python modules are independent except the same C source code is used for both.

Chapter one of this book will contain some basic information and an example in figure \ref{fig:addTwoVectors}. This chapter and chapter three are for a quick start for readers who want to get started programing. Chapter two is mostly a reference chapter containing C and \pyjv{} functions and usage information.  Chapter four will delve more deeply into the \ttbf{block} and \ttbf{view} structures.  Following chapters cover more complicated functions for signal processing and linear algebra.

%\subsection*{Example {\arabic{cexctr}}:  Add two vectors}
\clearpage
\begin{figure}[t]
\caption{Add Two Vectors}
\label{fig:addTwoVectors}
\setlength{\parskip}{.25cm}
\begin{minipage}[t][20cm][t]{.475\textwidth}
{\begin{center} \bfseries{c VSIPL} \end{center}}\setlength{\parskip}{.25cm}
\inputminted[linenos=true,resetmargins=true,xleftmargin=.75cm,fontfamily=tt, fontsize= \small]{c}{./c_examples/example1.c}
\end{minipage}
\begin{minipage}[t][10cm][t]{.475\textwidth}
{\begin{center} \bfseries{\pyjv} \end{center}}\setlength{\parskip}{.25cm}
\inputminted[linenos=true,resetmargins=true,xleftmargin=.75cm,fontfamily=tt,fontsize=\small]{python}{./pyJvsip_examples/example1a.py}\setlength{\parskip}{.25cm}
{\begin{center} \bfseries{Polymorphism with \pyjv} \end{center}}\setlength{\parskip}{.25cm}
\inputminted[linenos=true,resetmargins=true,xleftmargin=.75cm,fontfamily=tt,fontsize=\small]{python}{./pyJvsip_examples/example1b.py}
\end{minipage}
\end{figure}

\section*{\ttbf{C VSIP} versus \pyjv}\addcontentsline{toc}{section}{C VSIP versus \pyjv}
In this section I will make some comments about the difference between programing with c and the \ttbf{C VSIP} library and programing with \pyjv{} in the python environment. 

The VSIPL library has support mechanisms for blocks and views. The \pyjv{} module has support for blocks and views. The block and view in \pyjv{} are instantiations of class definitions. The block and view in c VSIPL are opaque structures created with c VSIPL support functions defined for that purpose. This means a \pyjv{} view is not the same as a VSIPL view even though I may write about them as if they are the same object. In general VSIPL objects (LUD, FFT, matrix view, etc.) created with create functions are contained inside a \pyjv{} object as an instance variable.

The VSIPL library has a requirement for initialization and finalization.  PyJvsip is written on top of the c VSIPL implementation so we still need to initialize it and finalize it. However for \pyjv{} I have abstracted that away so that when a \pyjv{} object is created the initialization of the object checks to see if C VSIPL has been initialized and will call \ilCode{vsip\_init} if it needs to. There is a special class object which keeps track of \pyjv{} objects and when no \pyjv{} objects are left then it calls \ilCode{vsip\_finalize}. So \pyjv{} has no explicit initialization/finalization other than the required python import statement.  

To avoid memory leaks there is a requirement for destruction of allocated objects after they are no longer needed in C VSIPL. For deallocation of VSIPL objects contained within a \pyjv{} object; when a \pyjv{} object has no reference left the python garbage collector will call the delete method.  This will destroy any c VSIPL objects that have been allocated for use with the \pyjv{} object. So \pyjv{} has no explicit destroy functions.

\subsection*{Polymorphism}\addcontentsline{toc}{subsection}{Polymorphism}
The encapsulation of c VSIPL using SWIG adds type information to the VSIPL python objects. Using this information, and information added to \pyjv{} objects, as keys for python dictionaries allows VSIPL to become polymorphic. Most functions and methods in \pyjv{} determine the underlying functionality using type information extracted from the calling object. Not every combination will necessarily work.  Someplace under the covers everything must be covered.  However it is generally possible to program in \pyjv{} in a manner so that once the initial type has been chosen the rest of the code is generic even to the point of covering both real and complex. 
\subsubsection*{Example}\addcontentsline{toc}{subsubsection}{Example}
We show a simple example in figure \ref{fig:addTwoVectors} where we add two vectors both in C VSIPL on the left and in python on the right using \pyjv. 

\clearpage
\section*{Depth, Shape, Precision; VSIPL Naming}\addcontentsline{toc}{section}{Depth, Shape, Precision; VSIPL Naming}
In order to understand \cvl one needs to understand something about the convention used when naming functions, types, structures, scalars, etc. in \cvl. This also will help one understand some of the reasons behind the \ttbf{block} and \ttbf{view} structures in \cvl. I try to maintain the same conventions in this document and extend them to cover \pyjv type strings.
%
\subsection*{Depth}\addcontentsline{toc}{subsection}{Depth}
A scalar element has a \ttbf{depth}.  For VSIPL this is pretty simple.  It is either complex or real. I suppose in the future it is possible other scalar depths could be defined. For instance a scalar defining a pixel in an image might have red, green, blue components. 

Note that \ttbf{depth} is an attribute of a \ttbf{block}. 
%
\subsection*{Precision}\addcontentsline{toc}{subsection}{Precision}
Precision indicates how accurate the numbers are. In C this would be indicated by \ilCode{float}, \ilCode{double}, \ilCode{int}, etc. \jv only supports standard ANSI C89 precisions but the naming conventions for the VSIP specification allow for just about any precision to be be declared if an implementation wants to support it.

Note that \ttbf{precision} is an attribute of a \ttbf{block}
%
\subsection*{Shape}\addcontentsline{toc}{subsection}{Shape}
A \ttbf{view} defines the \ttbf{shape} of a VSIPL object. A \ttbf{block} is basically an abstract notion of memory storage. It has a \ttbf{depth} and a \ttbf{precision} and provides to a view a linear array of scalar elements.  How the elements are defined on the underlying memory of the compute device is implementation dependent. The \ttbf{view} then places a shape on the block allowing one to access the data as a vector, matrix, or tensor. 

So the \ttbf{shape} is an attribute of the \ttbf{view}. Views are basically index sets.
%
\subsection*{Function Naming for \cvl}\addcontentsline{toc}{subsection}{Function Naming for \cvl}
\subsubsection*{Depth Affix}\addcontentsline{toc}{subsubsection}{Depth Affix}
Generally the prefix \ttbf{c} is used to indicate complex and the prefix \ttbf{r} is used to indicate real. For real the precision is frequently understood with no \ttbf{r} except in some cases where both real and complex are needed. For instance \ilCode{vsip\_vadd\_f} is for real vectors and \ilCode{vsip\_cvadd\_f} is for complex vectors. The function \ttbf{add} has been defined to allow for adding a real vector to a complex vector resulting in \ilCode{vsip\_rcvadd\_f}.

We also have an \ttbf{mi} depth type which goes in the precision place-holder. This is a matrix index type the scalar of which is defined as a structure in the \cvl specification (similar to the way complex is defined).
%
\subsubsection*{Precision Affix}\addcontentsline{toc}{subsubsection}{Precision Affix}
There are many precisions available for use in VSIPL. The ones used in \jv are contained in table \ref{tab:jvsipPrecisions}.  

We note that the matrix index has elements that are the same precision as the vector index. The matrix index comes in the precision place when naming but it is much like the complex type and actually indicates a \ttbf{depth}.

The type \ttbf{ue32} is used in the definition of \ttbf{VSIPL} random numbers. There are no blocks defined for it, only a scalar.  For JVSIP this is defined to be an \ilCode{unsigned int} which normally is 32 bits long; but I don't think this is required by the C89 specification. So in general the declaration of this type (in \ttbf{vsip.h}) will be implementation dependent.
\begin{table}[H]
\caption{Precision Affix in JVSIP}
\label{tab:jvsipPrecisions}
\begin{center}
\begin{tabular}{|l|l|l}
Precision & Affix  & Comment\\ \hline
float & \ttbf{f} & standard c \ilCode{float}\\
double & \ttbf{d} & standard c \ilCode{double}\\
int & \ttbf{i} & standard c signed \ilCode{int}\\
short & \ttbf{si} & standard c signed \ilCode{short int}\\
unsigned char & \ttbf{uc} & standard c \ilCode{unsigned char} \\
implementation dependent & \ttbf{vi} & Vector Index. For \jv \\* & &\ilCode{unsigned long int} \\
\cvl defined & \ttbf{mi} &  Matrix Index\\* & & Actually a \ttbf{depth}\\
exactly 32 bit unsigned & \ttbf{ue32} & For \jv \ilCode{unsigned int} \\
\end{tabular}
\end{center}
\label{default}
\end{table}%

%
\subsubsection*{Shape Affix}\addcontentsline{toc}{subsubsection}{Shape Affix}
Shapes in \cv library are basically indicated by an \ttbf{s} for a scalar, a \ttbf{v} for a vector, a \ttbf{m} for a matrix and a \ttbf{t} for a tensor.

\subsubsection*{Comments on Naming in C VSIPL specification}\addcontentsline{toc}{subsubsection}{Comments on Naming in C VSIPL specification}
In the \cvl specification we have characters in italic font for d (depth), \\*s (shape), p (any precision), f (any float), i (any integer), etc. 

These special characters indicate an overloaded specification telling the implementor what general types may be defined for a function in an implementation.   I don't use these character types in this document because I am talking about an actual implementation; not a specification.  The characters I use indicate what is actually implemented.