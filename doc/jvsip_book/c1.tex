\chapter{Introduction To JVSIP}
\section*{Introduction}
If you are new to VSIPL or find you are too confused by the various acronyms or some of the terms here are unfamiliar read the Preface chapter above. It contains information about the origins and meaning of JVSIP and VSIPL.

First the big picture.

The JVSIP distribution contains a C signal processing library implementing (most of) the c VSIPL specification. It also contains a python vsip module encapsulating the C library.  Once the {\bf{vsip}} module was done a new module called {\bf{vsiputils}} was done to provide function overloading and to reduce the name space.  One of the main purposes of the {\bf{vsiputils}} module was to help me to learn python programing; but a lot of work was done there and the module still survives although it may go away in the future.  Eventually I got around to defining python classes and created the {\bf{pyJvsip}} module.  In this document we mainly treat the python interface defined in the {\bf{pyJvsip}} module but you should be aware other modules exist.

The distribution is available on \href{https://github.com/rrjudd/jvsip?}{{github}}. The distribution only contains source code. You will need a C compiler (supporting C89) to make the C Library. You will need the same C compiler, a python distribution (2.7), and a free open source package called \href{http://www.swig.org}{SWIG} to help encapsulate C code into python modules. The C library and the Python modules are independent except the same C source code is used for both.

Chapter one of this book will contain information and examples for a quick start for readers who want to get started programing.  Chapter two is mostly a reference chapter containing C and pyJvsip functions and usage information.  Following chapter delve deeper and cover the more complicated functions for signal processing and linear algebra.

\section*{C VSIP versus pyJvsip}
In this section I will make some comments about the difference between programing with c and the C VSIPL library and programing with pyJvsip in the python environment. 

The VSIPL library has support mechanisms for blocks and views. The pyJvsip module has support for blocks and views. The block and view in pyJvsip are instantiations of class definitions. The block and view in c VSIPL are opaque structures created with c VSIPL support functions defined for that purpose. This means a pyJvsip view is not the same as a VSIPL view even though I may write about them as if they are the same object. In general VSIPL objects (LUD, FFT, matrix view, etc.) created with create functions are contained inside a pyJvsip object as an instance variable.

The VSIPL library has a requirement for initialization and finalization.  PyJvsip is written on top of the c VSIPL implementation so we still need to initialize it and finalize it. However for pyJvsip I have abstracted that away so that when a pyJvsip object is created the initialization of the object checks to see if C VSIPL has been initialized and will call \ilCode{vsip\_init} if it needs to. There is a special class object which keeps track of pyJvsip objects and when no pyJvsip objects are left then it calls \ilCode{vsip\_finalize}. So pyJvsip has no explicit initialization/finalization other than the required python import statement.  

To avoid memory leaks there is a requirement for destruction of allocated objects after they are no longer needed in C VSIPL. For deallocation of VSIPL objects contained within a pyJvsip object; when a pyJvsip object has no reference left the python garbage collector will call the delete method.  This will destroy any c VSIPL objects that have been allocated for use with the pyJvsip object. So pyJvsip has no explicit destroy functions.

\subsection*{Polymorphism}
The encapsulation of c VSIPL using SWIG adds type information to the VSIPL python objects. Using this information, and information added to pyJvsip objects, as keys for python dictionaries allows VSIPL to become polymorphic. Most functions and methods in pyJvsip determine the underlying functionality using type information extracted from the calling object. Not every combination will necessarily work.  Someplace under the covers everything must be covered.  However it is generally possible to program in pyJvsip in a manner so that once the initial type has been chosen the rest of the code is generic even to the point of covering both real and complex. 

\section*{Example}
We do a simple example; see figure 1. 

%\subsection*{Example {\arabic{cexctr}}:  Add two vectors}
\begin{figure}[t]
\caption{Add Two Vectors}
\setlength{\parskip}{.25cm}
\begin{minipage}[t][5cm][t]{.5\textwidth}
{\begin{center} \bfseries{c VSIPL} \end{center}}\setlength{\parskip}{.25cm}
\inputminted[linenos=true,resetmargins=true,xleftmargin=.75cm,fontfamily=tt, fontsize= \tiny]{c}{./c_examples/example1.c}
\end{minipage}
\begin{minipage}[t][5cm][t]{.5\textwidth}
{\begin{center} \bfseries{pyJvsip} \end{center}}\setlength{\parskip}{.25cm}
\inputminted[linenos=true,resetmargins=true,xleftmargin=.75cm,fontfamily=tt,fontsize=\tiny]{python}{./pyJvsip_examples/example1a.py}\setlength{\parskip}{.25cm}
{\begin{center} \bfseries{Polymorphism with pyJvsip} \end{center}}\setlength{\parskip}{.25cm}
\inputminted[linenos=true,resetmargins=true,xleftmargin=.75cm,fontfamily=tt,fontsize=\tiny]{python}{./pyJvsip_examples/example1b.py}
\end{minipage}
\end{figure}